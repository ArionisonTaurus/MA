\documentclass[a4paper, 11pt]{report}
\usepackage[utf8]{inputenc}
\usepackage[ngerman]{babel}
\usepackage{amsmath, amssymb, amsthm}
\usepackage{paralist}
\usepackage{lmodern}
\usepackage[T1]{fontenc}
\usepackage[arrow, matrix, curve]{xy}
\usepackage{graphicx}
\usepackage[percent]{overpic}
\usepackage{cite}
\usepackage{caption}
\usepackage[onehalfspacing]{setspace}
\usepackage{hyperref}
\usepackage{titlesec}
\usepackage{texilikechaps}
\usepackage{color}

\captionsetup[table]{labelformat=empty}


\newcommand{\setk}{\Bbbk}
\newcommand{\ldot}{\,.\,}
\newcommand{\fa}{~\forall}
\newcommand{\ex}{~\exists}
\newcommand{\fA}{\mathcal{A}}
\newcommand{\fB}{\mathcal{B}}
\newcommand{\fC}{\mathcal{C}}
\newcommand{\fM}{\mathcal{M}}
\newcommand{\fN}{\mathcal{N}}
\newcommand{\fF}{\mathcal{F}}
\newcommand{\fU}{\mathcal{U}}
\newcommand{\lingua}{\mathcal{L}}
\newcommand{\lld}{\mathcal{L}^{ld}}
\newcommand{\lf}{\mathcal{L}^f}
\newcommand{\lfc}{\mathcal{L}^{f,c}}
\newcommand{\sF}{\ensuremath{\mathcal{F}} }
\newcommand{\monster}{\mathbb{M}} %Monster-Modell-M 
\newcommand{\MM}{\mathbb{M}} %ebenfalls
\newcommand{\gdw}{\Leftrightarrow}
\newcommand{\Mod}{\mathcal{M}} %geschwungenes Modell-M
\newcommand{\Nod}{\mathcal{N}} %das gleiche für N
\newcommand{\leer}{\emptyset}
\newcommand{\In}{\in}
\newcommand{\setN}{\mathbb{N}}
\newcommand{\setZ}{\mathbb{Z}}
\newcommand{\setQ}{\mathbb{Q}}
\newcommand{\setR}{\mathbb{R}}
\newcommand{\setC}{\mathbb{C}}
\newcommand{\setH}{\mathbb{H}}
\newcommand{\Forall}{~\forall}
\newcommand{\Exists}{~\exists}
\newcommand{\abs}[1]{{\left| #1 \right|}}

\DeclareMathOperator{\ld}{ld}
\DeclareMathOperator{\ad}{ad}
\DeclareMathOperator{\acl}{acl}
\DeclareMathOperator{\dcl}{dcl}
\DeclareMathOperator{\tp}{tp}
\DeclareMathOperator{\tq}{T^2}
\DeclareMathOperator{\td}{T^d}
\DeclareMathOperator{\inn}{int}


\def\Ind#1#2{#1\setbox0=\hbox{$#1x$}\kern\wd0\hbox to 0pt{\hss$#1\mid$\hss}
	\lower.9\ht0\hbox to 0pt{\hss$#1\smile$\hss}\kern\wd0}

\def\ind{\mathop{\mathpalette\Ind{}}}

\def\notind#1#2{#1\setbox0=\hbox{$#1x$}\kern\wd0
	\hbox to 0pt{\mathchardef\nn=12854\hss$#1\nn$\kern1.4\wd0\hss}
	\hbox to 0pt{\hss$#1\mid$\hss}\lower.9\ht0 \hbox to 0pt{\hss$#1\smile$\hss}\kern\wd0}

\def\nind{\mathop{\mathpalette\notind{}}}


\theoremstyle{definition}
\newtheorem{theorem}[subsection]{Satz}
\newtheorem{corollary}[subsection]{Folgerung}
\newtheorem{proposition}[subsection]{Proposition}
\newtheorem{definition}[subsection]{Definition}
\newtheorem*{example}{Beispiel}
\newtheorem*{remark}{Bemerkung}
\newtheorem*{remarks}{Bemerkungen}
\newtheorem*{recall}{Erinnerung}
\newtheorem{satz}[subsection]{Satz}
\newtheorem*{satzleer}{Satz}
\newtheorem{kor}[subsection]{Folgerung}
\newtheorem{prop}[subsection]{Proposition}
\newtheorem{lemma}[subsection]{Lemma}
\newtheorem{Def}[subsection]{Definition}
\newtheorem{bsp}[subsection]{Beispiel}
\newtheorem{axiom}[subsection]{Axiom}
\newtheorem{propdef}[subsection]{Proposition/ Definition}
\newtheorem{bemdef}[subsection]{Bemerkung/Definition}
\newtheorem{lemdef}[subsection]{Lemma/Definition}
\newtheorem{theocol}[subsection]{Folgerung/Satz}
\newtheorem*{bem}{Bemerkung}
\newtheorem*{erinn}{Erinnerung}

\makeatletter
\newenvironment{proof2}[1][\proofname]{\par
	\pushQED{\hfill$\blacksquare$}%
	\normalfont \topsep6\p@\@plus6\p@\relax
	\trivlist
	\item\relax
	{\itshape
		#1\@addpunct{.}}\hspace\labelsep\ignorespaces
}{%
	\popQED\endtrivlist\@endpefalse
}
\makeatother

\txisection{chapter}{\normalfont \huge \bfseries }
\txisection{section}{\normalfont \Large \bfseries }

\usepackage[left=3.5cm,right=3cm,top=3.5cm,bottom=5cm]{geometry}
\setlength{\parindent}{0em}

\setlength\abovedisplayskip{4pt}
\setlength\belowdisplayskip{4pt}
\setlength\jot{4pt}

\newcommand{\lra}{\Leftrightarrow}
\newcommand{\xq}{ \bar{x}}

\def\Ind#1#2{#1\setbox0=\hbox{$#1x$}\kern\wd0\hbox to 0pt{\hss$#1\mid$\hss}
	\lower.9\ht0\hbox to 0pt{\hss$#1\smile$\hss}\kern\wd0}
\def\ua{\mathop{\mathpalette\Ind{}}}
\def\notind#1#2{#1\setbox0=\hbox{$#1x$}\kern\wd0
	\hbox to 0pt{\mathchardef\nn=12854\hss$#1\nn$\kern1.4\wd0\hss}
	\hbox to 0pt{\hss$#1\mid$\hss}\lower.9\ht0 \hbox to 0pt{\hss$#1\smile$\hss}\kern\wd0}
\def\nua{\mathop{\mathpalette\notind{}}}


\begin{document}
	\begin{titlepage}

		\centering

		$ $\par

		\vspace{4.5cm}

		\Huge{\textbf{Paare algebraischer Strukturen\\TODO: Titel absprechen}\par}

		\vspace{3cm}

		\large{Masterarbeit von \par}

		\vspace{0.4cm}

		\Large{Max Aaron Vollprecht\par}

		\vspace{0.6cm}

		\large{Betreuer: Prof. Dr. Amador Martin-Pizarro\par}

		\vfill

		\large{\today\par}

		\vspace{0.6cm}

		Albert-Ludwigs-Universität Freiburg im Breisgau\\

		Fakultät für Mathematik und Physik

	\end{titlepage}
\colorbox{green}{Ende Korrektur der Formatierung}
	%!TEX root = DieLoesungAllerMilleniumsprobleme.tex
\addcontentsline{toc}{section}{Vorwort}
\chapter*{Vorwort}
TODO: Hier Vorwort einfügen
	\tableofcontents
	\newpage
	%!TEX root = DieLoesungAllerMilleniumsprobleme.tex
\addcontentsline{toc}{section}{Notation}
\chapter*{Notation}
Im Folgenden seien, wenn nicht weiter erklärt, mit $i,j,k,l,m,n$ immer natürliche Zahlen gemeint, mit $\kappa$ immer unendliche Kardinalzahlen.\\
Oftmals wird nicht zwischen Strukturen und deren Trägermengen unterschieden, insbesondere bei Paaren von Strukturen. Wenn von $\lingua$-Definierbarkeit in einer $\lingua$-Struktur $\fM$ die Rede ist, ist Definierbarkeit mit $\lingua_M$-Formeln gemeint, bei $\lingua_S$-Definierbarkeit für ein $S\subseteq M$ nur Definierbarkeit mit $\lingua_S$-Formeln. Wenn von $\dcl$ und $\acl$ die Rede ist, ist die kleinste Sprache gemeint, falls mehrere verwendet werden.\\
Als Topologie wird die Ordnungstopologie bzw. deren Produkttopologie verstanden, mit \glqq{}Intervall\grqq{} ist ein offenes, nichtleeres Intervall mit Randpunkten in der Struktur oder $\pm\infty$ gemeint. Außerdem sei für $A\prec B$ und $X\subseteq B$ $A$-definierbar in $T$ die Menge $X_A$ die durch dieselbe definierende Formel in $B$ definierbare Menge (für $X\subseteq A$ und $X_B$ analog). Außerdem sei für Relationen $P$ mit \glqq{}$\exists/\forall x\in P(\dots)$\grqq{} die Formel $$\glqq{}\exists x(P(x)\land\dots)/\forall x(P(x)\rightarrow\dots)\grqq{}$$ gemeint und für eine durch $\varphi$ definierbare Menge $X$ mit \glqq{}$x\in X$\grqq{} die Formel $\varphi(x)$.\\
$\abs{\overline{a}}$ soll je nach Kontext unterschiedliches bedeuten, einerseits die Supremumsnorm von $\overline{a}$, andererseits die Anzahl der Einträge. Da das eine ein Element der Struktur ist und das andere eine natürliche Zahl, ist immer klar erkennbar, was gemeint ist. Im Allgemeinen wird auch nicht immer zwischen Tupeln und Elementen unterschieden, außer, wenn das für das Verständnis notwendig ist.\\
Im modelltheoretischen Kontext ungewöhnlich sind Mengen der Form $\{f=g\},\{f>g\}$ und $\{f<g\}$ für zwei Abbildungen $f,g:X\rightarrow Y$. Hierunter sollen die maßtheoretischen Interpretationen dieser Ausdrucksweise $$\{x\in X\mid f(x)=g(x)\},\{x\in X\mid f(x)>g(x)\}\text{ und }\{x\in X\mid f(x)<g(x)\}$$ verstanden werden.
	%!TEX root = DieLoesungAllerMilleniumsprobleme.tex
	\chapter{Paare algebraisch abgeschlossener Körper}
	\section{Algebraische und lineare Disjunktheit von Körpern}
	In diesem Teil richten wir uns im Aufbau etwas nach \cite{Delon} und in manchen Beweisen nach \cite{SergeLang}.
	
    \begin{definition}
    	Gegeben Körperinklusionen $C\subseteq K,L\subseteq M$ in Rautenform, nenne $K$ und $L$ \textbf{linear disjunkt über } $C$, falls alle Basen von $K$ als $C$-Vektorraum auch über $L$ linear unabhängig bleiben. Nenne $K$ und $L$ \textbf{algebraisch disjunkt über} $C$, falls alle Transzendenzbasen von $K$ über $C$ auch algebraisch unabhängig über $L$ bleiben. Schreibe $K\ld_CL$ bzw. $K\ad_CL$.
    \end{definition}
    
    \begin{remark}
    	Es reicht, lineare Disjunktheit für eine Basis zu zeigen. Denn wenn man die lineare Unabhängigkeit über $L$ für eine Basis verliert, verliert man sie per $C$-Basiswechsel auch für alle anderen.
    \end{remark}
    \begin{remark}
    	Es reicht, die Erhaltung der linearen/algebraischen Unabhängigkeit nur für beliebige endliche Mengen zu prüfen. Denn lineare/algebraische Unabhängigkeit einer Menge besteht genau dann, wenn sie für alle endlichen Teilmengen gilt.
    \end{remark}
    \begin{remark}
    	Der Körper $M$ kommt in der Definition nur vor, damit die Rechenoperationen zwischen $K$ und $L$ wohldefiniert sind. Die genaue Wahl ist irrelevant und daher nicht in der Notation berücksichtigt. Wir nehmen für die Zukunft einfach an, dass die Multiplikation klar definiert ist. Es wird sich sowieso herausstellen, dass im Fall $K\ld_CL$ die Operationen eindeutig bestimmt sind.
    \end{remark}
    
    \begin{lemma}\label{Fraktionskörper}
    	Sei $C$ ein Körper und $C\subseteq R,S$ Ringerweiterungen von Integritätsbereichen. Dann gilt $\operatorname{Frac}(R)\ld_C\operatorname{Frac}(S)$ genau dann, wenn linear unabhängige Mengen in $R$ über $C$ auch linear unabhängig über $S$ bleiben.
    \end{lemma}
    \begin{proof}
    	Die Hinrichtung folgt leicht aus $R\subseteq\operatorname{Frac}(R),S\subseteq\operatorname{Frac}(S)$. Für die Rückrichtung seien $r_1x_1^{-1},\dots,r_nx_n^{-1}\in\operatorname{Frac}(R)$ linear unabhängig über $C$, aber linear abhängig über $\operatorname{Frac}(S)$ mit nichttrivialer Linearkombination $$(s_1y_1^{-1})r_1x_1^{-1},\dots,(s_ny_n^{-1})r_nx_n^{-1}=0.$$\newpage
    	Dann gilt aber nach Multiplikation mit $\prod\limits_{i=1}^nx_iy_i\neq0$ die lineare Abhängigkeit über $S$ der über $C$ unabhängigen Elemente $((\prod\limits_{i\neq j}x_i)r_j)_{1,\dots,n}\in R$ mit der folgenden Gleichung: $$0=\sum\limits_{j=1}^n(\prod\limits_{i\neq j}x_i)r_j(\prod\limits_{i\neq j}y_i)s_j$$
    \end{proof}
    
    \begin{remark}
    	Es reicht in obiger Aussage wieder, sich auf endliche Mengen zu beschränken. Alternativ kann man es auch wieder nur für eine $C$-Basis von $R$ zeigen.
    \end{remark}
    
    \begin{lemma}\label{Tensoren}
    	Für Körper $C,K,L$ wie oben gilt $K\ld_CL$ genau dann wenn $K[L]=L[K]\cong K\otimes_CL$ mit kanonischem Isomorphismus, daher ist $\ld$ symmetrisch.
    \end{lemma}
    \begin{proof}
    	Der aufgespannte Ring erfüllt $$K[L]=\{\sum\limits_{(k,l)\in X}kl\mid X\subset K\times L\text{ endlich}\}=L[K].$$
    	Wenn $(k_i)_I,(l_j)_J$ Basen von $K,L$ über $C$ sind, ist $(k_i\otimes l_j)_{i\in I,j\in J}$ eine Basis von $K\otimes_CL$. Der $C$-Homomorphismus $$\sum\limits_{i\in I_0,j\in J_0} c_{ij}k_i\otimes l_j\mapsto \sum\limits_{i\in I_0,j\in J_0} c_{ij}k_il_j$$ für $I_0\subseteq I,J_0\subseteq J$ endlich ist immer surjektiv, da klarerweise $E:=(k_il_j)_{i\in I,j\in J}$ ein Erzeugendensystem von $K[L]$ ist. Er ist injektiv genau dann, wenn $E$ auch linear unabhängig über $C$ ist, also keine Linearkombination $$0=\sum\limits_{i\in I_0,j\in J_0}c_{ij}k_il_j=\sum\limits_{i\in I_0}(\sum\limits_{j\in J_0}c_{ij}l_j)k_i$$ existiert mit $c_{ij}\neq0$ für mindestens ein Paar $(i,j)$. Aber das ist genau dann der Fall, wenn keine $\tilde{c}_i=\sum\limits_{j\in J_0}c_{ij}l_j\in L$ existieren für $i\in I_0$ mit $\tilde{c}_i\neq0$ für mindestens ein $i$ und $0=\sum\limits_{i\in I_0}\tilde{c}_ik_i$, also wenn die $(k_i)_I$ linear unabhängig über $L$ sind.
    \end{proof}
    \begin{remark}
    	 Auch $\ad$ ist symmetrisch, denn $K\ad_CL$ genau dann, wenn $\dim(\overline{k}/C)=\dim(\overline{k}/L)$ für alle $\overline{k}\in K$ im Sinne der $\acl$-Dimension in ACF, also gerade dann, wenn $K$ und $L$ unabhängig über $C$ im modelltheoretischen Sinn als Teilmengen eines algebraisch abgeschlossenen Körpers.
    \end{remark}
    
    \begin{definition}
    	\begin{itemize}
    		\item Eine Körpererweiterung $K\subseteq L$ heiße \textbf{regulär}, wenn $\overline{K}\ld_KL$.
    		\item Für Körper $K,L\subseteq M$ sei $KL:=K(L)=L(K)$.
    	\end{itemize}
    \end{definition}
    
    Wir können einige Folgerungen und \glqq{}Rechenregeln\grqq{} aus den definierten Eigenschaften ziehen:
    
    \begin{lemma}\label{Stapellemma}
    	Gegeben die Körperinklusionen $C\subseteq L\subseteq M$ und $C\subseteq K$. Dann gilt $K\ld_CM$ genau dann, wenn $K\ld_CL$ und $KL\ld_LM$.
    \end{lemma}
    \begin{proof}
    	Sei $(k_h)_H$ eine Basis von $K$ über $C$, $(l_i)_I$ eine von $L$ über $C$ und $(m_j)_J$ eine von $M$ über $L$. $K\ld_CM$ bedeutet, dass die $C$-Basis von $K$ auch $M$-Basis von $K$ ist, aber dann ist sie natürlich auch $L$-Basis, da $C\subseteq L\subseteq M$. Dann kann man sich überlegen, dass $(l_im_j)_{I\times J}$ eine Basis von $M$ über $C$ ist und nach dem letzten Satz $(k_h)_H$ eine Basis von $L[K]$ als $L$-Vektorraum.\\
    	Wenn $KL\ld_LM$ nicht gelten würde, würde nach den Bemerkungen oben auch schon die Basis $(k_h)_H$ von $L[K]$ über $L$ linear abhängig über $M$ werden, es gäbe also eine Linearkombination $$\sum\limits_{h\in H}\lambda_hk_h=0,\ M\ni\lambda_h=0\text{ für fast alle, aber nicht alle } h\in H.$$
    	Da $(k_h)_H$ aber die Basis von $K$ über $C$ ist, widerspricht das $K\ld_CM$.\\
    	Für die Rückrichtung ist zu zeigen, dass $(l_im_j)_{I\times J}$ linear unabhängig über $K$ bleibt. Wenn das nicht so ist und die Linearkombination $$0=\sum\limits_{(i,j)\in I\times J}\lambda_{ij}l_im_j,\  K\ni\lambda_{ij}=0\text{ für fast alle, aber nicht alle } (i,j)\in I\times J$$ das bezeugt, schreibe $$\lambda_{ij}=:\sum\limits_{h\in H}c_{hij}k_h$$ als $C$-Basisdarstellung für alle $(i,j)\in I\times J$. Einsetzen und Umklammern bringt uns die Linearkombination $$0=\sum\limits_{(i,j)\in I\times J}\lambda_{ij}l_im_j=\sum\limits_{(i,j)\in I\times J}(\sum\limits_{h\in H} c_{hij}k_h)l_im_j=\sum\limits_{i\in I}(\sum\limits_{(h,j)\in H\times J}c_{hij}k_hl_i)m_j,$$ $$C\ni c_{hij}=0\text{ für fast alle, aber nicht alle } (h,i,j)\in H\times I\times J.$$\\
    	Da wir aber $KL\ld_CM$ annehmen, muss $$\sum\limits_{(h,j)\in H\times J}c_{hij}k_hl_i=0\text{ sein für alle }j\in J,$$ da wir $K\ld_CL$ annehmen, folgt daraus $c_{hij}=0$ für alle $h,i,j$.
    \end{proof}
    
    \begin{lemma}\label{Das komplizierte Lemma}
    	Wenn $C\subseteq K$ regulär ist und $K\ad_CL$, folgt $K\ld_CL$.
    \end{lemma}
    \begin{proof}
    	Geht mit Bewertungen, steht zum Beispiel in \cite{SergeLang} (Seite 57, Theorem 3).
    \end{proof}
    
    \begin{lemma}\label{Rechenregeln}
    	\begin{enumerate}
    		\item $K\ld_CL$ impliziert $K\cap L=C$.
    		\item Wenn $C\subseteq K$ algebraisch und $C\subseteq L$ regulär ist, dann gilt $K\ld_CL$.
    		\item $K\ld_CL$ impliziert $K\ad_CL$.
    		\item $K\ad_CL$ impliziert $\overline{K}\ad_C\overline{L}$.
    		\item Wenn $K\ld_CL, K\subset M$ und $X\subset M$ algebraisch unabhängig über $KL$, dann $K(X)\ld_KKL$.
    		\item Wenn $C\subseteq K$ regulär ist und $K\ld_CL$, folgt $K\ld_C\overline{L}$ bzw. $L\subseteq KL$ regulär.
    	\end{enumerate}
    \end{lemma}
    \begin{proof}
    	\begin{enumerate}
    		\item Die eine Inklusion gilt, denn $C\subseteq K,L$. Für die Rückrichtung sei\linebreak$x\in (K\cap L)\setminus C$. Dann ist $(1,x)\in K^2$ linear abhängig über $L$ und somit über $C$. Aber dann ist $x$ schon in $C$.
    		\item Wegen der Regularität gilt $L\ld_C\overline{C}$ und wegen $C\subseteq K\subseteq\overline{C}$ und dem Lemma \ref{Stapellemma} gilt $L\ld_C K$.
    		\item Seien $k_1,\dots,k_n\in K$ algebraisch abhängig über $L$, das heißt, es gibt ein Polynom $$0\neq f(X)=\sum\limits_{\abs{\alpha}\leq m}l_\alpha X^\alpha\in L[X_1,\dots,X_n]$$ mit $f(k)=0$, also insbesondere $(k^\alpha)_{\abs{\alpha}\leq m}$ linear abhängig über $L$, per Annahme also auch über $C$. Dann existiert aber eine Linearkombination $\sum\limits_{\abs{\alpha}\leq m}c_\alpha k^\alpha=0$ mit $c_\alpha\in C$ nicht alle Null, und diese bezeugt die algebraische Abhängigkeit über $C$.
    		\item Folgt aus $\dim(\cdot/L)=\dim(\cdot/acl(L))$ als Matroideneigenschaft.
    		\item Wegen algebraischer Unabhängigkeit ist $$\abs{\overline{x}}=\dim(\overline{x}/KL)\leq\dim(\overline{x}/K)\leq\abs{\overline{x}}$$ für alle $\overline{x}\in X$, also sind $X$ und $KL$ im modelltheoretischen Sinne unabhängig über $K$. Das gilt dann aber auch für $K\cup X$ und $KL$, $\acl(K\cup X)$ und $KL$ und auch für $K(X)$ und $KL$, also $K(X)\ad_KKL$.\\
    		$K(X)\supseteq K$ ist außerdem regulär, denn $K[X]$ hat als $K$-Basis $(x^n)_{x\in X,n\in\setN}$, wie man wegen algebraischer Unabhängigkeit sieht. Diese Basis bleibt aber linear unabhängig über $\overline{K}$, denn sonst wäre ein Polynom in $\overline{K}[X_1,X_2,\dots]$ gefunden, was ein $x\in X$ über den anderen Elementen algebraisiert, also $$x\in\acl(X\setminus\{x\}\cup\overline{K})=\acl(X\setminus\{x\}\cup K),$$ also wäre $X$ nicht mehr algebraisch unabhängig über $K$. Lemma \ref{Fraktionskörper} besagt dann $K(X)=\operatorname{Frac}(K[X])\ld_K\overline{K}$.\\
    		Also haben wir $K(X)\ad_KKL$ und $K(X)\supseteq K$ regulär, woraus nach Lemma \ref{Das komplizierte Lemma} $K(X)\ld_KKL$ folgt.
    		\item Mit 3. folgt $K\ad_CL$, mit 4. $\overline{K}\ad_C\overline{L}$, mit Lemma \ref{Stapellemma} $K\ad_C\overline{L}$, mit Lemma \ref{Das komplizierte Lemma} $K\ld_C\overline{L}$ (benutze $C\subseteq K$ regulär) und mit noch einmal Lemma \ref{Stapellemma} gilt schließlich für die Einbettungskette $C\subseteq L\subseteq\overline{L}$ die Regularitätsbedingung $LK\ld_L\overline{L}$.
    	\end{enumerate}
    \end{proof}
    \newpage
    \section{Paare algebraisch abgeschlossener Körper}
    Wir wollen Paare $(K,E_K)$ von Körpern betrachten, wobei $E_K\subseteq K$. In der richtigen Sprache haben lassen diese sich axiomatisieren, dort haben echte Paare (d.h. $K\neq E_K$) algebraisch abgeschlossener Körper mit fixierter Charakteristik sogar Quantorenelimination, sind vollständig und in einer kleineren Sprache modellvollständig. Diese Sprachen und einige der Folgerungen für ihre Strukturen wollen wir hier (angelehnt an \cite{Delon}) einführen:
    
    \begin{definition}
    	Wir definieren die Sprachen $\lld:=\{0,1,+,-,\cdot,(l_n)_{n\geq2}\},\lf:=\lld\cup\{f_{i,n}\mid n\geq2,1\leq i\leq n\}$ und $\lfc:=\lf\cup\{^{-1}\}$, wobei die $(l_n)_n$ $n$-stellige Relationen sein sollen und die $(f_{i,n})_{i,n}$ $n+1$-stellige partielle Funktionen.
    \end{definition}

    \begin{lemma}\label{Symbolik}
    	Beliebige Paare $(K,E_K)$ von Körpern werden kanonisch zu $\lld$-Strukturen, indem man folgendes setzt:
    	$$\models l_n(x_1,\dots,x_n):\Leftrightarrow x_1,\dots,x_n\text{ sind linear unabhängig über }E_K$$
    	Dann kann man die Substruktur $E_K$ auch definieren, da $x\in E_K\Leftrightarrow \models\neg l_2(1,x)=:E(x)$ und noch viel weitergehender auch $$y\in\langle\overline{x}\rangle_{E_K}\text{ für } x_1,\dots,x_n\text{ linear unabhängig über }E_K\Leftrightarrow\ \models l_n(\overline{x})\land\neg l_{n+1}(\overline{x},y)=:\phi(\overline{x},y).$$
    	Mit diesem Wissen setzt man jetzt in $\lf,\lfc$
    	$$\models (z=f_{n,i}(y,\overline{x})):\Leftrightarrow\ \models\phi(\overline{x},y)\text{ und }z\text{ ist die }i\text{-te Koordinate von }y\text{ in der Basisdarstellung},$$
    	wobei letzteres durch $$\exists z_1,\dots,z_n(z=z_i\land y=x_1z_1+\dots+x_nz_n\land z_1,\dots,z_n\in E_K)$$ oder aber auch $$\forall z_1,\dots,z_n(y=x_1z_1+\dots+x_nz_n\land z_1,\dots,z_n\in E_K\rightarrow z_i=z)$$ definierbar ist.
    \end{lemma}
    \newpage
    \begin{lemma}
    	Mit dem Ganzen sind echte algebraisch abgeschlossene Paare von Körpern definierbar in allen drei Sprachen $\lld,\lf,\lfc$, nenne die Theorien $$\operatorname{ACP}^{\lld},\operatorname{ACP}^{\lf},\operatorname{ACP}^{\lfc}.$$ Einzig nicht offensichtlich hierbei ist, dass man in der Theorie sagen muss, dass $\neg l_2(1,x)$ einen Körper definiert.
    \end{lemma}
    
    An sich wird alles sowieso interdefinierbar zwischen den Sprachen sein, aber manchmal ist es klüger, eine spezielle Sprache zu betrachten. Die intuitivste Sprache wäre dabei $\lingua^E:\lingua\cup\{E(x)\}$, aber diese hat leider nicht so schöne Eigenschaften (zum Beispiel ist sie nicht modellvollständig, denn lineare Unabhängigkeit über $E$ überträgt sich nicht auf Oberstrukturen). Das Ziel ist jetzt, zu beweisen, dass $\operatorname{ACP}^{\lf},\operatorname{ACP}^{\lfc}$ Quantorenelimination haben und $\operatorname{ACP}^{\lld}$ immerhin modellvollständig ist.\\
    Dazu müssen wir erst einmal verstehen, wie $\lfc$-Unterstrukturen von $\operatorname{ACP}^{\lfc}$-Modellen aussehen.
    
    \begin{lemma}
    	Betrachte einen Paar von Körpern $(K,E_K)$ und eine Teilmenge ${A\subseteq K}$ sowie eine $\lfc$-Struktur $$\fA:=(A,0,1,+,-,\cdot,^{-1},(l_n)_{n\geq2},(f_{i,n})_{n\geq2,1\leq i\leq n}).$$ Dann ist $\fA\subseteq_{\lfc}(K,E_K)$ genau dann, wenn $A$ ein Unterkörper von $K$ ist, $\fA=(A,E_A)$ für $E_A:=A\cap E_K$, und $A\ld_{E_A}E_K$.
    \end{lemma}
    \begin{proof}
    	$A$ ist genau dann Unterkörper von $K$, wenn es $0,1$ enthält, unter $+,-,\cdot,^{-1}$ abgeschlossen ist und die entsprechenden Abbildungsvorschriften erbt.\\
    	Außerdem ist $A\ld_{E_A}E_K$ genau dann, wenn für alle $\overline{a}\in A$ aus $\overline{a}$ linear abhängig über $E_K$ schon äquivalent zu linearer Abhängigkeit über $E_A$ ist; per kanonischer Definition also genau dann wenn $$(A,E_A)\models l_{\abs{\overline{a}}}(\overline{a}) \Leftrightarrow (K,E_K)\models l_{\abs{\overline{a}}}(\overline{a}).$$
    	Wenn $\fA$ Unterstruktur ist, dann gilt für alle $\overline{a}\in A$
    	\begin{align*}
    	&\fA\models l_{\abs{\overline{a}}}(\overline{a})\Leftrightarrow(K,E_K)\models l_{\abs{\overline{a}}}(\overline{a})\\
    	\Leftrightarrow&\ \overline{a}\text{ linear unabhängig über }E_K\Leftrightarrow\overline{a}\text{ linear unabhängig über }E_A,
    	\end{align*}\newpage
    	wobei die letzte Äquivalenz davon herrührt, dass die lineare Abhängigkeit bezeugende Linearkombination sich z.B. mit Projektionen und Nullen als Koeffizienten schreiben lässt und Projektionen von Elementen aus $A$ wegen Unterstruktureigenschaft wieder in $A$ sind. Also stimmen die Interpretationen der $l_n$ in $\fA$ mit denen in $(A,E_A)$ überein und es ist $A\ld_{E_A}E_K$. Da die definierende Formel der $(f_{i,n})$ bis auf die Angabe des Bildbereiches eine Ringformel ist, und da $f_{i,n}(\overline{a})\in E_K\cap A$ für alle $\overline{a}\in A$, für alle $n,i$, stimmen auch die Interpretationen der $f_{i,n}$ in $\fA$ und in $(A,E_A)$ überein. Damit ist $\fA=(A,E_A)$.\\
    	Die Rückrichtung folgt mit den ersten Zeilen dieses Beweises und, weil die Projektionen $f_{n,i}$ Elemente aus $A$ nach $A$ abbilden. Das sieht man daran, dass man die Koeffizienten einer Linearkombination in eine Abhängigkeitsbedingung umschreiben kann; wenn diese in $E_K$ erfüllt ist, muss sie wegen linearer Disjunktheit auch in $E_A$ erfüllt sein, also sind die Projektionen schon in $E_A$.
    \end{proof}
    
    \begin{lemma}\label{transz Erw}
    	Wenn $(A,E_A)\subseteq_{\lfc}(K,E_K)$ und $X\subseteq K$ algebraisch unabhängig über $AE_K$ ist, dann ist $$(A,E_A)\subseteq_{\lfc}(A(X),E_A)\subseteq_{\lfc}(K,E_K).$$
    \end{lemma}
    \begin{proof}
    	Nach vorigem Lemma ist $A\ld_{E_A}E_K$, daher gilt mit Lemma \ref{Rechenregeln} (5.)\linebreak$A(X)\ld_{E_A}E_K$ und da $A(X)$ Unterkörper von $K$ ist, gilt mit der Rückrichtung des letzten Lemmas $(A(X),E_A)\subseteq_{\lfc}(K,E_K)$.\\
    	$(A,E_A)\subseteq_{\lfc}(A(X),E_A)$ gilt, weil die Bedingung $A\ld_{E_A} E_A$ immer stimmt.
    \end{proof}
    
    \begin{lemma}\label{E-Erw}
    	Sei $(A,E_A)\subseteq_{\lfc}(K,E_K)$ und $E_A\subseteq B\subseteq E_K$ ein Zwischenkörper. Dann ist $$(A,E_A)\subseteq_{\lfc}(AB,B)\subseteq_{\lfc}(K,E_K).$$
    \end{lemma}
    \begin{proof}
    	Es ist nur $A\ld_{E_A}B,AB\ld_BE_K$ zu zeigen. Wegen $(A,E_A)\subseteq_{\lfc}(K,E_K)$ gilt $A\ld_{E_A}E_K$ und mit Lemma \ref{Stapellemma} gilt schon beides.
    \end{proof}
    \newpage
    \begin{lemma}\label{Fortsetzungslemma}
    	Im Falle, dass das vorige Lemma auf $$(A,E_A)\subseteq_{\lfc}(K,E_K),(\tilde{A},E_{\tilde{A}})\subseteq_{\lfc}(\tilde{K},E_{\tilde{K}}),E_A\subseteq B\subseteq E_K\text{ und }E_{\tilde{A}}\subseteq \tilde{B}\subseteq E_{\tilde{K}}$$ angewendet wird und dass gilt $$A\cong \tilde{A},B\cong \tilde{B},E_A\cong E_{\tilde{A}}$$ \--- wobei die ersten beiden Isomorphismen den dritten fortsetzen \---, sind $(AB,B)$ und $(\tilde{A}\tilde{B},\tilde{B})$ schon isomorph als $\lfc$-Strukturen.
    \end{lemma}
    \begin{proof}
    	Mit Lemma \ref{Tensoren} erhält man wegen $A\ld_{E_A}B,\tilde{A}\ld_{E_{\tilde{A}}}\tilde{B}$ den Isomorphismus $$A[B]\cong A\otimes_{E_A}B\cong \tilde{A}\otimes_{E_{\tilde{A}}}\tilde{B}\cong\tilde{A}[\tilde{B}],$$ der sich zu einem Isomorphismus der Quotientenkörper $AB$ und $\tilde{A}\tilde{B}$ fortsetzt und $A$ auf $\tilde{A}$, $B$ auf $\tilde{B}$ und $E_A$ auf $E_{\tilde{A}}$ abbildet. Als Körperisomorphismus überträgt er auch Linearkombinationen, also auch die Bedeutung der $(l_n)_n$ und $(f_{i,n})_{i,n}$, weswegen er ein $\lfc$-Isomorphismus ist.
    \end{proof}
    
    \begin{lemma}
    	Wenn $(A,E_A)\subseteq_{\lfc}(K,E_K)$ und $(K,E_K)$ ein Paar algebraisch abgeschlossener Körper ist, ist $E_A\subseteq A$ regulär.
    \end{lemma}
    \begin{proof}
    	Die Aussage folgt aus $A\ld_{E_A}E_K$ und der Körperinklusion $E_A\subseteq\overline{E_A}\subseteq E_K$ mit dem Lemma \ref{Stapellemma}.
    \end{proof}
    
    \begin{lemma}\label{alg Abschl}
    	Unter denselben Bedingungen wie im vorigen Lemma ist $(\overline{A},\overline{E_A})$ Zwischenstruktur.
    \end{lemma}
    \begin{proof}
    	Laut Lemma \ref{E-Erw} ist $(A\overline{E_A},\overline{E_A})$ Zwischenstruktur und damit insbesondere $$A\ld_{E_A}\overline{E_A},A\overline{E_A}\ld_{\overline{E_A}}E_K.$$ Klarerweise ist $$(A,E_A)\subseteq_{\lfc}(\overline{A\overline{E_A}},\overline{E_A})=(\overline{A},\overline{E_A}),$$ weil $A\overline{E_A}$ in der Bedingung $A\ld_{E_A}\overline{E_A}$ gar nicht vorkommt und die Erweiterung also nichts ändert.\newpage
    	Lemma \ref{Rechenregeln} (6.) ergibt wegen $\overline{E_A}\subseteq E_K$ regulär $$\overline{A}=\overline{A\overline{E_A}}\ld_{\overline{E_A}}E_K,$$ was $(\overline{A},\overline{E_A})\subseteq_{\lfc}(K,E_K)$ beweist.
    \end{proof}
    
    \begin{theorem}\label{QE}
    	$\operatorname{ACP}^{\lfc}$ hat Quantorenelimination und ist vollständig, wenn man noch eine Charakteristik vorgibt.
    \end{theorem}
    \begin{proof}
    	Gegeben sei eine beliebige unendliche Kardinalität $\kappa$.
    	Zeige die Aussage mit dem Back\&Forth-System der Isomorphismen zwischen maximal $\kappa$ großen Unterstrukturen von $\kappa^+$-saturierten Modellen $(K,E_K),(L,E_L)$:
    	Dieses ist nichtleer, denn wenn $\mathbb{P}$ der Primkörper der Charakteristik ist, ist $(\mathbb{P},\mathbb{P})$ Unterstruktur von allen Modellen (wegen Gleichheit des Paares ist lineare Disjunktheit klar), bilde das als Unterstruktur von $K$ auf sich selbst als Unterstruktur von $L$ ab.
    	Sei $(M,E_M)\rightarrow(N,E_N)$ im B\&F-System. $K\supset E_K,L\supset E_L$ haben Transzendenzgrad $\infty$.\\
    	Das kann man zum Beispiel erreichen, indem die Erweiterung offenkundig transzendent ist, und man dann jeweils den partiellen Typ über $\emptyset$ betrachtet, der die algebraische Unabhängigkeit von $n$ Elementen über $E_K$ bzw. $E_L$ beschreibt, dieser hat folgende Gestalt:
    	$$\{\forall \overline{e}\in E\setminus\{0\}(f(\overline{e},\overline{x})\neq0)\mid 0\neq f\in\mathbb{P}(T_1,T_2,\dots,\overline{x})\}$$
    	Er ist endlich erfüllbar, da für $m$ größer als der größte Polynomgrad im endlichen Teilfragment und $x$ transzendent über $E_K$ bzw. $E_L$ die Elemente $x,x^m,x^{m^2},\dots$ algebraisch unabhängig über Polynome von Grad kleiner $m$ sind.\\
    	\OE\ seien $(M,E_M)(N,E_N)$ jeweils algebraisch abgeschlossen. Denn Lemma \ref{E-Erw} erzeugt einen Isomorphismus zwischen den Zwischenstrukturen $$(ME_M^{\text{alg }K},E_M^{\text{alg }K})\text{ und }(NE_N^{\text{alg }L},E_N^{\text{alg }L}),$$ der sich auf einen Isomorphismus $$(M^{\text{alg }K},E_M^{\text{alg }K})\cong(N^{\text{alg }L},E_N^{\text{alg }L})$$ fortsetzt.\\
    	Sei jetzt $a\in K$. Wenn $a\in M$ liegt, dann kann man die Abbildung auf triviale Weise auf $a$ fortsetzen.\newpage
    	Wenn ansonsten $a$ algebraisch über $E_KM$ ist, ist $a\in\acl(E_KM)$, also existiert $X\subset E_K$ endlich mit $a\in\acl(MX)$. \OE\ sei $X$ jetzt schon ein Oberkörper von $E_M$, wichtig ist nur der endliche Transzendenzgrad über $E_M$. Für einen beliebigen algebraisch abgeschlossenen Zwischenkörper $E_N\subseteq Y\subseteq E_L$ von gleichem Transzendenzgrad (den gibt es, da die Saturation Transzendenzgrad $\infty$ von $E_N\subset E_L$ ergibt) kann $E_M\cong E_N$ fortgesetzt werden zu einem Isomorphismus $X\cong Y$. Diesen kann man wie in Lemma \ref{Fortsetzungslemma} fortsetzen zu einem $\lfc$-Isomorphismus zwischen den Zwischenstrukturen $(\overline{MX},X),(\overline{NY},Y)$, wobei die erste $a$ enthält.\\
    	Wenn $a$ transzendent über $E_KM$ ist, gibt es ein $b\in L$ transzendent über $E_LM$, denn der entsprechende Typ ist konsistent, wenn $\abs{L}>\abs{E_L}$. So etwas lässt sich aber in einer elementaren Oberstruktur erreichen (für die endliche Konsistenz ist die genaue Kardinalität des Transzendenzgrads von $E_L\subset L$ egal) und wenn der Typ dort konsistent ist, dann auch unten.\\
    	$a$ und $b$ erzeugen dann einen Isomorphismus $C:=\overline{M(a)}\overset{\phi}{\cong}\overline{N(b)}$. Setze $$E:=\overline{M(a)}\cap E_K\cap\phi^{-1}(E_L\cap\overline{N(b)}),$$ dann gilt $(C,E)\cong_{\lfc}(\phi(C),\phi(E))$ und $b\in\phi(C)$.\\
    	Zu zeigen ist nun nur noch $$(M,E_M)\subseteq_{\lfc}(C,E)\subseteq_{\lfc}(K,E_K),(N,E_N)\subseteq_{\lfc}(\phi(C),\phi(E))\subseteq_{\lfc}(K,E_K):$$\\
    	Aus $(M,E_M)\subseteq_{\lfc}(K,E_K)$ folgt mit Lemma \ref{transz Erw} $$(M,E_M)\subseteq_{\lfc}(M(a),E_M)\subseteq_{\lfc}(K,E_K),$$ daraus mit Lemma \ref{E-Erw} und $E_M\subseteq E\subseteq E_K$ $$(M,E_M)\subseteq_{\lfc}(M(a)E,E)\subseteq_{\lfc}(K,E_K),$$ daraus folgt mit Lemma \ref{alg Abschl} und $E\subseteq\overline{M(a)}=C$ schließlich $$(M,E_M)\subseteq_{\lfc}(C,E)\subseteq_{\lfc}(K,E_K).$$
    	Der Teil für $(\phi(C),\phi(E))$ geht analog und wegen $a\in C$ haben wir die gesuchte Fortsetzung gefunden.
    \end{proof}
    
    \newpage
    
    \begin{lemma}\label{Eliminierungsregeln}
    	Es gilt:\\
    	\begin{align*}\operatorname{ACP}^{\lfc}&\models\forall x_1,\dots,x_n\forall y_1,\dots,y_n\neq0(l_n(x_1y_1^{-1},\dots,x_ny_n^{-1})\\&\leftrightarrow l_n\left(x_1\prod\limits_{i=1\dots n,i\neq 1}y_i,\dots,x_n\prod\limits_{i=1\dots n,i\neq n}y_i\right))\\
    	\operatorname{ACP}^{\lfc}&\models\forall x_1,\dots,x_n\forall e_1,\dots,e_n\in E\left(l_n(e_1x_1,\dots,e_nx_n)\leftrightarrow l_n(x_1,\dots,x_n)\right)\\
    	\operatorname{ACP}^{\lfc}&\models\forall x_1,\dots,x_n,y\forall e_1,\dots,e_n,e\in E(f_{n,i}(ey,e_1x_1,\dots,e_nx_n)=\frac{e}{e_i}f_{n,i}(y,x_1,\dots,x_n))\\
    	\operatorname{ACP}^{\lfc}&\models\forall a,b,x_2,\dots,x_n(\neg t_n(a+b,x_2,\dots,x_n)\leftrightarrow\neg t_{n-1}(x_2,\dots,x_n)\lor\\
    	&t_{n-1}(x_2,\dots,x_n)\land((l_n(b,x_2,\dots,x_n)\land\neg l_{n+1}(a,b,x_2,\dots,x_n))\lor\\
    	&(\neg l_n(b,x_2,\dots,x_n)\land l_n(a,x_2,\dots,x_n))))\\
    	\operatorname{ACP}^{\lfc}&\models\forall a,b,x_1,\dots,x_n(f_{i,n}(a+b,x_1,\dots,x_n)=f_{i,n}(a,x_1,\dots,x_n)+f_{i,n}(b,x_1,\dots,x_n))\\
    	&\text{für alle }i=1,\dots,n\\
    	\operatorname{ACP}^{\lfc}&\models\forall a,b,x_1,\dots,x_{i-1},x_{i+1},\dots,x_n,z(f_{i,n}(z,x_1,\dots,x_{i-1},a+b,x_{i+1},\dots,x_n)=\\
    	&\left\{\begin{array}{ll}
    	f_{i,n+1}(z,x_1,\dots,x_{i-1},a,b,x_{i+1},\dots,x_n)& a,b,\overline{x}\text{ unabhängig}\\
    	f_{i,n}(z,x_1,\dots,x_{i-1},a,x_{i+1},\dots,x_n)&\text{wenn nicht und }b\in\langle\overline{x}\rangle_E\\
    	f_{i,n-1}(z,x_1,\dots,x_{i-1},x_{i+1},\dots,x_n)&\text{wenn nicht und }z\in\langle\overline{x}\rangle_E\\
    	\frac{f_{i,n}(z,x_1,\dots,x_{i-1},b,x_{i+1},\dots,x_n)}{1+f_{i,n}(a,x_1,\dots,x_{i-1},b,x_{i+1},\dots,x_n)}&\text{wenn nicht}
    	\end{array}\right.)\\
    	&\text{für alle }i=1,\dots,n\\
    	\operatorname{ACP}^{\lfc}&\models\forall a,b,x_1,\dots,x_{i-1},x_{i+1},\dots,x_n,z(f_{j,n}(z,x_1,\dots,x_{i-1},a+b,x_{i+1},\dots,x_n)=\\
    	&\left\{\begin{array}{ll}
    	f_{j+1_{j>i},n+1}(z,x_1,\dots,x_{i-1},a,b,x_{i+1},\dots,x_n)\\
    	\ \ a,b,\overline{x}\text{ unabhängig}\\
    	f_{j,n}(z,x_1,\dots,x_{i-1},a,x_{i+1},\dots,x_n)\\
    	-f_{i,n}(z,x_1,\dots,x_{i-1},a,x_{i+1},\dots,x_n)f_{j-1_{j>i},n-1}(b,x_1,\dots,x_{i-1},x_{i+1},\dots,x_n)\\
    	\ \ \text{wenn nicht und }b\in\langle\overline{x}\rangle_E\\
    	f_{j,n}(z,x_1,\dots,x_{i-1},b,x_{i+1},\dots,x_n)\\
    	-f_{i,n}(z,x_1,\dots,x_{i-1},a+b,x_{i+1},\dots,x_n)f_{j,n}(a,x_1,\dots,x_{i-1},b,x_{i+1},\dots,x_n)\\
    	\ \ \text{wenn nicht}
    	\end{array}\right.)\\
    	&\text{für alle }i=1,\dots,n\text{, für alle }j\neq i\\
    	\end{align*}
    \end{lemma}
    \begin{proof}
    	Ist reine Fallunterscheidung und Rechenarbeit.
    \end{proof}

    \begin{definition}
    Wenn es keine Rolle spielt, in welcher Sprache man gerade ist, schreibe einfach \textbf{ACP} für die drei Theorien.
    \end{definition}
    
    \begin{corollary}\label{Formel-Vereinfachung}
    	Man kann jede Formel mit Parametern aus $X$ modulo $\operatorname{ACP}$ schreiben als boolesche Kombination aus $$\glqq{}l_n(\text{Monome mit Koeffizienten von Produkten aus }X)\grqq{}$$ und
    	\begin{align*}
    	\text{\glqq{}}(\text{Polynom in }\mathbb{P}[X])(&f_{i_1,n_1}(\text{Monome in Produkten aus }X),\\\dots,&f_{i_m,n_m}(\text{Monome in Produkten aus }X))=0\grqq{},
    	\end{align*}
        wobei $\mathbb{P}$ den Primkörper der entsprechenden Charakteristik bezeichne.
    \end{corollary}
    \begin{corollary}
    	In jedem Modell $(K,E_K)$ ist jede definierbare Menge $X\subset E_K^n$ schon beschreibbar als $X=E_K^n\cap Z$, wobei $Z$ definierbar in der Ringsprache über denselben Parametern ist. Das liegt daran, dass die $(f_{i,n}),(l_n)$ nicht definiert bzw. trivial sind, wenn man Werte aus $E_K$ einsetzt.\\ Dementsprechend ist $E(x)$ eine streng minimale Formel und für $e_1,\dots,e_n\in E_K$ ist $\operatorname{RM}(\overline{e}/X)=\dim(\overline{e}/X)$ für alle $X\subseteq K$.
    \end{corollary}
    \begin{corollary}
    	$\operatorname{ACP}^{\lf}$ hat für fixierte Charakteristik ebenfalls Quantorenelimination, da nach Folgerung \ref{Formel-Vereinfachung} die Operation \glqq{}$^{-1}$\grqq{} eliminiert werden kann. Da in Lemma \ref{Symbolik} gezeigt wurde, dass die $(f_{i,n})$ sowohl existenziell als auch universell definierbar sind, ist jede Formel modulo $\operatorname{ACP}^{\lld}$ immerhin universell und $\operatorname{ACP}^{\lld}$ ist modellvollständig.
    \end{corollary}
    
    \begin{corollary}
    	Man kann sich leicht überlegen, dass ACP$\cup\{$\glqq{}Charakteristik=p\grqq{}$\}$ das Primmodell $(\overline{\mathbb{P}(e)},\overline{\mathbb{P}})$ hat für $\mathbb{P}$ als Primkörper und ein beliebiges $e$ transzendent über $\mathbb{P}$.
    \end{corollary}
    \newpage
    \begin{theorem}
    	ACP ist $\omega$-stabil.
    \end{theorem}
    \begin{proof}
    	Betrachte die Menge der Typen in einem Modell $(K,E_K)$ über einer vorgegebenen Menge $S\subset K$ und wähle als Sprache $\lfc$. \OE\ ist das Modell schon $\abs{S}^+$-saturiert und $S$ Träger einer $\lfc$-Unterstruktur. Aus dem B\&F-System in Satz \ref{QE} geht hervor, dass es die folgenden Typen über $S$ gibt:
    	\begin{itemize}
    		\item Den Typ eines Elementes in $S$
    		\item Die Typen eines Elementes $a$ in $\overline{SE_K}\setminus S$ (bestimmt durch den endlichen Trans\-zen\-denz\-grad über $S$ des minimalen Unterkörpers $E_S\subseteq X\subseteq E_K$, sodass $a\in\overline{SX}$, sowie durch den Isomorphietyp des Minimalpoynoms von $a$ über $SX$)
    		\item Den Typ eines Elementes in $K\setminus\overline{SE_K}$
    	\end{itemize}
        Der erste Typ hat klarerweise Morleyrang 0. Sei $a$ ein Realisator eines Typen der zweiten Art. Dann ist $a$ algebraisch über $S\overline{e}$ für gewisse $e_1,\dots,e_n\in E_K$, also $$\operatorname{RM}(a/S)\leq\operatorname{RM}(\overline{e}/S)=\dim(\overline{e}/S)<\omega.$$
        Der dritte Typ kann keinen Morleyrang $>\omega$ haben, denn dann müsste es einen Typen mit Morleyrang $\omega$ geben, die anderen Arten von Typen haben aber endlichen Rang. Sei $(a_n)$ eine Folge von Elementen, sodass $$n<\operatorname{RM}(a_n)<\omega\text{ für alle }n\in\setN$$ (TODO: sowas gibt es). Dann ist im Stoneraum $\lim\limits_{n\rightarrow\infty}\tp(a_n)=p$ für den Typen $p$ der dritten Art: Denn $a_n$ kann jeweils nicht mehr algebraisch über $SX$ sein für alle $E_S\subset X$ von Transzendenzgrad $\leq n$. Also sind für jede Umgebung, die das Nichterfüllen einer bestimmten Art von Polynom über $SE_K$ beschreibt, fast alle $a_n$ enthalten. Aber diese Umgebungen bestimmen den Typen $p$ eindeutig, also sind für jedes $\phi\in p$ fast alle $\tp(a_n)$ in $\fU_\phi$.\\
        Jedes $\phi\in S_1(S)$ mit $\operatorname{RM}(\phi)<\omega$ ist dann nur in endlich vielen $\tp(a_n)$ enthalten (nämlich maximal $\operatorname{RM}(\phi)$ vielen), also ist $\operatorname{RM}(p)\geq\omega$, damit herrscht Gleichheit.\\
        Da alle Typen Morleyrang $<\infty$ haben, haben alle Formeln in einer freien Variable Morleyrang $<\infty$ und die Theorie ist $\omega$-stabil.\\
        Ein Alternativbeweis wäre auch, dass es nicht mehr Typen über $S$ gibt als $\abs{S}+\aleph_0$, auch das beweist die $\omega$-Stabilität. Und auch mit algebraischen Methoden ist ein Beweis möglich, siehe dazu den Anhang.
    \end{proof}
    %!TEX root = DieLoesungAllerMilleniumsprobleme.tex
\chapter{Dichte Paare o-minimaler Strukturen}
\section{O-Minimale Theorien und ihre Eigenschaften}

Viele Techniken und Gedanken aus dem vorigen Kapitel werden jetzt auf o-minimale Theorien übertragen. Einführend folgen hier zunächst die wichtigsten Erkenntnisse über diese Theorien, basierend auf \cite{vdDZellzerlegung}.

\begin{definition}
	Eine $\lingua$-Struktur $\fM$ heißt \textbf{o-minimal}, wenn $\lingua$ eine zweistellige Relation $<$ enthält, deren Interpretation die einer linearen Ordnung ist, und die definierbaren Teilmengen von $\fM$ endliche Vereinigungen von Intervallen und Punkten sind.
\end{definition}

Induktiv kann man den Begriff einer Zelle in einer solchen Struktur definieren.
\begin{definition}
	Sei $\sigma$ eine $\{0,1\}$-wertige endliche Folge der Länge $n$. Eine $\sigma$-Zelle $Z$ in der o-minimalen Struktur $\fM$ ist eine (definierbare) Teilmenge von $M^n$, sodass genau eine der folgenden Möglichkeiten gilt:
	\begin{itemize}
		\item $\sigma$ ist die leere Folge, $n=0$ und $Z=M^0=\{0\}$.
		\item $n=1$ und $Z$ ist entweder Intervall ($\sigma=(1)$) oder einelementig ($\sigma=(0)$).
		\item Es existiert ein $\sigma'$, eine $\sigma'$-Zelle Z' und eine definierbare stetige Funktion\linebreak$f:Z'\rightarrow M$, sodass $Z=\operatorname{Graph}(f)$. Außerdem ist $\sigma=\sigma'\textasciicircum0$.
		\item Es existiert ein $\sigma'$, eine $\sigma'$-Zelle Z' und zwei definierbare stetige Funktionen $f,g:M^{Z'}\cup\{\pm\infty\}$, sodass $f<g$ überall und $$Z=\{(x,y)\in Z'\times M\mid f(x)<y<g(x)\}.$$ Außerdem ist $\sigma=\sigma'\textasciicircum1$.
	\end{itemize}
\end{definition}
\begin{definition}
	Sei $\fM$ o-minimal. Eine Menge $Z\subseteq M^n$ heißt \textbf{Zelle}, wenn es eine $\{0,1\}$-Folge $\sigma$ der Länge $n$ gibt, sodass $Z$ eine $\sigma$-Zelle ist.
\end{definition}

Es gibt \glqq{}schöne\grqq{} Zellen, nämlich die offenen. Glücklicherweise ist jede Zelle definierbar homöomorph zu einer offenen Zelle.
\begin{lemma}
	Jede $\sigma$-Zelle ist homöomorph zu einer offenen Zelle. Der kanonische Homöomorphismus entsteht dabei durch Weglassen der Koordinaten, in denen $\sigma$ den Wert 0 hat. Insbesondere ist der Homöomorphismus definierbar über der selben Menge wie die Zelle.
\end{lemma}

Zellen haben auch zwei andere wichtige topologische Eigenschaften:
\begin{lemma}
	Zellen sind \textbf{lokal abgeschlossen}, das heißt, sie sind offen in der Spurtopologie ihres Abschlusses. Sie sind außerdem \textbf{definierbar zusammenhängend}, das heißt, sie enthalten keine definierbaren clopen Teilmengen.
\end{lemma}

Wichtigste Grundlage der Arbeit mit o-minimalen Theorien sind die folgenden Aussagen zur \textbf{Zellzerlegung}. Sie stellen quasi eine Verallgemeinerung der Definition von O-Minimalität auf höhere Dimensionen dar und geben die Struktur definierbarer Funktionen an.
\begin{theorem}
	In einer o-minimalen Struktur $\fM$ ist jede definierbare Menge eine endliche disjunkte Vereinigung von Zellen. Für jede definierbare (möglicherweise partielle) Funktion gibt es eine Zerlegung des Definitionsbereiches in Zellen, sodass die Funktion auf jeder Zelle stetig ist. Wenn es eine Funktion in einer Variable ist, ist sie nicht nur stückweise stetig, sondern es gibt auch eine Zerlegung, sodass die Funktion auf jedem Stück entweder streng monoton oder konstant ist.
\end{theorem}
\begin{remark}
	Wenn die Menge bzw. die Funktion über einer bestimmten Menge definierbar sind, kann man sich die Zerlegung ebenso über dieser Menge definierbar wählen. Außerdem kann man endlich viele Mengen simultan zerlegen und sie sogar partitionieren. Eine Partition sei dabei definiert als
	\begin{itemize}
		\item Eine Zerlegung einer Teilmenge von $M$ oder
		\item Eine Zerlegung einer Teilmenge von $M^{n+1}$, so dass die Projektion auf die ersten $n$ Koordinaten eine Partition einer Teilmenge von $M^n$ erzeugt.
	\end{itemize}
\end{remark}

\begin{corollary}
	O-Minimale Strukturen eliminieren $\exists^\infty$. Damit ist für jede definierbare Menge in einer Variable die Menge der zu vereinigenden Punkte und Intervalle uniform beschränkt. Also lässt sich o-Minimalität als Formelmenge beschreiben und ist somit Eigenschaft einer vollständigen Theorie.
\end{corollary}

\begin{definition}
	Eine vollständige Theorie sei o-minimal, wenn eines ihrer Modelle es ist.
\end{definition}
\newpage
Ab jetzt geben wir uns ein o-minimales Modell $\fM$ einer vollständigen $\lingua$-Theorie T so vor, dass die Ordnung dicht ist. Zellen sind sehr ähnlich zu Mannigfaltigkeiten, genauso wie diesen kann man ihnen eine Dimension zuweisen.
\begin{lemdef}
	Die endliche Folge $\sigma$ zur jeweiligen Zelle $Z$ ist eindeutig bestimmt und daher der Wert $\dim(Z)=\sum\limits_{i=1}^n\sigma(i)$ wohldefiniert. Er wird \textbf{Dimension der Zelle} genannt. Für allgemeine definierbare nichtleere Mengen $X$ sei die \textbf{Dimension} $\dim(X)$ definiert als das Maximum der Zelldimensionen in einer beliebigen Zellzerlegung, die Dimension der leeren Menge sei 0. Das ist wohldefiniert und rein topologisch definierbar: $\dim(X)\leq n$ genau dann, wenn eine Projektion von $X$ nach $M^n$ nichtleeres Inneres hat. Damit ist die Dimension auch $\lingua$-definierbar.
\end{lemdef}

Man kann die Dimension auch anders definieren:
\begin{lemma}
	Es erfüllt $\acl=\dcl$ das Austauschprinzip und induziert also einen Dimensionsbegriff für Tupel. Wenn $\fM$ hinreichend saturiert ist und $X\subseteq M^n$ definierbar über $Y$, ist $\dim(X)=\max\{\dim(a/Y)\mid a\in X\}$.
\end{lemma}

O-Minimale Gruppen haben eine enorm wichtige Eigenschaft, aus der viele weitere (hier nicht bedeutende) Dinge folgen.
\begin{lemma}
	Wenn $\fM$ eine 0-definierbare, durch $<$ angeordnete Gruppenoperation mit einem 0-definierbaren positiven Element (üblicherweise 1 genannt) besitzt, gilt \textbf{Definable Choice}: Aus jeder definierbaren Menge lässt sich kanonisch, uniform und definierbar ein Element auswählen. Insbesondere gibt es zu jeder Menge eine definierbare Skolemfunktion und jede definierbare Familie von uniform parametrisierten Mengen hat eine definierbare Auswahlfunktion. Die Auswahl ist dabei definierbar über derselben Menge wie die ursprüngliche(n) Menge(n).
\end{lemma}
\newpage
\section{Anforderungen an die hier betrachteten Theorien}

Im Folgenden halten wir eine vollständige o-minimale Theorie T mit dichter Ordnung in der Sprache $\lingua$ fest und betrachten die Theorie $\tq$ in der Sprache $\lingua^P:=\lingua\cup\{P(x)\}$, sodass die Modelle von $\tq$ Modelle von T sind und in jedem Modell $\fM$ die Menge $P(M)$ ebenfalls Modell von T ist. Schreibe so ein Paar dann als $(B,A)$ mit $A=P(B)$.\\
T erweitere RCF, insbesondere gilt Definable Choice. Dann sind Skolemfunktionen definierbar und \OE\ ist $\lingua$ schon so eine definitorische Erweiterung, dass T Quantorenelimination hat und universell axiomatisierbar ist (wobei bei einzelnen Theorien die Frage interessant wäre, welche Skolemfunktionen man dafür überhaupt hinzufügen muss).\\
Aus T universell mit Quantorenelimination folgt, dass Unterstrukturen von Modellen von T schon elementare Unterstrukturen sind. Also ist für jede Teilmenge $S$ eines Modells $\dcl(S)$ schon eine elementare Substruktur; zur Vereinfachung bezeichne in Zukunft $AB:=\dcl(A\cup B)=\langle A\cup B\rangle_\lingua$ für zwei Teilmengen $A,B$ eines Modells.\\
$P$ beschreibt also eine elementare Unterstruktur, mit $\td$ wird nun die Theorie beschrieben, die ausdrückt, dass $P$ eine dichte echte Unterstruktur ist (diese zwei Aussagen oder deren Gegenteil müssen auf jeden Fall von der Theorie beschrieben werden, wenn sie vollständig sein soll). Klar ist dann, dass Unterstrukturen von $\td$ automatisch Modelle von $\tq$ sind.\\\\
Der Arbeit \cite{VanDenDries} folgend, wird in den nächsten Abschnitten die Vollständigkeit und eine Art von Quantorenelimination für $\td$ gezeigt, die mittels eines Back\&Forth-Systems bewiesen wird. Danach werden einige Erkenntnisse aus dem B\&F-System gezogen, bevor die Hauptaussagen dieser Arbeit bewiesen werden. Zunächst ist aber eine genauere Betrachtung von sogenannten kleinen Mengen vonnöten.
\newpage
\section{Kleine Mengen, Dichtheit und saturierte Modelle}
\begin{definition}
	Sei $(B,A)\models\td$, dann ist eine $\lingua^P$-definierbare Menge $S\subseteq B$ \textbf{klein}, wenn eine $\lingua$-definierbare Funktion $f:B^n\rightarrow B$ existiert mit $S\subseteq f(A^n)$.
\end{definition}

\begin{lemma}
	Kleine Mengen in einem dichten Paar bilden ein Ideal in den $\lingua^P$-definierbaren Mengen bzgl. $\glqq{}\subseteq\grqq{}$.
\end{lemma}
\begin{proof}
	Es gibt kleine Mengen, zum Beispiel Punktmengen oder der kleinere Teil eines Paares, außerdem sind Teilmengen von kleinen Mengen offensichtlich wieder klein.\\
	Sei jetzt $(B,A)$ ein dichtes Paar und $X,Y$ kleine Mengen, seien $f,g:B^n\rightarrow B$ definierbar mit $$X\subseteq f(A^n),Y\subseteq g(A^n).$$ \OE\ bilden $f,g$ dabei schon von demselben $B^n$ ab, ansonsten muss man beide mit Dummyvariablen vergrößern. Definiere
	\begin{align*}
	h:B^{n+1}&\rightarrow B,\\
	h(\overline{x},y):=&\left\{\begin{array}{ll}
	f(\overline{x})&y=0\\
	g(\overline{x})&\text{sonst}\\
	\end{array}\right.,
	\end{align*}
	das ist $\lingua$-definierbar und es gilt $X\cup Y\subseteq h(A^{n+1})$, also ist $X\cup Y$ auch klein.\\
	Dass nicht jede Menge klein ist, wird im Folgenden noch bewiesen werden.
\end{proof}

In dem folgenden Lemma meint $+,\cdot$ nicht unbedingt die Operationen aus $\lingua$, sondern neue, beliebige Operationen. Die Anordnung hingegen soll die gewöhnliche bleiben.

\begin{lemma}\label{Hilfsaussage Kleinheit}
	Seien $A\prec B\models\operatorname{T}$ hinreichend hoch saturiert, $f:B^{n+1}\rightarrow B\ A\text{-definierbar},\linebreak b\in B\setminus A,\ \beta,\gamma\in A\cup\{\pm\infty\}$ mit $\beta<b<\gamma$ und einer angeordneten $A$-definierbaren Körperstruktur $(\cdot,+)$ auf $(\beta,\gamma)=:I$. Dann existieren $a_0,\dots,a_n\in I_A$ mit $$a_nb^n+a_{n-1}b^{n-1}+\dots,a_0\in I\setminus f(A^n\times\{b\}).$$
\end{lemma}
\newpage
\begin{proof}
	Wenn die Aussage nicht gilt, dann gilt mit $$p(x,y):=x_ny^n+x_{n-1}y^{n-1}+\dots,x_0,$$ dass für jedes $a\in (I_A)^{n+1}$ ein $\alpha\in A^n$ existiert mit $p(a,b)=f(\alpha,b)$. Es muss für festes $a\in I_A$ ein Intervall um $b$ in $I_A$ mit dieser Eigenschaft geben, denn sonst wäre $b\in\dcl(A)=A$.\\
	Sei jetzt $a$ nicht mehr fixiert, dann existiert mit Definable Choice in $I_A$ eine definierbare Zuordnung $a\mapsto\alpha(a)$, sodass $p(a,\cdot)=f(\alpha(a),\cdot)$ auf einem Intervall gilt. Da jedes $a$ $n+1$ viele Einträge hat und jedes $\alpha(a)$ $n$ viele, müssen unendlich viele $a\in (I_A)^{n+1}$ existieren, die durch $\alpha$ auf das selbe Element abgebildet werden. Denn wenn das nicht so wäre, wäre wegen Saturation ein generisches Element über all den vorigen zu Definitionen benutzten Parametern aus $(I_A)^{n+1}$ algebraisch über einem Element aus $A^n$, was der Generizität widerspricht. Da es unendlich viele Elemente gibt, sodass $\alpha$ auf ihnen konstant ist, gibt es schon eine Zelle von Dimension $>0$ mit der Eigenschaft und damit insbesondere eine Zelle $E$ von Dimension 1 (als Teilmenge einer Zelle lässt sich immer eine von kleinerer Dimension finden). Nenne den konstanten Wert dann $\alpha^*$.\\
	Da also gilt: für alle $a\in E$ existiert ein Intervall $J$ mit $p(a,\cdot)=f(\alpha^*,\cdot)$ auf $J$, existieren mit Definable Choice $\beta^*,\gamma^*:E\rightarrow I_A$, sodass $p(a,\cdot)=f(\alpha^*,\cdot)$ auf $(\beta^*(a),\gamma^*(a))$ gilt. \OE\ seien $\beta^*$ und $\gamma^*$ jetzt schon stetig auf $E$ und ein $e\in E$ beliebig· Dann existiert für $\varepsilon$ hinreichend klein eine $E$-Umgebung $U$ um $e$, sodass $$\beta^*<\frac{1}{2}(\beta^*(e)+\gamma^*(e))-\varepsilon,\ \frac{1}{2}(\beta^*(e)+\gamma^*(e))+\varepsilon<\gamma^*$$ auf $U$, also $$p(a,x)=f(\alpha^*,x)\text{ für alle }a\in U,x\in(\frac{1}{2}(\beta^*(e)+\gamma^*(e))-\varepsilon,\frac{1}{2}(\beta^*(e)+\gamma^*(e))+\varepsilon)$$ gilt. Es kann aber nicht $p(a-a',x)=p(a,x)-p(a',x)=f(\alpha^*,x)-f(\alpha^*,x)=0$ für $a,a'\in U$ verschieden und unendlich viele $x$ sein, weil ein Nichtnullpolynom nicht unendlich viele Nullstellen haben kann.
\end{proof}

\newpage

\begin{corollary}
	Es sei $(A,B)\models\td,\ f:B^{n+1}\rightarrow B$ $A$-definierbar in $B$ und $b\in B\setminus A$. Dann enthält $f(A^n\times\{b\})$ kein Intervall um $b$.
\end{corollary}
\begin{proof}
	Nimm an, dass das Gegenteil gelte für das Intervall $J$ (\OE\ mit Randpunkten $c<d\in A$): Dann ist $$x\mapsto\frac{1}{c-x}+\frac{1}{d-x}$$ eine ordnungstreue $A$-definierbare Bijektion $(c,d)\rightarrow A$. Diese erzeugt in $A$ eine definierbare angeordnete Körperstruktur auf $(c,d)=J_A$, also auch auf $J$ eine $A$-definierbare angeordnete Körperstruktur. Mit dem vorigen Lemma existiert ein Element aus\linebreak$J\setminus f(A^n\times\{b\})$, denn es spielt für die Aussage keine Rolle, ob man zu einer genügend saturierten Elementarerweiterung von $(B,A)$ übergeht.
\end{proof}

\begin{theorem}\label{Kleinheit}
	Wenn $(B,A)\models\td$, dann ist kein Intervall eine kleine Teilmenge.
\end{theorem}
\begin{proof}
	Sei $f:B^n\rightarrow B$ eine durch $\varphi(x,y,b)$ definierbare Abbildung mit $\varphi$ einer\linebreak$\lingua_A$-Formel und $b\in B^m$ für ein $m\in\setN$ definiert. Für $\dim(b/A)=0$ ist $f(A^n)\subseteq A$ klar, deswegen sei \OE\ $\dim(b/A)\geq1$. Definiere
	\begin{align*}
	g(x,z):=\left\{\begin{array}{ll}
	\text{das eindeutige }y\in B &\text{für alle z, für die }\varphi(x,y,z)\\
	\text{mit }B\models\varphi(x,y,z) &\text{bei festem }z\text{ eine Funktion definiert}\\
	\ &\ \\
	0 &\text{sonst}
	\end{array}\right.,
	\end{align*}
	Dann ist $g$ in $B$ $A$-definierbar und $g(\cdot,b)=f$. Falls $\dim(b/A)>1$, füge genug Komponenten von $b$ zu $A$ hinzu, sodass $\dim(b/A)=1$. Das Hinzufügen einer Komponente $b_j$ zu $A$ führt zu einem neuen Gegenbeispiel zur Aussage in einem dichten Paar, denn $Ab_j$ ist nach den Eingangsbemerkungen Modell von T und $Ab_j$ ist erst recht dicht in, aber nicht gleich $B$ (sonst hätte man die Dimension mit diesem Schritt schon zu sehr verkleinert).\\
	Finde also $b_i$, sodass $A$-definierbare $(h_j)$ existieren mit $b_j=h_j(b_i)$ für alle $j$. Wenn jetzt $J\subseteq f(A^n)=g(A^n,b)=g(A^n,h(b_i))$ für ein Intervall $J$, dann widerspricht das der Aussage des letzten Lemmas für die Funktion $(x,y)\mapsto g(x,h(y))$.
\end{proof}

\begin{definition}
	Schreibe ab jetzt $P(\overline{x}):=\bigwedge\limits_{i=1}^\abs{x}P(x_i)$.
\end{definition}

\newpage

Zum Abschluss dieses Abschnittes sollen hier noch ein paar weitere Fakten über dichte Paare aufgeführt werden, die später eine Bedeutung bekommen.

\begin{lemma}
	Wenn $(B,A)$ für ein unendliches $\kappa>\abs{\operatorname{T}}$ ein $\kappa$-saturiertes Modell von $\td$ ist, ist $\dim(B/A)\geq\kappa$.
\end{lemma}
\begin{proof}
	Sei $S$ eine Basis von $B/A$ mit $\abs{S}<\kappa$; zeige nun, dass es kein Erzeugendensystem sein kann. Das folgt aus der Saturation angewandt auf den partiellen Typen $$\{\forall\overline{y}\in P(x\neq t(\overline{y}))\mid t\ \lingua_S\text{-Term}\},$$ der endlich erfüllbar ist, weil die Negation jeder dieser Formeln \glqq{}$x$ ist in einer kleinen Menge\grqq{} impliziert. Wenn der Typ also nicht endlich erfüllbar wäre, würde eine endliche Vereinigung von kleinen Mengen ganz $B$ überdecken. Das kann aber nicht gelten, denn eine endliche Vereinigung von kleinen Mengen ist wieder klein.
\end{proof}

\begin{corollary}\label{Finden transz Elte}
	Da Intervalle nicht klein sind, zeigt ein abgewandelter Beweis sogar, dass in einem $\kappa$-saturiertem Modell $(B,A)\models\td$, gegeben Mengen $S,S',S''\subset B$ mit $$\abs{S},\abs{S'},\abs{S''}<\kappa,$$ ein transzendentes Element $b$ über $SA$ gefunden werden kann mit $a<b$ für alle $a\in S'$ und $b<c$ für alle $c\in S''$, sofern dieser Ordnungstyp von $b$ überhaupt konsistent ist.
\end{corollary}

\begin{lemma}\label{Kodichte von A}
	Sei $(B,A)\models\td$. Dann ist $A$ auch kodicht in $B$.
\end{lemma}
\begin{proof}
	Zu zeigen ist, dass für alle $a,c\in B$ ein $b\in B\setminus A$ existiert mit $a<b<c$. Durch Translation und additive Inversion kann man annehmen, dass $a=0$. Wähle jetzt ein $d\in B\setminus A$ beliebig und $e\in A$ mit $d-c<e<d$. Dann ist $d-e$ nicht in $A$ (denn sonst wäre es $d$) und $$0=e-e<d-e<d-(d-c)=c.$$
\end{proof}

\newpage

\section{Formelreduzierung in $\td$}
In diesem Abschnitt wird gezeigt, dass sich $\lingua^P$-Formeln modulo $\td$ sehr stark vereinfachen lassen und dass $\td$ vollständig ist. Indem dieses mit einem Back\&Forth-System gezeigt wird, erhält man zusätzlich eine sehr große Klasse von elementaren Abbildungen zwischen Modellen von $\td$.\\
In diesem Kontext wird wieder die $\acl$-Unabhängigkeit (TODO: Sollte man das vielleicht doch definieren?) in einem Modell von T relevant, die im ersten Kapitel mit \glqq{}algebraisch disjunkt\grqq{} bezeichnet wurde. Man kann sich dafür folgende (teilweise schon bekannte) Fakten überlegen.

\begin{lemma}\label{Unabhängigkeitsregeln}
	Seien $A,B,C,D$ Mengen in irgendeinem Modell von T.
	\begin{enumerate}
		\item Wenn $A$ und $B$ unabhängig über $C$ sind, sind $B$ und $A$ unabhängig über $C$ und $A\cap B\subseteq\acl(C)$ (in fast allen betrachteten Fällen wird sowieso $A,B\supseteq C$ und $C=\acl(C)$ gelten, sodass dann Gleichheit herrscht).
		\item Wenn $A$ und $B$ unabhängig über $C$ sind und $S\subseteq B$, dann sind auch $A\cup S$ und $B$ unabhängig über $C\cup S$.
		\item Wenn $A$ und $B$ unabhängig über $C$ sind, $$A\subseteq S\subseteq\acl(A),B\subseteq S'\subseteq\acl(B),C\subseteq S''\subseteq\acl(C),$$ dann sind $S$ und $S'$ unabhängig über $S''$.
		\item Wenn $A$ und $B$ unabhängig über $C$ sind und $D$ (algebraisch) unabhängig über $AB$, dann sind $A\cup D$ und $B$ unabhängig über $C$.
		\item Wenn $(D,C)\preceq(B,A)\models\tq$, dann sind $A$ und $D$ unabhängig über $C$.
		\item Wenn $(D,C)\subseteq(B,A)\models\tq,\ S\subseteq A$ und $A$ und $D$ unabhängig über $C$ sind, dann sind $A$ und $DS$ unabhängig über $CS$, $\langle D\cup S\rangle_{\lingua^P}=(DS,CS)$ und $$(D,C)\subseteq(DS,CS)\subseteq(B,A).$$
		\item Wenn $(D,C)\subseteq(B,A)\models\tq$ und $S\subseteq B$ unabhängig über $DA$ ist, dann sind $A$ und $DS$ unabhängig über $C$, $\langle D\cup S\rangle_{\lingua^P}=(DS,C)$ und $$(D,C)\subseteq(DS,C)\subseteq(B,A).$$
	\end{enumerate}
\end{lemma}
\newpage
\begin{proof}
	1.-4. sind bekannt. TODO: Vielleicht doch besser ausführen.
	\item[5.] Wenn $\overline{d}\in D$ algebraisch unabhängig über $C$ ist, aber nicht über $A$, dann existiert eine $\lingua_A$-Formel $\varphi(\overline{x},\overline{a})$, sodass \OE\ $d_1$ von $\varphi(x_1,d_2,d_3\dots,\overline{a})$ algebraisiert wird (\OE\ wird $d_1$ schon durch $\varphi$ definiert). Also erfüllt $\overline{d}$ die $\lingua^P$-Formel $$\exists \overline{y}\in P(\varphi(\overline{x},\overline{y})\land\forall z_2,z_3,\dots\exists! z_1(\varphi(\overline{z},\overline{y})))$$ in $(B,A)$, also auch in $(D,C)$. Es existiert also $\overline{c}\in C$ mit $$B\models\varphi(\overline{d},\overline{c})\land\forall z_2,z_3,\dots\exists! z_1(\varphi(\overline{z},\overline{c})),$$ was im Widerspruch zur Unabhängigkeit von $\overline{d}$ über $C$ steht.
	\item[6.] Dass $A$ und $DS$ unabhängig über $CS$ sind, ergibt sich in der Kombination von 2. und dann 3.\\
	Dass die Trägermenge von $\langle D\cup S\rangle_{\lingua^P}$ die Menge $DS$ ist, ergibt sich direkt per Definition als $DS=\dcl(D\cup S)=\langle D\cup S\rangle_\lingua$. Weil $A$ und $DS$ unabhängig über $CS$ sind, folgt $$P(\langle D\cup S\rangle_{\lingua^P})=DS\cap P(B)=DS\cap A=CS.$$
	\item[7.] Es ergibt sich aus 4. und 3., dass $A$ und $DS$ unabhängig über $CS$ sind. Der Rest folgt analog zu 6.
\end{proof}

Zu bemerken ist, dass ein Spezialfall von Unabhängigkeit viele nützliche Eigenschaften hat. Auf diesen wird später noch oft zurückgegriffen werden.
\begin{definition}
	Seien $(D,C)\subseteq(B,A)$ zwei Modelle von $\tq$. Dann heiße diese Inklusion \textbf{frei}, wenn $D$ und $A$ unabhängig über $C$ sind.
\end{definition}

\newpage
Für die Konstruktion des gewünschten Back\&Forth-Systems sei $\kappa>\abs{\operatorname{T}}$ eine beliebige, aber feste Kardinalzahl und $(B,A),(D,C)\models\td$ zwei $\kappa$-saturierte Modelle.
\begin{theorem}\label{BackForth}
	Sei $S$ die Menge aller partiellen Isomorphismen zwischen Unterstrukturen $(B',A')$ von $(B,A)$ und $(D',C')$ von $(D,C)$ der Mächtigkeit $<\kappa$, sodass die Inklusionen frei sind. Dann bildet $S$ ein nichtleeres B\&F-System und $\td$ ist insbesondere vollständig.
\end{theorem}
\begin{proof}
	Das System ist nichtleer, denn es gibt ein Primmodell $\fM$ von T. T ist bekanntlich vollständig und in jedem Modell $A$ werden alle Eigenschaften von $\fM_A:=\langle\emptyset\rangle_\lingua$ in T beschrieben, also sind alle Modelle $(\fM_A)_A$ isomorph. Klarerweise ist $\abs{M}=\abs{\operatorname{T}}<\kappa$. Der Isomorphismus $(M_A,M_A)\cong(M_C,M_C)$ liegt in $S$, denn Unabhängigkeit ist bei zwei gleichen Mengen offensichtlich.\\
	Sei jetzt $S\ni i:(B',A')\rightarrow(D',C')$ und $b\in B$; zu zeigen ist, dass es eine Erweiterung auf $b$ gibt. Wenn $b\in B'$ ist, ist nichts zu zeigen. Wenn $b\in A\setminus B'$, betrachte den partiellen Typ über $D'$ $$\{\alpha<x\mid i^{-1}(\alpha)<b\}\cup\{x<\beta\mid b<i^{-1}(\beta)\}\cup\{P(x)\}.$$
	Dieser ist konsistent, da $i$ ein Isomorphismus ist und $C$ dicht in $D$; mit Saturation existiert ein $d\in C\setminus D'$ mit diesem Ordnungstyp. $i$ setzt sich dann eindeutig zu einem Isomorphismus $$i':(B'b,A'b)\rightarrow(D'd,C'd)\text{ mit }i(b)=d$$ fort, der gegeben ist durch die Abbildung $t(b)\mapsto i(t)(d)$ für $t$ einen $\lingua_{B'}$-Term und $i(t)$ den durch $i$ geshifteten Term. Die Surjektivität dieser Abbildung ist klar, ebenso dass $i'(A'b)=C'd$. Wohldefiniertheit, Injektivität und Isomorphismuseigenschaft gelten, denn:\\
	$Rt_1(b)\dots t_n(b)$ gilt für $\lingua_{B'}$-Terme $t_1,\dots,t_n$ und eine Relation $R$ genau dann, wenn es ein $B'$-definierbares Intervall $I$ um $b$ mit dieser Eigenschaft gibt (denn sonst wäre $b$ definierbar über $B'$ und somit in $B'$). Schickt man $I\cap B'$ mit $i$ nach $$J:=i(I\cap B')=(i(\inf I),i(\sup I))\cap D',$$ so gilt für alle Elemente $z\in J$, dass $Ri(t_1)(z)\dots i(t_n)(z)$, da $i$ ein Isomorphismus ist. Das gilt wegen o-Minimalität sogar auch noch für alle Elemente $z\in(i(\inf I),i(\sup I))$, weil dieses Intervall $D'$-definierbar ist und man sonst mit Definable Choice ein Element aus $$(i(\inf I),i(\sup I))\cap D'$$ finden könnte, für das das nicht gilt.
	\newpage
	Da $i(\inf I)<d<i(\sup I)$ aus dem Typ hervorgeht, gilt also auch $Ri(t_1)(d)\dots i(t_n)(d)$.\\
	Die Rückrichtung der Äquivalenz geht analog.\\
	Zu zeigen ist nun, dass $B'b$ und $A$ frei über $A'b$ sowie $D'd$ und $C$ frei über $C'd$ sind, ebenso zu zeigen ist noch, dass $$(B'b,A'b)\subseteq(B,A),(D'd,C'd)\subseteq(D,C).$$ Das alles folgt aber aus Lemma \ref{Unabhängigkeitsregeln} (6.). Außerdem gilt $\abs{D'd}=\abs{B'b}=\abs{B'}+\abs{\operatorname{T}}<\kappa$.\\
	Sei jetzt $b\in B'A\setminus(A\cup B')$. Dann gibt es $\overline{a}\in A$ mit $b\in B'\overline{a}$. Erweitere wie schon bekannt $i$, sodass $\overline{a}\in\operatorname{dom}(i)$; dann ist schon ganz $B'\overline{a}\subseteq\operatorname{dom}(i)$, also auch $b$.\\
	Abschließend sei $b\in B\setminus B'A$;  wie oben erfülle dann den mit $i$ geshifteten Ordnungstyp von $b$ über $B'$ mit einem Element $d\in D\setminus D'C$ (mit Folgerung \ref{Finden transz Elte} geht das). Wieder kann $i$ dann auf einen Isomorphismus $(B'b,A')\rightarrow(D'd,C')$ fortgesetzt werden und nach Lemma \ref{Unabhängigkeitsregeln} (7.) erfüllen $(B'b,A'),(D'd,C')$ auch die hinreichenden Eigenschaften.
\end{proof}

Das eben aufgestellte B\&F-System beweist die Formelreduzierung in $\td$.

\begin{theorem}\label{Formelreduzierung}
	Jede $\lingua^P$-Formel ist modulo $\td$ äquivalent zu einer booleschen Kombination von Formeln der Gestalt
	$$\exists\overline{y}\in P(\phi(\overline{x},\overline{y}))$$
	für $\phi$ eine $\lingua$-Formel. Nenne eine solche Formel eine \textbf{gute Formel in Reinform} und eine boolesche Kombination davon eine \textbf{gute Formel}.
\end{theorem}
\begin{proof}
	\underline{Hilfsaussage:}\\
	Es reicht zu zeigen, dass für alle Modelle $(B,A),(D,C)\models\td$ und für alle $b\in B^n$, $d\in D^n$ gilt: Wenn $b$ und $d$ dieselben guten Formeln erfüllen, sind ihre Typen in $(B,A)$ und $(D,C)$ dieselben.\\
	Dass dies ausreicht, erkennt man mit dem Ziegler'schen Trennungslemma:\linebreak
	Sei $\psi\in\fF_n(\lingua^P)$ nicht äquivalent zu einer guten Formel und nenne die Menge aller guten Formeln in $n$ freien Koordinaten $K$. Dann ist $K$ abgeschlossen unter $\land,\lor$ und enthält $\top,\bot$. Wenn $\psi$ nicht äquivalent zu einer Formel aus $K$ ist, sind $\td\cup\{\psi\}$ und $\td\cup\{\neg\psi\}$ nicht durch $K$ trennbar, also existieren $(B,A),(D,C)\models\td,b\in B^n,d\in D^n$, sodass $(B,A)\models\psi(b)$ und $(D,C)\models\neg\psi(d)$, aber $(B,A)\models\chi(b)$ genau dann, wenn $(D,C)\models\chi(d)$, und das für alle $\chi\in K$. Dann erfüllen $b$ und $d$ dieselben guten Formeln, aber haben nicht denselben Typ - ein Widerspruch!
	\newpage
	\begin{proof2}[Beweis der Hilfsaussage]
		Seien $b,d$ wie verlangt und $(B,A),(D,C)$ schon \OE\linebreak $\abs{\operatorname{T}}^+$-saturiert, denn das ändert nichts an Typen und dem Erfüllen von guten Formeln. Sei $a\in A^m$ für ein hinreichend großes $m$, mit der Eigenschaft dass $$\dim(b/a)\leq\dim(b/A).$$ So etwas gibt es, wenn man endlich viele $\lingua_A$-Formeln betrachtet, sodass jede Interdefinierbarkeit in $b$ durch diese Formeln bezeugt wird; als $a$ kann man dann die vereinigte Parametermenge dieser ganzen Formeln nehmen. Es folgt dann Gleichheit der Dimensionen, da über einer kleineren Menge nicht mehr interdefinierbar werden kann. Für $$A':=\dcl(a),B':=\dcl(a,b)$$ gilt dann, dass $A$ und $B'$ unabhängig über $A'$ sind. Es sind nämlich per Definition von $a$ die Mengen $A$ und $b$ unabhängig über $a$ (eben wegen $\dim(b/a)=\dim(b/A)$), mit Lemma \ref{Unabhängigkeitsregeln} (2.) sind dann auch $A$ und $a\cup b$ unabhängig über $a$ und mit (3.) sind $A$ und $B'=\dcl(a,b)$ unabhängig über $A'=\dcl(a)$. Außerdem sind $A'$ und $B'$ maximal $\abs{\operatorname{T}}$ groß und $(B',A')\subseteq (B,A)$.\\
		Wenn man den partiellen $\lingua^P$-Typ $\tp_\lingua(a/b)\cup\{P(\overline{x})\}$ betrachtet, bleibt er konsistent unter der Ersetzung $b\rightarrowtail d$ in den Formeln. Seien nämlich ${\psi_1(\overline{x},b),\dots,\psi_n(\overline{x},b)\in\tp_\lingua(a/b)}$, dann ist $$\exists\overline{x}\in P(\bigwedge\limits_{i=1}^n\psi_i(\overline{x},\overline{y}))$$ eine gute Formel, die von $b$ und daher auch von $d$ erfüllt wird. Also ist der ersetzte partielle Typ endlich konsistent, wegen Saturation habe er den Erfüller $c\in C$ und es gilt $\tp_\lingua(a,b)=\tp_\lingua(c,d)$. Wegen der Typengleichheit folgt insbesondere $$\dim(b/a)=\dim(d/c);$$ es bleibt noch zu zeigen, dass $\dim(b/A)=\dim(d/C)$, damit dann gilt $$\dim(d/C)=\dim(b/A)=\dim(b/a)=\dim(d/c)$$ und wie oben $C$ und $D':=\dcl(c,d)$ frei über $C':=\dcl(c)$ sind sowie $(D',C')\subseteq(D,C)$.
		\newpage
		Die Gleichheit $\dim(b/A)=\dim(d/C)$ gilt aber, da für jede $\lingua$-Formel $\psi$ und $j_1,\dots,j_n\in\setN$ die Formel zu $$\glqq{}\text{es existiert }\overline{y}\in P\text{, sodass }\psi(\overline{x},\overline{y})\ x_i\text{ über }x_{j_1},\dots,x_{j_m}\text{ definiert}\grqq{}$$ eine gute Formel in Reinform ist, die also genau dann von $b$ erfüllt wird, wenn sie von $d$ erfüllt wird.\\
		Da $(a,b)$ und $(c,d)$ den gleichen $\lingua$-Typ haben, gibt es einen partiellen Isomorphismus $i$ von $B'=\dcl(a,b)$ nach $D'=\dcl(c,d)$ durch Abbilden der $\lingua_{a,b}$-Terme auf $\lingua_{c,d}$-Terme. Dieser erfüllt $i((a,b))=(c,d)$, die Einschränkung auf $A'=\dcl(a)$ bildet einen Isomorphismus nach $C'=\dcl(c)$. Also ist $i$ partieller Isomorphismus $(B',A')\rightarrow(D',C')$, damit im B\&F-System, also elementare Abbildung, weswegen $b$ und $d$ denselben $\lingua^P$-Typen haben.
	\end{proof2}
\end{proof}

\begin{corollary}\label{Definierbarkeit aus A}
	Für ein dichtes Paar $(B,A)$ und $S\subseteq B^n$ eine $\lingua^P_{A_0}$-definierbare Menge (wobei $A_0\subseteq A$) ist $S\cap A^n$ eine $\lingua_{A_0}$-definierbare Teilmenge von $A$.
\end{corollary}
\begin{proof}
	Nach der Formelreduzierung sei $S$ \OE\ durch eine gute Formel definiert. Da Definierbarkeit abgeschlossen unter booleschen Kombinationen ist, reicht es, eine Formel in Reinform zu betrachten.\\
	Für jede $\lingua_{A_0}$-Formel $\varphi(x,y,a')$ und jedes $a\in A^n$ sind die Aussagen $$\glqq{}\text{Es existiert ein }y\in A^m\text{ mit }(B,A)\models\varphi(a,y,a')\grqq{},$$ $$\glqq{}\text{Es existiert ein }y\in A^m\text{ mit }B\models\varphi(a,y,a')\grqq{},$$ $$\glqq{}\text{Es existiert ein }y\in A^m\text{ mit }A\models\varphi(a,y,a')\grqq{}$$ äquivalent wegen $\varphi$ als $\lingua$-Formel und $A\prec B$. Es folgt, dass $$(\exists y\in P(\varphi(x,y,a')))(B)\cap A^n=(\exists y(\varphi(x,y,a')))(A).$$
\end{proof}
\newpage
\section{Folgen der Existenz des B\&F-Systems}
Im Folgenden werden einige Anordnungen von wechselseitigen Inklusionen von Modellen von T betrachtet, in denen Gleichheit von bestimmten Typen folgt.

\begin{lemma}\label{freie Inklusionen}
	Für dichte Paare $(B,A),(D,C)$ mit $(D,C)\subseteq(B,A)$ sind folgende Eigenschaften äquivalent:
	\begin{enumerate}
		\item $(D,C)\preceq(B,A)$
		\item Die Inklusion ist frei.
	\end{enumerate}
\end{lemma}
\begin{proof}
	$\glqq{}1.\Rightarrow2.\grqq{}:$ Diese Richtung ist schon aus Lemma \ref{Unabhängigkeitsregeln} (5.) bekannt.\\
	$\glqq{}2.\Rightarrow1.\grqq{}:$ Finde $(\abs{B}+\abs{\operatorname{T}})^+$-saturierte Strukturen $$(B,A)\preceq(B',A'),(D,C)\preceq(D',C');$$ es ist dann $(D,C)$ eine gemeinsame Unterstruktur und $(D,C)\subseteq(D',C')$ ist frei nach dem Beweis der Gegenrichtung. Außerdem sind nach Voraussetzung $D$ und $A$ unabhängig über $C$, also wird Unabhängigkeit von Tupeln in $D$ über $C$ \glqq{}hochgegeben\grqq{} über $A$, da aber Unabhängigkeit von Tupeln in $D\subseteq B$ über $A$ auch über $A'$ erhalten bleibt (da $(B',A')$ elementare Oberstruktur), ist auch $(D,C)\subseteq(B',A')$ frei. Also ist die Identität auf $(D,C)$ im Back\&Forth-System, daher elementare Abbildung. Daraus folgt für alle $\lingua^P_D$-Formeln $\varphi$, dass $$(D,C)\models\phi\Leftrightarrow(D',C')\models\varphi\Leftrightarrow(B',A')\models\varphi\Leftrightarrow(B,A)\models\varphi.$$
\end{proof}

\begin{lemma}\label{Gemeinsame Unterstruktur}
	Seien $(B_1,A_1),(B_2,A_2)\models\td$ und $(B,A)$ eine gemeinsame Unterstruktur, sodass die Inklusionen frei sind. Wenn $a\in (A_1)^n$ und $b\in (A_2)^n$ denselben $\lingua$-Typen über $B$ erfüllen, erfüllen sie auch denselben $\lingua^P$-Typen über $B$.
\end{lemma}
\begin{proof}
	\OE\ seien $(B_1,A_1)$ und $(B_2,A_2)$ schon genügend saturiert, das ändert nichts an Typen über $B$ und (nach derselben Argumentation wie im vorigen Lemma) auch nichts an der Unabhängigkeit. Da $a$ und $b$ denselben Typen über $B$ erfüllen, kann man wieder $\lingua_B$-Terme mit eingesetztem $a$ auf $\lingua_B$-Terme mit eingesetztem $b$ abbilden (Wohldefiniertheit und Injektivität wird durch die Typengleichheit ermöglicht) und bekommt einen partiellen Isomorphismus $i:Ba\cong Bb$, dessen Einschränkung auf die $\lingua_A$-Terme einen partiellen Isomorphismus $Aa\cong Ab$ induziert und sodass $i(a)=b$. Also gilt $i:(Ba,Aa)\cong(Bb,Ab)$, da außerdem die Inklusionen $(Ba,Aa)\subseteq(B_1,A_1)$ und $(Bb,Ab)\subseteq(B_2,A_2)$ frei sind nach Lemma \ref{Unabhängigkeitsregeln} (6.), ist $i$ im Back\&Forth-System, also elementar, also haben $a$ und $b$ denselben $\lingua^P$-Typen über $B$.
\end{proof}

\begin{corollary}
	Wenn man sich solch ein Paar $(a,b)$ beliebig wählt (z.B. $a=b=0$), sind in dem Typen auch die parameterfreien $\lingua^P_B$-Formeln, die in $(B_1,A_1)$ bzw. $(B_2,A_2)$ gelten. Also gelten dieselben Formeln, was als $(B_1,A_1)\equiv_B(B_2,A_2)$ geschrieben wird und Lemma \ref{freie Inklusionen} verallgemeinert.
\end{corollary}

\begin{lemma}\label{selber Schnitt}
	Seien $(B_1,A_1),(B_2,A_2)$ zwei dichte Paare und $A\subseteq A_1\cap A_2$ eine gemeinsame Substruktur, sowie $a\in B_1\setminus A_1,\ b\in B_2\setminus A_2$, die den gleichen Ordnungstyp über $A$ haben. Dann haben $a$ und $b$ sogar den gleichen $\lingua^P$-Typ über $A$.
\end{lemma}
\begin{proof}
	Es sind trivialer Weise $A_i$ und $A$ unabhängig über $A$ für $i=1,2$, außerdem ist $a$ transzendent über $A_1$ und $b$ transzendent über $A_2$. Nach Lemma \ref{Unabhängigkeitsregeln} (4.) sind also die Einbettungen $$(Aa,A)\subseteq(B_1,A_1)\text{ und }(Ab,A)\subseteq(B_2,A_2)$$ frei. Nach dem Beweis zu Satz \ref{BackForth} gibt es einen Isomorphismus $Aa\cong Ab$, der $A\cong A$ fortsetzt, und für den $i(a)=b$ gilt, also einen Isomorphismus $i:(Aa,A)\cong(Ab,A)$.\\ Wenn \OE\ die beiden Modelle von $\td$ genügend saturiert sind, ist $i$ im B\&F-System, also erfüllen $a$ und $b$ dieselben $\lingua^P_A$-Formeln.
\end{proof}

\begin{lemma}\label{Speziell definierbare kleine Mengen}
	Sei $(D,C)\subseteq(B,A)$ frei und $(B,A)\models\td$, sowie $X\subseteq B$ eine kleine $\lingua^P_D$-definierbare Menge. Dann gibt es sogar eine $\lingua_D$-definierbare Funktion $f:B^n\rightarrow B$, die die Kleinheit von $X$ bezeugt.
\end{lemma}
\begin{proof}
	Man kann für $(B,A)$ schon hinreichende Saturiertheit annehmen. Wenn es keine solche Funktion gäbe, wäre $$\{x\in X\}\cup\{x\notin f(A^n)\mid f:B^n\rightarrow B\ D\text{-definierbar}\}$$ eine konsistente Formelmenge über $D$; ihr Erfüller sei $b\in B\setminus DA$. Da $X$ klein ist und als solches kein Intervall enthält, kann man den Ordnungstyp von $b$ über $D$ auch durch ein Element $$\tilde{b}\in B\setminus(X\cup DA)$$ erfüllen wie in Folgerung \ref{Finden transz Elte}. Nach Konstruktion des B\&F-Systems haben $b$ und $\tilde{b}$ dann aber den selben $\lingua^P_D$-Typ über $D$, was im Widerspruch zu $b\in X,\tilde{b}\notin X$ steht.
\end{proof}

\section{Definierbare Teilmengen von $A^n$}
Wir interessieren uns für die Gestalt von $\lingua^P$-definierbaren Teilmengen von $A^n$. Dafür braucht man zuerst eine Hilfsaussage für definierbare Mengen in o-minimalen Strukturen.

\begin{lemma}
	Sei $\fM$ eine o-minimale Struktur, die eine angeordnete 0-definierbare Gruppenoperation $+$ mit positivem Element 1 hat und $Y\subseteq M^n$ definierbar. Dann ist $Y$ eine endliche Vereinigung von Mengen der Form $$\{f(b,\cdot)=0,g(b,\cdot)>0\},$$ wobei $b\in M^m$ und $f,g$ stetige, 0-definierbare Abbildungen $M^{m+n}\rightarrow M$ sind.
\end{lemma}
\begin{proof}
	Schreibe $Y=\phi(b,\fM)$ für ein $b\in M^m$ und definiere $Z:=\phi(\fM)$. Wenn man $Z$ in 0-definierbare Zellen $(Z_i)_i$ zerlegt, erhält man $Y$ als endliche Vereinigung der Fasern $((Z_i)_b)_i$.  Es sei also \OE\ $Z$ schon eine 0-definierbare Zelle.\\
	Definiere $$f(x):=\left\{\begin{array}{ll}
	\inf\{\abs{x-d}\mid d\in Z\}&Z\text{ nichtleer}\\
	1&\text{sonst}
	\end{array}\right.,$$
	$$g(x):=\left\{\begin{array}{ll}
	\inf\{\abs{x-d}\mid d\in \overline{Z}\setminus Z\}&Z\text{ nichtleer}\\
	1&\text{sonst}
	\end{array}\right.,$$ das sind lipschitzstetige Funktionen.\\
	Klar ist, dass $\overline{Z}=\{f=0\}$; da Zellen lokal abgeschlossen sind, ist $$\overline{Z}\setminus Z=\overline{\overline{Z}\setminus Z}=\{g=0\}.$$ Also erhalten wir $$Z=\overline{Z}\setminus(\overline{Z}\setminus Z)=\{f=0\}\setminus\{g=0\}=\{f=0\}\cap\{g>0\}$$ und $$Y=Z_b=\{f(b,\cdot)=0\}\cap\{g(b,\cdot)>0\}.$$
\end{proof}
\newpage
\begin{theorem}\label{Definierbare Mengen}
	Für ein dichtes Paar $(B,A)$ und $Y\subseteq A^n$ ist folgendes äquivalent:
	\begin{enumerate}
		\item $Y$ ist $\lingua^P$-definierbar.
		\item Es existiert ein $\lingua$-definierbares $Z\subseteq B^n$, sodass $Y=Z\cap A^n$.
		\item $Y$ ist definierbar in $(A,(R_b)_{b\in B})$ mit der Interpretation $A\models R_b(a)$ genau dann, wenn $0<a<b$ in $B$.
	\end{enumerate}
\end{theorem}
\begin{proof}
	$\glqq{}1.\Rightarrow 2.\grqq{}:$ Sei $\varphi$ eine $\lingua^P_B$-Formel mit $\varphi(B)=Y$. Zu zeigen ist, dass eine $\lingua_B$-Formel $\psi$ existiert mit $(B,A)\models P(x)\rightarrow(\varphi(x)\leftrightarrow\psi(x))$;
	 das ist genau dann der Fall, wenn $$\mathfrak{Th}(B,A)_B\cup\{P(x)\}\cup\{\varphi(x)\}\text{ und }\mathfrak{Th}(B,A)_B\cup\{P(x)\}\cup\{\neg\varphi(x)\}$$ in $\lingua_B$ getrennt werden können. Nach dem Trennungslemma gilt das genau dann, wenn für alle $$(B,A)\preceq(D_1,C_1),(D_2,C_2)$$ und alle $c_i\in C_i\ (i=1,2)$ mit $$(D_1,C_1)\models\varphi(c_1),(D_2,C_2)\models\neg\varphi(c_2)$$ eine $\lingua_B$-Formel $\chi$ existiert mit $$(D_1,C_1)\models\chi(c_1),(D_2,C_2)\models\neg\chi(c_2).$$
	Seien solche $(D_i,C_i)$ und $c_i$, die die Voraussetzungen von oben erfüllen. Dann ist das die Situation aus Lemma \ref{Gemeinsame Unterstruktur}, denn elementare Erweiterungen sind frei. Also muss ein trennendes $\chi$ wie verlangt existieren, denn ansonsten würden $c_1$ und $c_2$ denselben $\lingua$-Typ erfüllen, aber nicht denselben $\lingua^P$-Typ.\\
	$\glqq{}2.\Rightarrow 3.\grqq{}:$ Sei $Y=Z\cap A^n$. Nach dem letzten Lemma ist $Z$ eine boolesche Kombination aus Mengen der Form $\{f(b,\cdot)=0\}$ und $\{g(b,\cdot)>0\}$ für stetige 0-$\lingua$-definierbare Funktionen $f,g$ und passende $b\in B^m$. Es reicht also die Aussage für Mengen in diesen Formen zu zeigen.\\
	Wegen der Stetigkeit der Funktionen und $A$ dicht in $B$, gilt aber in $B$
	\begin{align*}
	f(b,z)=0\Leftrightarrow\ &\text{Für alle }0<\varepsilon\in A\text{ existiert }A^m\ni a<b\text{ (koordinatenweise),}\\&\text{sodass für alle }a'\in A^m\text{ mit }a<a'<b\text{ (koordinatenweise)}\\&\text{gilt, dass }\abs{f(a',z)}<\varepsilon,\\
	g(b,z)>0\Leftrightarrow\ &\text{Es existiert ein }0<\varepsilon\in A\text{ und ein }A^m\ni a<b\text{ (koordinatenweise),}\\&\text{sodass für alle }a'\in A^m\text{ mit }a<a'<b\text{ (koordinatenweise)}\\&\text{gilt, dass }g(a',z)>\varepsilon.
	\end{align*}
	Die rechten Bedingungen sind jeweils in $(A,(R_b)_{b\in B})$ definierbar.\\
	$\glqq{}3.\Rightarrow 1.\grqq{}:$ Da $A$ und alle $R_b$ in $(B,A)$ definierbar sind, ist $Y$ auch in $(B,A)$ definierbar.
\end{proof}

\begin{definition}
	Sei $(B,A)\models\tq$. Dann heißt $S\subseteq B$ speziell, wenn $S$ und $A$ frei über $A\cap S$ sind.
\end{definition}

\begin{lemma}\label{Spezialität dcl}
	Wenn $S$ in $(B,A)$ speziell ist, ist auch $\dcl(S)$ speziell und $\dcl(S)\cap A=\dcl(S\cap A)$.
\end{lemma}
\begin{proof}
	Sei $a\in\dcl(S)\cap A$, also ist $a\in A$ und abhängig (als einelementiges Tupel) über $S$. Wegen Unabhängigkeit von $A$ und $S$ ist $a$ abhängig über $S\cap A$, also in $\dcl(S\cap A)$. Andererseits gilt $$\dcl(S\cap A)\subseteq\dcl(S)\cap\dcl(A)=\dcl(S)\cap A$$ sowieso und es folgt die zweite Behauptung. Mit Lemma \ref{Unabhängigkeitsregeln} (3.) folgt, dass $\dcl(S)$ und $A$ unabhängig über $\dcl(S\cap A)$ sind.
\end{proof}

\begin{lemma}\label{Definierbare Mengen Spezialität}
	Der Fall 2. aus dem Satz \ref{Definierbare Mengen} lässt sich wie folgt verallgemeinern: Wenn $Y\subseteq A^n$ eine $\lingua^P_D$-definierbare Menge ist für ein spezielles $D\subseteq B$, dann existiert ein $\lingua_D$-definierbares $Z\subseteq B^n$ mit $Y=Z\cap A^n$.
\end{lemma}
\begin{proof}
	Da eine Ersetzung $D\rightarrowtail\dcl(D)$ nichts ändert, können wir gleich davon ausgehen, dass $D$ schon ein Modell von T ist.\\
	Der Beweis geht dann analog, nur müssen diesmal $$\mathfrak{Th}(B,A)_D\cup\{P(x)\}\cup\{\varphi(x)\}\text{ und }\mathfrak{Th}(B,A)_D\cup\{P(x)\}\cup\{\neg\varphi(x)\}$$ in $\lingua_D$ getrennt werden. Das ergibt anstatt von $(B,A)\preceq(D_1,C_1),(D_2,C_2)$ hier $$(D_1,C_1),(D_2,C_2)\models\mathfrak{Th}(B,A)_D.$$
	Wegen der Spezialität von $D$ gilt, dass die Inklusionen $(D,D\cap A)\subseteq(D_1,C_1),(D_2,C_2)$ frei sind: Wenn $\varphi(c,x)$ nämlich in $(D_i,C_i)$ für ein $i$ die Abhängigkeit von $d\in D^n$ über $C_i$ bezeugt, muss $$\glqq{}\text{Es existiert ein }a'\in P\text{ mit }\varphi(a',x)\text{ bezeugt die Abhängigkeit von }d\text{ über }P\grqq{}$$ in $\mathfrak{Th}(B,A)_D$ gelten, also ist $d$ abhängig über $A$, also wegen Spezialität von $D$ auch abhängig über $D\cap A$. Ab dann kann man den Beweis wie bekannt weiterführen und muss nur $(B,A)$ durch $(D,D\cap A)$ ersetzen.
\end{proof}

\section{Definierbare Mengen in einer Variablen}
In diesem Abschnitt wird eine zur o-Minimalität ähnliche Charakterisierung von Mengen in einer Variable in Modellen von $\td$ hergeleitet. Ab jetzt sei $(B,A)$ ein beliebiges dichtes Paar und $S\subseteq B$ eine spezielle Menge. Da $B$ selbst eine spezielle Menge ist, gelten die folgenden Aussagen insbesondere alle auch für Definierbarkeit unabhängig von irgendwelchen Parametermengen. Angelehnt ist die folgende Vorgehensweise (auch im nächsten Abschnitt) an die Modifikation in \cite{Piz} von \cite{VanDenDries}, Teile konnten noch präziser für spezielle Mengen gemacht werden.

\begin{lemma}
	Sei $Y\subseteq B^n$ $\lingua_S$-definierbar und $(U_y)_{y\in Y}\subseteq B$ eine Familie von offenen, uniform $\lingua_S$-definierbaren Mengen. Dann ist $$X:=\bigcup\limits_{y\in Y\cap A^n}U_y$$ auch $\lingua_S$-definierbar.
\end{lemma}
\begin{proof}
	Wir führen eine Induktion über $\dim Y$, man kann schon annehmen, dass alle $U_y$ nichtleer sind, sonst muss man entsprechend aus $Y$ aussondern. Wenn $\dim Y=0$ ist, ist $Y$ endlich und es ist schon $Y\cap A^n$ definierbar mit Parametern aus $S$ (zähle z.B. auf, das wievielte Element aus $Y$ man nimmt), also auch $X$.\\
	Zerlege $Y$ in Zellen, das ändert nichts an irgendwelchen Definierbarkeiten, also kann man annehmen, dass $Y$ selbst schon Zelle ist. Wenn $Y$ keine offene Zelle ist und ${\pi:Y\rightarrow Z}$ die kanonische homöomorphe Projektion zu einer offenen Zelle von Dimension $m<\dim Y$, existiert nach Satz \ref{Definierbare Mengen Spezialität} ein $\lingua_S$-definierbares $Z'\subseteq B^m$ mit $$\pi(Y\cap A^n)=Z'\cap A^m,$$ da $\pi(Y\cap A^n)\subseteq A^m$ eine $\lingua^P_S$-definierbare Menge ist.\\
	Definiere dann $Y':=Z\cap Z'$, sodass immer noch $\pi(Y\cap A^n)=Y'\cap A^m$ gilt und man die Menge umparametrisieren kann:
	$$\bigcup\limits_{y'\in Y'\cap A^m}U_{\pi^{-1}(y')}=\bigcup\limits_{y\in\pi^{-1}(Y'\cap A^m)}U_y=\bigcup\limits_{y\in Y\cap A^n}U_y=S$$
	Da $(U_{\pi^{-1}(y')})_{y'\in Y'}$ die gleichen Dinge erfüllt wie $(U_y)_{y\in Y}$ und $\dim Y'\leq\dim Z<\dim Y$, gilt die Aussage per Induktion.\\
	Für den Beweis für offene Zellen definiere die vier $\lingua_S$-definierbaren Mengen $$U:=\bigcup\limits_{y\in Y}U_y,$$ $$Y_x:=\{y\in Y\mid x\in U_y\},$$ $$C:=\{x\in U\mid\inn(Y_x)=\emptyset\},$$ $$D:=\bigcup\limits_{x\in\inn C}Y_x.$$
	Es gilt $$D=\bigcup\limits_{x\in U}\{y\in Y\mid x\in U_y\cap\inn C\}=\{y\in Y\mid U_y\cap\inn C\neq\emptyset\}.$$
	Wir wollen jetzt herleiten, dass $\inn D$ leer sein muss. Seien dafür mit Definable Choice die $\lingua_S$-definierbaren Funktionen $f,g_1,g_2:D\rightarrow B$ gegeben mit $$f(y)\in(g_1(y),g_2(y))\subseteq U_y\cap\inn C.$$
	Wenn $\inn D\neq\emptyset$, schränke $D$ ein, sodass es offen ist und $f,g_1,g_2$ stetig auf $D$ sind. Sei außerdem $d,e\in U_y\cap\inn C, (c,h)\subseteq U_y\cap\inn C$ (das ist offen), sodass $$c<g_1(y)<d<f(y)<e<g_2(y)<h.$$ Setze $$V:=g_1^{-1}((c,d))\cap f^{-1}((d,e))\cap g_2^{-1}((e,h)),$$ das ist dann eine offene Umgebung um $y$ in $B^n$. Für alle $z\in V$ ist $$f(y)\in(d,e)\subseteq(g_1(z),g_2(z))\subseteq U_z,$$ also ist $z\in Y_{f(y)}$.\newpage
	Das heißt, es gilt $V\subseteq Y_{f(y)}$; aber das ist unmöglich, weil $V$ offen in dem (eingeschränkten) Definitionsbereich, also auch in $B^n$ ist und also $Y_{f(y)}$ nichtleeres Inneres hätte, was $f(y)\in C$ widerspricht.\\
	$D$ hat als Menge ohne Inneres eine kleinere Dimension als $n$. Also ist induktiv auch $$\bigcup\limits_{y\in D\cap A^n}U_y$$ $S$-definierbar in $B$. Es ist außerdem
	\begin{align}\label{align 1}
	\bigcup\limits_{y\in (Y\setminus D)\cap A^n}U_y\cup((C\setminus\inn C)\cap\bigcup\limits_{y\in Y\setminus D}U_y)=\bigcup\limits_{y\in Y\setminus D}U_y.
	\end{align}
	Dass die linke Seite Teilmenge der Rechten ist, ist sowieso klar per Definition; sei nun $$x\in\bigcup\limits_{y\in Y\setminus D}U_y\setminus (C\setminus\inn C)$$ und $y\in Y\setminus D$ mit $x\in U_y$. Wegen $x\notin C\setminus\inn C$ hat $Y_x$ nichtleeres Inneres ($x\notin C$) oder $x\in\inn C$ und daher $y\in D$ (denn $\{x\}\subseteq U_y\cap\inn C$). Da zweiteres ausgeschlossen wurde, hat $Y_x$ nichtleeres Inneres, also auch $Y_x\setminus D$ (denn $D$ war niedrigdimensional) und enthält damit ein Element $z\in A^n\cap(Y_x\setminus D)$ wegen Dichtheit. $z$ bezeugt, dass $x\in\bigcup\limits_{y\in (Y\setminus D)\cap A^n}U_y$ liegt.\\
	Aus \ref{align 1} folgt, dass $$\bigcup\limits_{y\in (Y\setminus D)\cap A^n}U_y=\left(\bigcup\limits_{y\in Y\setminus D}U_y\setminus(C\setminus\inn C)\right)\cup\left((C\setminus\inn C)\cap\bigcup\limits_{y\in (Y\setminus D)\cap A^n}U_y\right)$$ $\lingua_S$-definierbar ist (der letzte Teil ist es wegen Endlichkeit von $C\setminus\inn C$). Also ist auch $$X=\bigcup\limits_{y\in Y\cap A^n}U_y=\bigcup\limits_{y\in (Y\setminus D)\cap A^n}U_y\cup\bigcup\limits_{y\in D\cap A^n}U_y$$ $\lingua_S$-definierbar in $B$.
\end{proof}
\newpage
\begin{theorem}
	Sei eine $\lingua^P_S$-definierbare Menge $X\subseteq B$. Dann stimmt $X$ bis auf eine kleine $\lingua^P_S$-definierbare Menge mit einer $\lingua_S$-definierbaren Menge $X'$ überein.
\end{theorem}
\begin{proof}
	Sei zunächst $X$ gegeben durch $\exists y\in P(\psi(x,y))$ für eine Formel $\psi\in\fF_{1+n}(\lingua_S)$. Die Mengen $$F_y:=\psi(B,y)\cap\partial\psi(B,y)$$ sind endlich für jedes $y\in M^n$ und weil o-minimale Theorien $\exists^\infty$ eliminieren, ist deren Mächtigkeit uniform beschränkt, sagen wir durch $k$. Sei $$Y:=(\exists x\psi(x,y))(B),Y':=\{y\in Y\mid F_y\neq\emptyset\}$$ und $\lingua_S$-definierbare Funktionen $g_1,\dots,g_k:Y'\rightarrow F_y$, sodass $F_y=\{g_1(y),\dots,g_k(y)\}$ für alle $y\in Y'$. Als diese Funktionen kann man zum Beispiel die angeordnete Aufzählung der Elemente in $F_y$ nehmen. Dann gilt
	\begin{align*}
	X&=\left(\exists y\in P(\psi(x,y))\right)(B,A)=\bigcup\limits_{y\in A^n}\psi(B,y)=\bigcup\limits_{y\in Y\cap A^n}\psi(B,y)\\&=\bigcup\limits_{y\in Y\cap A^n}(\psi(B,y)\cap \partial\psi(B,y))\cup\bigcup\limits_{y\in Y\cap A^n}\inn\psi(B,y)\\&=\bigcup\limits_{y\in Y'\cap A^n}\{g_1(y),\dots,g_k(y)\}\cup\bigcup\limits_{y\in Y\cap A^n}\inn\psi(B,y)\\&=\bigcup\limits_{i=1}^kg_i(Y'\cap A^n)\cup \bigcup\limits_{y\in Y\cap A^n}\inn\psi(B,y).
	\end{align*}
	Da $X':=\bigcup\limits_{i=1}^kg_i(Y'\cap A^n)$ klein ist und $S'':=\bigcup\limits_{y\in Y\cap A^n}\inn\psi(B,y)$ nach dem letzten Lemma $\lingua_S$-definierbar, stimmen $X$ und $X'$ bis auf die kleine $\lingua^P_S$-definierbare Menge $S''$ überein. Da Darstellungen $$\glqq{}(X\setminus  S')\cup S'',X\ \lingua_S\text{-definierbar, }S',S''\text{ klein, }\lingua^P_S\text{-definierbar}\grqq{}$$ unter booleschen Kombinationen erhalten bleiben (Zur Erinnerung: Kleine Mengen bilden ein Ideal), folgt die Aussage für gute Formeln und daher für alle Mengen. Dabei ist zu beachten, dass Satz \ref{Formelreduzierung} auch hergibt, dass $\lingua^P_S$-Formeln zu guten $\lingua^P_S$-Formeln äquivalent sind.
\end{proof}
\newpage
\begin{lemma}
	Sei $X\subseteq B$ eine kleine $\lingua^P_S$-definierbare Teilmenge. Dann ist $X$ eine endliche Vereinigung von Mengen der Form $f(A^n\cap E)$ für $E$ eine offene $\lingua_S$-definierbare Zelle und $f:E\rightarrow B$ eine stetige $\lingua_S$-definierbare Funktion.
\end{lemma}
\begin{proof}
	Wenn $X$ klein ist, existiert ein $\lingua$-definierbares $g:B^m\rightarrow B$, sodass $X\subseteq g(A^m)$, nach Lemma \ref{Speziell definierbare kleine Mengen} kann man schon annehmen, dass $g$ über $\dcl(S)$ definierbar ist (denn $(\dcl(S),\dcl(S)\cap A)\subseteq (B,A)$ ist frei nach Lemma \ref{Spezialität dcl}), also auch über $S$. Setze $$X':=g^{-1}(X)\cap A^m=(g\upharpoonright A^m)^{-1}(X),$$ das ist $\lingua^P_S$-definierbar und es gilt $g(X')=X$ wegen $X\subseteq\operatorname{im}(g\upharpoonright A^m)$.\\
	Beweise die Aussage jetzt induktiv über $m$: Wenn $m=0$ ist, ist $X$ endlich und eine Vereinigung der Mengen $f(\{0\})$ für konstante Funktionen $f$. Wegen Endlichkeit gilt $X\subseteq\dcl(S)$, also sind die Funktionen $f$ über $S$ definierbar.\\
	Wenn $m>0$ ist, schreibe $X'$ als Teilmenge von $A^m$ wegen Lemma \ref{Definierbare Mengen Spezialität} in der Form $Y\cap A^m$ für ein $\lingua_S$-definierbares $Y$. Sei eine Zerlegung $\mathfrak{Z}$ von $Y$ in $\lingua_S$-definierbare Zellen gegeben, auf denen $g$ jeweils stetig ist, dann ist $$X=g(Y\cap A^m)=\bigcup\limits_{Z\in\mathfrak{Z}}g(Z\cap A^m).$$
	Für offene Zellen $Z$ ist so eine Darstellung also schon gefunden. Sei $Z$ nun eine Zelle der Dimension $d<m$ und $\pi:B^d\rightarrow B^m$ eine $\lingua_S$-definierbare Fortsetzung des kanonischen Homöomorphismus der entsprechenden offenen Zelle $Z'$ nach $Z$. Mit Lemma \ref{Definierbare Mengen Spezialität} kann man ein $\lingua_S$-definierbares $E\subseteq B^d$ findet mit $$Z'\cap\pi^{-1}(A^m)=A^d\cap E,$$ denn es gilt $$A^d\supseteq Z'\cap\pi^{-1}(A^m),$$ weil $\pi|_{Z'}$ die Umkehrabbildung einer Projektion ist.\\
	Es ergibt sich dann
	$$g(Z\cap A^m)=(g\circ\pi)(Z'\cap\pi^{-1}(A^m))=(g\circ\pi)(A^d\cap E)$$
	Die Gleichung ist aber schon der Fall eines kleineren $m$ und mit Induktion folgt die Aussage.
\end{proof}
\newpage
\begin{theorem}\label{Satz 4}
	Sei $X\subseteq B$ eine $\lingua^P_S$-definierbare Menge. Dann existiert eine endliche $\lingua_S$-definierbare Unterteilung von $B$, sodass für jedes dadurch erzeugte offene Intervall $I$ genau einer der folgenden Fälle gilt:
	\begin{itemize}
		\item $I$ ist disjunkt zu $X$.
		\item $I$ ist Teilmenge von $X$.
		\item $X\cap I$ ist dicht \& kodicht in $I$ und entweder ist $X\cap I$ klein oder $I\setminus X$.
	\end{itemize}
	Für kleine $X$ entfallen die Fälle \glqq{}Teilmenge\grqq{} und \glqq{}koklein\grqq{}.
\end{theorem}
\begin{proof}
	Sei $X$ zunächst klein und gegeben als $X=f(A^n\cap E)$ für ein stetiges, $\lingua_S$-definierbares $f:E\rightarrow B$ und eine offene $\lingua_S$-definierbare Zelle $E\subseteq B^n$. Da definierbarer Zusammenhang unter definierbaren stetigen Funktionen erhalten bleibt, ist $I:=f(E)$ auch definierbar zusammenhängend, weil es $\lingua_S$-definierbar ist, ist es Vereinigung von Intervallen und Punkten, also muss es ein Intervall (ausnahmsweise seien hier auch nichtoffene Intervalle mitgemeint). Wenn $I$ endlich ist, ist auch $X$ endlich und es ist nichts zu zeigen. $X$ ist disjunkt zu $B\setminus I$, nun muss nur noch gezeigt werden, dass $X=X\cap I$ dicht und kodicht in $I$ ist. Der Satz verlangt eigentlich Dichte und Kodichte in einem offenen Intervall, allerdings ist Dichte und Kodichte in nichtoffenen unendlichen Intervallen äquivalent zu Dichte und Kodichte in deren Innerem. Aber dichte Teilmengen werden durch stetige Abbildungen auf dichte Teilmengen abgebildet; da $A^n$ dicht in $B^n$ ist, ist auch $A^n\cap E$ dicht in $E$ und folglich $X$ dicht in $I$. Andererseits muss auch das Komplement von $X$ dicht in $I$ sein, da es sonst ein Intervall ganz in $X$ gäbe, was der Kleinheit mit Satz \ref{Kleinheit} widerspricht.\\
	Die gesuchte Eigenschaft für kleine Mengen bleibt unter Vereinigungen erhalten. Man muss nur eine Verfeinerung der Unterteilung durchführen und dann ausnutzen, dass endliche Vereinigungen von kleinen Mengen klein sind. Also gilt für $Y$ und $Z$ als kleine Teilmengen eines Intervalls $I$:
	\begin{itemize}
		\item $Y\cup Z$ klein
		\item $Y\cup Z=Y$, wenn $Z=\emptyset$
		\item $Y\cup Z$ dicht in $I$, wenn $Y$ und $Z$ dicht in $I$
		\item $Y\cup Z$ kodicht in $I$ als kleine Menge (s. Begr. oben)
	\end{itemize}
    Also gilt mit dem letzten Lemma die Aussage für kleine Mengen allgemein.\newpage
	Sei jetzt $X$ nicht mehr klein, dann stimmt es aber bis auf eine kleine Menge mit einer $\lingua$-definierbaren Menge $X'$ überein. Schreibe also $X=(X'\cup Y)\setminus Z$ für $Y$ und $Z$ klein und disjunkt. $X'$ ist als $\lingua$-definierbare Menge eine endliche Vereinigung von Punkten und Intervallen; da $Y,Z$  als kleine Mengen die oben beschriebene Gestalt haben, entspricht $X'\cup Y$ der Darstellung ohne den Fall koklein. In der Darstellung der drei Fälle wird der erste durch Subtraktion von $Z$ nicht geändert, der zweite bleibt oder wird zum dritten (2. Teil) und der dritte bleibt auch, da er nur von $Y$ kommt, aber $Y$ und $Z$ disjunkt sind. Also hat $X$ auch so eine Darstellung.
\end{proof}

\begin{corollary}
	Jede $\lingua^P_S$-definierbare Menge $X\subseteq B$ hat ein Supremum und Infimum in $\dcl(S)\cup\{\pm\infty\}$.
\end{corollary}

\begin{lemma}
	$\td$ eliminiert $\exists^\infty$: Wenn $S\subseteq B^{m+n}$ eine $\lingua^P$-definierbare Menge ist mit $S_x$ endlich für alle $x\in B^m$, dann ist $(\abs{S_x})_{x\in B^m}$ beschränkt.
\end{lemma}
\begin{proof}
	Für $n=1$ gilt das nach Bemerkung 5.33. aus \cite{Lukas}. Denn $S_x$ ist endlich, genau dann, wenn $S_x$ diskret in $B$ ist; und das ist uniform $\lingua^P$-ausdrückbar. Die Äquivalenz zur Diskretheit sieht man ein, indem man eine Aufteilung von $S_x$ wie im letzten Satz vornimmt. Dann ist $S_x$ genau dann endlich, wenn nur der Fall $\glqq{}S_x\cap I=\emptyset\grqq{}$ vorkommt; weil dichte Teilmengen und Intervalle nicht diskret sind, gilt das wiederum genau dann, wenn $S_x$ diskret ist.\\
	Sei $n>1$. Dann sind für $\pi_{i_1,\dots,i_k}$ als Projektionsabbildung auf die Koordinaten $i_1,\dots,i_k$ jeweils auch $$Y_x:=(\pi_{1,\dots,m+1}(S))_x=\pi_{m+1}(S_x)$$ und $$Z_x:=(\pi_{1,\dots,m,m+2,\dots,m+n}(S))_x=\pi_{m+2,\dots,m+n}(S_x)$$ endlich und daher ist nach Induktionsvoraussetzung die Mächtigkeit jeweils uniform beschränkt durch irgendwelche $K,L\in\setN$. Dann gilt aber $$\abs{S_x}\leq \abs{Y_x\times Z_x}=\abs{Y_x}\abs{Z_x},$$ was uniform durch $KL$ beschränkt ist.
\end{proof}
\newpage
\section{Definierbare Funktionen und definierbarer Abschluss}
Um definierbare Funktionen besser zu verstehen, ist es notwendig, sich mit dem definierbaren Abschluss in $\lingua^P$ zu beschäftigen. Damit kann man zeigen, dass definierbare Funktionen in einer Variablen \glqq{}fast überall\grqq{} $\lingua$-definierbar sind. Wir erinnern uns, dass ein dichtes Paar $(B,A)$ fixiert war und $S\subseteq B$ speziell. Mit \glqq{}definierbar abgeschlossen\grqq{} ist im Folgenden \glqq{}$\lingua^P$-definierbar abgeschlossen\grqq{} gemeint. Zuerst kommen zwei einfach zu beweisende Aussagen, um sie dann stark zu verallgemeinern und daraus Wissen über Funktionen zu erhalten.

\begin{lemma}\label{A definierbar abgeschl}
	$A$ ist definierbar abgeschlossen.
\end{lemma}
\begin{proof}
	Sei $b\in B\setminus A$ und $(B,A)\preceq(D,C)$ eine genügend saturierte Elementarerweiterung. Dann wird der Ordnungstyp von $b$ über $A$ auch von einem Element $D\setminus C\ni d\neq b$ realisiert wegen Dichtheit von $D\setminus C$ in $D$ und Saturation.\\
	Nach Lemma \ref{selber Schnitt} haben $b$ und $d$ dann den selben $\lingua^P$-Typen über $A$, weswegen $b$ nicht definierbar über $A$ in $(D,C)$ sein kann, also auch nicht in $(B,A)$.
\end{proof}

\begin{corollary}
	Sei $A_0\preceq A$. Dann ist $A_0$ definierbar abgeschlossen.
\end{corollary}
\begin{proof}
	Sei $b$ definierbar über $A_0$. Dann ist $b$ insbesondere definierbar über $A$, also in $A$. Nach Folgerung \ref{Definierbarkeit aus A} ist dann $\{b\}=\{b\}\cap A$ schon $\lingua$-definierbar in $A$ über $A_0$, also in $A_0$, da $A_0$ elementare Substruktur ist.
\end{proof}

\begin{lemma}\label{Freie Definierbarkeit}
	Sei $(D,C)\subseteq(B,A)$ frei, dann ist $D$ definierbar abgeschlossen in $(B,A)$.
\end{lemma}
\begin{proof}
	Für eine beliebige Struktur $(B',A')\succeq (B,A)$ gelten die Voraussetzungen ebenso, da $\acl$-Abhängigkeit über $A$ als Teil vom Typen äquivalent ist zu $\acl$-Abhängigkeit über $A'$. Ebenso ist $D$ in $(B,A)$ definierbar abgeschlossen genau dann, wenn es in $(B',A')$ definierbar abgeschlossen ist. Also sei $(B,A)$ jetzt schon \OE\ hinreichend saturiert und $b\in B\setminus D$.\\
	Nach dem Beweis zur Existenz des B\&F-Systems kann der partielle Isomorphismus $$(B,A)\supseteq(D,C)\cong(D,C)\subseteq(B,A)$$ insbesondere auf mehrere Weisen auf $b$ fortgesetzt werden: Wenn $b\in A$ oder $b\in B\setminus AD$, ging es nur um die Erfüllung von transzendenten Ordnungstypen, dort hat man also mehrere Optionen. Wenn $b\in AD\setminus(A\cup D)$ ist und $a\in A^n$ unabhängig über $D$ mit $b\in Da$, dann finde $a'_1$ transzendent über $Da$ von passendem Ordnungstyp über $D$, $a'_2$ transzendent über $Daa'_1$ von passendem Ordnungstyp über $Da'_1$, usw.\newpage
	So kann man den Isomorphismus auf $Da$ fortsetzen mit Bild $Da'$; da aber $a$ und $a'$ per Konstruktion unabhängig über $D$ waren, sind auch $Da$ und $Da'$ unabhängig über $D$, also $Da\cap Da'=D$ und das Bild von $b$ kann nicht $b$ selbst sein. Also gibt es auch in diesem Fall mehrere Möglichkeiten für eine Fortsetzung, also mehrere elementare Abbildungen, daher kann $b$ nicht definierbar über $D$ sein.
\end{proof}

Das vorige Lemma kann man unter bestimmten Bedingungen umkehren, so zum Beispiel für $\operatorname{RCF^d}$.
\begin{definition}
	$\td$ sei \textbf{zahm für Paare}, wenn jede definierbar abgeschlossene Menge in jedem Modell $(B,A)\models\td$ speziell ist.
\end{definition}

\begin{lemma}\label{RCF zahm}
	$\operatorname{RCF^d}$ ist zahm für Paare.
\end{lemma}
\begin{proof}
	Sei $(B,A)$ ein Modell und $S$ definierbar abgeschlossen. Dann ist $S$ Träger einer $\lingua$-Unterstruktur von $B$, also auch Modell von RCF. Sei $\overline{B}$ ein algebraischer Abschluss von $B$, dann stimmt der RCF-$\acl$ auf $B$ mit dem auf $B$ eingeschränkten ACF-$\acl$ auf $\overline{B}$ überein, wir brauchen sie also nicht zu unterscheiden. Zu zeigen ist dann $$S\ad_{S\cap A}A.$$
	Es gilt aber $$S\ld_{S\cap A}A,$$ denn wenn $\overline{x}\in S$ linear abhängig über $A$ ist und \OE\ $x_1,\dots,x_k$ eine Basis des Tupels über $A$ sind, dann kann jedes $x_i\in\overline{x}$ eindeutig als $A$-Linearkombination davon geschrieben werden. Also ist diese Linearkombination $\lingua^P_S$-definierbar, also in $S\cap A$; somit ist $\overline{x}$ linear abhängig über $S\cap A$. Mit Lemma \ref{Rechenregeln} folgt $S\ad_{S\cap A}A$.
\end{proof}

Nun aber zu den definierbaren Funktionen. Es gibt mehrere Möglichkeiten, $\lingua^P$-definierbare Funktionen mit $\lingua$-definierbaren Funktionen in Verbindung zu setzen.

\begin{lemma}
	Sei $F:A^n\rightarrow A$ eine $\lingua^P_S$-definierbare Funktion. Dann gibt es $S\cap A$-definierbare $f_1,\dots,f_k:A^n\rightarrow A$ in $A$, sodass für alle $a\in A^n$ ein $f_i$ existiert mit $F(a)=f_i(a)$.
\end{lemma}
\begin{proof}
	Man kann annehmen, dass $S$ schon Modell von T ist, ansonsten geht man zu $\dcl(S)$ über. Nach Lemma \ref{Spezialität dcl} ist $\dcl(S)\cap A=\dcl(S\cap A)$, also ändert das die Aussage nicht.\newpage
	Wenn das Lemma nicht gälte, gilt für alle $k\in\setN$ und alle in $A$ über $S\cap A$ definierbaren $f_1,\dots,f_k:A^n\rightarrow A$, dass ein $a\in A$ existiert mit $f_i(a)\neq F(a)$ für alle $i$. Also ist der partielle Typ $$\{P(x)\}\cup\{F(x)\neq f(x)\mid f:A^n\rightarrow A\ \lingua_{S\cap A}\text{-definierbar}\}$$ konsistent und es existiert $(B,A)\preceq(B',A')$ und $a'\in A'^n$ mit $F(a')\neq f(a')$ für alle $\lingua_{S\cap A}$-definierbaren $f:A'^n\rightarrow A'$.\\
	Allerdings ist die Inklusion $$(S,S\cap A)\subseteq (B,A)$$ frei, wegen Elementarität auch $$(S,S\cap A)\subseteq (B',A').$$ Nach Lemma \ref{Unabhängigkeitsregeln} (6.) ist wegen $a'\in A'^n$ die Inklusion $$(Sa',(S\cap A)a')\subseteq(B',A')$$ frei, nach Lemma \ref{Freie Definierbarkeit} ist $Sa'$ also $\lingua^P$-definierbar abgeschlossen. Da $F(a')$ $\lingua^P$-definierbar über $Sa'$ ist, liegt es in $Sa'$, wegen $F:A'^n\rightarrow A'$ ist $F(a')\in A'$. Wegen Unabhängigkeit liegt also $$F(a')\in Sa'\cap A'=(S\cap A)a'$$ und es gibt eine $\lingua_{S\cap A}$-definierbare Abbildung $f:A'^n\rightarrow A'$ mit $f(a')=F(a')$ - ein Widerspruch!
\end{proof}

\begin{lemma}
	Sei $F:B\rightarrow B$ eine $\lingua^P_S$-definierbare Funktion. Dann gibt es $\lingua_S$-definierbare $f_1,\dots,f_k:B\rightarrow B$ und eine kleine $\lingua^P_S$-definierbare Menge $X\subseteq B$, sodass für alle $b\in B\setminus X$ ein $f_i$ existiert mit $F(b)=f_i(b)$.
\end{lemma}
\begin{proof}
	\OE\ ist wieder $S$ ein Modell.\\
	Wenn die Aussage nicht gilt, gilt für alle $k\in\setN$, alle kleinen $\lingua^P_S$-definierbaren Mengen $X\subseteq B$ und alle $\lingua_S$-definierbaren $f_1,\dots,f_k:B\rightarrow B$, dass ein $b\in B\setminus X$ existiert mit $f_i(b)\neq F(b)$ für alle $i$.\newpage
	Also ist der partielle Typ $$\{x\notin X\mid X\text{ klein, }\lingua^P_S\text{-definierbar}\}\cup\{F(x)\neq f(x)\mid f:B\rightarrow B\ \lingua_S\text{-definierbar}\}$$ (endlich) konsistent und es existiert $(B,A)\preceq(B',A')$ und $$b'\in B'\setminus\bigcup\limits_{f:B'^n\rightarrow B'\ \lingua_S\text{-definierbar}}f(A'^n)=B'\setminus A'S$$ mit $F(b')\neq f(b')$ für alle $\lingua_S$-definierbaren $f:B'\rightarrow B'$.\\
	Allerdings ist $$(S,S\cap A)\subseteq(B',A')$$ frei und nach Lemma \ref{Unabhängigkeitsregeln} (7.) wegen $b'\in B'\setminus A'S$ die Inklusion $$(Sb',S\cap A)\subseteq(B',A')$$ ebenso, nach Lemma \ref{Freie Definierbarkeit} ist $Sb'$ also $\lingua^P$-definierbar abgeschlossen. Da $F(b')$ $\lingua^P$-definierbar über $Sb'$ ist, ist es in $Sb'$, also existiert eine $\lingua_S$-definierbare Abbildung $f:B'\rightarrow B'$ mit $f(b')=F(b')$ - ein Widerspruch!
\end{proof}

\begin{theorem}\label{Satz 3}
	Sei $F:B\rightarrow B$ eine $\lingua^P_S$-definierbare Funktion. Dann stimmt $F$ auf bis auf eine $\lingua^P_S$-definierbare kleine Menge mit einer $\lingua_S$-definierbaren Funktion überein.
\end{theorem}
\begin{proof}
	Nach dem vorigen Lemma existieren $\lingua_S$-definierbare $f_1,\dots,f_k:B\rightarrow B$ und ein kleines $\lingua^P_S$-definierbares $X$, sodass für alle $b\in B\setminus X$ ein $i$ existiert mit $F(b)=f_i(b)$. Wenn $k=1$ ist, ist das die gewünschte Aussage, wenn nicht, zeige dass man $k$ weiter reduzieren kann. Dafür partitioniere $B$ mit Satz \ref{Satz 4} in $\lingua_S$-definierbare ${E,T,D,K,C\subseteq B}$, sodass $E$ endlich ist, alle anderen Mengen dafür offen, 
	$$T\subseteq\{F=f_1\}, D\cap\{F=f_1\}=\emptyset,K':=K\cap\{F=f_1\}\text{ klein, dicht und kodicht in }K$$ $$\text{sowie }C':=C\cap\{F=f_1\}\text{ koklein, dicht und kodicht in }C.$$
	Definiere dann $$f:x\mapsto\left\{\begin{array}{ll}
	f_1(x)&x\in T\cup C\\
	f_2(x)&\text{sonst}
	\end{array}\right.$$\newpage
	Die Menge $X':=X\cup E\cup K'\cup (C\setminus C')$ ist klein als Vereinigung von kleinen Mengen, außerdem $\lingua^P_S$-definierbar, und wenn $x\in B\setminus X'$ mit $F(x)\neq f_i(x)$ für $i=3,\dots,k$, dann gibt es folgende Möglichkeiten:
	\begin{itemize}
		\item $x\in T\cup C'\subseteq\{F=f_1\}$: Dann ist $F(x)=f_1(x)=f(x)$.
		\item $x\in D\cup(K\setminus K')$, also insbesondere $x\notin\{F=f_1\}$: Dann ist $f(x)=f_2(x)$ und wegen $F(x)\neq f_1(x)$ ist $F(x)=f(x)$.
	\end{itemize}
    Also nimmt $F$ auf $B\setminus X'$ immer die Werte von $f,f_3,\dots,f_k$ an und induktiv ist die Aussage gezeigt.
\end{proof}

\begin{lemma}\label{Stückweise stetige Abbildungen}
	Sei $f:B\rightarrow B$ stückweise stetig und $\lingua^P_S$-definierbar. Dann ist $f$ schon $\lingua_S$-definierbar.
\end{lemma}
\begin{proof}
	Nach Satz $\ref{Satz 3}$ stimmt $f$ bis auf eine kleine $\lingua^P_S$-definierbare Menge $X$ mit einer $\lingua_S$-definierbaren Funktion $f'$ überein. Wegen o-Minimalität von T ist $f'$ auch stückweise stetig. Unterteile $X$ wie in Satz \ref{Satz 4} und verfeinere die Unterteilung, so dass $f,f'$ auf jedem Intervall stetig sind. Für jedes Intervall $I$ dieser Unterteilung gilt dann entweder, dass $X\cap I$ dicht und kodicht in $I$ ist oder, dass $X\cap I=\emptyset$. Der erste Fall kann nie eintreten, da zwei stetige Funktionen, die auf der dichten Teilmenge $I\setminus X$ übereinstimmen, schon auf ganz $I$ übereinstimmen. Also ist $X$ endlich und $f$ kann mit $\lingua$ definiert werden (nämlich durch $f'$ außerhalb von $X$ und ansonsten manuell). Da diese Unterteilung $\lingua_S$-definierbar ist, ist auch $f$ schon $\lingua_S$-definierbar.
\end{proof}

\section{Offene und abgeschlossene $\lingua^P$-definierbare Mengen}
Nach der Beschreibung der Struktur von $\lingua^P$-definierbaren Teilmengen von $B$ liegt die Vermutung nahe, dass offene und abgeschlossene $\lingua^P$-definierbare Mengen schon $\lingua$-definierbar sind.\\
Dies lässt sich in vielen Situationen auch beweisen. Im Folgenden werden mehrere Möglichkeiten dafür angegeben. Hierbei sei $(B,A)$ wieder ein dichtes Paar und $S$ speziell.\\
Als erstes ist festzuhalten, dass die Aussage für Mengen in einer Variable leicht zu sehen ist und man sie nur für offene Mengen zeigen muss.
\newpage
\begin{lemma}\label{Offenheit einfache Fälle}\ 
	\begin{itemize}
		\item Sei $Z\subseteq B$ offen und $\lingua^P_S$-definierbar. Dann ist $Z$ schon $\lingua_S$-definierbar.
		\item Wenn alle offenen $\lingua^P_S$-definierbaren Teilmengen von $B^n$ auch $\lingua_S$-definierbar sind, sind es alle abgeschlossenen solchen Mengen auch.
	\end{itemize}
\end{lemma}
\begin{proof}\ 
	\begin{itemize}
		\item In der Darstellung von Satz \ref{Satz 4} kann der Fall $Z\cap I$ dicht und kodicht nicht auftreten. Die Menge $Z\cap I$ ist nämlich offen und kann daher nicht kodicht in $I$ sein (denn sonst würde sie ein Element ihres eigenen Komplements enthalten). Also sind die $\lingua^P_S$-definierbaren offenen Teilmengen von $B$ gerade die endlichen Vereinigungen von Intervallen mit Rand aus $\dcl(S)\cup\{\pm\infty\}$ und Punkten aus $\dcl(S)$ und das ist $\lingua_S$-definierbar.
		\item Wenn $Z\subseteq B^n$ abgeschlossen und $\lingua^P_S$-definierbar ist, ist $Z^c$ offen und $\lingua^P_S$-definierbar, also $\lingua_S$-definierbar per Voraussetzung. Damit ist $Z$ dann selbst $\lingua_S$-definierbar.
	\end{itemize}
\end{proof}


Für mehrstellige Mengen kann man folgenden Satz aus \cite{Piz} benutzen, die Voraussetzungen dafür sind allerdings nicht zwangsläufig erfüllt. Für einige Theorien wie z.B. $\operatorname{RCF^d}$ ist das aber ein gangbarer Weg (siehe auch Lemma \ref{RCF zahm}).
\begin{theorem}\label{Kurven}
	Sei $F:B^n\rightarrow B$ eine $\lingua^P_S$-definierbare Funktion und $\td$ sei zahm für Paare. Dann ist $F$ definierbar in $\lingua_S$ genau dann, wenn für alle speziellen $X\supseteq S$ und alle $\lingua_X$-definierbaren partiellen Funktionen $\alpha: Y\rightarrow B^n$ mit $Y$ offen in $B$ auch $F\circ\alpha$ schon $\lingua_X$-definierbar ist.
\end{theorem}

Mit diesem Satz kann man dann die gewünschte Eigenschaft beweisen.

\begin{theorem}
	Offene und abgeschlossene $\lingua^P_S$-definierbare Mengen sind $\lingua_S$-definierbar, sofern $\td$ zahm für Paare ist.
\end{theorem}
\begin{proof}
	Es reicht, das für offene Mengen zu zeigen; genauer reicht es schon, die Definierbarkeit für charakteristische Funktionen solcher Mengen zu zeigen. Sei $Z\subseteq B^n$ eine offene $\lingua^P_S$-definierbare Menge und $X\supseteq S$ speziell, $\alpha:Y\rightarrow B^n$ eine $\lingua_X$-definierbare, partielle Funktion und $Y\subseteq B$ offen. Zu zeigen ist für die Anwendung von Satz \ref{Kurven}, dass $\chi_Z\circ\alpha$ definierbar in $\lingua_X$ ist.\newpage
	Man kann annehmen, dass $\alpha$ stetig ist, sonst zerlege $Y$ in Intervalle, auf denen $\alpha$ stetig ist, und Punkte; das ändert nichts an irgendwelchen Definierbarkeiten. Es ist festzustellen, dass $$1=\chi_Z\circ\alpha(x)\Leftrightarrow\alpha(x)\in Z\Leftrightarrow x\in\alpha^{-1}(Z).$$ Da aber $\alpha^{-1}(Z)$ eine $\lingua^P_X$-definierbare Teilmenge von $B$ ist, offen wegen Stetigkeit von $\alpha$ und Offenheit von $Z$ und zusätzlich $X$ speziell ist, ist $\alpha^{-1}(Z)$ nach Lemma \ref{Offenheit einfache Fälle} $\lingua_X$-definierbar. Also ist $\chi_Z\circ\alpha$ definierbar in $\lingua_X$ durch $$\chi_Z\circ\alpha(x)=\left\{\begin{array}{ll}
	1&x\in\alpha^{-1}(Z)\\
	0&\text{sonst}
	\end{array}\right..$$
\end{proof}

\begin{corollary}
	Der Abschluss und das Innere von $\lingua^P_S$-definierbaren Mengen sind unter dieser Anforderung an die Theorie $\lingua_S$-definierbar.
\end{corollary}

Es gibt auch andere Möglichkeiten, die gesuchte Aussage für manche Theorien zu beweisen. Für Teilmengen von $\setR^n$ in einer o-minimalen Erweiterung von RCF mit Grundmenge $\setR$ wird das zum Beispiel in \cite{VanDenDries} beschrieben. Es wäre auch zu betrachten, ob die Argumentation in \cite{DieMoeglicheLoesung} eine Lösung dieses Problems hergibt. Man könnte sich überlegen, wie Satz \ref{Satz 4} und Satz \ref{Satz 3} ins mehrdimensionale zu verallgemeinern wären und dann so vorgehen, wie beim Beweis der Zellzerlegung in \cite{vdDZellzerlegung}. Aber dies wäre eine Aufgabe für eine weitere Arbeit zu dem Thema.
    \newpage
    \appendix
    %!TEX root = DieLoesungAllerMilleniumsprobleme.tex
\addcontentsline{toc}{chapter}{Anhang}
\renewcommand\thesection{\Alph{section}}
\section{Ein Alternativbeweis zur $\omega$-Stabilität von ACP}

\begin{proof}
	Nach Folgerung \ref{Formel-Vereinfachung} kann man jede Formel mit Parametern aus $X$ modulo ACP schreiben als als boolesche Kombination aus $$\glqq{}l_n(\text{Monome mit Koeffizienten von Produkten aus }X)\grqq{}$$ und
	\begin{align*}
	\text{\glqq{}}(\text{Polynom in }\mathbb{P}[X])(&f_{i_1,n_1}(\text{Monome in Produkten aus }X),\\\dots,&f_{i_m,n_m}(\text{Monome in Produkten aus }X))=0\grqq{}.
	\end{align*}
	Diese beiden Arten von Formeln lassen sich verallgemeinern
    zu Formeln der Art $${\exists\overline{e}\in E(f(\overline{e},\overline{x})=0)}$$ für  $f(\overline{T},\overline{x})\in\mathbb{P}(\overline{T})[\overline{x}]$ und der Art
	\begin{align*}
	&\exists z_{1,2},\dots,z_{1,n_1+1},z_{2,2},\dots,z_{2,n_2+1},\dots\in E(p(z_{1,1},\dots,z_{k,1})=0\\
	&\bigwedge\limits_{i=1}^km_{i,1}(\overline{x})=z_{i,2}m_{i,2}(\overline{x})+\dots+z_{i,n_i}m_{i,n_i}(\overline{x}))
	\end{align*}
	für Monome $(m_{i,j})\in\mathbb{P}(X)[\overline{x}]$ und ein Polynom $p\in\mathbb{P}(X)[\overline{x}]$. Nenne die Menge aller Formeln der ersten Art $A$ und die aller Formeln der zweiten Art $B$. Insbesondere wurde die Menge der \glqq{}interessanten\grqq{} Formeln nur vergrößert, das heißt, dass ein Typ $p$ eindeutig durch $$(p\cap A)\cup(p\cap B)\cup(p\cap\neg A)\cup(p\cap\neg B)$$ festgelegt wird.\\
	Das bedeutet, in einem vorgegebenen Modell $\fM$ mit $X\subseteq M$ (\OE unendlich) zerfällt $S_n(X)$ in folgende Teilmengen:\\
	\begin{enumerate}
		\item Typen, die eine Formel aus $A$ und eine aus $B$ enthalten.
		\item Typen, die eine Formel aus $A$ und keine aus $B$ enthalten.
		\item Typen, die eine Formel aus $B$ und keine aus $A$ enthalten.
		\item Typen, die keine Formel aus $A\cup B$ enthalten.
	\end{enumerate}
	Für einen Typen $p$ ist im Fall 3./4. $p\cap A=\emptyset$, also $p\cap(A\cup\neg A)$ eindeutig gegeben durch die Verneinung aller möglichen Formeln in $A$ mit $n$ freien Variablen (denn das ist endlich konsistent, TODO: vielleicht beschreiben, warum). Analog ist in Fall 2./4. $p\cap(B\cup\neg B)$ eindeutig bestimmt.\\
	Es bleibt nun noch zu zeigen, dass es im Fall 1./2. jeweils nur $\abs{X}$ viele Möglichkeiten für Einschränkungen $p\cap(A\cup\neg A)$ geben kann und im Fall 1./3. nur $\abs{X}$ viele Möglichkeiten für Einschränkungen $p\cap(B\cup\neg B)$.\\
	Zunächst zum ersten Teil: Definiere für ein Polynom $g\in E(X)[\overline{x}]$ die Relation
	\begin{align*}&g\in\in p:\Leftrightarrow\text{ es existiert }f(\overline{T},\overline{x})\in\mathbb{P}(\overline{T})[\overline{x}],\text{ es existieren }a_1,\dots,a_n\in E\text{ mit }\\
	&f(\overline{a},\overline{x})=g(\overline{x})\text{ und }\exists\overline{e}\in E(f(\overline{e},\overline{x})=0)\in p.
	\end{align*}
	$I:=\{g\in E(X)[\overline{x}]\mid g\in\in p\}$ ist offenbar ein Ideal im Noetherschen Ring $E(X)[\overline{x}]$ (es ist nichtleer im Fall 1./2.) und daher endlich erzeugt durch $h_1,\dots,h_m\in I$. Da jedes Element $g\in I$ mit einem Element $$\exists\overline{e}\in E(\overline{g}(\overline{e},\overline{x})=0)\in p$$ korrespondiert, ist $p\cap(A\cup\neg A)$ isoliert durch die übertragenen Erzeuger $$\exists\overline{e}\in E(\overline{h_1}(\overline{e},\overline{x})=0),\dots,\exists\overline{e}\in E(\overline{h_m}(\overline{e},\overline{x})=0),$$ also gibt es nur $\abs{X}$ viele Möglichkeiten für $p\cap(A\cup\neg A)$.\\
	Formeln der zweiten Art kann man in Konjunktionen von Formeln der ersten Art umwandeln, indem man $(z_{l,1})_{l=1\dots k}$ zu freien Variablen macht. Auf diese Weise kann man partielle Typen in $B$ zu partiellen Typen in $A$ in mehr Variablen umformen (am angenehmsten geht es wahrscheinlich, wenn man annimmt, dass $\fM$ schon hinreichend saturiert ist und einen Erfüller $\overline{a}$ von einem $p$ der Art 1./3. betrachtet, eine Belegung $(b_{i,j})$ für die $(z_{i,j})$ in einer der Formeln findet und dann $$\operatorname{tp}(\overline{a},(b_{l,1})_{l=1\dots k}/X)\cap(A\cup\neg A)$$ betrachtet). Aber wie auch immer man das macht, es können nur mehr Typen werden, dafür landet man wieder in Fall 1./2., wo wir wissen, dass es nur $\abs{X}$ viele Möglichkeiten gibt. Also ist ACP $\kappa$-stabil für alle unendlichen $\kappa$.
\end{proof}
    \newpage
    \bibliography{quellen}{}
    \addtocontents{toc}{\bigskip}
    \addcontentsline{toc}{section}{Literaturverzeichnis}
    \bibliographystyle{alpha}
    \newpage
    %!TEX root = DieLoesungAllerMilleniumsprobleme.tex
\addcontentsline{toc}{section}{Eigenständigkeitserklärung}
\chapter*{Eigenständigkeitserklärung}
\vspace{0.8cm}
Hiermit versichere ich, Max Aaron Vollprecht, dass ich die Masterarbeit selbstständig verfasst habe. Es wurden keine anderen als die angegebenen Quellen und Hilfsmittel benutzt.
\\ \\
Freiburg, den \today
\end{document}