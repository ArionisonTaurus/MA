\documentclass[a4paper, 11pt]{report}
\usepackage[utf8]{inputenc}
\usepackage[ngerman]{babel}
\usepackage{amsmath, amssymb, amsthm}
\usepackage{paralist}
\usepackage{lmodern}
\usepackage[T1]{fontenc}
\usepackage[arrow, matrix, curve]{xy}
\usepackage{graphicx}
\usepackage[percent]{overpic}
\usepackage{cite}
\usepackage{caption}
\usepackage[onehalfspacing]{setspace}
\usepackage{hyperref}
\usepackage{titlesec}
\usepackage{texilikechaps}

\captionsetup[table]{labelformat=empty}


\newcommand{\setk}{\Bbbk}
\newcommand{\ldot}{\,.\,}
\newcommand{\fa}{~\forall}
\newcommand{\ex}{~\exists}
\newcommand{\fA}{\mathcal{A}}
\newcommand{\fB}{\mathcal{B}}
\newcommand{\fC}{\mathcal{C}}
\newcommand{\fM}{\mathcal{M}}
\newcommand{\fN}{\mathcal{N}}
\newcommand{\fF}{\mathcal{F}}
\newcommand{\fU}{\mathcal{U}}
\newcommand{\lingua}{\mathcal{L}}
\newcommand{\lld}{\mathcal{L}^{ld}}
\newcommand{\lf}{\mathcal{L}^f}
\newcommand{\lfc}{\mathcal{L}^{f,c}}
\newcommand{\sF}{\ensuremath{\mathcal{F}} }
\newcommand{\monster}{\mathbb{M}} %Monster-Modell-M 
\newcommand{\MM}{\mathbb{M}} %ebenfalls
\newcommand{\gdw}{\Leftrightarrow}
\newcommand{\Mod}{\mathcal{M}} %geschwungenes Modell-M
\newcommand{\Nod}{\mathcal{N}} %das gleiche für N
\newcommand{\leer}{\emptyset}
\newcommand{\In}{\in}
\newcommand{\setN}{\mathbb{N}}
\newcommand{\setZ}{\mathbb{Z}}
\newcommand{\setQ}{\mathbb{Q}}
\newcommand{\setR}{\mathbb{R}}
\newcommand{\setC}{\mathbb{C}}
\newcommand{\setH}{\mathbb{H}}
\newcommand{\Forall}{~\forall}
\newcommand{\Exists}{~\exists}
\newcommand{\abs}[1]{{\left| #1 \right|}}

\DeclareMathOperator{\ld}{ld}
\DeclareMathOperator{\ad}{ad}
\DeclareMathOperator{\acl}{acl}
\DeclareMathOperator{\dcl}{dcl}
\DeclareMathOperator{\tp}{tp}
\DeclareMathOperator{\tq}{T^2}
\DeclareMathOperator{\td}{T^d}

\def\Ind#1#2{#1\setbox0=\hbox{$#1x$}\kern\wd0\hbox to 0pt{\hss$#1\mid$\hss}
	\lower.9\ht0\hbox to 0pt{\hss$#1\smile$\hss}\kern\wd0}

\def\ind{\mathop{\mathpalette\Ind{}}}

\def\notind#1#2{#1\setbox0=\hbox{$#1x$}\kern\wd0
	\hbox to 0pt{\mathchardef\nn=12854\hss$#1\nn$\kern1.4\wd0\hss}
	\hbox to 0pt{\hss$#1\mid$\hss}\lower.9\ht0 \hbox to 0pt{\hss$#1\smile$\hss}\kern\wd0}

\def\nind{\mathop{\mathpalette\notind{}}}


\theoremstyle{definition}
\newtheorem{theorem}[subsection]{Satz}
\newtheorem{corollary}[subsection]{Folgerung}
\newtheorem{proposition}[subsection]{Proposition}
\newtheorem{definition}[subsection]{Definition}
\newtheorem*{example}{Beispiel}
\newtheorem*{remark}{Bemerkung}
\newtheorem*{remarks}{Bemerkungen}
\newtheorem*{recall}{Erinnerung}
\newtheorem{satz}[subsection]{Satz}
\newtheorem{kor}[subsection]{Folgerung}
\newtheorem{prop}[subsection]{Proposition}
\newtheorem{lemma}[subsection]{Lemma}
\newtheorem{Def}[subsection]{Definition}
\newtheorem{bsp}[subsection]{Beispiel}
\newtheorem{axiom}[subsection]{Axiom}
\newtheorem{propdef}[subsection]{Proposition/ Definition}
\newtheorem{bemdef}[subsection]{Bemerkung/Definition}
\newtheorem{theocol}[subsection]{Folgerung/Satz}
\newtheorem*{bem}{Bemerkung}
\newtheorem*{erinn}{Erinnerung}

\txisection{chapter}{\normalfont \huge \bfseries }
\txisection{section}{\normalfont \Large \bfseries }

\usepackage[left=3.5cm,right=3cm,top=3.5cm,bottom=5cm]{geometry}
\setlength{\parindent}{0em}

\setlength\abovedisplayskip{4pt}
\setlength\belowdisplayskip{4pt}
\setlength\jot{4pt}

\newcommand{\lra}{\Leftrightarrow}
\newcommand{\xq}{ \bar{x}}

\def\Ind#1#2{#1\setbox0=\hbox{$#1x$}\kern\wd0\hbox to 0pt{\hss$#1\mid$\hss}
	\lower.9\ht0\hbox to 0pt{\hss$#1\smile$\hss}\kern\wd0}
\def\ua{\mathop{\mathpalette\Ind{}}}
\def\notind#1#2{#1\setbox0=\hbox{$#1x$}\kern\wd0
	\hbox to 0pt{\mathchardef\nn=12854\hss$#1\nn$\kern1.4\wd0\hss}
	\hbox to 0pt{\hss$#1\mid$\hss}\lower.9\ht0 \hbox to 0pt{\hss$#1\smile$\hss}\kern\wd0}
\def\nua{\mathop{\mathpalette\notind{}}}


\begin{document}
	\section{Notation}
	Im Folgenden seien, wenn nicht weiter erklärt, mit $k,l,m,n$ immer natürliche Zahlen gemeint, mit hebräischen Buchstaben immer Kardinalzahlen.\\
	Oftmals wird nicht zwischen Strukturen und deren Trägermengen unterschieden, insbesondere bei Paaren von Strukturen.\\
	Als Topologie wird immer die Ordnungstopologie bzw. deren Produkttopologie verstanden, mit \glqq{}Intervall\grqq{} ist immer ein offenes, nichtleeres Intervall mit Randpunkten in der Struktur oder $\pm\infty$ gemeint. Außerdem sei für $A\prec B$ und $X\subseteq B$ $A$-definierbar in $T$ die Menge $X_A$ die durch dieselbe definierende Formel in $B$ definierbare Menge (für $X\subseteq A$ und $X_B$ natürlich analog). Außerdem sei für Relationen $P$ mit \glqq{}$\exists/\forall x\in P(\dots)$\grqq{} die Formel \glqq{}$\exists x(P(x)\land\dots)/\forall x(P(x)\rightarrow\dots)$\grqq{}.\\
	Mit $\abs{\overline{a}}$ ist je nach Kontext unterschiedliches gemeint, einerseits die Supremumsnorm von $\overline{a}$, andererseits die Anzahl der Einträge. Da das eine ein Element der Struktur ist und das andere eine natürliche Zahl, ist immer klar erkennbar, was gemeint ist. Im Allgemeinen wird auch nicht immer zwischen Tupeln und Elementen unterschieden.
	
	%!TEX root = DieLoesungAllerMilleniumsprobleme.tex
	\chapter{Paare algebraisch abgeschlossener Körper}
	Der hier vorgestellte Beweis stammt ursprünglich aus einem Werk (\cite{Robinson}) von Robinson, dem es in erster Linie um die Vollständigkeit und Entscheidbarkeit dieser Theorie ging. In diesem Werk soll es aber vorwiegend um Stabilität und eine gewisse Form der Formelreduktion gehen.
	
	\section{Modelltheoretische Grundlagen}
	Im Folgenden werden einige Aussagen über abzählbare und vollständige Theorien T in einer Sprache $\lingua$ aufgezählt, die die Grundlage für die Stabilitätstheorie bilden. Als Quelle soll dabei \cite{Marker} dienen. Diese Erkenntnisse werden wichtig werden, um die $\omega$-Stabilität der Paare algebraisch abgeschlossener Körper zu zeigen.
	
	\begin{definition}
		Sei $\fM$ ein $\aleph_0$-saturiertes Modell von T und $X$ eine beliebige $\lingua_M$-definierbare Menge in $M^n$. Der \textbf{Morleyrang} $\RM^\fM(X)$ von $X$ in $\fM$ sei dann induktiv definiert
		\begin{itemize}
			\item durch $-1$ genau dann, wenn $X$ leer ist
			\item durch $0$ genau dann, wenn $X$ endlich und nichtleer ist
			\item als $\RM^\fM(X)\geq\alpha+1$ für eine Ordinalzahl $\alpha$ genau dann, wenn es unendlich viele paarweise disjunkte $\lingua_M$-definierbare Teilmengen von $X$ gibt, die jeweils Morleyrang größer als oder gleich $\alpha$ haben
			\item als $\RM^\fM(X)\geq\gamma$ für eine Limesordinalzahl $\gamma$ genau dann, wenn er größer als jede Zahl kleiner $\gamma$ ist.
		\end{itemize}
	    Ist $\RM^\fM(X)$ größer als oder gleich $\alpha$ für eine Ordinalzahl $\alpha$, aber nicht größer als oder gleich $\alpha+1$, so sei $\RM^\fM(X):=\alpha$, ist $\RM^\fM(X)$ größer als alle Ordinalzahlen, sei $\RM^\fM(X):=\infty$ (oder alternativ: der Rang ist nicht definiert).
	\end{definition}
	
	\begin{definition}
		Für ein $\aleph_0$-saturiertes Modell $\fM$ von T und $\varphi$ eine $\lingua_M$-Formel sei $$\RM^\fM(\varphi):=\RM^\fM(\varphi(\fM)).$$
	\end{definition}
	\newpage
	\begin{factdef}
		Für allgemeine Modelle $\fM'$ und eine $\aleph_0$-saturierte Elementarerweiterung $\fM$ sowie eine $\lingua_{M'}$-Formel $\varphi$ ist der Morleyrang von $\varphi$ in $\fM$ nicht von der konkreten Wahl der Erweiterung abhängig.\\
		Definiere daher in einem Modell $\fM'$ den Morleyrang $\RM(\varphi)$ als den Morleyrang $\RM^\fM(\varphi)$ in einer beliebigen $\aleph_0$-saturierten Erweiterung $\fM$ und $\RM(X):=\RM(\varphi)$ für eine durch $\varphi$ definierbare Menge $X\subseteq(M')^n$.
	\end{factdef}
	
	\begin{factdef}
		In jedem Modell $\fM$ existiert für jede $\lingua_M$-Formel $\varphi$ mit $\RM(\varphi)\neq\infty$ eine natürliche Zahl $n$, sodass es in jeder Elementarerweiterung $\fM'$ maximal $n$ viele disjunkte Teilmengen von $\varphi(M')$ mit demselben Morleyrang gibt. Das kleinste solche $n$ nennt man den Morleygrad $\DM$.
	\end{factdef}
	
	\begin{definition}
		Für einen Typen $p$ definiere $$\RM(p):=\min\limits_{\psi\in p}\RM(\psi),\ \DM(p):=\min\limits_{\psi\in p,\RM(\psi)=\RM(p)}\DM(\psi).$$
		Als Kurzschreibweise stehe außerdem bei einem gegebenen Modell $\fM$, $\overline{a}$ in $M$ und $S$ als Teilmenge von $M$ die Bezeichnung $\RM(\overline{a}/S)$ für $\RM(\tp(\overline{a}/S))$.\\
		Analog sei $\DM(p)$ definiert.
	\end{definition}
	\begin{remark}
		Der Rang und Grad eines Typen werden in einer Formel aus dem Typen angenommen, nenne diese \textbf{minimal}.
	\end{remark}
	
	\begin{fact}
		Es sei in einem Modell $\fM$ eine definierbare Menge $X$ streng minimal. Dann ist für jedes Tupel $\overline{a}$ in $X$ sowie jede Teilmenge $S$ von $M$, über der $X$ definierbar ist, der Morleyrang schon bekannt: $$\RM(\overline{a}/S)=\dim(\overline{a}/S)$$
		Außerdem ist eine Formel $\varphi$ streng minimal genau dann, wenn ihr Rang und Grad beide Eins sind.
	\end{fact}
	
	\begin{fact}
		Sei $\fM$ ein Modell und $S\subseteq M$, außerdem $F$ eine konsistente und unter Konjunktion abgeschlossene Menge von $\lingua_S$-Formeln in $n$ freien Variablen, die eine Formel mit definiertem Morleyrang enthält. Dann gibt es für jede hinreichend große Obermenge $S'$ von $S$ genau $\DM(F)$ Fortsetzungen von $F$ zu einem Typen über $S'$ mit Morleyrang $\RM(F)$ (die Definition von Rang und Grad werde entsprechend von Typen verallgemeinert).\\
		Wenn der Morleygrad Eins ist, nenne den partiellen Typen \textbf{stationär}. 
	\end{fact}
	
	\begin{fact}\label{Stabilität Morleyrang}
		Die Theorie T ist $\omega$-stabil genau dann, wenn jede Formel definierten Morleyrang hat. Äquivalent ist, dass jede Formel in einer Variable definierten Morleyrang hat oder dass jeder Typ in einer Variable definierten Morleyrang hat.
	\end{fact}
	
	\begin{fact}\label{Anfangsstück}
		In jedem Modell ist für alle natürlichen Zahlen $n$ die Menge der angenommenen Morleyränge von Formeln oder Typen in $n$ Variablen ein Anfangsstück von $\textbf{On}\cup\{\infty\}$.
	\end{fact}
	
	\begin{fact}
		Sei $\fM$ ein Modell und seien $\overline{a},\overline{b}$ Tupel in $M$ sowie $S$ eine Teilmenge von $M$. Wenn $\overline{a}$ und $\overline{b}$ interalgebraisch über $S$ sind, folgt $$\RM(\overline{a}/S)=\RM(\overline{b}/S).$$
	\end{fact}
	
	\section{Algebraische und lineare Disjunktheit von Körpern}
	In diesem Teil richten wir uns im Aufbau und der Lemma-übergreifenden Strategie im Großen und Ganzen nach \cite{Delon}, wohingegen die konkreten Beweise meist von \cite{SergeLang} inspiriert sind. Ziel ist es, zwei algebraische Relationen zu verstehen, die Körper zueinander haben können.
	
    \begin{definition}
    	Gegeben Körperinklusionen $C\subseteq K,L\subseteq M$ in Rautenform, nenne $K$ und $L$ \textbf{linear disjunkt über} $C$, falls alle Basen von $K$ als $C$-Vektorraum auch über $L$ linear unabhängig bleiben. Nenne $K$ und $L$ \textbf{algebraisch disjunkt über} $C$, falls alle Transzendenzbasen von $K$ über $C$ auch algebraisch unabhängig über $L$ bleiben. Schreibe $K\ld_CL$ bzw. $K\ad_CL$.
    \end{definition}
    
    \begin{remark}
    	\ 
    	\begin{itemize}
    		\item Algebraische Disjunktheit ist nichts anderes als die Einschränkung der Unabhängigkeit aus dem letzten Kapitel auf bestimmte Mengen (nämlich Körpern in Rautenanordnung). Denn Körper $K$ und $L$ sind algebraisch disjunkt über $C$ genau dann, wenn sie unabhängig über $C$ sind im modelltheoretischen Sinn als Teilmengen eines algebraisch abgeschlossenen Körpers.\\
    		Wir bezeichnen es dennoch anders, um Verwirrung zu vermeiden. Wenn nämlich alle betrachteten Mengen schon Körper sind und die beiden Mengen, zwischen denen Unabhängigkeit gilt, schon Obermengen der dritten sind, kann man etwas weitergehende Eigenschaften feststellen, die im Normalfall so nicht gelten.
    		\item Es reicht, lineare Disjunktheit für eine Basis zu zeigen. Denn wenn man die lineare Unabhängigkeit über $L$ für eine Basis verliert, verliert man sie per $C$-Basiswechsel auch für alle anderen.
    		\item Es reicht, die Erhaltung der linearen/algebraischen Unabhängigkeit nur für beliebige endliche Mengen zu prüfen. Denn lineare/algebraische Unabhängigkeit einer Menge besteht genau dann, wenn sie für alle endlichen Teilmengen gilt.
    		\item Der Körper $M$ kommt in der Definition nur vor, damit die Rechenoperationen zwischen $K$ und $L$ wohldefiniert sind. Die genaue Wahl ist irrelevant und daher nicht in der Notation berücksichtigt. Wir nehmen für die Zukunft einfach an, dass die Multiplikation klar definiert ist. Es wird sich sowieso herausstellen, dass im Fall $K\ld_CL$ die Operationen eindeutig bestimmt sind.
    	\end{itemize}
    \end{remark}
    
    \begin{example}
    	Wir wollen Beispiele für die verschiedenen Konstellationen aus linearer und algebraischer Disjunktheit geben: 
    	\begin{itemize}
    		\item Trivialerweise gilt für alle Körper $C\subseteq K$, dass $C$ und $K$ sowohl linear als auch algebraisch disjunkt über $C$ sind.\\
    		Ein weniger leichtes Beispiel wäre, dass $\setQ(\sqrt{2})$ und $\setQ(\sqrt{3})$ linear und algebraisch disjunkt über $\setQ$ sind. Die lineare Disjunktheit folgt, wenn man feststellt, dass $1$ und $\sqrt{2}$ nicht linear abhängig über $\setQ(\sqrt{3})$ sein können, da sonst $\sqrt{2}$ in $\setQ(\sqrt{3})$ wäre. Für algebraische Disjunktheit ist nichts zu zeigen, da $\dim(\setQ(\sqrt{2})/\setQ)$ ohnehin schon $0$ ist.
    		\item Als Beispiel für algebraische und fehlende lineare Disjunktheit können die Körper $\setQ(\sqrt{2})$ und $\setQ(\sqrt{2})$ über $\setQ$ dienen. Algebraische Disjunktheit folgt dabei wie oben und lineare Disjunktheit gilt nicht, da $1$ und $\sqrt{2}$ nicht linear unabhängig über $\setQ(\sqrt{2})$ sind.
    		\item Ausstehend ist noch der Fall, in dem keine Art von Disjunktheit gilt: Das ist zum Beispiel bei den Körpern $\setQ(\pi)$ und $\setC$ über $\setQ$ so, da $\setQ(\pi)$ Teilmenge von $\setC$ ist und daher lineare Dimension $1$ und algebraische Dimension $0$ über $\setC$ hat, aber lineare Dimension $\aleph_0$ und algebraische Dimension $1$ über $\setQ$ besitzt.
    	\end{itemize}
        Es fällt auf, dass die Kombinationsmöglichkeit \glqq{}lineare, aber keine algebraische Disjunktheit\grqq{} nicht behandelt wurde. Das liegt daran, dass sie nicht möglich ist, wie später erklärt werden wird.
    \end{example}
    \newpage
    \begin{lemma}\label{Fraktionskörper}
    	Sei $C$ ein Körper und seien $C\subseteq R,S$ Ringerweiterungen des Körpers auf Integritätsbereiche. Dann sind $\operatorname{Frac}(R)$ und $\operatorname{Frac}(S)$ linear disjunkt über $C$ genau dann, wenn linear unabhängige Mengen in $R$ über $C$ auch linear unabhängig über $S$ bleiben.
    \end{lemma}
    \begin{proof}
    	Die Hinrichtung folgt leicht aus $R\subseteq\operatorname{Frac}(R),S\subseteq\operatorname{Frac}(S)$. Für die Rückrichtung seien $r_1x_1^{-1},\dots,r_nx_n^{-1}$ in $\operatorname{Frac}(R)$ linear unabhängig über $C$, aber linear abhängig über $\operatorname{Frac}(S)$ mit nichttrivialer Linearkombination $$(s_1y_1^{-1})r_1x_1^{-1},\dots,(s_ny_n^{-1})r_nx_n^{-1}=0,$$ wobei alle $s_iy_i^{-1}$ aus $\operatorname{Frac}(S)$ seien.\\
    	Durch Multiplikation mit $\prod\limits_{i=1}^nx_iy_i\neq0$ erhält man die Gleichung $$0=\sum\limits_{j=1}^n(\prod\limits_{i\neq j}x_i)r_j(\prod\limits_{i\neq j}y_i)s_j.$$ Diese bezeugt die lineare Abhängigkeit über $S$ der Elemente $((\prod\limits_{i\neq j}x_i)r_j)_{1,\dots,n}$ in $R$, die über $C$ unabhängig sind.
    \end{proof}
    
    \begin{remark}
    	Es reicht in obiger Aussage wieder, sich auf endliche Mengen zu beschränken. Alternativ kann man die Erhaltung der linearen Unabhängigkeit auch wieder nur für eine $C$-Basis von $R$ zeigen.
    \end{remark}
    
    Das folgende Lemma sagt insbesondere aus, dass lineare Disjunktheit symmetrisch ist. Noch viel wichtiger ist aber die Aussage über die Struktur der zueinander als Ringe adjungierten Körper.
    
    \begin{lemma}\label{Tensoren}
    	Für Körper $C,K,L$ wie oben gilt $K\ld_CL$ genau dann, wenn $$K[L]=L[K]\cong K\otimes_CL$$ mit kanonischem Isomorphismus.
    \end{lemma}
    \begin{proof}
    	Der aufgespannte Ring erfüllt $$K[L]=\{\sum\limits_{(k,l)\in X}kl\mid X\subseteq K\times L\text{ endlich}\}=L[K].$$
    	Wenn $(k_i)_I,(l_j)_J$ Basen von $K,L$ über $C$ sind, ist $(k_i\otimes l_j)_{i\in I,j\in J}$ eine Basis von $K\otimes_CL$.\newpage
    	Der $C$-Homomorphismus $$\sum\limits_{i\in I_0,j\in J_0} c_{ij}k_i\otimes l_j\mapsto \sum\limits_{i\in I_0,j\in J_0} c_{ij}k_il_j$$ für endliche, beliebige Teilmengen $I_0\subseteq I,J_0\subseteq J$ ist immer surjektiv, da klarerweise $E:=(k_il_j)_{i\in I,j\in J}$ ein Erzeugendensystem von $K[L]$ ist.\\
    	Er ist injektiv genau dann, wenn $E$ auch linear unabhängig über $C$ ist, also keine Linearkombination $$0=\sum\limits_{i\in I_0,j\in J_0}c_{ij}k_il_j=\sum\limits_{i\in I_0}(\sum\limits_{j\in J_0}c_{ij}l_j)k_i$$ existiert mit $c_{ij}\neq0$ für mindestens ein Paar $(i,j)$. Aber das ist genau dann der Fall, wenn keine $\tilde{c}_i=\sum\limits_{j\in J_0}c_{ij}l_j$ in $L$ existieren mit $i$ in $I_0$ und so, dass $\tilde{c}_i\neq0$ für mindestens ein $i$ und $0=\sum\limits_{i\in I_0}\tilde{c}_ik_i$ gilt; also wenn die $(k_i)_I$ linear unabhängig über $L$ sind.
    \end{proof}
    
    \begin{corollary}\label{Isomorphismen linear disjunkt}
    	Seien im Folgenden die Körper $C,K,L,C',K',L'$ gegeben, sodass $K$ und $L$ linear disjunkt über $C$ sind sowie $K'$ und $L'$ über $C'$. Es existiere ein Isomorphismus $C\cong C'$, der eine Fortsetzung auf $\varphi_1:K\cong K'$ und eine andere auf $\varphi_2:L\cong L'$ habe. Dann gibt es eine \glqq{}fusionierte\grqq{} Fortsetzung auf $K[L]\cong K'[L']$, denn die zwei Fortsetzungen induzieren einen Isomorphismus $$K\otimes_CL\cong K'\otimes_C'L',$$ durch die Abbildungsvorschrift $$k\otimes l\mapsto\varphi_1(k)\otimes\varphi_2(l).$$
    	Dieser setzt die vorigen Abbildungen von den Teilkörpern $$K\cong K\otimes_C1,L\cong1\otimes_CL,C\cong C\otimes_C1=1\otimes_CC$$ fort.\\
    	Insbesondere gibt es auch nur eine Möglichkeit, die Verknüpfung von Elementen aus linear disjunkten Körpern zu definieren (was wir zu Beginn dieses Kapitels schon ohne Beweis erwähnt hatten).
    \end{corollary}
    \newpage
    \begin{definition}
    	\ 
    	\begin{itemize}
    		\item Eine Körpererweiterung $K\subseteq L$ heiße \textbf{regulär}, wenn $\overline{K}\ld_KL$.
    		\item Für Körper $K,L\subseteq M$ sei $KL:=K(L)=L(K)$.
    	\end{itemize}
    \end{definition}
    
    \begin{example}
    	Die simpelsten Möglichkeiten für eine reguläre Körpererweiterung $K\subseteq L$ wäre einerseits, wenn $K$ selbst schon algebraisch abgeschlossen ist; andererseits auch die Adjunktion eines transzendenten Elements $e$ zu einem Körper $K$.\\
    	Im ersten Fall gilt die lineare Disjunktheit schon wegen Körpergleichheit, im zweiten Fall ist die $K$-Basis $\dots,e^{-1},1,e,e^2,\dots$ linear unabhängig über $\overline{K}$, sonst wäre $e$ nicht transzendent.
    \end{example}
    
    Wir können einige Folgerungen und \glqq{}Rechenregeln\grqq{} aus den definierten Eigenschaften ziehen, die insbesondere zeigen, unter welchen Bedingungen sich lineare und algebraische Disjunktheit gegenseitig implizieren und was für Regeln bei Inklusionsketten gelten.
    
    \begin{lemma}\label{Stapellemma}
    	Gegeben die Körperinklusionen $C\subseteq L\subseteq M$ und $C\subseteq K$. Dann gilt $K\ld_CM$ genau dann, wenn $K\ld_CL$ und $KL\ld_LM$ gilt.
    \end{lemma}
    \begin{proof}
    	Sei $(k_h)_H$ eine Basis von $K$ über $C$, $(l_i)_I$ eine Basis von $L$ über $C$ und $(m_j)_J$ eine Basis von $M$ über $L$. Die Aussage $K\ld_CM$ bedeutet, dass die $C$-Basis von $K$ auch eine $M$-Basis von $K$ ist, aber dann ist sie natürlich auch eine $L$-Basis wegen den Inklusionen $C\subseteq L\subseteq M$. Dies folgt, da die Eigenschaft, Erzeugendensystem über dem Körper zu sein, sich nach oben vererbt, die für die lineare Unabhängigkeit sich dafür nach unten vererbt. Also gilt $K\ld_CL$.\\
    	Außerdem ist $(l_im_j)_{I\times J}$ eine Basis von $M$ über $C$ und $(k_h)_H$ eine Basis von $L[K]$ als $L$-Vektorraum. Die erste Aussage ist aus dem Beweis der Multiplikativität von Körpererweiterungsgraden bekannt und bei der zweiten folgt aus der Definition von $\ld$ die Unabhängigkeit, die Eigenschaft als Erzeugendensystem ist klar.\\
    	Wenn $KL\ld_LM$ nicht gelten würde, müsste nach Lemma \ref{Fraktionskörper} und der anschließenden Bemerkung schon die $L$-Basis $(k_h)_H$ von $L[K]$ linear abhängig über $M$ sein, es gäbe also eine $M$-Linearkombination $$\sum\limits_{h\in H}\lambda_hk_h=0,\ \lambda_h=0\text{ für fast alle, aber nicht alle } h\text{ in }H.$$
    	Da $(k_h)_H$ aber die Basis von $K$ über $C$ ist, widerspricht das $K\ld_CM$.\newpage
    	Für die Rückrichtung ist zu zeigen, dass $(l_im_j)_{I\times J}$ linear unabhängig über $K$ bleibt. Wenn nicht, sei $$0=\sum\limits_{(i,j)\in I\times J}\lambda_{ij}l_im_j,\text{ wobei }\lambda_{ij}=0\text{ für fast alle, aber nicht alle } (i,j)\text{ in }I\times J$$ eine $K$-Linearkombination, die das bezeugt. Schreibe $$\lambda_{ij}=:\sum\limits_{h\in H}c_{hij}k_h$$ als $C$-Basisdarstellung für alle $(i,j)$ in $I\times J$.\\
    	Einsetzen und Umklammern führt uns zur Linearkombination $$0=\sum\limits_{(i,j)\in I\times J}\lambda_{ij}l_im_j=\sum\limits_{(i,j)\in I\times J}\left(\sum\limits_{h\in H} c_{hij}k_h\right)l_im_j=\sum\limits_{i\in I}\left(\sum\limits_{(h,j)\in H\times J}c_{hij}k_hl_i\right)m_j,$$ $$c_{hij}=0\text{ für fast alle, aber nicht alle } (h,i,j)\text{ in }H\times I\times J.$$
    	Da wir aber $KL\ld_CM$ annehmen, muss $$\sum\limits_{(h,j)\in H\times J}c_{hij}k_hl_i=0\text{ sein für alle }j\text{ in }J,$$ und da wir $K\ld_CL$ annehmen, folgt daraus $c_{hij}=0$ für alle $h,i,j$.
    \end{proof}
    
    Für weitere Regeln brauchen wir einen eingeschobenen Fakt aus \cite{SergeLang} (Seite 57, Theorem 3), der mit Bewertungen bewiesen wird.
    \begin{fact}\label{Das komplizierte Lemma}
    	Wenn $C\subseteq K$ regulär ist und $K\ad_CL$, folgt $K\ld_CL$.
    \end{fact}
    
    Damit lassen sich dann einige Regeln für das \glqq{}Herumschieben\grqq{} von linearer und algebraischer Disjunktheit beweisen.
    \begin{lemma}\label{Rechenregeln}
    	Seien $C,K,L,M$ Körper.
    	\begin{enumerate}
    		\item Wenn $K$ und $L$ linear disjunkt über $C$ sind, so ist $K\cap L=C$.
    		\item Wenn die Erweiterung $C\subseteq K$ algebraisch und die Erweiterung $C\subseteq L$ regulär ist, dann sind $K$ und $L$ linear disjunkt über $C$.
    		\item Wenn $K$ und $L$ linear disjunkt über $C$ sind, sind sie auch algebraisch disjunkt.
    		\item Wenn $K$ und $L$ algebraisch disjunkt über $C$ sind, dann sind es auch $\overline{K}$ und $\overline{L}$.
    		\item Wenn $K$ und $L$ linear disjunkt über $C$ sind, $K\subseteq M$ und $X\subseteq M$ algebraisch unabhängig über $KL$, dann folgt $K(X)\ld_KKL$.
    		\item Wenn die Erweiterung $C\subseteq K$ regulär ist und $K$ und $L$ linear disjunkt über $C$ sind, folgt $K\ld_C\overline{L}$ und außerdem, dass die Erweiterung $L\subseteq KL$ regulär ist.
    	\end{enumerate}
    \end{lemma}
    \begin{proof}
    	\ 
    	\begin{enumerate}
    		\item Die eine Inklusion gilt, denn $C\subseteq K,L$. Für die Inklusion in Gegenrichtung sei $x$ in $(K\cap L)\setminus C$. Dann ist $(1,x)$ in $K$ linear abhängig über $L$ und somit auch über $C$. Aber dann ist $x$ schon in $C$, weil es als $C$-Vielfaches von $1$ geschrieben werden kann.
    		\item Wegen der Regularität gilt $L\ld_C\overline{C}$ und wegen $C\subseteq K\subseteq\overline{C}$ und dem Lemma \ref{Stapellemma} gilt $L\ld_C K$.
    		\item Seien $k_1,\dots,k_n$ in $K$ algebraisch abhängig über $L$, das heißt, es gibt ein Polynom $$0\neq f(X)=\sum\limits_{\abs{\alpha}\leq m}l_\alpha X^\alpha\text{ in }L[X_1,\dots,X_n]$$ mit $f(k)=0$, also insbesondere $(k^\alpha)_{\abs{\alpha}\leq m}$ linear abhängig über $L$, per Annahme also auch über $C$. Dann existiert aber eine Linearkombination $\sum\limits_{\abs{\alpha}\leq m}c_\alpha k^\alpha=0$ mit $c_\alpha$ in $C$ nicht alle Null, und diese bezeugt die algebraische Abhängigkeit über $C$.
    		\item Diese Aussage ist exakt der Spezialfall von Lemma \ref{Unabhängigkeit acl} in unserem Setting von Körpern als Teilmengen eines algebraisch abgeschlossenen Körpers.
    		\item Wegen algebraischer Unabhängigkeit ist $$\abs{\overline{x}}=\dim(\overline{x}/KL)\leq\dim(\overline{x}/K)\leq\abs{\overline{x}}$$ für alle $\overline{x}$ in $X$, also sind $X$ und $KL$ im modelltheoretischen Sinne unabhängig über $K$. Das gilt dann nach den Regeln aus Kapitel \ref{Kapitel 0} auch für $K\cup X$ und $KL$, für $\acl(K\cup X)$ und $KL$ und auch für $K(X)$ und $KL$, also $K(X)\ad_KKL$.\\
    		Die Erweiterung $K(X)\supseteq K$ ist außerdem regulär, denn $K[X]$ hat als $K$-Basis $$\left\{\prod\limits_{i=1}^kx_i^{n_i}\right\}_{\{x_1,\dots,x_k\}\subseteq X\text,\ n\in\setN^k},$$ wie man wegen algebraischer Unabhängigkeit sieht.\newpage
    		Diese Basis bleibt aber linear unabhängig über $\overline{K}$, denn sonst wäre ein Polynom in $\overline{K}[X_1,X_2,\dots]$ gefunden, was ein $x$ aus $X$ über den anderen Elementen algebraisiert, also $$x\in\acl(X\setminus\{x\}\cup\overline{K})=\acl(X\setminus\{x\}\cup K),$$ demnach wäre $X$ nicht mehr algebraisch unabhängig über $K$. Lemma \ref{Fraktionskörper} besagt dann $$K(X)=\operatorname{Frac}(K[X])\ld_K\overline{K}.$$
    		Also haben wir $K(X)\ad_KKL$ und $K(X)\supseteq K$ regulär, woraus nach Fakt \ref{Das komplizierte Lemma} $K(X)\ld_KKL$ folgt.
    		\item Mit 3. folgt $K\ad_CL$, mit 4. $\overline{K}\ad_C\overline{L}$, mit Lemma \ref{Stapellemma} $K\ad_C\overline{L}$, mit Fakt \ref{Das komplizierte Lemma} $K\ld_C\overline{L}$ (benutze $C\subseteq K$ regulär) und mit noch einmal Lemma \ref{Stapellemma} gilt schließlich für die Einbettungskette $C\subseteq L\subseteq\overline{L}$ die Regularitätsbedingung $LK\ld_L\overline{L}$.
    	\end{enumerate}
    \end{proof}
    
    Es hat nicht nur die algebraische Disjunktheit, sondern auch die Regularität einer Erweiterung und die lineare Disjunktheit eine modelltheoretische Bedeutung.
    
    \begin{lemma}
    	Es sei eine Körpererweiterung $K\subseteq L$ gegeben und ein Tupel $a$ aus $L$. Dann ist $K\subseteq K(a)$ genau dann regulär, wenn $\tp(a/K)$ stationär im Modell $\overline{L}$ von ACF ist.
    \end{lemma}
    \begin{proof}
    	Die minimale Formel dieses Typen ist eine, die beschreibt, welche Elemente algebraisch übereinander sind und wie sie es sind. Ohne Einschränkungen sei $a$ so angeordnet, dass die ersten $k$ Einträge algebraisch unabhängig über $K$ sind und alle späteren im algebraischen Abschluss davon. Die minimale Formel beschreibe dann ohne Einschränkungen Polynome von minimalem Grad derart, dass $a_{k+1}$ über $K[a_1,\dots,a_k]$ algebraisiert wird, $a_{k+2}$ über $K[a_1,\dots,a_{k+1}]$ und so weiter.\\
    	Hierbei seien die auftretenden Potenzen der $a_i$ in den Koeffizienten schon maximal reduziert, also für alle $i$ größer als $k$ auf jeden Fall kleiner als der Grad des Minimalpolynoms von $a_i$ über $K(a_1,\dots,a_{i-1})$, dieser sei mit $n_i$ bezeichnet. Dann ist $\tp(a/K)$ stationär genau dann, wenn keines dieser Polynome über $K[a_1,\dots,a_i]$ beim Übergang zu einer größeren Parametermenge als $K$ reduzibel wird.\newpage
    	Da ACF modellvollständig ist, ist das äquivalent dazu, dass kein solches Polynom beim Übergang zu $\overline{K}$ reduzibel wird und das wiederum ist dasselbe wie die Aussage, dass die Erzeuger $$(a_1^{m_1}\cdot\dots\cdot a_\abs{a}^{m_\abs{a}})_{m_1,\dots,m_k\in\setN,0\leq m_i<n_i\text{ für }i>k}$$ von $K[a]$ über $\overline{K}$ linear unabhängig bleiben, also dass $K(a)$ und $\overline{K}$ linear disjunkt über $K$ sind.
    \end{proof}
    
    \begin{remark}
    	Damit ist die Folgerung \ref{Isomorphismen linear disjunkt} das algebraische Analogon zu folgender Aussage aus der Stabilitätstheorie: Sei $\fM,\fN$ zwei Modelle einer $\omega$-stabilen vollständigen, abzählbaren Theorie und Mengen $S\subseteq M,S'\subseteq N$, sodass es eine elementare Abbildung $$\varphi:S\cong S'$$ gibt, außerdem seien $a,b$ Tupel in $M$ und $a',b'$ Tupel in $N$, sodass gelte: $$\varphi(\tp(a/S))=\tp(a'/S'),\varphi(\tp(b/S))=\tp(b'/S).$$
    	Wenn $a$ und $b$ unabhängig über $S$ sind sowie $a'$ und $b'$ unabhängig über $S'$, außerdem $\tp(b'/S)$ stationär, dann kann der gemeinsame Typ von $a$ und $b$ ebenso durch $\varphi$ übertragen werden: $$\varphi(\tp(a,b/S))=\tp(a',b'/A).$$
    	Dass die Aussagen einander fast entsprechen, sieht man daran, dass die Übertragung der Typen durch $\varphi$ Isomorphismen zwischen den erzeugten Körpern $\langle S,a\rangle$ und $\langle S',a'\rangle$ sowie zwischen $\langle S,b\rangle$ und $\langle S',b'\rangle$ entspricht, die sich auf einen Isomorphismus $$\langle S,a,b\rangle\cong\langle S',a',b'\rangle$$ fortsetzen lassen. Der einzige Unterschied besteht darin, dass in der modelltheoretischen Version die Unabhängigkeit von $a$ und $b$ über $S$ sowie von $a'$ und $b'$ über $S'$ eine algebraische Disjunktheit auf beiden Seiten erzeugt, die wegen Stationarität des Typen nur einseitig zur linearen Disjunktheit wird.
    \end{remark}

    \newpage
    \section{Paare algebraisch abgeschlossener Körper}
    Wir wollen Paare $(K,E_K)$ von Körpern betrachten, wobei $E_K\subseteq K$ ist. In der richtigen Sprache lassen diese sich axiomatisieren, dort haben echte Paare (d.h. $K\neq E_K$) algebraisch abgeschlossener Körper mit fixierter Charakteristik sogar Quantoren\-elimination, sie sind vollständig und in einer kleineren Sprache modellvollständig. Diese Sprachen und einige der Folgerungen für ihre Strukturen wollen wir hier (angelehnt an \cite{Delon}) einführen.
    
    \begin{definition}
    	Wir definieren die Sprachen $\lld:=\{0,1,+,-,\cdot,(l_n)_{n\geq2}\},\lf:=\lld\cup\{f_{i,n}\mid n\geq2,1\leq i\leq n\}$ und $\lfc:=\lf\cup\{^{-1}\}$, wobei die $(l_n)_n$ $n$-stellige Relationen sein sollen und die $(f_{i,n})_{i,n}$ $n+1$-stellige Funktionen.
    \end{definition}
    
    Es kommt im Folgenden zu einem kleineren Problem: Eigentlich benötigt man für $^{-1}$ und die $f_{i,n}$ eine Auffassung als partielle Funktionen und im Folgenden werden sie auch so behandelt. Da das aber grundsätzlich nicht in unseren Axiomen der Logik vorgesehen ist, muss man die Funktionen durch $0$ fortsetzen. Rein formal gesehen gibt es dann den Unterschied zwischen der Interpretation von $$\glqq{}f_{i,n}(\overline{x})=0\grqq{}$$ als logische Formel in der Sprache und als für uns relevante Aussage, der gerade in der Aussage $$\glqq{}\overline{x}\text{ liegt nicht im partiellen Definitionsbereich von }f_{i,n}\grqq{}$$ besteht.\\
    Da aber sowohl der Definitionsbereich von $^{-1}$ als auch von jedem $f_{i,n}$ quantorenfrei\linebreak 0-definierbar ist, wird dieser Unterschied im Folgenden ignoriert werden, weil der Übergang zwischen diesen Interpretationen immer möglich ist, ohne in irgendeiner Form Bedingungen oder Aussagen von Sätzen zu verändern. Diese Tatsache ist stets im Hinterkopf zu behalten, insbesondere auch in Folgerung \ref{Formel-Vereinfachung}. Dort reicht quantorenfrei nämlich nicht aus, mehr dazu aber an der gegebenen Stelle.\\\\
    Nach der Klärung dieser Probleme ist es jetzt möglich, überhaupt zur Bedeutung dieser Sprache zu kommen. Es sei noch festgehalten, dass das weitere Vorgehen auch ohne Benutzung von \glqq{}$^{-1}$\grqq{} möglich wäre. Allerdings besteht das Problem mit den partiellen Funktionen ohnehin, und daher kann man die Behebung auch auf diese Funktion anwenden, wenn man so ein Vorgehen benutzen muss.
    
    \newpage
    
    \begin{lemma}\label{Symbolik}
    	Beliebige Paare $(K,E_K)$ von Körpern werden kanonisch zu $\lld$-Strukturen, indem man folgendes setzt:
    	$$\models l_n(x_1,\dots,x_n):\Leftrightarrow x_1,\dots,x_n\text{ sind linear unabhängig über }E_K.$$
    	Dann kann man die Substruktur $E_K$ auch definieren, da $x$ in $E_K$ ist genau dann, wenn $$\models\neg l_2(1,x)=:E(x)$$ und noch viel weitergehender auch $$y\in\langle\overline{x}\rangle_{E_K}\text{ für } x_1,\dots,x_n\text{ linear unabhängig über }E_K\Leftrightarrow\ \models l_n(\overline{x})\land\neg l_{n+1}(\overline{x},y)=:\phi(\overline{x},y).$$
    	Mit diesem Wissen setzt man jetzt in $\lf$ bzw. $\lfc$
    	$$\models (z=f_{i,n}(y,\overline{x})):\Leftrightarrow\ \models\phi(\overline{x},y)\text{ und }z\text{ ist die }i\text{-te Koordinate von }y\text{ in der Basisdarstellung},$$
    	wobei letzteres durch $$\exists z_1,\dots,z_n(z=z_i\land y=x_1z_1+\dots+x_nz_n\land z_1,\dots,z_n\in E)$$ oder aber auch $$\forall z_1,\dots,z_n(y=x_1z_1+\dots+x_nz_n\land z_1,\dots,z_n\in E\rightarrow z_i=z)$$ definierbar ist.
    \end{lemma}
    
    \begin{lemma}
    	Mit diesen Vorarbeiten sind echte algebraisch abgeschlossene Paare von Körpern definierbar in allen drei Sprachen $\lld,\lf,\lfc$, nenne die Theorien $$\operatorname{ACP}^{\lld},\operatorname{ACP}^{\lf},\operatorname{ACP}^{\lfc}.$$ Besonders zu beachten hierbei ist, dass man in der Theorie sagen muss, dass $\neg l_2(1,x)$ einen Körper definiert.
    \end{lemma}
    \newpage
    Alle verwendeten Sprachen und ihre Interpretationen für Paare von Körpern sind interdefinierbar, aber für bestimmte Fragestellungen sind manche Sprachen passender als andere. Die intuitivste Sprache wäre dabei $\lingua^E:=\lingua\cup\{E(x)\}$ mit der obigen Bedeutung für $E$, aber diese hat leider nicht so starke Eigenschaften (zum Beispiel ist sie nicht modellvollständig, denn lineare Unabhängigkeit über $E$ überträgt sich nicht zwangsläufig auf Obermodelle). Das Ziel ist jetzt, zu beweisen, dass $\operatorname{ACP}^{\lf}$ und $\operatorname{ACP}^{\lfc}$ Quantoren\-eli\-mi\-na\-tion haben sowie dass $\operatorname{ACP}^{\lld}$ immerhin modellvollständig ist.\\
    Dazu müssen wir erst einmal verstehen, wie $\lfc$-Unterstrukturen von $\operatorname{ACP}^{\lfc}$-Modellen aussehen.
    
    \begin{lemma}
    	Betrachte ein Paar von Körpern $(K,E_K)$ und eine Teilmenge ${A\subseteq K}$ sowie eine $\lfc$-Struktur $$\fA:=(A,0,1,+,-,\cdot,^{-1},(l_n)_{n\geq2},(f_{i,n})_{n\geq2,1\leq i\leq n}).$$ Dann ist $\fA$ eine $\lfc$-Unterstruktur von $(K,E_K)$ genau dann, wenn $A$ ein Unterkörper von $K$ ist und außerdem $\fA=(A,E_A)$ für $E_A:=A\cap E_K$ gilt sowie $A$ und $E_K$ linear disjunkt über $E_A$ sind.
    \end{lemma}
    \begin{proof}
    	Die Menge $A$ ist genau dann Unterkörper von $K$, wenn es $0,1$ enthält, unter $+,-,\cdot,^{-1}$ abgeschlossen ist und die entsprechenden Abbildungsvorschriften erbt.\\
    	Außerdem sind $A$ und $E_K$ linear disjunkt über $E_A$ genau dann, wenn für alle $\overline{a}$ in $A$ aus $\overline{a}$ linear abhängig über $E_K$ schon äquivalent zu linearer Abhängigkeit über $E_A$ ist; per kanonischer Definition also genau dann, wenn $$(A,E_A)\models l_{\abs{\overline{a}}}(\overline{a}) \Leftrightarrow (K,E_K)\models l_{\abs{\overline{a}}}(\overline{a}).$$
    	Wenn $\fA$ eine Unterstruktur von $(K,E_K)$ ist, gilt für alle $\overline{a}$ in $A$, dass
    	\begin{align*}
    	&\fA\models l_{\abs{\overline{a}}}(\overline{a})\Leftrightarrow(K,E_K)\models l_{\abs{\overline{a}}}(\overline{a})\\
    	\Leftrightarrow&\ \overline{a}\text{ linear unabhängig über }E_K\Leftrightarrow\overline{a}\text{ linear unabhängig über }E_A,
    	\end{align*}
    	wobei die letzte Äquivalenz davon herrührt, dass in jeder die lineare Abhängigkeit bezeugenden Linearkombination von $\overline{a}$ die Koeffizienten, die nicht $0$ sind, durch Anwendung von Funktionen $f_{i,n}$ auf Teile von $\overline{a}$ extrahiert werden können. Projektionen von Elementen aus $A$ sind aber wegen der Unterstruktureigenschaft wieder in $A$.\newpage
    	Also stimmen die Interpretationen der $l_n$ in $\fA$ mit denen in $(A,E_A)$ überein und es ist $A\ld_{E_A}E_K$. Da die definierende Formel der $(f_{i,n})$ bis auf die Angabe des Bildbereiches eine $\lingua_{\text{Ring}}$-Formel ist, und da $f_{i,n}(\overline{a})$ in $E_K\cap A$ für alle $\overline{a}$ in $A$ und für alle $n,i$, stimmen auch die Interpretationen der $f_{i,n}$ in $\fA$ und in $(A,E_A)$ überein. Damit ist $\fA=(A,E_A)$.\\
    	Die Rückrichtung folgt mit den ersten Zeilen dieses Beweises und, weil die Projektionen $f_{i,n}$ Elemente aus $A$ nach $A$ abbilden. Das erkennt man dadurch, dass man die Koeffizienten einer Linearkombination in eine Abhängigkeitsbedingung umschreiben kann; wenn diese in $E_K$ erfüllt ist, muss sie wegen linearer Disjunktheit auch in $E_A$ erfüllt sein, also sind die Projektionen schon in $E_A$.
    \end{proof}
    
    \begin{lemma}\label{transz Erw}
    	Wenn $(A,E_A)$ eine $\lfc$-Unterstruktur von $(K,E_K)$ ist und $X\subseteq K$ algebraisch unabhängig über $AE_K$, dann erhalten wir die folgende Kette von Inklusionen: $$(A,E_A)\subseteq_{\lfc}(A(X),E_A)\subseteq_{\lfc}(K,E_K).$$
    \end{lemma}
    \begin{proof}
    	Nach dem vorigen Lemma sind $A$ und $E_K$ linear disjunkt über $E_A$, daher gilt mit Lemma \ref{Rechenregeln} (5.) $$A(X)\ld_{E_A}E_K.$$ Da $A(X)$ Unterkörper von $K$ ist, gilt mit der Rückrichtung des letzten Lemmas, dass $$(A(X),E_A)\subseteq_{\lfc}(K,E_K).$$
    	Die Struktur $(A,E_A)$ ist eine $\lfc$-Unterstruktur von $(A(X),E_A)$, weil die Bedingung $A\ld_{E_A} E_A$ immer erfüllt ist.
    \end{proof}
    
    \begin{lemma}\label{E-Erw}
    	Sei $(A,E_A)\subseteq_{\lfc}(K,E_K)$ und $E_A\subseteq B\subseteq E_K$ ein Zwischenkörper. Dann ist $$(A,E_A)\subseteq_{\lfc}(AB,B)\subseteq_{\lfc}(K,E_K).$$
    \end{lemma}
    \begin{proof}
    	Es sind nur die Bedingungen $$A\ld_{E_A}B\text{ und }AB\ld_BE_K$$ zu zeigen. Wegen $(A,E_A)\subseteq_{\lfc}(K,E_K)$ sind $A$ und $E_K$ linear disjunkt über $E_A$ und mit Lemma \ref{Stapellemma} gelten schon beide gesuchten Aussagen.
    \end{proof}
    \newpage
    \begin{lemma}\label{Fortsetzungslemma}
    	Im Falle, dass das vorige Lemma auf die Inklusionen $$(A,E_A)\subseteq_{\lfc}(K,E_K)\text{ und }(\tilde{A},E_{\tilde{A}})\subseteq_{\lfc}(\tilde{K},E_{\tilde{K}})$$ sowie die Zwischenkörper $$E_A\subseteq B\subseteq E_K\text{ und }E_{\tilde{A}}\subseteq \tilde{B}\subseteq E_{\tilde{K}}$$ angewendet wird und dass gilt $$A\cong \tilde{A},B\cong \tilde{B},E_A\cong E_{\tilde{A}}$$ \--- wobei die ersten beiden Isomorphismen den dritten fortsetzen sollen\---, sind $(AB,B)$ und $(\tilde{A}\tilde{B},\tilde{B})$ schon isomorph als $\lfc$-Strukturen.
    \end{lemma}
    \begin{proof}
    	Mit Folgerung \ref{Isomorphismen linear disjunkt} erhält man den Isomorphismus $$A[B]\cong\tilde{A}[\tilde{B}],$$ der sich zu einem Isomorphismus der Quotientenkörper $AB$ und $\tilde{A}\tilde{B}$ fortsetzt und $A$ auf $\tilde{A}$ sowie $B$ auf $\tilde{B}$ und $E_A$ auf $E_{\tilde{A}}$ abbildet. Als Körperisomorphismus überträgt er auch Linearkombinationen, also auch die Interpretationen der $(l_n)_n$ und $(f_{i,n})_{i,n}$, weswegen er ein $\lfc$-Isomorphismus ist.
    \end{proof}
    
    \begin{lemma}\label{Unterstruktur regulär}
    	Sei die Inklusion $(A,E_A)\subseteq_{\lfc}(K,E_K)$ und $(K,E_K)$ ein Paar algebraisch abgeschlossener Körper gegeben, dann ist $E_A\subseteq A$ regulär.
    \end{lemma}
    \begin{proof}
    	Die Aussage folgt aus $A\ld_{E_A}E_K$ und der Körperinklusion $E_A\subseteq\overline{E_A}\subseteq E_K$ mit dem Lemma \ref{Stapellemma}.
    \end{proof}
    
    \begin{lemma}\label{alg Abschl}
    	Unter denselben Bedingungen wie im vorigen Lemma ist $(\overline{A},\overline{E_A})$ Zwischenstruktur.
    \end{lemma}
    \begin{proof}
    	Laut Lemma \ref{E-Erw} ist $(A\overline{E_A},\overline{E_A})$ Zwischenstruktur und damit insbesondere $$A\ld_{E_A}\overline{E_A},A\overline{E_A}\ld_{\overline{E_A}}E_K.$$ Klarerweise ist $$(A,E_A)\subseteq_{\lfc}(\overline{A\overline{E_A}},\overline{E_A})=(\overline{A},\overline{E_A}),$$ weil $A\overline{E_A}$ in der Bedingung $A\ld_{E_A}\overline{E_A}$ gar nicht vorkommt und die Erweiterung deshalb nichts ändert.\newpage
    	Lemma \ref{Rechenregeln} (6.) ergibt wegen der Regularität der Erweiterung $\overline{E_A}\subseteq E_K$ (die gemäß Lemma \ref{Unterstruktur regulär} vorliegt) die Konstellation $$\overline{A}=\overline{A\overline{E_A}}\ld_{\overline{E_A}}E_K,$$ was $(\overline{A},\overline{E_A})\subseteq_{\lfc}(K,E_K)$ beweist.
    \end{proof}
    
    \begin{theorem}\label{QE}
    	Die Theorie $\operatorname{ACP}^{\lfc}$ hat Quantorenelimination und ist vollständig, wenn man eine Charakteristik vorgibt.
    \end{theorem}
    \begin{proof}
    	Gegeben sei eine beliebige unendliche Kardinalzahl $\kappa$.
    	Zeige die Aussage mit dem Back\&Forth-System der Isomorphismen zwischen maximal $\kappa$ großen Unterstrukturen von $\kappa^+$-saturierten Modellen $(K,E_K),(L,E_L)$:
    	Dieses ist nichtleer, denn wenn $\mathbb{P}$ der Primkörper der Charakteristik ist, ist $(\mathbb{P},\mathbb{P})$ Unterstruktur von allen Modellen (wegen Gleichheit des Paares ist lineare Disjunktheit klar), bilde das als Unterstruktur von $K$ auf sich selbst als Unterstruktur von $L$ ab.
    	Sei $(M,E_M)\rightarrow(N,E_N)$ im B\&F-System. Die Erweiterungen $K\supseteq E_K$ und $L\supseteq E_L$ haben Transzendenzgrad $\infty$.\\
    	Dies kann man zum Beispiel feststellen, indem die Erweiterung offenkundig transzendent ist, und man dann jeweils den partiellen Typ über $\emptyset$ betrachtet, der die algebraische Unabhängigkeit von $n$ Elementen über $E_K$ bzw. $E_L$ beschreibt. Dieser hat folgende Gestalt:
    	$$\{\forall \overline{e}\in E\setminus\{0\}(f(\overline{e},\overline{x})\neq0)\mid 0\neq f\in\mathbb{P}[T_1,T_2,\dots,\overline{x}]\}.$$
    	Er ist endlich erfüllbar, da für $m$ größer als der größte Polynomgrad im endlichen Teilfragment und $x$ transzendent über $E_K$ bzw. $E_L$ die Elemente $x,x^m,x^{m^2},\dots$ algebraisch unabhängig über Polynomen von Grad kleiner $m$ sind.\\
    	Ohne Einschränkungen seien $(M,E_M)$ und $(N,E_N)$ jeweils algebraisch abgeschlossene Paare. Das kann man annehmen, denn die Lemmata \ref{E-Erw} und \ref{Fortsetzungslemma} besagen, dass es einen Isomorphismus zwischen den Zwischenstrukturen $$(M\overline{E_M},\overline{E_M})\text{ und }(N\overline{E_N},\overline{E_N}),$$ gibt der sich mit \ref{alg Abschl} auf einen Isomorphismus $$(\overline{M},\overline{E_M})\cong(\overline{N},\overline{E_N})$$ fortsetzt.\\
    	Sei jetzt $a$ in $K$. Wenn $a$ in $M$ liegt, dann kann man die Abbildung auf triviale Weise auf $a$ fortsetzen.\newpage
    	Wenn ansonsten $a$ algebraisch über $E_KM$ ist, ist $a$ in $\acl(E_KM)$, also existiert $X\subseteq E_K$ endlich mit $a$ in $\acl(MX)$. Ohne Einschränkungen sei $X$ jetzt schon ein Oberkörper von $E_M$, wichtig ist nur der endliche Transzendenzgrad über $E_M$. Wegen der Saturation hat die Erweiterung $E_N\subseteq E_L$ Transzendenzgrad $\infty$ und für einen beliebigen algebraisch abgeschlossenen Zwischenkörper $E_N\subseteq Y\subseteq E_L$ von gleichem Transzendenzgrad wie der von $X$ über $E_M$ kann $E_M\cong E_N$ fortgesetzt werden zu einem Isomorphismus $X\cong Y$. Diesen kann man wie oben mit Lemma \ref{Fortsetzungslemma} und Lemma \ref{alg Abschl} fortsetzen zu einem $\lfc$-Isomorphismus zwischen den Zwischenstrukturen $(\overline{MX},X)$ und $(\overline{NY},Y)$, wobei die erste $a$ enthält.\\
    	Wenn $a$ transzendent über $E_KM$ ist, gibt es ein über $E_LN$ transzendentes $b$ in $L$, denn der entsprechende Typ ist konsistent, wenn $\abs{L}$ größer ist als $\abs{E_L}$. So etwas lässt sich aber in einer elementaren Oberstruktur erreichen (für die endliche Konsistenz ist die genaue Kardinalität des Transzendenzgrads von $E_L\subseteq L$ egal) und wenn der Typ dort konsistent ist, dann auch unten.\\
    	Die Elemente $a$ und $b$ erzeugen einen Isomorphismus $C:=\overline{M(a)}\overset{\phi}{\cong}\overline{N(b)}$. Setze $$E:=\overline{M(a)}\cap E_K\cap\phi^{-1}(E_L\cap\overline{N(b)}),$$ dann gilt $(C,E)\cong_{\lfc}(\phi(C),\phi(E))$ und $b$ ist in $\phi(C)$.\\
    	Zu zeigen ist nun nur noch $$(M,E_M)\subseteq_{\lfc}(C,E)\subseteq_{\lfc}(K,E_K)\text{ und }(N,E_N)\subseteq_{\lfc}(\phi(C),\phi(E))\subseteq_{\lfc}(K,E_K).$$\\
    	Aus $(M,E_M)\subseteq_{\lfc}(K,E_K)$ folgt mit Lemma \ref{transz Erw} $$(M,E_M)\subseteq_{\lfc}(M(a),E_M)\subseteq_{\lfc}(K,E_K),$$ daraus mit Lemma \ref{E-Erw} und $E_M\subseteq E\subseteq E_K$ $$(M,E_M)\subseteq_{\lfc}(M(a)E,E)\subseteq_{\lfc}(K,E_K),$$ daraus folgt mit Lemma \ref{alg Abschl} und $E\subseteq\overline{M(a)}=C$ schließlich $$(M,E_M)\subseteq_{\lfc}(C,E)\subseteq_{\lfc}(K,E_K).$$
    	Der Beweis der Behauptung der Zwischenstruktureigenschaft für $(\phi(C),\phi(E))$ geht analog und weil $a$ aus $C$ kommt, haben wir die gesuchte Fortsetzung gefunden.
    \end{proof}
    
    \newpage
    
    \begin{definition}
    	Wenn es keine Rolle spielt, in welcher Sprache man gerade ist, schreibe einfach \textbf{ACP} für die Theorie.
    \end{definition}
    
    Bis jetzt haben wir zwar die Quantorenelimination erreicht, allerdings kann man Formeln modulo ACP noch weiter reduzieren. Insbesondere wäre eine \glqq{}Schachtelung\grqq{} von mehreren $f_{i,n}$ ineinander im Weiteren nicht leicht zu behandeln. Nach weiterer Reduktion der Formeln braucht man das aber auch nicht, da die Funktionen in den kleineren Körper abbilden. Der folgende Fakt lässt sich durch einige Rechnungen und Fallunterscheidungen beweisen, aus Gründen des Leseflusses werden diese aber hier weggelassen.
    
    \begin{fact}\label{Eliminierungsregeln}
    	In jedem Modell $(K,E_K)$ von ACP gelten die nachfolgenden Äquivalenzen für alle $i,n$. Hierbei seien die Variablen aus $K$ beliebig, ausgenommen die $e,e_1,\dots,e_n$, diese seien in $E_K$ beliebig. Im Falle der partiellen Funktionen seien nur Situationen betrachtet, in denen die betrachteten Argumente im partiellen Definitionsbereich liegen.\\
    	Es ist möglich, \glqq{}$^{-1}$\grqq{} in gewissen Situationen zu eliminieren:
    	$$l_n(x_1y_1^{-1},\dots,x_ny_n^{-1})\text{ gilt genau dann, wenn }l_n\left(x_1\prod\limits_{i=1\dots n,i\neq 1}y_i,\dots,x_n\prod\limits_{i=1\dots n,i\neq n}y_i\right)$$
    	$$\text{und }f_{i,n}(x_0y_0^{-1},x_1y_1^{-1},\dots,x_ny_n^{-1})= f_{i,n}\left(x_0\prod\limits_{i=0\dots n,i\neq 0}y_i,\dots,x_n\prod\limits_{i=0\dots n,i\neq n}y_i\right)$$
    	Auch lassen sich Vorfaktoren aus $E_K$ nach außen bringen/eliminieren:
    	$$l_n(e_1x_1,\dots,e_nx_n)\text{ gilt genau dann, wenn }l_n(x_1,\dots,x_n)$$
    	$$\text{und }f_{i,n}(ey,e_1x_1,\dots,e_nx_n)=e{e_i}^{-1}f_{i,n}(y,x_1,\dots,x_n)$$
    	Beim \glqq{}$+$\grqq{} ist die Situation schon komplexer. In der ersten Koordinate ist das noch recht leicht:
    	\begin{align*}
    	&\neg l_n(a+b,x_2,\dots,x_n)\text{ gilt genau dann, wenn }\\&\neg l_{n-1}(x_2,\dots,x_n)\text{ oder }(l_{n-1}(x_2,\dots,x_n)\text{ und }\\&(((l_n(b,x_2,\dots,x_n)\text{ und }\neg l_{n+1}(a,b,x_2,\dots,x_n))\text{ oder }\\&(\neg l_n(b,x_2,\dots,x_n)\text{ und }l_n(a,x_2,\dots,x_n))))),
    	\end{align*}
    	$$\text{außerdem ist }f_{i,n}(a+b,x_1,\dots,x_n)=f_{i,n}(a,x_1,\dots,x_n)+f_{i,n}(b,x_1,\dots,x_n)$$\newpage
    	Bei $l_n$ funktioniert diese Äquivalenz analog auch in den anderen Koordinaten, um \glqq{}$+$\grqq{} aus einem $f_{i,n}$ herauszuziehen, bedarf es aber einer größeren Fallunterscheidung:
    	\begin{align*}
    	&f_{i,n}(z,x_1,\dots,x_{i-1},a+b,x_{i+1},\dots,x_n)=\\
    	&\left\{\begin{array}{ll}
    	f_{i,n+1}(z,x_1,\dots,x_{i-1},a,b,x_{i+1},\dots,x_n)& a,b,\overline{x}\text{ unabhängig}\\
    	f_{i,n}(z,x_1,\dots,x_{i-1},a,x_{i+1},\dots,x_n)&\text{wenn nicht und }b\in\langle\overline{x}\rangle_{E_K}\\
    	f_{i,n-1}(z,x_1,\dots,x_{i-1},x_{i+1},\dots,x_n)&\text{wenn nicht und }z\in\langle\overline{x}\rangle_{E_K}\\
    	\frac{f_{i,n}(z,x_1,\dots,x_{i-1},b,x_{i+1},\dots,x_n)}{1+f_{i,n}(a,x_1,\dots,x_{i-1},b,x_{i+1},\dots,x_n)}&\text{ansonsten}
    	\end{array}\right.,
    	\end{align*}\begin{align*}
    	&f_{j,n}(z,x_1,\dots,x_{i-1},a+b,x_{i+1},\dots,x_n)=\\
    	&\left\{\begin{array}{ll}
    	f_{j+1_{j>i},n+1}(z,x_1,\dots,x_{i-1},a,b,x_{i+1},\dots,x_n)&a,b,\overline{x}\text{ unabhängig}\\\\
    	f_{j,n}(z,x_1,\dots,x_{i-1},a,x_{i+1},\dots,x_n)\\
    	-f_{i,n}(z,x_1,\dots,x_{i-1},a,x_{i+1},\dots,x_n)&\text{wenn nicht und }b\in\langle\overline{x}\rangle_{E_K}\\\cdot f_{j-1_{j>i},n-1}(b,x_1,\dots,x_{i-1},x_{i+1},\dots,x_n)\\\\
    	f_{j,n}(z,x_1,\dots,x_{i-1},b,x_{i+1},\dots,x_n)\\
    	-f_{i,n}(z,x_1,\dots,x_{i-1},a+b,x_{i+1},\dots,x_n)&\text{ansonsten}\\\cdot f_{j,n}(a,x_1,\dots,x_{i-1},b,x_{i+1},\dots,x_n)
    	\end{array}\right.\\
    	&(\text{hierbei sei }j\neq i)
    	\end{align*}
    \end{fact}
    
    Es steht noch eine Erklärung der genauen Verwendung dieses Faktes aus:\\
    Anschaulich betrachtet, kann man damit Formeln, die im Inneren eines $l_n$ oder eines $f_{i,n}$ ein $+,-,^{-1}$ oder ein weiteres $f_{i',n'}$ enthalten, in boolesche Kombinationen von Formeln umwandeln, die diese Art der Schachtelung nicht enthalten.\\
    Also erhält man quantorenfreie Formeln, in denen $+,-,^{-1}$ und alle $f_{i,n}$ wenn überhaupt, dann nur außerhalb aller $l_n$ und $f_{i,n}$ vorkommen. Da man aber quantorenfreie\linebreak$\lingua_\text{Ring}\cup\{^{-1}\}$-Formeln in quantorenfreie $\lingua_\text{Ring}$-Formeln umwandeln kann, ist es möglich, das \glqq{}$^{-1}$\grqq{} vollständig zu eliminieren; man erhält eine leicht handhabbare Form der Formeln.\\
    An dieser Stelle sei darauf hingewiesen, dass hier nicht mehr ausreicht, dass die Definitionsbereiche der $f_{i,n}$ quantorenfrei definierbar sind. Vielmehr wird benötigt, dass sie selbst in einer Form wie in der Folgerung definierbar sind. Das ist aber der Fall, wie man sich weiter vorne versichern kann.
    \newpage
    
    \begin{corollary}\label{Formel-Vereinfachung}
    	Man kann in jedem Modell $(K,E_K)$ jede Formel mit Parametern aus $X\subseteq K$ modulo ACP schreiben als boolesche Kombination aus Formeln der Form $$\glqq{}l_n(\text{Monome mit Koeffizienten von Produkten aus }X)\grqq{}$$ und
    	\begin{align*}
    	\text{\glqq{}}(\text{Polynom in }\mathbb{P}[X])(&f_{i_1,n_1}(\text{Monome in Produkten aus }X),\\\dots,&f_{i_m,n_m}(\text{Monome in Produkten aus }X))=0\grqq{},
    	\end{align*}
        wobei $\mathbb{P}$ den Primkörper der entsprechenden Charakteristik bezeichne.
    \end{corollary}
    \begin{corollary}
    	In jedem Modell $(K,E_K)$ ist jede definierbare Menge $X\subseteq E_K^n$ schon beschreibbar als $X=E_K^n\cap Z$, wobei $Z$ definierbar in der Ringsprache über denselben Parametern ist. Das liegt daran, dass die $(f_{i,n}),(l_n)$ nicht trivial sind, wenn man Werte aus $E_K$ einsetzt.\\ Dementsprechend ist $E(x)$ eine streng minimale Formel und für $e_1,\dots,e_n$ in $E_K$ ist $\RM(\overline{e}/X)=\dim(\overline{e}/X)$ für alle $X\subseteq K$.
    \end{corollary}
    \begin{corollary}
    	Die Theorie $\operatorname{ACP}^{\lf}$ hat für fixierte Charakteristik ebenfalls Quantorenelimination, da nach Folgerung \ref{Formel-Vereinfachung} die Operation \glqq{}$^{-1}$\grqq{} eliminiert werden kann. Da in Lemma \ref{Symbolik} gezeigt wurde, dass die $(f_{i,n})$ sowohl existenziell als auch universell definierbar sind, ist jede Formel modulo $\operatorname{ACP}^{\lld}$ immerhin universell und $\operatorname{ACP}^{\lld}$ ist modellvollständig.
    \end{corollary}
    
    \begin{remark}
    	Man kann sich leicht überlegen, dass ACP$\cup\{\glqq{}\text{Charakteristik}=p\grqq{}\}$ das Primmodell $(\overline{\mathbb{P}(e)},\overline{\mathbb{P}})$ hat für $\mathbb{P}$ als Primkörper und ein beliebiges $e$ transzendent über $\mathbb{P}$. Diese Struktur findet man in einem beliebigen Modell $(K,E_K)$ wieder, indem man ein $z$ in $K\setminus E_K$ wählt und das Paar $(\overline{\mathbb{P}(z)},\overline{\mathbb{P}})$ betrachtet. Klarerweise ist das isomorph zu $(\overline{\mathbb{P}(e)},\overline{\mathbb{P}})$ und es ist in allen betrachteten Sprachen eine Substruktur von $(K,E_K)$, weil $\overline{\mathbb{P}(z)}$ und $E_K$ linear disjunkt über $\overline{\mathbb{P}}$ sind. Dies folgt nämlich mit Lemma \ref{Rechenregeln} (5.), da $z$ frei über $E_K$ ist sowie $\mathbb{P}$ und $E_K$ linear disjunkt über $\mathbb{P}$ sind.
    \end{remark}
    \newpage
    Im folgenden Lemma werden einige Polynomumformungen vorgenommen. Weil diese recht technisch sind, sollen sie zuerst an einem Beispiel vorgeführt werden: Seien $a,b,c$ drei über $\setQ$ algebraisch unabhängige Zahlen aus $\setC$, betrachte die ACP-Modelle $$(\overline{\setQ(a)},\overline{\setQ})\preceq(\setC,\overline{\setQ(b,c)})$$ Sei $d$ eine Nullstelle von $$f(X):=X^2-\frac{ac}{b^{-1}+\sqrt{c}}.$$ Man kann dann $f$ durch Multiplikation mit Elementen aus $\setQ(a)$ zu dem Polynom $$b^{-1}X^2+\sqrt{c}X^2-a\sqrt{c}^2$$ umformen, das die Eigenschaft hat, irreduzibel im Polynomring $\setQ(a)[b^{-1},\sqrt{c},X]$ zu sein und $d$ als Nullstelle zu haben. Eine weitere Umformung ergibt das Polynom $$(bc)^{-1}X^2+\sqrt{c}^{-1}X^2-a$$ aus $\setQ(a)[(bc)^{-1},\sqrt{c}^{-1},X]$, das das einzige Polynom von Grad 2 ist, sodass $\xi_1,\xi_2$ aus $\overline{\setQ(b,c)}$ existieren und das Polynom irreduzibel als Element von $\overline{\setQ(a)}[\xi_1,\xi_2,X]$ ist, einen konstanten Term von festgelegtem Wert $a$ ungleich Null hat sowie außerdem $d$ als Nullstelle hat.\\
    Mit dieser Eigenschaft sind seine Koeffizienten als Polynom in $X$ in $\lfc$ definierbar über $\setQ(a)\cup\{d\}$, denn nach TODO ist Irreduzibilität eine elementare Eigenschaft des zugrundeliegenden Körpers. Aus den Koeffizienten sind aber die verwendeten Potenzen von $\xi_1,\xi_2$ über $\overline{\setQ(a)}$ definierbar wegen linearer Disjunktheit von $\overline{\setQ(a)}$ und $\overline{\setQ(b,c)}$, also sind $\xi_1$ und $\xi_2$ algebraisch über $\overline{\setQ(a)}\cup\{d\}$. Da natürlich auch andersherum $d$ algebraisch über $\overline{\setQ(a)}\cup\{\xi_1,\xi_2\}$ ist, gilt für den Morleyrang $$\RM(d/\overline{\setQ(a)})=\RM(\xi_1,\xi_2/\overline{\setQ(a)})=\dim(\xi_1,\xi_2/\overline{\setQ(a)})=\dim(b,c/\overline{\setQ(a)})=2.$$
    Bei diesem Wert handelt es sich genau um die Mächtigkeit des kleinsten Tupels $\overline{e}$ aus $\overline{\setQ(b,c)}$, sodass $d$ algebraisch über $\overline{\setQ(a)},\overline{e}$ ist.\\\\
    Mit diesen Techniken ist es jetzt möglich, die Stabilität unserer Theorie zu beweisen.
    \newpage
    \begin{theorem}
    	Die Theorie ACP ist $\omega$-stabil.
    \end{theorem}
    \begin{proof}
    	Betrachte die Menge der Typen in einem Modell $(K,E_K)$ über einer vorgegebenen Menge $S\subseteq K$ und wähle als Sprache $\lfc$, zunächst sei $S$ keine Teilmenge von $E_K$. Ohne Einschränkungen sei das Modell schon $\abs{S}^+$-saturiert und $S$ sogar $\lfc$-algebraisch abgeschlossen. Insbesondere ist $S$ Träger einer elementaren Unterstruktur, denn $\operatorname{ACP}^{\lfc}$ ist modellvollständig und $S$ ist Modell, da es keine Teilmenge von $E_K$ ist. Aus dem B\&F-System in Satz \ref{QE} geht hervor, dass es die folgenden Typen über $S$ gibt:
    	\begin{itemize}
    		\item Den Typ eines Elementes in $S$
    		\item Die Typen eines Elementes $a$ in $\overline{SE_K}\setminus S$ (bestimmt durch die Größe eines minimalen Tupels $\overline{e}$ aus $E_K$, sodass $a$ in $\overline{S\overline{e}}$ ist, sowie durch den Isomorphietyp des Minimalpoynoms von $a$ über $S\overline{e}$)
    		\item Den Typ eines Elementes in $K\setminus\overline{SE_K}$
    	\end{itemize}
        Der erste Typ hat klarerweise Morleyrang 0. Im Fall des zweiten Typs für ein Element $a$ erhalten wir mit einem analogen Vorgehen zu oben eine Umformung des Minimalpolynoms von $a$ über $SE_K$. Damit ergibt sich der Morleyrang als die Mächtigkeit eines minimalen Tupels aus $E_K$, sodass $a$ algebraisch über $SE_K$ ist.\\
        Der dritte Typ kann keinen Morleyrang größer als $\omega$ haben, denn dann müsste es nach Fakt \ref{Anfangsstück} einen Typen mit Morleyrang $\omega$ geben, die anderen Arten von Typen haben aber endlichen Rang. Sei $(a_n)$ eine Folge von Elementen, sodass $$n=\RM(a_n)\text{ für alle }n\text{ in }\setN.$$ Man könnte zum Beispiel über $S$ algebraisch unabhängige $x_1,x_2,\dots$ aus $E_K\setminus S$ nehmen, die wegen Saturiertheit existieren müssen; ebenso kann man ein $z$ aus $S\setminus E_K$ wählen, das nach unseren Annahmen an $S$ existiert und das transzendent über $E_K$ ist. Die Elemente $1,z,z^2,\dots$ aus $S$ sind dann linear unabhängig über $E_K$ und wir setzen $a_n:=x_1+zx_2+\dots+z^nx_n$. Dann ist $a_n$ interdefinierbar mit $x_1,\dots,x_n$ über $z$, also hat es Morleyrang $n$ wegen der algebraischen Unabhängigkeit der $x_i$.\newpage
        Im Stoneraum gilt \--- bezüglich der in \cite{Lukas} beschriebenen Topologie \--- $$\lim\limits_{n\rightarrow\infty}\tp(a_n)=p$$ für den Typen $p$ der dritten Art: Denn $a_n$ ist transzendent über $SX$ für alle $E_S\subset X$ von Transzendenzgrad $\leq n-1$. Also sind für jede Umgebung, die das Nichterfüllen einer bestimmten Art von Polynom über $SE_K$ beschreibt, fast alle $a_n$ enthalten. Aber diese Umgebungen bestimmen den Typen $p$ eindeutig, also sind für jedes $\phi$ in $p$ fast alle $\tp(a_n)$ in der von $\phi$ erzeugten Umgebung $\fU_\phi$ im Stoneraum.\\
        Jedes $\phi$ in $\fF_1(\lingua_S)$ mit endlichem Rang ist dann nur in endlich vielen $\tp(a_n)$ enthalten (nämlich maximal $\operatorname{RM}(\phi)$ vielen), also ist $\RM(p)\geq\omega$, damit herrscht Gleichheit.\\
        Da alle Typen definierten Morleyrang haben, ist die Theorie $\omega$-stabil  nach Fakt \ref{Stabilität Morleyrang}.\\
        Ein Alternativbeweis wäre auch, dass es nicht mehr Typen über $S$ gibt als $\abs{S}+\aleph_0$, auch das beweist die $\omega$-Stabilität. Und auch mit algebraischen Methoden ist ein Beweis möglich, siehe dazu den Anhang.
    \end{proof}
    %!TEX root = DieLoesungAllerMilleniumsprobleme.tex
\chapter{Dichte Paare o-minimaler Strukturen}
\section{Allgemeine Betrachtungen und Anforderungen an die Theorien}

Viele Techniken und Gedanken aus dem vorigen Kapitel werden jetzt auf o-minimale Theorien übertragen: Im Folgenden halten wir einfach eine vollständige o-minimale Theorie T in der Sprache $\lingua$ fest und betrachten die Theorie $\tq$ in der Sprache $\lingua_P:=\lingua\cup\{P(x)\}$, sodass die Modelle von $\tq$ Modelle von T sind und in jedem Modell $\fM$ die Menge $P(M)$ ebenfalls Modell von T ist. Schreibe so ein Paar dann auch als $(B,A)$ mit $A=P(B)$.\\
Wir setzen voraus, dass $T$ eine durch $+$ dicht und linear angeordnete abelsche Gruppe mit einem positiven Element $1$ beschreibt, sodass Definable Choice gilt. Dann sind die Skolemfunktionen definierbar und \OE\ ist $\lingua$ schon so eine definitorische Erweiterung, dass $T$ Quantorenelimination hat und universell axiomatisierbar ist (wobei bei einzelnen Theorien die Frage interessant wäre, welche Skolemfunktionen man dafür überhaupt hinzufügen muss). Außerdem seien alle Modelle genug saturiert, dass die üblichen Rechenregeln für die Dimension gelten.\\
Aus T universell mit Quantorenelimination folgt, dass Unterstrukturen von Modellen von T schon elementare Unterstrukturen sind. Also ist für jede Teilmenge $S$ eines Modells $\dcl(S)$ schon eine elementare Substruktur; zur Vereinfachung bezeichne in Zukunft $AB:=\dcl(A\cup B)$ für zwei Teilmengen $A,B$ eines Modells.\\
$P$ beschreibt also eine elementare Unterstruktur, mit $\td$ wird nun die Theorie beschrieben, die ausdrückt, dass $P$ eine dichte echte Unterstruktur ist (diese zwei Sachen oder deren Gegenteil müssen auf jeden Fall von der Theorie beschrieben werden, wenn sie vollständig sein soll). Klar ist dann, dass Unterstrukturen von $\td$ automatisch Modelle von $\tq$ sind.\\
Dem Verlauf in \cite{VanDenDries} folgend, wird zunächst die Vollständigkeit und eine Art von Quantorenelimination für $\td$ gezeigt, wofür aber eine genauere Betrachtung von sogenannten kleinen Mengen vonnöten ist.

\section{Kleine Mengen}
\begin{definition}
	Sei $(B,A)\models\td$, dann ist eine $\lingua_P$-definierbare Menge $S\subseteq B$ \textbf{klein}, wenn eine $\lingua$-definierbare Funktion $f:B^n\rightarrow B$ existiert mit $S\subseteq f(A^n)$.
\end{definition}

Das Ziel ist jetzt, zu zeigen, dass definierbare Intervalle nicht klein sind.\newpage
Im folgenden Lemma meint $(\cdot,+)$ nicht unbedingt die Operationen in $\lingua$, sondern nur irgendwelche definierbaren Verknüpfungen.

\begin{lemma}\label{Hilfsaussage Kleinheit}
	Seien $A\prec B\models\operatorname{T},\ f:B^{n+1}\rightarrow B\ A\text{-definierbar},\ b\in B\setminus A,\ {\beta,\gamma\in A}$ mit $\beta<b<\gamma$ und einer angeordneten $A$-definierbaren Körperstruktur $(\cdot,+)$ auf $(\beta,\gamma)=:I$. Dann existieren $a_0,\dots,a_n\in I_A$ mit $$a_nb^n+a_{n-1}b^{n-1}+\dots,a_0\in I\setminus f(A^n\times\{b\}).$$
\end{lemma}
\begin{proof}
	Wenn die Aussage nicht gilt, dann gilt mit $p(x,y):=x_ny^n+x_{n-1}y^{n-1}+\dots,x_0$, dass für jedes $a\in (I_A)^{n+1}$ ein $\alpha\in A^n$ existiert mit $p(a,b)=f(\alpha,b)$. Es muss für festes $a\in I_A$ ein Intervall um $b$ in $I_A$ mit dieser Eigenschaft geben, denn sonst wäre $b\in\dcl(A)=A$.\\
	Sei jetzt $a$ nicht mehr fixiert, dann existiert mit Definable Choice eine definierbare Zuordnung $a\mapsto\alpha(a)$, sodass $p(a,\cdot)=f(\alpha(a),\cdot)$ auf einem Intervall gilt. Da jedes $a$ $n+1$ viele Einträge hat und jedes $\alpha(a)$ $n$ viele, müssen unendlich viele $a\in (I_A)^{n+1}$ existieren, die durch $\alpha$ auf das selbe Element abgebildet werden. Denn wenn das nicht so wäre, wäre ein generisches Element aus $(I_A)^{n+1}$ algebraisch über einem Element aus $A^n$, was der Generizität widerspricht. Da es unendlich viele Elemente gibt, sodass $\alpha$ auf ihnen konstant ist, gibt es schon eine Zelle von Dimension $>0$ mit der Eigenschaft und damit insbesondere eine Zelle $E$ von Dimension 1 (Als Teilmenge einer Zelle lässt sich immer eine von kleinerer Dimension finden). Nenne den konstanten Wert dann $\alpha^*$.\\
	Da also gilt: Für alle $a\in E$ existiert ein Intervall $J$ mit $p(a,\cdot)=f(\alpha^*,\cdot)$ auf $J$; existieren mit Definable Choice $\beta^*,\gamma^*:E\rightarrow I_A$, sodass $p(a,\cdot)=f(\alpha^*,\cdot)$ auf $(\beta^*(a),\gamma^*(a))$ gilt. \OE\ seien $\beta^*$ und $\gamma^*$ jetzt schon stetig auf $E$ und ein $e\in E$ beliebig· Dann existiert für $\varepsilon$ hinreichend klein eine $E$-Umgebung $U$ um $e$, sodass $$\beta^*<\frac{1}{2}(\beta^*(e)+\gamma^*(e))-\varepsilon,\ \frac{1}{2}(\beta^*(e)+\gamma^*(e))+\varepsilon<\gamma^*$$ auf $U$, also $$p(a,x)=f(\alpha^*,x)\text{ für alle }a\in U,x\in(\frac{1}{2}(\beta^*(e)+\gamma^*(e))-\varepsilon,\frac{1}{2}(\beta^*(e)+\gamma^*(e))+\varepsilon)$$ gilt. Es kann aber nicht $p(a-a',x)=p(a,x)-p(a',x)=f(\alpha^*,x)-f(\alpha^*,x)=0$ für $a,a'\in U$ verschieden und unendlich viele $x$ sein, weil ein Nichtnullpolynom nicht unendlich viele Nullstellen haben kann.
\end{proof}

\newpage

\begin{lemma}
	Sei $\fA$ eine o-minimale Erweiterung eines angeordneten Vektorraums über einem angeordneten Körper $F$ und $g:A^{p+1}\rightarrow A$ definierbar, außerdem existiere für unendlich viele $\lambda\in F$ ein $a_\lambda\in A^p$ mit $g(a_\lambda,x)=\lambda x$ für unendlich viele $x\in A$. Dann existiert ein Intervall $I$ in $A$, sodass auf $I$ eine $A$-definierbare Körperstruktur existiert, die mit $<$ kompatibel ist (was automatisch einen reell abgeschlossenen Körper impliziert).
\end{lemma}
\begin{proof}
	TODO: Geht irgendwie aus \cite{PeterStarch} hervor.
\end{proof}

\begin{lemma}
	Es sei $(A,B)\models\td,\ f:B^{n+1}\rightarrow B$ $A$-definierbar in $B$ und $b\in B\setminus A$. Dann enthält $f(A^n\times\{b\})$ kein Intervall um $b$.
\end{lemma}
\begin{proof}
	Nimm an, dass das Gegenteil gelte für das Intervall $J$ (\OE\ mit Randpunkten in $A$): Dann existiert insbesondere für jedes $q\in\setQ$ hinreichend nahe bei $1$ ein $a_q\in A^n$ mit $f(a_q,b)=qb$. Dann existiert wieder ein Intervall $I_q\subseteq J_A$ mit $f(a_q,x)=qx$ für alle $x\in I_q$. \OE\ ist dieses Intervall schon beschränkt und die Randpunkte seien $c_q<d_q$. Definiere dann $$r_q:=\frac{c_q+d_q}{2},s_q:=\frac{d_q-c_q}{2}\in A,$$ $$g:(u,v,x)\mapsto f(u,v+x)-f(u,v)\ \ \ \ u\in A^n,v,x\in A.$$
	Dann gilt für alle $x\in(-s_q,s_q)$ $$g(a_q,r_q,x)=f(a_q,r_q+x)-f(a_k,r_q)=q(r_q+x)-qr_q=qx.$$
	Also existiert nach dem letzten Lemma ein Intervall in $A$ mit einer $A$-definierbaren Körperstruktur als RCF. Durch Translation (benutze Dichtheit) nehme an, dass $b\in I_B$ liegt. Dann existiert nach Lemma \ref{Hilfsaussage Kleinheit} ein Element $c\in I_B\setminus f(A^n\times\{b\})$. \OE\ sei schon $\inf J,\sup J\in I$, sonst ersetze $J$ durch ein kleineres Intervall.\\
	Seien $d,e\in I$ mit $d<c<e$ und $\varphi$ die orientierungserhaltende, $A$-definierbare affine Abbildung in $I$ mit $\varphi(d)=\inf J,\varphi(e)=\sup J$. Dann ist $\varphi(c)\in J\setminus(\phi\circ f)(A^n\times\{b\})$ und da das Verketten mit einer $A$-definierbaren invertierbaren Abbildung nichts an der Aussage ändert, gibt es einen Widerspruch.
\end{proof}

\newpage

\begin{theorem}\label{Kleinheit}
	Wenn $(B,A)\models\td$, dann ist kein Intervall eine kleine Teilmenge.
\end{theorem}
\begin{proof}
	Sei $f:B^n\rightarrow B$ eine durch $\varphi(x,y,b)$ definierbare Abbildung mit $\varphi$ eine\linebreak$\lingua_A$-Formel und $b\in B^m$ für ein $m\in\setN$ definiert. Für $\dim(b/A)=0$ ist $f(A^n)\subseteq A$ klar, deswegen sei \OE\ $\dim(b/A)\geq1$. Definiere
	\begin{align*}
	g(x,z):=\left\{\begin{array}{ll}
	\text{das eindeutige }y\in B &\text{für alle z, für die }\varphi(x,y,z)\\
	\text{ mit }B\models\varphi(x,y,z) &\text{ bei festem }\text{ eine Funktion definiert}\\
	\ &\ \\
	0 &\text{sonst}
	\end{array}\right.,
	\end{align*}
	Dann ist $g$ in $B$ $A$-definierbar und $g(\cdot,b)=f$. Falls $\dim(b/A)>1$, füge genug Komponenten von $b$ zu $A$ hinzu, sodass $\dim(b/A)=1$. Das Hinzufügen ändert nichts, denn $Ab_i$ ist nach den Eingangsbemerkungen Modell von T und $Ab_i$ ist erst recht dicht in, aber nicht gleich $B$ (sonst hätte man die Dimension mit diesem Schritt schon zu sehr verkleinert).\\
	Finde also $b_i$, sodass $A$-definierbare $(h_j)$ existieren mit $b_j=h_j(b_i)$ für alle $j$. Wenn jetzt $J\subseteq f(A^n)=g(A^n,b)=g(A^n,h(b_i))$ für ein Intervall $J$, dann widerspricht das der Aussage des letzten Lemmas für die Funktion $(x,y)\mapsto g(x,h(y))$.
\end{proof}

\begin{definition}
	Schreibe ab jetzt $P(\overline{x}):=\bigwedge\limits_{i=1}^\abs{x}P(x_i)$.
\end{definition}

\begin{lemma}
	Wenn $(B,A)$ für ein unendliches $\kappa>\abs{\operatorname{T}}$ ein $\kappa$-saturiertes Modell von $\td$ ist, ist $\dim(B/A)\geq\kappa$.
\end{lemma}
\begin{proof}
	Sei $S$ eine Basis von $B/A$ mit $\abs{S}<\kappa$; zeige nun, dass es kein Erzeugendensystem sein kann. Das folgt aus der Saturation angewandt auf den partiellen Typen $$\{\forall\overline{y}\in P(x\neq t(\overline{y}))\mid t\ \lingua_E\text{-Term}\},$$ der endlich erfüllbar ist, weil die Negation jeder dieser Formeln \glqq{}$x$ ist in einer kleinen Menge\grqq{} impliziert. Wenn der Typ also nicht endlich erfüllbar wäre, würde eine endliche Vereinigung von kleinen Mengen ganz $B$ überdecken. Das kann aber nicht gelten, denn eine endliche Vereinigung von kleinen Mengen ist wieder klein (in der das bezeugenden Abbildung kann man das durch Erhöhen der Dimension des Urbildes und Fallunterscheidung über eine Koordinate beweisen).
\end{proof}
\newpage
\begin{corollary}\label{Finden transz Elte}
	Der Beweis zeigt sogar, dass in einem $\kappa$-saturiertem Modell $(B,A)\models\td$, gegeben Menge, $S,S',S''\subset B$ mit $\abs{S},\abs{S'},\abs{S''}<\kappa$, ein transzendentes Element $b$ über $SA$ gefunden werden kann mit $a<b$ für alle $a\in S'$ und $b<c$ für alle $c\in S''$, sofern dieser Ordnungstyp von $b$ überhaupt konsistent ist.
\end{corollary}

\section{Formelreduzierung in $\td$}
In diesem Abschnitt wird gezeigt, dass sich $\lingua_P$-Formeln modulo $\td$ sehr stark vereinfachen lassen. Indem dieses mit einem Back\&Forth-System gezeigt wird, erhält man zusätzlich eine sehr große Klasse von elementaren Abbildungen zwischen Modellen von $\td$.\\
In diesem Kontext wird wieder die $\acl$-Unabhängigkeit in einem Modell von T relevant, die im ersten Kapitel mit \glqq{}algebraisch disjunkt\grqq{} bezeichnet wurde. Man kann sich dafür folgende (teilweise schon bekannte) Fakten überlegen.

\begin{lemma}\label{Unabhängigkeitsregeln}
	Seien $A,B,C,D$ Mengen in irgendeinem Modell von $T$.
	\begin{enumerate}
		\item Wenn $A$ und $B$ unabhängig über $C$ sind, sind $B$ und $A$ unabhängig über $C$ und $A\cap B\subseteq\acl(C)$ (in fast allen betrachteten Fällen wird sowieso $A,B\supseteq C$ und $C=\acl(C)$ gelten).
		\item Wenn $A$ und $B$ unabhängig über $C$ sind und $S\subseteq B$, dann sind auch $A\cup S$ und $B$ unabhängig über $C\cup S$.
		\item Wenn $A$ und $B$ unabhängig über $C$ sind, $A\subseteq S\subseteq\acl(A),B\subseteq S'\subseteq\acl(B)$, dann sind $S$ und $S'$ unabhängig über $C$.
		\item Wenn $A$ und $B$ unabhängig über $C$ sind und $D$ (algebraisch) unabhängig über $AB$, dann sind $A\cup D$ und $B$ unabhängig über $C$.
		\item Wenn $(D,C)\preceq(B,A)\models\tq$, dann sind $A$ und $D$ unabhängig über $C$.
		\item Wenn $(D,C)\subseteq(B,A)\models\tq,\ S\subseteq A$ und $A$ und $D$ unabhängig über $C$ sind, dann sind $A$ und $DS$ unabhängig über $CS$, $\langle D\cup S\rangle_{\lingua_P}=(DS,CS)$ und $$(D,C)\subseteq(DS,CS)\subseteq(B,A).$$
		\item Wenn $(D,C)\subseteq(B,A)\models\tq$ und $S\subseteq B$ unabhängig über $DA$ ist, dann sind $A$ und $DS$ unabhängig über $C$, $\langle D\cup S\rangle_{\lingua_P}=(DS,C)$ und $$(D,C)\subseteq(DS,C)\subseteq(B,A).$$
	\end{enumerate}
\end{lemma}
\newpage
\begin{proof}
	1.-4. sind bekannt.
	\item[5.] Wenn $\overline{d}\in D$ algebraisch unabhängig über $C$ ist, aber nicht über $A$, dann existiert eine $\lingua_A$-Formel $\varphi(\overline{x},\overline{a})$, sodass \OE\ $d_1$ von $\varphi(x_1,d_2,d_3\dots,\overline{a})$ algebraisiert wird (\OE\ wird $d_1$ schon durch $\varphi$ definiert). Also erfüllt $\overline{d}$ die $\lingua_P$-Formel $$\exists \overline{y}\in P(\varphi(\overline{x},\overline{y})\land\forall z_2,z_3,\dots\exists! z_1(\varphi(\overline{z},\overline{y})))$$ in $(B,A)$, also auch in $(D,C)$. Es existiert also $\overline{c}\in C$ mit $$B\models\varphi(\overline{d},\overline{c})\land\forall z_2,z_3,\dots\exists! z_1(\varphi(\overline{z},\overline{c})),$$ was im Widerspruch zur Unabhängigkeit von $\overline{d}$ über $C$ steht.
	\item[6.] Dass $A$ und $DS$ unabhängig über $CS$ sind, ergibt sich in der Kombination von 2. und dann 3.\\
	Dass die Trägermenge von $\langle D\cup S\rangle_{\lingua_P}$ die Menge $DS$ ist, ergibt sich direkt per Definition als $DS=\dcl(D\cup S)=\langle D\cup S\rangle_\lingua$. Weil $A$ und $DS$ unabhängig über $CS$ sind, folgt $$P(\langle D\cup S\rangle_{\lingua_P})=DS\cap P(B)=DS\cap A=CS.$$
	\item[7] Es ergibt sich aus 4. und 3. dass $A$ und $DS$ unabhängig über $CS$ sind. Der Rest geht analog zu 6.
\end{proof}

Zu bemerken ist, dass ein Spezialfall von Unabhängigkeit viele nützliche Eigenschaften hat. Auf diesen wird später noch oft zurückgegriffen werden.
\begin{definition}
	Seien $(D,C)\subseteq(B,A)$ zwei Modelle von $\tq$. Dann heiße diese Erweiterung \textbf{frei}, wenn $D$ und $A$ unabhängig über $C$ sind.
\end{definition}

\begin{lemma}\label{Kodichte von A}
	Sei $(B,A)\models\td$. Dann ist $A$ auch kodicht in $B$.
\end{lemma}
\begin{proof}
	Zu zeigen ist, dass für alle $a,c\in B$ ein $b\in B\setminus A$ existiert mit $a<b<c$. Durch Translation und Inversion kann man annehmen, dass $a=0$. Wähle jetzt ein $d\in B\setminus A$ beliebig und $e\in A$ mit $d-c<e<d$. Dann ist $d-e$ nicht in $A$ (denn sonst wäre es $d$) und $0=e-e<d-e<d-(d-c)=c$.
\end{proof}

\newpage
Für die Konstruktion des gewünschten Back\&Forth-Systems sei $\kappa>\abs{T}$ eine beliebige, aber feste Kardinalzahl und $(B,A),(D,C)\models\td$ zwei $\kappa$-saturierte Modelle.
\begin{theorem}\label{BackForth}
	Sei $S$ die Menge aller partiellen Isomorphismen zwischen Unterstrukturen $(B',A')$ von $(B,A)$ und $(D',C')$ von $(D,C)$ der Mächtigkeit $<\kappa$, sodass die Erweiterungen frei sind. Dann bildet $S$ ein nichtleeres B\&F-System und $\td$ ist insbesondere vollständig.
\end{theorem}
\begin{proof}
	Das System ist nichtleer, denn es gibt ein Primmodell $\fM$ von $T$, weil $T$ vollständig ist und in jedem Modell $A$ alle Eigenschaften von $\fM_A:=\langle\emptyset\rangle_\lingua$ in $T$ beschrieben werden. Klarerweise ist $\abs{M}=\abs{T}<\kappa$. Der Isomorphismus $(\fM_A,\fM_A)\cong(\fM_C,\fM_C)$ liegt in $S$, denn Unabhängigkeit ist bei zwei gleichen Mengen offensichtlich.\\
	Sei jetzt $S\ni i:(B',A')\rightarrow(D',C')$ und $b\in B$. Wenn $b\in B'$ ist, ist nichts zu zeigen. Wenn $b\in A\setminus B'$, betrachte den partiellen Typ $$\{\alpha<x\mid i^{-1}(\alpha)<b\}\cup\{x<\beta\mid b<i^{-1}(\beta)\}\cup\{P(x)\}.$$
	Dieser ist konsistent, da $i$ ein Isomorphismus ist und $C$ dicht in $D$; mit Saturation existiert ein $d\in C\setminus D'$ mit diesem Ordnungstyp. $i$ setzt sich dann eindeutig zu einem Isomorphismus $i':(B'b,A'b)\rightarrow(D'd,C'd)$ mit $i(b)=d$ fort, der gegeben ist durch die Abbildung $t(b)\mapsto i(t)(d)$ für $t$ einen $\lingua_{B'}$-Term und $i(t)$ den durch $i$ geshifteten Term. Die Surjektivität dieser Abbildung ist klar, ebenso dass $i'(A'b)=C'd$. Wohldefiniertheit, Injektivität und Isomorphismuseigenschaft gelten, denn:\\
	$Rt_1(b)\dots t_n(b)$ gilt für $\lingua_{B'}$-Terme $t_1,\dots,t_n$ und eine Relation $R$ genau dann, wenn es ein $B'$-definierbares Intervall $I$ um $b$ mit dieser Eigenschaft gibt (denn sonst wäre $b$ definierbar über $B'$ und somit in $B'$). Schickt man $I\cap B'$ mit $i$ nach $J:=i(I\cap B')$, so gilt für alle Elemente $z\in J$, dass $Ri(t_1)(z)\dots i(t_n)(z)$, da $i$ ein Isomorphismus ist. Wäre jetzt nicht $Ri(t_1)(d)\dots i(t_n)(d)$, so gäbe es ein $D'$-definierbares Intervall $I'$ um $d$, sodass das nicht gilt; insbesondere ist $I'$ disjunkt zu $J$. Allerdings ist $$b\in I'\cap\operatorname{convex}(J)=I'\cap(i(\inf I),i(\sup I)),$$ also können $I'$ und $J$ nicht disjunkt sein. Es gilt also $Ri(t_1)(d)\dots i(t_n)(d)$.\\
	Die Rückrichtung geht analog.\\
	Zu zeigen ist nun, dass $B'b$ und $A$ frei über $A'b$ sowie $D'd$ und $C$ frei über $C'd$ sind, ebenso zu zeigen ist noch, dass $(B'b,A'b)\subseteq(B,A),(D'd,C'd)\subseteq(D,C)$. Das alles folgt aber aus Lemma \ref{Unabhängigkeitsregeln} (6.). Außerdem gilt $\abs{D'd}=\abs{B'b}=\abs{B'}+\abs{T}<\kappa$.\\
	Sei jetzt $b\in B'A\setminus(A\cup B')$. Dann gibt es $\overline{a}\in A$ mit $b\in B'\overline{a}$. Erweitere wie schon bekannt $i$, sodass $\overline{a}\in\operatorname{dom}(i)$; dann ist schon ganz $B'\overline{a}\subseteq\operatorname{dom}(i)$, also auch $b$.\\
	Abschließend sei $b\in B\setminus B'A$;  wie oben erfülle dann den mit $i$ geshifteten Ordnungstyp von $b$ über $B'$ mit einem Element $d\in D\setminus D'C$ (mit Folgerung \ref{Finden transz Elte} geht das). Wie oben kann $i$ dann auf einen Isomorphismus $(B'b,A')\rightarrow(D'd,C')$ fortgesetzt werden und nach Lemma \ref{Unabhängigkeitsregeln} (7.) erfüllen $(B'b,A'),(D'd,C')$ auch die hinreichenden Eigenschaften.
\end{proof}

Dieses B\&F-System beweist die Formelreduzierung in $\td$.
\begin{theorem}
	Jede $\lingua_P$-Formel ist modulo $\td$ äquivalent zu einer booleschen Kombination von Formeln der Gestalt
	$$\exists\overline{y}\in P(\phi(\overline{x},\overline{y}))$$
	für $\phi$ eine $\lingua$-Formel. Nenne eine solche boolesche Kombination eine \textbf{gute Formel} und eine Formel der Gestalt wie beschrieben eine \textbf{gute Formel in Reinform}.
\end{theorem}
\begin{proof}
	\underline{Hilfsaussage:}\\
	Es reicht zu zeigen, dass für alle Modelle $(B,A),(D,C)\models\td$ und für alle $b\in B^n,d\in D^n$ gilt: Wenn $b$ und $d$ dieselben guten Formeln erfüllen, sind ihre Typen in $(B,A)$ und $(D,C)$ dieselben.\\
	Dass dies ausreicht, erkennt man mit dem Ziegler'schen Trennungslemma: Sei $\psi\in\fF_n(\lingua_P)$ nicht äquivalent zu einer guten Formel und nenne die Menge aller guten Formeln in $n$ freien Koordinaten $K$. Dann ist $K$ abgeschlossen unter $\land,\lor$ und enthält $\top,\bot$. Wenn $\psi$ nicht äquivalent zu einer Formel aus $K$ ist, sind $\td\cup\{\psi\}$ und $\td\cup\{\neg\psi\}$ nicht durch $K$ trennbar, also existieren $(B,A),(D,C)\models\td,b\in B^n,d\in D^n$, sodass $(B,A)\models\psi(b)$ und $(D,C)\models\neg\psi(d)$, aber $(B,A)\models\chi(b)$ genau dann, wenn $(D,C)\models\chi(d)$ für alle $\chi\in K$. Dann erfüllen $b$ und $d$ dieselben guten Formeln, aber haben nicht denselben Typ - ein Widerspruch!\\
	\begin{proof}[Beweis der Hilfsaussage]
		Seien $b,d$ wie verlangt und $(B,A),(D,C)$ schon \OE\linebreak $\abs{T}^+$-saturiert (das ändert nichts an Typen und dem Erfüllen von guten Formeln). Sei $a\in A^m$ für ein hinreichend großes $m$, mit der Eigenschaft dass $\dim(b/a)\leq\dim(b/A)$ (es folgt dann Gleichheit, da über einer kleineren Menge nicht mehr interdefinierbar werden kann). Für $A':=\dcl(a),B':=\dcl(a,b)$ gilt dann, dass $A$ und $B'$ unabhängig über $A'$ sind. Es sind nämlich per Definition von $a$ die Mengen $A$ und $b$ unabhängig über $a$ (eben wegen $\dim(b/a)=\dim(b/A)$), mit Lemma \ref{Unabhängigkeitsregeln} (2.) sind dann auch $A$ und $b\cup(A')$ unabhängig über $A'$ und mit 3. sind $A$ und $B'=\dcl(b\cup A')$ unabhängig über $A'$. Außerdem sind $A'$ und $B'$ maximal $\abs{T}$ groß.\newpage
		Wenn man den partiellen $\lingua_P$-Typ $\tp_\lingua(a/b)\cup\{P(\overline{x})\}$ betrachtet, bleibt er konsistent unter der Ersetzung $b\rightarrowtail d$ in den Formeln. Seien nämlich $\psi_1(\overline{x},b),\dots,\psi_n(\overline{x},b)\in\tp_\lingua(a/b)$, dann ist $$\exists\overline{x}\in P(\bigwedge\limits_{i=1}^n\psi_i(\overline{x},\overline{y}))$$ eine gute Formel, die von $b$ und daher auch von $d$ erfüllt wird. Also ist der ersetzte partielle Typ endlich konsistent, wegen Saturation habe er den Erfüller $c\in C$ und es gilt $\tp_\lingua(a,b)=\tp_\lingua(c,d)$. Wegen der Typengleichheit folgt insbesondere $\dim(b/a)=\dim(d/c)$; es bleibt noch zu zeigen, dass $\dim(b/A)=\dim(d/C)$, damit dann gilt $\dim(d/C)=\dim(b/A)=\dim(b/a)=\dim(d/c)$ und wie oben $C$ und $D':=\dcl(c,d)$ frei über $C':=\dcl(c)$ sind. Die Gleichheit $\dim(b/A)=\dim(d/C)$ gilt aber, da für jede $\lingua$-Formel $\psi$ und $j_1,\dots,j_n\in\setN$ die Formel zu $$\glqq{}\text{es existiert }\overline{y}\in P\text{, sodass }\psi(\overline{x},\overline{y})\ x_i\text{ über }x_{j_1},\dots,x_{j_m}\text{ definiert}\grqq{}$$ eine gute Formel ist, die also genau dann von $b$ erfüllt wird, wenn sie von $d$ erfüllt wird.\\
		Da $(a,b)$ und $(c,d)$ den gleichen $\lingua$-Typ haben, gibt es einen partiellen Isomorphismus $i$ von $B'=\dcl(a,b)$ nach $D'=\dcl(c,d)$ mit $i((a,b))=(c,d)$, die Einschränkung auf $A'=\dcl(a)$ bildet einen Isomorphismus nach $C'=\dcl(c)$. Also ist $i$ partieller Isomorphismus $(B,A)\rightarrow(D,C)$, damit im B\&F-System, also elementare Abbildung, weswegen $b$ und $d$ denselben $\lingua_P$-Typen haben.
	\end{proof}
\end{proof}

\begin{corollary}\label{Definierbarkeit aus A}
	Für ein dichtes Paar $(B,A)$ und $S\subseteq B^n$ eine $A_0$-definierbare Menge in $\lingua_P$ (wobei $A_0\subseteq A$) ist $S\cap A^n$ eine $A_0$-definierbare Menge in $\lingua$.
\end{corollary}
\begin{proof}
	Nach der Formelreduzierung sei $S$ \OE\ durch eine gute Formel definiert. Da die Definierbarkeit abgeschlossen unter booleschen Kombinationen ist, reicht es, eine Formel in Reinform zu betrachten.\newpage
	Da aber für jede $\lingua_{A_0}$-Formel $\varphi(x,y,a')$ und jedes $a\in A^n$ die Aussagen $$\glqq{}\text{Es existiert ein }y\in A^m\text{ mit }(B,A)\models\varphi(a,y,a')\grqq{},$$ $$\glqq{}\text{Es existiert ein }y\in A^m\text{ mit }B\models\varphi(a,y,a')\grqq{},$$ $$\glqq{}\text{Es existiert ein }y\in A^m\text{ mit }A\models\varphi(a,y,a')\grqq{}$$ äquivalent sind wegen $\varphi$ als $\lingua$-Formel und $A\prec B$, folgt, dass $$\exists y\in P(\varphi(x,y,a'))(B)\cap A^n=\exists y(\varphi(x,y,a'))(A).$$
\end{proof}

\section{Folgen der Existenz des B\&F-Systems}
Im Folgenden werden einige Anordnungen von wechselseitigen Inklusionen von Modellen von T betrachtet, in der Gleichheit von bestimmten Typen folgt.

\begin{lemma}\label{freie Erweiterungen}
	Für dichte Paare $(B,A),(D,C)$ mit $(D,C)\subseteq(B,A)$ sind folgende Eigenschaften äquivalent:
	\begin{enumerate}
		\item $(D,C)\preceq(B,A)$
		\item Die Erweiterung ist frei.
	\end{enumerate}
\end{lemma}
\begin{proof}
	$\glqq{}1.\Rightarrow2.\grqq{}:$ Diese Richtung ist schon aus Lemma \ref{Unabhängigkeitsregeln} (5.) bekannt.\\
	$\glqq{}2.\Rightarrow1.\grqq{}:$ Finde $(\abs{B}+\abs{T})^+$-saturierte Strukturen $$(B,A)\preceq(B',A'),(D,C)\preceq(D',C');$$ es ist dann $(D,C)$ eine gemeinsame Unterstruktur und $(D,C)\subseteq(D',C')$ ist frei nach dem Beweis der Gegenrichtung. Außerdem sind nach Voraussetzung $D$ und $A$ unabhängig über $C$, da aber Unabhängigkeit von Tupeln in $D$ über $A$ auch über $A'$ erhalten bleibt (da $(B',A')$ elementare Oberstruktur), ist auch $(D,C)\subseteq(B',A')$ frei. Also ist die Identität auf $(D,C)$ im Back\&Forth-System, daher elementare Abbildung. Daraus folgt für alle $(\lingua_P)_D$-Formeln $\varphi$, dass $$(D,C)\models\phi\Leftrightarrow(D',C')\models\varphi\Leftrightarrow(B',A')\models\varphi\Leftrightarrow(B,A)\models\varphi.$$
\end{proof}

\begin{lemma}\label{Gemeinsame Unterstruktur}
	Seien $(B_1,A_1),(B_2,A_2)\models\td$ und $(B,A)$ eine gemeinsame Unterstruktur, sodass die Inklusionen frei sind. Wenn $a\in (A_1)^n$ und $b\in (A_2)^n$ denselben $\lingua$-Typen über $B$ erfüllen, erfüllen sie auch denselben $\lingua_P$-Typen über $B$.
\end{lemma}
\begin{proof}
	\OE\ seien $(B_1,A_1)$ und $(B_2,A_2)$ schon genügend saturiert, das ändert nichts an Typen über $B$ und (nach derselben Argumentation wie im vorigen Lemma) auch nichts an der Unabhängigkeit. Da $a$ und $b$ denselben Typen über $B$ erfüllen, kann man wieder $\lingua_B$-Terme mit eingesetztem $a$ auf $\lingua_B$-Terme mit eingesetztem $b$ abbilden (Wohldefiniertheit und Injektivität wird durch die Typengleichheit ermöglicht) und bekommt einen partiellen Isomorphismus $i:Ba\cong Bb$, dessen Einschränkung auf die $\lingua_A$-Terme einen partiellen Isomorphismus $Aa\cong Ab$ induziert und sodass $i(a)=b$. Also gilt $i:(Ba,Aa)\cong(Bb,Ab)$, da außerdem die Erweiterungen $(Ba,Aa)\subseteq(B_1,A_1)$ und $(Bb,Ab)\subseteq(B_2,A_2)$ frei sind nach Lemma \ref{Unabhängigkeitsregeln} (6.), ist $i$ im Back\&Forth-System, also elementar, also haben $a$ und $b$ denselben $\lingua_P$-Typen über $B$.
\end{proof}

\begin{corollary}
	Wenn man sich solch ein Paar $(a,b)$ beliebig wählt (z.B. $a=b=0$), sind in dem Typen auch die parameterfreien $(\lingua_P)_B$-Formeln, die in $(B_1,A_1)$ bzw. $(B_2,A_2)$ gelten. Also gelten dieselben Formeln, was als $(B_1,A_1)\equiv_B(B_2,A_2)$ geschrieben wird.
\end{corollary}

\begin{lemma}\label{selber Schnitt}
	Seien $(B_1,A_1),(B_2,A_2)$ zwei dichte Paare und $A\subseteq A_1\cap A_2$ eine gemeinsame Substruktur, sowie $a\in B_1\setminus A_1,\ b\in B_2\setminus A_2$, die den gleichen Ordnungstyp über $A$ haben. Dann haben $a$ und $b$ sogar den gleichen $\lingua_P$-Typ über $A$.
\end{lemma}
\begin{proof}
	Es sind trivialer Weise $A_i$ und $A$ unabhängig über $A$ für $i=1,2$, außerdem ist $a$ transzendent über $A_1$ und $b$ transzendent über $A_2$. Nach Lemma \ref{Unabhängigkeitsregeln} (4.) sind also die Einbettungen $(Aa,A)\subseteq(B_1,A_1)$ und $(Ab,A)\subseteq(B_2,A_2)$ frei. Nach dem Beweis zu Satz \ref{BackForth} gibt es also einen Isomorphismus $Aa\cong Ab$, der $A\cong A$ fortsetzt und für den $i(a)=b$ gilt, also gibt es einen Isomorphismus $i:(Aa,A)\cong(Ab,A)$.\\ Wenn \OE\ die beiden Modelle von $\td$ genügend saturiert sind, ist $i$ im B\&F-System, also erfüllen $a$ und $b$ dieselben Formeln.
\end{proof}

\newpage

\section{Definierbare Teilmengen von $A^n$}
Wir interessieren uns für die Gestalt von $\lingua_P$-definierbaren Teilmengen von $A^n$. Dafür braucht man zuerst eine Hilfsaussage für definierbare Mengen in o-minimalen Strukturen.

\begin{lemma}
	Sei $\fM$ eine o-minimale Struktur, die eine angeordnete Gruppenoperation $+$ mit positivem Element 1 hat und $Y\subseteq M^n$ definierbar. Dann ist $Y$ eine endliche Vereinigung von Mengen der Form $\{f(b,\cdot)=0,g(b,\cdot)>0\}$, wobei $b\in M^m$ und $f,g$ stetige, 0-definierbare Abbildungen $M^{m+n}\rightarrow M$ sind.
\end{lemma}
\begin{proof}
	Schreibe $Y=\phi(b,\fM)$ für ein $b\in M^m$ und definiere $Z:=\phi(\fM)$. Wenn man $Z$ in Zellen $(Z_i)_i$ zerlegt, erhält man $Y$ als endliche Vereinigung von $((Z_i)_b)_i$.  Es sei also o.B.d.A. $Z$ schon eine 0-definierbare Zelle.\\
	Definiere $$f(x):=\left\{\begin{array}{ll}
	\inf\{\abs{x-d}\mid d\in Z\}&Z\text{ nichtleer}\\
	1&\text{sonst}
	\end{array}\right.,$$
	$$g(x):=\left\{\begin{array}{ll}
	\inf\{\abs{x-d}\mid d\in \overline{Z}\setminus Z\}&Z\text{ nichtleer}\\
	1&\text{sonst}
	\end{array}\right.,$$ das sind lipschitzstetige Funktionen.\\
	Klar ist, dass $\overline{Z}=\{f=0\}$; da Zellen lokal abgeschlossen sind, ist $\overline{Z}\setminus Z=\overline{\overline{Z}\setminus Z}=\{g=0\}$. Also erhalten wir $$Z=\overline{Z}\setminus(\overline{Z}\setminus Z)=\{f=0\}\setminus\{g=0\}=\{f=0\}\cap\{g>0\}$$ und $$Y=Z_b=\{f(b,\cdot)=0\}\cap\{g(b,\cdot)>0\}.$$
\end{proof}

\begin{theorem}\label{Definierbare Mengen}
	Für ein dichtes Paar $(B,A)$ und $Y\subseteq A^n$ ist folgendes äquivalent:
	\begin{enumerate}
		\item $Y$ ist $\lingua_P$-definierbar.
		\item Es existiert ein $\lingua$-definierbares $Z\subseteq B^n$, sodass $Y=Z\cap A^n$.
		\item $Y$ ist definierbar in $(A,(R_b)_{b\in B})$ mit der Interpretation $A\models R_b(a)$ genau dann, wenn $0<a<b$ in $B$.
	\end{enumerate}
\end{theorem}
\newpage
\begin{proof}
	$\glqq{}1.\Rightarrow 2.\grqq{}:$ Sei $\varphi$ eine $(\lingua_P)_B$-Formel mit $\varphi(B)=Y$. Zu zeigen ist, dass eine $\lingua_B$-Formel $\psi$ existiert mit $(B,A)\models P(x)\rightarrow(\varphi(x)\leftrightarrow\psi(x))$; das ist genau dann der Fall, wenn $\mathfrak{Th}(B,A)_B\cup\{P(x)\}\cup\{\varphi(x)\}$ und $\mathfrak{Th}(B,A)_B\cup\{P(x)\}\cup\{\neg\varphi(x)\}$ in $\lingua_B$ getrennt werden können. Nach dem Trennungslemma gilt das genau dann, wenn für alle $(B,A)\preceq(D_1,C_1),(D_2,C_2)$ und alle $c_i\in C_i\ (i=1,2)$ mit $(D_1,C_1)\models\varphi(c_1),(D_2,C_2)\models\neg\varphi(c_2)$ eine $\lingua_B$-Formel $\chi$ existiert mit $(D_1,C_1)\models\chi(c_1),(D_2,C_2)\models\neg\chi(c_2)$.\\
	Seien solche $(D_i,C_i)$ und $c_i$, die die Voraussetzungen von oben erfüllen. Dann ist das die Situation aus Lemma \ref{Gemeinsame Unterstruktur}, denn elementare Erweiterungen sind frei. Also muss ein trennendes $\chi$ wie verlangt existieren, denn ansonsten würden $c_1$ und $c_2$ denselben $\lingua$-Typ erfüllen, aber nicht denselben $\lingua_P$-Typ.\\
	$\glqq{}2.\Rightarrow 3.\grqq{}:$ Sei $Y=Z\cap A^n$. Nach dem letzten Lemma ist $Z$ eine boolesche Kombination aus Mengen der Form $\{f(b,\cdot)=0\}$ und $\{g(b,\cdot)>0\}$ für stetige 0-$\lingua$-definierbare Funktionen $f,g$ und passende $b\in B^m$. Es reicht also die Aussage für Mengen in diesen Formen zu zeigen. Wegen der Stetigkeit der Funktionen und $A$ dicht in $B$ gilt aber in $B$
	\begin{align*}
	f(b,z)=0\Leftrightarrow\ &\text{Für alle }0<\varepsilon\in A\text{ existiert }A^m\ni a<b\text{ (koordinatenweise),}\\&\text{sodass für alle }a'\in A^m\text{ mit }a<a'<b\text{ (koordinatenweise)}\\&\text{gilt, dass }\abs{f(a',z)}<\varepsilon,\\
	g(b,z)>0\Leftrightarrow\ &\text{Es existiert ein }0<\varepsilon\in A\text{ und ein }A^m\ni a<b\text{ (koordinatenweise),}\\&\text{sodass für alle }a'\in A^m\text{ mit }a<a'<b\text{ (koordinatenweise)}\\&\text{gilt, dass }\abs{f(a',z)}>\varepsilon.
	\end{align*}
	Die rechten Bedingungen sind jeweils in $(A,(R_b)_{b\in B})$ definierbar.\\
	$\glqq{}3.\Rightarrow 1.\grqq{}:$ Da $A$und alle $R_b$ in $(B,A)$ definierbar sind, ist $Y$ auch in $(B,A)$ definierbar.
\end{proof}

\section{Definierbare Funktionen}
Um definierbare Funktionen besser zu verstehen, ist es notwendig, sich mit dem definierbaren Abschluss zu beschäftigen.

\begin{lemma}\label{A definierbar abgeschl}
	In jedem dichten Paar $(B,A)$ ist $A$ definierbar abgeschlossen.
\end{lemma}
\begin{proof}
	Sei $b\in B\setminus A$ und $(B,A)\preceq(D,C)$ eine genügend saturierte Erweiterung. Dann wird der Ordnungstyp von $b$ über $A$ auch von einem Element $D\setminus C\ni d\neq b$ realisiert wegen Dichtheit von $D\setminus C$ in $D$ und Saturation.\newpage
	Nach Lemma \ref{selber Schnitt} haben $b$ und $d$ dann den selben $\lingua_P$-Typen über $A$, weswegen $b$ nicht definierbar über $A$ in $(D,C)$ sein kann, also auch nicht in $(B,A)$.
\end{proof}

\begin{corollary}
	Sei $(B,A)$ ein dichtes Paar und $A_0\preceq A$. Dann ist $A_0$ definierbar abgeschlossen.
\end{corollary}
\begin{proof}
	Sei $b$ definierbar über $A_0$. Dann ist $b$ insbesondere definierbar über $A$, also in $A$. Nach Folgerung \ref{Definierbarkeit aus A} ist dann $\{a\}=\{a\}\cap A$ schon $\lingua$-definierbar aus $A_0$, also in $A_0$, da $A_0$ elementare Substruktur ist.
\end{proof}

\begin{lemma}\label{Freie Definierbarkeit}
	Sei $(D,C)\subseteq(B,A)$ frei und $(B,A)$ dichtes Paar. Dann ist $D$ definierbar abgeschlossen in $(B,A)$.
\end{lemma}
\begin{proof}
	Wenn $D=C$, ist $D\preceq A$ und die Aussage daher klar nach der vorigen Folgerung. Die Erweiterung $(AD,A)\subseteq(B,A)$ ist trivialerweise frei (zwei gleiche Mengen in der Unabhängigkeit), außerdem ist $A\preceq AD$ dicht (da $A$ dicht in $B\supseteq AD$) und eine echte Inklusion, da für $D=A$ wegen Unabhängigkeit von $D$ und $A$ ansonsten $D=C$ folgen würde. Nach Lemma \ref{freie Erweiterungen} ist also $(AD,A)\preceq(B,A)$ und daher ist $\dcl(D)\subseteq\dcl(AD)=AD$, da $AD$ definierbar abgeschlossen nach Lemma \ref{A definierbar abgeschl}.\\
	Sei jetzt $d\in AD$ $\lingua_P$-definierbar über $D$ und $a\in A^n$ minimal mit $d\in Da$ (insbesondere ist $a$ unabhängig über $D$). Im Folgenden wird gezeigt, dass dann $a$ schon das leere Tupel, also $d\in D$ ist.\\
	Nimm an, dass $n>0$ und sei $f:B^n\rightarrow B$ die definierende Funktion von $d$, also ist sie $D$-definierbar und $f(a)=d$. Seien $$S_1:=\{x\in B^n\mid f(x_1,\dots,x_{n-1},\cdot)\text{ ist streng monoton wachsend auf einem Intervall um }x_n\},$$ $$S_2:=\{x\in B^n\mid f(x_1,\dots,x_{n-1},\cdot)\text{ ist streng monoton fallend auf einem Intervall um }x_n\},$$ $$S_3:=\{x\in B^n\mid f(x_1,\dots,x_{n-1},\cdot)\text{ ist konstant auf einem Intervall um }x_n\}.$$
	$S_1\cup S_2\cup S_3$ ist groß, denn wenn eine offene Menge $U\subseteq B^n\setminus(S_1\cup S_2\cup S_3)$ existiert, wähle $x\in U$ beliebig und ein Intervall $I$ um $x_n$ mit $\{(x_1,\dots,x_{n-1})\}\times I\subset U$. Nach der Charakterisierung o-minimaler definierbarer Funktionen existiert ein Subintervall $J\subseteq I$, sodass $f(x_1,\dots,x_{n-1},\cdot)$ entweder streng monoton wachsend, fallend oder konstant ist auf $J$. Also ist $x\in S_1\cup S_2\cup S_3$ im Widerspruch zu $x\in U$.\newpage
	Da $a$ generisch ist, muss es also in der großen Menge liegen.
	\begin{itemize}
		\item Wenn $a$ in $S_1$ liegt, nehmen wir an, dass $(B,A)$ schon hinreichend saturiert ist (das ändert nichts, da $(B,A)$ ja nur irgendeine Oberstruktur und Modell von $\td$ sein muss) und finden in $A\setminus Da_1\dots a_{n-1}$ ein $a'\neq a_n$ mit demselben Ordnungstyp über $Da_1\dots a_{n-1}$ (ansonsten wäre $a_n$ definierbar über $a_1,\dots,a_{n-1}$). Insbesondere ist $a_1,\dots,a_{n-1},a'\in S_1$, weil die Menge aller solchen Elemente $a'$ $Da_1\dots a_{n-1}$-definierbar ist und daher eine $Da_1\dots a_{n-1}$-definierbare Umgebung von $a_n$ dort drin liegt, in der $a'$ liegen muss. Da $f$ streng monoton ist, ist $f(a_1,\dots,a_{n-1},a')\neq f(a)=d\in D$.\\
		Allerdings ist $d$ $\lingua_P$-definierbar über $D$, also ist $$f(a_1,\dots,a_{n-1},x)=d\in\tp_{\lingua_P}(a/Da_1\dots a_{n-1})\setminus\tp_{\lingua_P}(a'/Da_1\dots a_{n-1})$$ (oder zumindest mit der definierenden Formel für $d$ eingesetzt), die Typen sind daher nicht gleich.\\
		Da $a_n,a'\in A$ aber den gleichen Ordnungstyp über $Da_1\dots a_{n-1}$ haben, haben sie auch den gleichen $\lingua$-Typ über $Da_1\dots a_{n-1}$ nach dem Beweis von Satz \ref{BackForth}. Außerdem ist $(Da_1\dots a_{n-1},Ca_1\dots a_{n-1})\subseteq(B,A)$ nach Lemma \ref{Unabhängigkeitsregeln} (6.) frei, weswegen aus Lemma \ref{Gemeinsame Unterstruktur} folgt, dass $a_n,a'$ denselben $\lingua_P$-Typ über $Da_1\dots a_{n-1}$ haben - Widerspruch!
		\item Das Fall $a\in S_2$ geht analog, es wurde eben auch nur streng monoton benutzt.
		\item Im Falle $a\in S_3$ ist $d$ $\lingua$-definierbar über $Da_1\dots a_{n-1}$ durch $$\glqq{}d=f(a_1,\dots,a_{n-1},x)\text{ für irgendein }(a_1,\dots,a_{n-1},x)\in S_3.$$
	\end{itemize}
\end{proof}

\fcolorbox{red}{red}{Ab hier müsst ihr (Stand 20.1.) nicht mehr korrekturlesen. (TODO: Entfernen, wenn unnötig)}

\begin{theorem}
	Sei $(B,A)\models\td$ und sei $F:B\rightarrow B$ eine $\lingua_P$-definierbare Funktion. Dann stimmt $F$ auf bis auf eine kleine Menge mit einer $\lingua$-definierbaren Funktion überein.
\end{theorem}
\begin{proof}
	TODO: Muss noch warten
\end{proof}
\begin{corollary}
	Auch jede $\lingua_P$-definierbare Menge $S\subseteq B^n$ stimmt bis auf eine kleine Menge mit einer $\lingua$-definierbaren Menge $S'$ überein. $S'$ kann man zum Beispiel finden, indem man für die $\lingua_P$-definierbare charakteristische Funktion $\chi_S$ die $\lingua$-definierbare Annäherung $f$ findet und $S':=\{f=1\}$ setzt.
\end{corollary}

\begin{lemma}
	Sei $(B,A)\models\td$ und sei $F:A^n\rightarrow A$ eine $\lingua_P$-definierbare Funktion. Dann gibt es $\lingua_A$-definierbare $f_1,\dots,f_k:A^n\rightarrow A$, sodass für alle $a\in A^n$ ein $f_i$ existiert mit $F(a)=f_i(a)$.
\end{lemma}
\begin{proof}
	Wenn die Aussage nicht gilt, gilt für alle $k\in\setN$ und alle $\lingua_A$-definierbaren $f_1,\dots,f_k:A^n\rightarrow A$, dass ein $a\in A$ existiert mit $f_i(a)\neq F(a)$ für alle $i$. Also ist der partielle Typ $$\{F(x)\neq f(x)\mid f:A^n\rightarrow A\ \lingua_A\text{-definierbar}\}$$ konsistent und es existiert $(B,A)\preceq(B',A')$ und $a'\in A'^n$ mit $F(a')\neq f(a')$ für alle $\lingua_A$-definierbaren $f:A'^n\rightarrow A'$.\\
	Allerdings ist nach Lemma \ref{Unabhängigkeitsregeln} (6.) wegen $a'\in A'^n$ die Erweiterung $(Ba',Aa')\subseteq(B',A')$ frei, nach Lemma \ref{Freie Definierbarkeit} ist $Ba'$ also $\lingua_P$-definierbar abgeschlossen. Da $F(a')$ $\lingua_P$-definierbar über $Ba'$ ist, ist es in $Ba'$, wegen $F:A'^n\rightarrow A$ ist $F(a')\in A'$. Wegen Unabhängigkeit liegt also $F(a')\in Ba'\cap A'=Aa'$ und es gibt eine $\lingua_A$-definierbare Abbildung $f:A'^n\rightarrow A'$ mit $f(a')=F(a')$ - ein Widerspruch!
\end{proof}

\section{Definierbare eindimensionale Mengen}
TODO: Hier noch einleitende Worte finden\\
In diesem Teil sei $(B,A)$ ein dichtes Paar, als Konvention nehmen wir an, dass $A^0=\{0\}$.

\begin{lemma}
	Für jede $\lingua$-definierbare Menge $S\subseteq B^m$ und Funktion $g:B^m\rightarrow B^k$ gibt es eine $\lingua$-definierbare Teilmenge $S'\subseteq S$, sodass $$A^m\cap S\cap g^{-1}(A^k)=A^m\cap S'.$$
\end{lemma}
\begin{proof}
	Für $S=\emptyset$, wähle $S'=\emptyset$. Ansonsten führen wir eine Induktion über $(m,k,\dim S)$ mit elementweiser Halbordnung (die ist fundiert):\\
	Wenn $m=0,k=0$ oder $\dim S=0$, ist $A^m\cap S\cap g^{-1}(A^k)$ endlich und daher $\lingua$-definierbar, also kann man $S'=A^m\cap S\cap g^{-1}(A^k)$ wählen. Sei also $(m,k,\dim S)>(0,0,0)$.
	\begin{itemize}
		\item Wenn $k>1$ gilt und $g$ die Koordinatenfunktionen $g_1,\dots,g_k$ hat, so existieren $(S'_i)_{i\leq k}$ mit $S'_i\subseteq S$ und $A^m\cap S\cap g_k^{-1}(A)=A^m\cap S'_i$ für alle $i$ per Induktionsvoraussetzung. Dann gilt
		\begin{align*}
		A^m\cap S\cap g^{-1}(A^k)=\bigcap\limits_{i=1}^k A^m\cap S\cap g_i^{-1}(A)=\bigcap\limits_{i=1}^k A^m\cap S'_i=A^m\cap(\bigcap\limits_{i=1}^k S'_i),
		\end{align*}
		also erfüllt $S':=\bigcap\limits_{i=1}^k S'_i$ das Gewünschte.
		\item Wenn $k=1$ gilt, zerlege $S$ in Zellen $(Z_i)$, deren Dimension natürlich $\leq\dim S$ ist. Wenn man da das Problem löst (induktiv bzw. von Hand) und jeweils ein passendes $S'_i$ findet, löst $\bigcup\limits_i S'_i$ das Problem für $S$.\\ Sei also $S$ jetzt schon eine Zelle.
		\begin{itemize}
			\item Wenn $n:=\dim S<m$ ist und $\pi$ die entsprechende homöomorphe Projektion auf eine offene Zelle in $B^n$  bzw. eine $\lingua$-definierbare Fortsetzung davon auf ganz $B^m$, sei $\lambda$ eine $\lingua$-definierbare Fortsetzung der Umkehrfunktion dieser Projektion. Wähle die Fortsetzung $\lambda$ dabei so, dass $\lambda(\pi(S))$ und $\lambda(B^n\setminus\pi(S))$ disjunkt sind. Das ermöglicht die Gleichheit $\lambda(C\cap D)=\lambda(C)\cap\lambda(D)$ für $C\subseteq\pi(S)$. Löse dann mit einem $\lingua$-definierbaren $S''\subseteq \pi(S)$ das Problem $$A^n\cap\pi(S)\cap\lambda^{-1}(A^m)\cap (g\circ\lambda)^{-1}(A)=A^n\cap S''.$$
			Das Problem entspricht im Übrigen den Anforderungen, weil man\linebreak $\lambda^{-1}(A^m)\cap (g\circ\lambda)^{-1}(A)$ wie im Fall $k>1$ umschreiben kann. Schneidet man das mit $\lambda^{-1}(A^m)$ und wendet darauf $\lambda$ an, erhält man (mit schrittweiser Verwendung des $\cap$-Herausziehens)
			\begin{align*}
			\lambda(A^n)\cap S\cap A^m\cap g^{-1}(A)&=\lambda(A^n\cap\pi(S)\cap\lambda^{-1}(A^m)\cap (g\circ\lambda)^{-1}(A))\\
			&=\lambda(\lambda^{-1}(A^m)\cap A^n\cap S'')\\
			&=A^m\cap\lambda(A^n)\cap\lambda(S''),
			\end{align*}
			wegen $A^m\cap S\subseteq\lambda(A^n)$ aufgrund der Projektionseigenschaft von $\pi$, kann man $\lambda(A^n)$ weglassen und erhält $$A^m\cap S\cap g^{-1}(A)=A^m\cap\lambda(S''),$$
			also löst $\lambda(S'')$ das Problem für $S$.
			\newpage
			\item Wenn $\dim S=m$, finde eine $\lingua_A$-definierbare Funktion $G:B^{m+n}\rightarrow B$ mit $g=G(\cdot,b)$ für ein über $A$ unabhängiges Tupel $b\in B^n$. Als nächstes betreiben wir Induktion über $n$. Wenn $n=0$, dann ist nichts zu tun, weil dann $g$ schon $A$-definierbar ist, also $g^{-1}(A)=A^m$ und man dann $S'=S$ wählen kann. Ansonsten zerlege $S$ wie schon im Beweis von Lemma \ref{Freie Definierbarkeit} in Mengen
			\begin{align*}
			S_1:=\{x\in B^{m+n}\mid&G(x_1,\dots,x_{m+n-1},\cdot)\text{ ist streng monoton wachsend }\\&\text{auf einem Intervall um }x_n\},\\S_2:=\{x\in B^{m+n}\mid&G(x_1,\dots,x_{m+n-1},\cdot)\text{ ist streng monoton fallend }\\&\text{auf einem Intervall um }x_n\},\\S_3:=\{x\in B^{m+n}\mid&G(x_1,\dots,x_{m+n-1},\cdot)\text{ ist konstant auf einem Intervall um }x_n\}
			\end{align*}
			und den Rest $S_4:=B^{m+n}\setminus(S_1\cup S_2\cup S_3)$, partitioniere diese Mengen dann noch in $A$-definierbare Zellen $(Z_i)_i$ und definiere $Z'_i:=\{x\in B\mid (x,b)\in Z_i\}$ für alle $i$. Dann ist für jede offene Zelle $G$ in der letzten Koordinate entweder streng monoton steigend, fallend oder konstant jeweils auf der ganzen Zelle; das folgt, indem analog zum Beweis von Lemma \ref{Freie Definierbarkeit} offene Zellen schon Teilmenge von $S_1,S_2$ oder $S_3$ sind, die lokale Definition dieser Mengen überträgt sich durch Supremumsbildung auf die gesamte Zelle.\\
			Löse das Problem jetzt für alle $(Z'_i)_i$, wegen $S:=\bigcup\limits_i Z'_i$ ist es dann auch für $S$ gelöst: Für nicht-offene Zellen geht das per Induktion bzw. genauso wie im vorigen Unterpunkt. Wenn $Z'_i$ nun eine offene Zelle ist, gilt für ein generisches Element $x$ über $A,b$, dass $(x,b)$ generisch von $B^{m+n}$ ist, also in $S_1\cup S_2\cup S_3$. Also ist $Z_i$ entweder in $S_1,S_2$ oder $S_3$ enthalten.
			\begin{itemize}
				\item Wenn $Z_i\subseteq S_3$ ist, definiere $$\tilde{G}(\overline{x})=z:\Leftrightarrow z=G(\overline{x},y)\text{ für ein }y\text{ mit }(\overline{x},y)\in Z_i,$$ dann gilt $g=\tilde{G}(\cdot,b_1,\dots,b_{n-1})$ und per Induktion kann man das Problem für $n-1$ lösen.
				\item Wenn $Z_i\subseteq S_1,S_2$, also $G$ auf $Z_i$ injektiv in der letzten Koordinate ist, wird das Problem durch $\emptyset$ gelöst: Denn sei $a\in A^m\cap S'_i\cap g^{-1}(A)$, also existiert $a'\in A$ mit $a'=g(a)=G(a,b)$, weil $a\in Z'_i$ ist, ist $(a,b)\in Z_i$, also ist wegen Injektivität von $G$ in der letzten Koordinate $b_n$ eindeutig bestimmt mit $(a,b)\in Z_i$ und $a'=G(a,b)$. Das ist aber $A,b_1,\dots,b_{n-1}$-definierbar, also ist $b$ nicht unabhängig über $A$.
			\end{itemize}
		\end{itemize}
	\end{itemize}
\end{proof}

\begin{lemma}
	Sei $X\subseteq B$ eine kleine Teilmenge. Dann ist $X$ eine endliche Vereinigung von Mengen $f(A^n\cap E)$ für $E$ eine offene Zelle und $f:E\rightarrow B$ eine stetige $\lingua$-definierbare Funktion.
\end{lemma}
\begin{proof}
	Wenn $X$ klein ist, existiert ein $\lingua$-definierbares $g:B^m\rightarrow B$, sodass $X\subseteq g(A^m)$. Setze $X':=g^{-1}(X)\cap A^m=(g\upharpoonright A^m)^{-1}(X)$, das ist $\lingua_P$-definierbar und es gilt $g(X')=X$ wegen $X\subseteq\operatorname{im}(g\upharpoonright A^m)$.\\
	Beweise die Aussage jetzt induktiv über $m$: Wenn $m=0$ ist, ist $X$ maximal einelementig und entweder gleich $f(\{0\})$ für eine konstante Funktion $f$ oder schon die leere Vereinigung.\\
	Wenn $m>0$ ist, schreibe $X'$ als Teilmenge von $A^m$ wegen Satz \ref{Definierbare Mengen} in der Form $Y\cap A^m$ für ein $\lingua$-definierbares $Y$. Sei eine Zerlegung $\mathfrak{Z}$ von $Y$ in Zellen gegeben, auf denen $g$ jeweils stetig ist, dann ist $$X=g(Y\cap A^m)=\bigcup\limits_{Z\in\mathfrak{Z}}g(Z\cap A^m).$$
	Für offene Zellen $Z$ ist so eine Darstellung also schon gefunden. Sei $Z$ nun eine Zelle der Dimension $d<m$ und $\pi:B^d\rightarrow B^m$ eine $\lingua$-definierbare Fortsetzung des kanonischen Homöomorphismus der entsprechenden offenen Zelle $Z'$ nach $Z$, die so gewählt ist, dass $\pi(Z')$ und $\pi(M^d\setminus Z')$ disjunkt sind. Dann gilt mit derselben Argumentation wie im letzten Beweis und mit Anwendung des daraus resultierenden Lemmas $$f(Z\cap A^m)=(f\circ\pi)(A^d\cap Z\cap\pi^{-1}(A^m))=(f\circ\pi)(A^d\cap S)$$ für ein passendes $\lingua$-definierbares $S$. Das ist dann aber schon der Fall eines kleineren $m$ und mit der Induktionsbehauptung folgt die Aussage.
\end{proof}

\begin{theorem}
	Sei $X\subseteq B$ eine $\lingua_P$-definierbare Menge. Dann existiert eine endliche Unterteilung von $B$, sodass jedes dadurch erzeugte offene Intervall $I$ entweder disjunkt zu $X$ ist, Teilmenge von $X$ ist oder $X\cap I$ dicht \& kodicht in $I$ ist. Für kleine $X$ entfällt der Fall \glqq{}Teilmenge\grqq{} natürlich.
\end{theorem}
\begin{proof}
	Die gesuchte Eigenschaft bleibt unter endlichen Vereinigungen von Mengen erhalten, man muss nur eine Verfeinerung der Unterteilung durchführen. Deswegen sei $X$ zunächst klein und nach dem letzten Lemma gegeben als $X=f(A^n\cap E)$ für ein stetiges, $\lingua$-definierbares $f:E\rightarrow B$ und eine offene Zelle $E\subseteq B^n$. Da definierbarer Zusammenhang unter definierbaren stetigen Funktionen erhalten bleibt, ist $I:=f(E)$ auch definierbar zusammenhängend, weil es $\lingua$-definierbar ist, hat es auch ein Supremum und Infimum in $B\cup\{\pm\infty\}$, ist also ein Intervall (ausnahmsweise sei hier auch ein nichtoffenes Intervall mitgemeint). Wenn $I$ endlich ist, ist auch $X$ endlich und es ist nichts zu zeigen. $X$ ist disjunkt zu $B\setminus I$, nun muss nur noch gezeigt werden, dass $X=X\cap I$ dicht und kodicht in $I$ ist, denn Dichte und Kodichte in nichtoffenen unendlichen Intervallen ist äquivalent zu Dichte und Kodichte in deren Innerem. Aber für dichte Teilmengen werden durch stetige Abbildungen auf dichte Teilmengen abgebildet; da $A^n$ dicht in $B^n$ ist, ist auch $A^n\cap E$ dicht in $E$ und folglich $X$ dicht in $I$. Andererseits muss auch das Komplement von $X$ dicht in $I$ sein, da es sonst ein Intervall ganz in $X$ gäbe, was der Kleinheit mit Lemma \ref{Kleinheit} widerspricht.\\
	Sei jetzt $X$ nicht mehr klein, dann stimmt es aber bis auf eine kleine Menge mit einer $\lingua$-definierbaren Menge $X$ überein.
\end{proof}
    \newpage
    \appendix
    
    %!TEX root = DieLoesungAllerMilleniumsprobleme.tex
\addcontentsline{toc}{chapter}{Anhang}
\renewcommand\thesection{\Alph{section}}
\section{Ein Alternativbeweis zur $\omega$-Stabilität von ACP}

\begin{proof}
	Nach Folgerung \ref{Formel-Vereinfachung} kann man jede Formel mit Parametern aus $X$ modulo ACP schreiben als als boolesche Kombination aus $$\glqq{}l_n(\text{Monome mit Koeffizienten von Produkten aus }X)\grqq{}$$ und
	\begin{align*}
	\text{\glqq{}}(\text{Polynom in }\mathbb{P}[X])(&f_{i_1,n_1}(\text{Monome in Produkten aus }X),\\\dots,&f_{i_m,n_m}(\text{Monome in Produkten aus }X))=0\grqq{}.
	\end{align*}
	Diese beiden Arten von Formeln lassen sich verallgemeinern
    zu Formeln der Art $${\exists\overline{e}\in E(f(\overline{e},\overline{x})=0)}$$ für  $f(\overline{T},\overline{x})\in\mathbb{P}(\overline{T})[\overline{x}]$ und der Art
	\begin{align*}
	&\exists z_{1,2},\dots,z_{1,n_1+1},z_{2,2},\dots,z_{2,n_2+1},\dots\in E(p(z_{1,1},\dots,z_{k,1})=0\\
	&\bigwedge\limits_{i=1}^km_{i,1}(\overline{x})=z_{i,2}m_{i,2}(\overline{x})+\dots+z_{i,n_i}m_{i,n_i}(\overline{x}))
	\end{align*}
	für Monome $(m_{i,j})\in\mathbb{P}(X)[\overline{x}]$ und ein Polynom $p\in\mathbb{P}(X)[\overline{x}]$. Nenne die Menge aller Formeln der ersten Art $A$ und die aller Formeln der zweiten Art $B$. Insbesondere wurde die Menge der \glqq{}interessanten\grqq{} Formeln nur vergrößert, das heißt, dass ein Typ $p$ eindeutig durch $$(p\cap A)\cup(p\cap B)\cup(p\cap\neg A)\cup(p\cap\neg B)$$ festgelegt wird.\\
	Das bedeutet, in einem vorgegebenen Modell $\fM$ mit $X\subseteq M$ (\OE unendlich) zerfällt $S_n(X)$ in folgende Teilmengen:\\
	\begin{enumerate}
		\item Typen, die eine Formel aus $A$ und eine aus $B$ enthalten.
		\item Typen, die eine Formel aus $A$ und keine aus $B$ enthalten.
		\item Typen, die eine Formel aus $B$ und keine aus $A$ enthalten.
		\item Typen, die keine Formel aus $A\cup B$ enthalten.
	\end{enumerate}
	Für einen Typen $p$ ist im Fall 3./4. $p\cap A=\emptyset$, also $p\cap(A\cup\neg A)$ eindeutig gegeben durch die Verneinung aller möglichen Formeln in $A$ mit $n$ freien Variablen. Analog ist in Fall 2./4. $p\cap(B\cup\neg B)$ eindeutig bestimmt.\\
	Es bleibt nun noch zu zeigen, dass es im Fall 1./2. jeweils nur $\abs{X}$ viele Möglichkeiten für Einschränkungen $p\cap(A\cup\neg A)$ geben kann und im Fall 1./3. nur $\abs{X}$ viele Möglichkeiten für Einschränkungen $p\cap(B\cup\neg B)$.\\
	Zunächst zum ersten Teil: Definiere für ein Polynom $g\in E(X)[\overline{x}]$ die Relation
	\begin{align*}&g\in\in p:\Leftrightarrow\text{ es existiert }f(\overline{T},\overline{x})\in\mathbb{P}(\overline{T})[\overline{x}],\text{ es existieren }a_1,\dots,a_n\in E\text{ mit }\\
	&f(\overline{a},\overline{x})=g(\overline{x})\text{ und }\exists\overline{e}\in E(f(\overline{e},\overline{x})=0)\in p.
	\end{align*}
	$I:=\{g\in E(X)[\overline{x}]\mid g\in\in p\}$ ist offenbar ein Ideal im Noetherschen Ring $E(X)[\overline{x}]$ (es ist nichtleer im Fall 1./2.) und daher endlich erzeugt durch $h_1,\dots,h_m\in I$. Da jedes Element $g\in I$ mit einem Element $$\exists\overline{e}\in E(\overline{g}(\overline{e},\overline{x})=0)\in p$$ korrespondiert, ist $p\cap(A\cup\neg A)$ isoliert durch die übertragenen Erzeuger $$\exists\overline{e}\in E(\overline{h_1}(\overline{e},\overline{x})=0),\dots,\exists\overline{e}\in E(\overline{h_m}(\overline{e},\overline{x})=0),$$ also gibt es nur $\abs{X}$ viele Möglichkeiten für $p\cap(A\cup\neg A)$.\\
	Formeln der zweiten Art kann man in Konjunktionen von Formeln der ersten Art umwandeln, indem man $(z_{l,1})_{l=1\dots k}$ zu freien Variablen macht. Auf diese Weise kann man partielle Typen in $B$ zu partiellen Typen in $A$ in mehr Variablen umformen (am angenehmsten geht es wahrscheinlich, wenn man annimmt, dass $\fM$ schon hinreichend saturiert ist und einen Erfüller $\overline{a}$ von einem $p$ der Art 1./3. betrachtet, eine Belegung $(b_{i,j})$ für die $(z_{i,j})$ in einer der Formeln findet und dann $$\operatorname{tp}(\overline{a},(b_{l,1})_{l=1\dots k}/X)\cap(A\cup\neg A)$$ betrachtet). Aber wie auch immer man das macht, es können nur mehr Typen werden, dafür landet man wieder in Fall 1./2., wo wir wissen, dass es nur $\abs{X}$ viele Möglichkeiten gibt. Also ist ACP $\kappa$-stabil für alle unendlichen $\kappa$.
\end{proof}
    
    \newpage
    \bibliography{quellen}{}
    \addtocontents{toc}{\bigskip}
    \addcontentsline{toc}{section}{Literaturverzeichnis}
    \bibliographystyle{alpha}
    
\end{document}