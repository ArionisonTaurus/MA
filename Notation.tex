%!TEX root = DieLoesungAllerMilleniumsprobleme.tex
\addcontentsline{toc}{section}{Notation}
\chapter*{Notation}
Im Folgenden seien, wenn nicht weiter erklärt, mit $i,j,k,l,m,n$ immer natürliche Zahlen gemeint, mit $\kappa$ immer unendliche Kardinalzahlen.\\
Oftmals wird nicht zwischen Strukturen und deren Trägermengen unterschieden, insbesondere bei Paaren von Strukturen. Wenn von $\lingua$-Definierbarkeit in einer $\lingua$-Struktur $\fM$ die Rede ist, ist Definierbarkeit mit $\lingua_M$-Formeln gemeint, bei $\lingua_S$-Definierbarkeit für ein $S\subseteq M$ nur Definierbarkeit mit $\lingua_S$-Formeln. Wenn von $\dcl$ und $\acl$ die Rede ist, ist die kleinste Sprache gemeint, falls mehrere verwendet werden.\\
Als Topologie wird die Ordnungstopologie bzw. deren Produkttopologie verstanden, mit \glqq{}Intervall\grqq{} ist ein offenes, nichtleeres Intervall mit Randpunkten in der Struktur oder $\pm\infty$ gemeint. Außerdem sei für $A\prec B$ und $X\subseteq B$ $A$-definierbar in $T$ die Menge $X_A$ die durch dieselbe definierende Formel in $B$ definierbare Menge (für $X\subseteq A$ und $X_B$ analog). Außerdem sei für Relationen $P$ mit \glqq{}$\exists/\forall x\in P(\dots)$\grqq{} die Formel $$\glqq{}\exists x(P(x)\land\dots)/\forall x(P(x)\rightarrow\dots)\grqq{}$$ gemeint und für eine durch $\varphi$ definierbare Menge $X$ mit \glqq{}$x\in X$\grqq{} die Formel $\varphi(x)$.\\
$\abs{\overline{a}}$ soll je nach Kontext unterschiedliches bedeuten, einerseits die Supremumsnorm von $\overline{a}$, andererseits die Anzahl der Einträge. Da das eine ein Element der Struktur ist und das andere eine natürliche Zahl, ist immer klar erkennbar, was gemeint ist. Im Allgemeinen wird auch nicht immer zwischen Tupeln und Elementen unterschieden, außer, wenn das für das Verständnis notwendig ist. Auch werden Tupel an gewissen Stellen als Menge eingesetzt, dann sei jeweils die Menge aus den Tupeleinträgen gemeint.\\
Im modelltheoretischen Kontext ungewöhnlich sind Mengen der Form $\{f=g\},\{f>g\}$ und $\{f<g\}$ für zwei Abbildungen $f,g:X\rightarrow Y$. Hierunter sollen die maßtheoretischen Interpretationen dieser Ausdrucksweise $$\{x\in X\mid f(x)=g(x)\},\{x\in X\mid f(x)>g(x)\}\text{ und }\{x\in X\mid f(x)<g(x)\}$$ verstanden werden.