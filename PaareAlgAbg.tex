%!TEX root = DieLoesungAllerMilleniumsprobleme.tex
	\chapter{Paare algebraisch abgeschlossener Körper}
	Der hier vorgestellte Beweis stammt ursprünglich aus einem Werk (\cite{Robinson}) von Robinson, dem es in erster Linie um die Vollständigkeit und Entscheidbarkeit dieser Theorie ging. In diesem Werk soll es aber vorwiegend um Stabilität und eine gewisse Form der Formelreduktion gehen.
	
	\section{Modelltheoretische Grundlagen}
	Im Folgenden werden einige Aussagen über abzählbare und vollständige Theorien T in einer Sprache $\lingua$ aufgezählt, die die Grundlage für die Stabilitätstheorie bilden. Als Quelle soll dabei \cite{Marker} dienen. Diese Erkenntnisse werden wichtig werden, um die $\omega$-Stabilität der Paare algebraisch abgeschlossener Körper zu zeigen.
	
	\begin{definition}
		Sei $\fM$ ein $\aleph_0$-saturiertes Modell von T und $X$ eine beliebige $\lingua_M$-definierbare Menge in $M^n$. Der \textbf{Morleyrang} $\RM^\fM(X)$ von $X$ in $\fM$ sei dann induktiv definiert
		\begin{itemize}
			\item durch $-1$ genau dann, wenn $X$ leer ist
			\item durch $0$ genau dann, wenn $X$ endlich und nichtleer ist
			\item als $\RM^\fM(X)\geq\alpha+1$ für eine Ordinalzahl $\alpha$ genau dann, wenn es unendlich viele paarweise disjunkte $\lingua_M$-definierbare Teilmengen von $X$ gibt, die jeweils Morleyrang größer als oder gleich $\alpha$ haben
			\item als $\RM^\fM(X)\geq\gamma$ für eine Limesordinalzahl $\gamma$ genau dann, wenn er größer als jede Zahl kleiner $\gamma$ ist.
		\end{itemize}
	    Ist $\RM^\fM(X)$ größer als oder gleich $\alpha$ für eine Ordinalzahl $\alpha$, aber nicht größer als oder gleich $\alpha+1$, so sei $\RM^\fM(X):=\alpha$, ist $\RM^\fM(X)$ größer als alle Ordinalzahlen, sei $\RM^\fM(X):=\infty$ (oder alternativ: der Rang ist nicht definiert).
	\end{definition}
	
	\begin{definition}
		Für ein $\aleph_0$-saturiertes Modell $\fM$ von T und $\varphi$ eine $\lingua_M$-Formel sei $$\RM^\fM(\varphi):=\RM^\fM(\varphi(\fM)).$$
	\end{definition}
	\newpage
	\begin{factdef}
		Für allgemeine Modelle $\fM'$ und eine $\aleph_0$-saturierte Elementarerweiterung $\fM$ sowie eine $\lingua_{M'}$-Formel $\varphi$ ist der Morleyrang von $\varphi$ in $\fM$ nicht von der konkreten Wahl der Erweiterung abhängig.\\
		Definiere daher in einem Modell $\fM'$ den Morleyrang $\RM(\varphi)$ als den Morleyrang $\RM^\fM(\varphi)$ in einer beliebigen $\aleph_0$-saturierten Erweiterung $\fM$ und $\RM(X):=\RM(\varphi)$ für eine durch $\varphi$ definierbare Menge $X\subseteq(M')^n$.
	\end{factdef}
	
	\begin{factdef}
		In jedem Modell $\fM$ existiert für jede $\lingua_M$-Formel $\varphi$ mit $\RM(\varphi)\neq\infty$ eine natürliche Zahl $n$, sodass es in jeder Elementarerweiterung $\fM'$ maximal $n$ viele disjunkte Teilmengen von $\varphi(M')$ mit demselben Morleyrang gibt. Das kleinste solche $n$ nennt man den Morleygrad $\DM$.
	\end{factdef}
	
	\begin{definition}
		Für einen Typen $p$ definiere $$\RM(p):=\min\limits_{\psi\in p}\RM(\psi),\ \DM(p):=\min\limits_{\psi\in p,\RM(\psi)=\RM(p)}\DM(\psi).$$
		Als Kurzschreibweise stehe außerdem bei einem gegebenen Modell $\fM$, $\overline{a}$ in $M$ und $S$ als Teilmenge von $M$ die Bezeichnung $\RM(\overline{a}/S)$ für $\RM(\tp(\overline{a}/S))$.\\
		Analog sei $\DM(p)$ definiert.
	\end{definition}
	\begin{remark}
		Der Rang und Grad eines Typen werden in einer Formel aus dem Typen angenommen, nenne diese \textbf{minimal}.
	\end{remark}
	
	\begin{fact}
		Es sei in einem Modell $\fM$ eine definierbare Menge $X$ streng minimal. Dann ist für jedes Tupel $\overline{a}$ in $X$ sowie jede Teilmenge $S$ von $M$, über der $X$ definierbar ist, der Morleyrang schon bekannt: $$\RM(\overline{a}/S)=\dim(\overline{a}/S)$$
		Außerdem ist eine Formel $\varphi$ streng minimal genau dann, wenn ihr Rang und Grad beide Eins sind.
	\end{fact}
	
	\begin{fact}
		Sei $\fM$ ein Modell und $S\subseteq M$, außerdem $F$ eine konsistente und unter Konjunktion abgeschlossene Menge von $\lingua_S$-Formeln in $n$ freien Variablen, die eine Formel mit definiertem Morleyrang enthält. Dann gibt es für jede hinreichend große Obermenge $S'$ von $S$ genau $\DM(F)$ Fortsetzungen von $F$ zu einem Typen über $S'$ mit Morleyrang $\RM(F)$ (die Definition von Rang und Grad werde entsprechend von Typen verallgemeinert).\\
		Wenn der Morleygrad Eins ist, nenne den partiellen Typen \textbf{stationär}. 
	\end{fact}
	
	\begin{fact}\label{Stabilität Morleyrang}
		Die Theorie T ist $\omega$-stabil genau dann, wenn jede Formel definierten Morleyrang hat. Äquivalent ist, dass jede Formel in einer Variable definierten Morleyrang hat oder dass jeder Typ in einer Variable definierten Morleyrang hat.
	\end{fact}
	
	\begin{fact}\label{Anfangsstück}
		In jedem Modell ist für alle natürlichen Zahlen $n$ die Menge der angenommenen Morleyränge von Formeln oder Typen in $n$ Variablen ein Anfangsstück von $\textbf{On}\cup\{\infty\}$.
	\end{fact}
	
	\begin{fact}
		Sei $\fM$ ein Modell und seien $\overline{a},\overline{b}$ Tupel in $M$ sowie $S$ eine Teilmenge von $M$. Wenn $\overline{a}$ und $\overline{b}$ interalgebraisch über $S$ sind, folgt $$\RM(\overline{a}/S)=\RM(\overline{b}/S).$$
	\end{fact}
	
	\section{Algebraische und lineare Disjunktheit von Körpern}
	In diesem Teil richten wir uns im Aufbau und der Lemma-übergreifenden Strategie im Großen und Ganzen nach \cite{Delon}, wohingegen die konkreten Beweise meist von \cite{SergeLang} inspiriert sind. Ziel ist es, zwei algebraische Relationen zu verstehen, die Körper zueinander haben können.
	
    \begin{definition}
    	Gegeben Körperinklusionen $C\subseteq K,L\subseteq M$ in Rautenform, nenne $K$ und $L$ \textbf{linear disjunkt über} $C$, falls alle Basen von $K$ als $C$-Vektorraum auch über $L$ linear unabhängig bleiben. Nenne $K$ und $L$ \textbf{algebraisch disjunkt über} $C$, falls alle Transzendenzbasen von $K$ über $C$ auch algebraisch unabhängig über $L$ bleiben. Schreibe $K\ld_CL$ bzw. $K\ad_CL$.
    \end{definition}
    
    \begin{remark}
    	\ 
    	\begin{itemize}
    		\item Algebraische Disjunktheit ist nichts anderes als die Einschränkung der Unabhängigkeit aus dem letzten Kapitel auf bestimmte Mengen (nämlich Körpern in Rautenanordnung). Denn Körper $K$ und $L$ sind algebraisch disjunkt über $C$ genau dann, wenn sie unabhängig über $C$ sind im modelltheoretischen Sinn als Teilmengen eines algebraisch abgeschlossenen Körpers.\\
    		Wir bezeichnen es dennoch anders, um Verwirrung zu vermeiden. Wenn nämlich alle betrachteten Mengen schon Körper sind und die beiden Mengen, zwischen denen Unabhängigkeit gilt, schon Obermengen der dritten sind, kann man etwas weitergehende Eigenschaften feststellen, die im Normalfall so nicht gelten.
    		\item Es reicht, lineare Disjunktheit für eine Basis zu zeigen. Denn wenn man die lineare Unabhängigkeit über $L$ für eine Basis verliert, verliert man sie per $C$-Basiswechsel auch für alle anderen.
    		\item Es reicht, die Erhaltung der linearen/algebraischen Unabhängigkeit nur für beliebige endliche Mengen zu prüfen. Denn lineare/algebraische Unabhängigkeit einer Menge besteht genau dann, wenn sie für alle endlichen Teilmengen gilt.
    		\item Der Körper $M$ kommt in der Definition nur vor, damit die Rechenoperationen zwischen $K$ und $L$ wohldefiniert sind. Die genaue Wahl ist irrelevant und daher nicht in der Notation berücksichtigt. Wir nehmen für die Zukunft einfach an, dass die Multiplikation klar definiert ist. Es wird sich sowieso herausstellen, dass im Fall $K\ld_CL$ die Operationen eindeutig bestimmt sind.
    	\end{itemize}
    \end{remark}
    
    \begin{example}
    	Wir wollen Beispiele für die verschiedenen Konstellationen aus linearer und algebraischer Disjunktheit geben: 
    	\begin{itemize}
    		\item Trivialerweise gilt für alle Körper $C\subseteq K$, dass $C$ und $K$ sowohl linear als auch algebraisch disjunkt über $C$ sind.\\
    		Ein weniger leichtes Beispiel wäre, dass $\setQ(\sqrt{2})$ und $\setQ(\sqrt{3})$ linear und algebraisch disjunkt über $\setQ$ sind. Die lineare Disjunktheit folgt, wenn man feststellt, dass $1$ und $\sqrt{2}$ nicht linear abhängig über $\setQ(\sqrt{3})$ sein können, da sonst $\sqrt{2}$ in $\setQ(\sqrt{3})$ wäre. Für algebraische Disjunktheit ist nichts zu zeigen, da $\dim(\setQ(\sqrt{2})/\setQ)$ ohnehin schon $0$ ist.
    		\item Als Beispiel für algebraische und fehlende lineare Disjunktheit können die Körper $\setQ(\sqrt{2})$ und $\setQ(\sqrt{2})$ über $\setQ$ dienen. Algebraische Disjunktheit folgt dabei wie oben und lineare Disjunktheit gilt nicht, da $1$ und $\sqrt{2}$ nicht linear unabhängig über $\setQ(\sqrt{2})$ sind.
    		\item Ausstehend ist noch der Fall, in dem keine Art von Disjunktheit gilt: Das ist zum Beispiel bei den Körpern $\setQ(\pi)$ und $\setC$ über $\setQ$ so, da $\setQ(\pi)$ Teilmenge von $\setC$ ist und daher lineare Dimension $1$ und algebraische Dimension $0$ über $\setC$ hat, aber lineare Dimension $\aleph_0$ und algebraische Dimension $1$ über $\setQ$ besitzt.
    	\end{itemize}
        Es fällt auf, dass die Kombinationsmöglichkeit \glqq{}lineare, aber keine algebraische Disjunktheit\grqq{} nicht behandelt wurde. Das liegt daran, dass sie nicht möglich ist, wie später erklärt werden wird.
    \end{example}
    \newpage
    \begin{lemma}\label{Fraktionskörper}
    	Sei $C$ ein Körper und seien $C\subseteq R,S$ Ringerweiterungen des Körpers auf Integritätsbereiche. Dann sind $\operatorname{Frac}(R)$ und $\operatorname{Frac}(S)$ linear disjunkt über $C$ genau dann, wenn linear unabhängige Mengen in $R$ über $C$ auch linear unabhängig über $S$ bleiben.
    \end{lemma}
    \begin{proof}
    	Die Hinrichtung folgt leicht aus $R\subseteq\operatorname{Frac}(R),S\subseteq\operatorname{Frac}(S)$. Für die Rückrichtung seien $r_1x_1^{-1},\dots,r_nx_n^{-1}$ in $\operatorname{Frac}(R)$ linear unabhängig über $C$, aber linear abhängig über $\operatorname{Frac}(S)$ mit nichttrivialer Linearkombination $$(s_1y_1^{-1})r_1x_1^{-1},\dots,(s_ny_n^{-1})r_nx_n^{-1}=0,$$ wobei alle $s_iy_i^{-1}$ aus $\operatorname{Frac}(S)$ seien.\\
    	Durch Multiplikation mit $\prod\limits_{i=1}^nx_iy_i\neq0$ erhält man die Gleichung $$0=\sum\limits_{j=1}^n(\prod\limits_{i\neq j}x_i)r_j(\prod\limits_{i\neq j}y_i)s_j.$$ Diese bezeugt die lineare Abhängigkeit über $S$ der Elemente $((\prod\limits_{i\neq j}x_i)r_j)_{1,\dots,n}$ in $R$, die über $C$ unabhängig sind.
    \end{proof}
    
    \begin{remark}
    	Es reicht in obiger Aussage wieder, sich auf endliche Mengen zu beschränken. Alternativ kann man die Erhaltung der linearen Unabhängigkeit auch wieder nur für eine $C$-Basis von $R$ zeigen.
    \end{remark}
    
    Das folgende Lemma sagt insbesondere aus, dass lineare Disjunktheit symmetrisch ist. Noch viel wichtiger ist aber die Aussage über die Struktur der zueinander als Ringe adjungierten Körper.
    
    \begin{lemma}\label{Tensoren}
    	Für Körper $C,K,L$ wie oben gilt $K\ld_CL$ genau dann, wenn $$K[L]=L[K]\cong K\otimes_CL$$ mit kanonischem Isomorphismus.
    \end{lemma}
    \begin{proof}
    	Der aufgespannte Ring erfüllt $$K[L]=\{\sum\limits_{(k,l)\in X}kl\mid X\subseteq K\times L\text{ endlich}\}=L[K].$$
    	Wenn $(k_i)_I,(l_j)_J$ Basen von $K,L$ über $C$ sind, ist $(k_i\otimes l_j)_{i\in I,j\in J}$ eine Basis von $K\otimes_CL$.\newpage
    	Der $C$-Homomorphismus $$\sum\limits_{i\in I_0,j\in J_0} c_{ij}k_i\otimes l_j\mapsto \sum\limits_{i\in I_0,j\in J_0} c_{ij}k_il_j$$ für endliche, beliebige Teilmengen $I_0\subseteq I,J_0\subseteq J$ ist immer surjektiv, da klarerweise $E:=(k_il_j)_{i\in I,j\in J}$ ein Erzeugendensystem von $K[L]$ ist.\\
    	Er ist injektiv genau dann, wenn $E$ auch linear unabhängig über $C$ ist, also keine Linearkombination $$0=\sum\limits_{i\in I_0,j\in J_0}c_{ij}k_il_j=\sum\limits_{i\in I_0}(\sum\limits_{j\in J_0}c_{ij}l_j)k_i$$ existiert mit $c_{ij}\neq0$ für mindestens ein Paar $(i,j)$. Aber das ist genau dann der Fall, wenn keine $\tilde{c}_i=\sum\limits_{j\in J_0}c_{ij}l_j$ in $L$ existieren mit $i$ in $I_0$ und so, dass $\tilde{c}_i\neq0$ für mindestens ein $i$ und $0=\sum\limits_{i\in I_0}\tilde{c}_ik_i$ gilt; also wenn die $(k_i)_I$ linear unabhängig über $L$ sind.
    \end{proof}
    
    \begin{corollary}\label{Isomorphismen linear disjunkt}
    	Seien im Folgenden die Körper $C,K,L,C',K',L'$ gegeben, sodass $K$ und $L$ linear disjunkt über $C$ sind sowie $K'$ und $L'$ über $C'$. Es existiere ein Isomorphismus $C\cong C'$, der eine Fortsetzung auf $\varphi_1:K\cong K'$ und eine andere auf $\varphi_2:L\cong L'$ habe. Dann gibt es eine \glqq{}fusionierte\grqq{} Fortsetzung auf $K[L]\cong K'[L']$, denn die zwei Fortsetzungen induzieren einen Isomorphismus $$K\otimes_CL\cong K'\otimes_C'L',$$ durch die Abbildungsvorschrift $$k\otimes l\mapsto\varphi_1(k)\otimes\varphi_2(l).$$
    	Dieser setzt die vorigen Abbildungen von den Teilkörpern $$K\cong K\otimes_C1,L\cong1\otimes_CL,C\cong C\otimes_C1=1\otimes_CC$$ fort.\\
    	Insbesondere gibt es auch nur eine Möglichkeit, die Verknüpfung von Elementen aus linear disjunkten Körpern zu definieren (was wir zu Beginn dieses Kapitels schon ohne Beweis erwähnt hatten).
    \end{corollary}
    \newpage
    \begin{definition}
    	\ 
    	\begin{itemize}
    		\item Eine Körpererweiterung $K\subseteq L$ heiße \textbf{regulär}, wenn $\overline{K}\ld_KL$.
    		\item Für Körper $K,L\subseteq M$ sei $KL:=K(L)=L(K)$.
    	\end{itemize}
    \end{definition}
    
    \begin{example}
    	Die simpelsten Möglichkeiten für eine reguläre Körpererweiterung $K\subseteq L$ wäre einerseits, wenn $K$ selbst schon algebraisch abgeschlossen ist; andererseits auch die Adjunktion eines transzendenten Elements $e$ zu einem Körper $K$.\\
    	Im ersten Fall gilt die lineare Disjunktheit schon wegen Körpergleichheit, im zweiten Fall ist die $K$-Basis $\dots,e^{-1},1,e,e^2,\dots$ linear unabhängig über $\overline{K}$, sonst wäre $e$ nicht transzendent.
    \end{example}
    
    Wir können einige Folgerungen und \glqq{}Rechenregeln\grqq{} aus den definierten Eigenschaften ziehen, die insbesondere zeigen, unter welchen Bedingungen sich lineare und algebraische Disjunktheit gegenseitig implizieren und was für Regeln bei Inklusionsketten gelten.
    
    \begin{lemma}\label{Stapellemma}
    	Gegeben die Körperinklusionen $C\subseteq L\subseteq M$ und $C\subseteq K$. Dann gilt $K\ld_CM$ genau dann, wenn $K\ld_CL$ und $KL\ld_LM$ gilt.
    \end{lemma}
    \begin{proof}
    	Sei $(k_h)_H$ eine Basis von $K$ über $C$, $(l_i)_I$ eine Basis von $L$ über $C$ und $(m_j)_J$ eine Basis von $M$ über $L$. Die Aussage $K\ld_CM$ bedeutet, dass die $C$-Basis von $K$ auch eine $M$-Basis von $K$ ist, aber dann ist sie natürlich auch eine $L$-Basis wegen den Inklusionen $C\subseteq L\subseteq M$. Dies folgt, da die Eigenschaft, Erzeugendensystem über dem Körper zu sein, sich nach oben vererbt, die für die lineare Unabhängigkeit sich dafür nach unten vererbt. Also gilt $K\ld_CL$.\\
    	Außerdem ist $(l_im_j)_{I\times J}$ eine Basis von $M$ über $C$ und $(k_h)_H$ eine Basis von $L[K]$ als $L$-Vektorraum. Die erste Aussage ist aus dem Beweis der Multiplikativität von Körpererweiterungsgraden bekannt und bei der zweiten folgt aus der Definition von $\ld$ die Unabhängigkeit, die Eigenschaft als Erzeugendensystem ist klar.\\
    	Wenn $KL\ld_LM$ nicht gelten würde, müsste nach Lemma \ref{Fraktionskörper} und der anschließenden Bemerkung schon die $L$-Basis $(k_h)_H$ von $L[K]$ linear abhängig über $M$ sein, es gäbe also eine $M$-Linearkombination $$\sum\limits_{h\in H}\lambda_hk_h=0,\ \lambda_h=0\text{ für fast alle, aber nicht alle } h\text{ in }H.$$
    	Da $(k_h)_H$ aber die Basis von $K$ über $C$ ist, widerspricht das $K\ld_CM$.\newpage
    	Für die Rückrichtung ist zu zeigen, dass $(l_im_j)_{I\times J}$ linear unabhängig über $K$ bleibt. Wenn nicht, sei $$0=\sum\limits_{(i,j)\in I\times J}\lambda_{ij}l_im_j,\text{ wobei }\lambda_{ij}=0\text{ für fast alle, aber nicht alle } (i,j)\text{ in }I\times J$$ eine $K$-Linearkombination, die das bezeugt. Schreibe $$\lambda_{ij}=:\sum\limits_{h\in H}c_{hij}k_h$$ als $C$-Basisdarstellung für alle $(i,j)$ in $I\times J$.\\
    	Einsetzen und Umklammern führt uns zur Linearkombination $$0=\sum\limits_{(i,j)\in I\times J}\lambda_{ij}l_im_j=\sum\limits_{(i,j)\in I\times J}\left(\sum\limits_{h\in H} c_{hij}k_h\right)l_im_j=\sum\limits_{i\in I}\left(\sum\limits_{(h,j)\in H\times J}c_{hij}k_hl_i\right)m_j,$$ $$c_{hij}=0\text{ für fast alle, aber nicht alle } (h,i,j)\text{ in }H\times I\times J.$$
    	Da wir aber $KL\ld_CM$ annehmen, muss $$\sum\limits_{(h,j)\in H\times J}c_{hij}k_hl_i=0\text{ sein für alle }j\text{ in }J,$$ und da wir $K\ld_CL$ annehmen, folgt daraus $c_{hij}=0$ für alle $h,i,j$.
    \end{proof}
    
    Für weitere Regeln brauchen wir einen eingeschobenen Fakt aus \cite{SergeLang} (Seite 57, Theorem 3), der mit Bewertungen bewiesen wird.
    \begin{fact}\label{Das komplizierte Lemma}
    	Wenn $C\subseteq K$ regulär ist und $K\ad_CL$, folgt $K\ld_CL$.
    \end{fact}
    
    Damit lassen sich dann einige Regeln für das \glqq{}Herumschieben\grqq{} von linearer und algebraischer Disjunktheit beweisen.
    \begin{lemma}\label{Rechenregeln}
    	Seien $C,K,L,M$ Körper.
    	\begin{enumerate}
    		\item Wenn $K$ und $L$ linear disjunkt über $C$ sind, so ist $K\cap L=C$.
    		\item Wenn die Erweiterung $C\subseteq K$ algebraisch und die Erweiterung $C\subseteq L$ regulär ist, dann sind $K$ und $L$ linear disjunkt über $C$.
    		\item Wenn $K$ und $L$ linear disjunkt über $C$ sind, sind sie auch algebraisch disjunkt.
    		\item Wenn $K$ und $L$ algebraisch disjunkt über $C$ sind, dann sind es auch $\overline{K}$ und $\overline{L}$.
    		\item Wenn $K$ und $L$ linear disjunkt über $C$ sind, $K\subseteq M$ und $X\subseteq M$ algebraisch unabhängig über $KL$, dann folgt $K(X)\ld_KKL$.
    		\item Wenn die Erweiterung $C\subseteq K$ regulär ist und $K$ und $L$ linear disjunkt über $C$ sind, folgt $K\ld_C\overline{L}$ und außerdem, dass die Erweiterung $L\subseteq KL$ regulär ist.
    	\end{enumerate}
    \end{lemma}
    \begin{proof}
    	\ 
    	\begin{enumerate}
    		\item Die eine Inklusion gilt, denn $C\subseteq K,L$. Für die Inklusion in Gegenrichtung sei $x$ in $(K\cap L)\setminus C$. Dann ist $(1,x)$ in $K$ linear abhängig über $L$ und somit auch über $C$. Aber dann ist $x$ schon in $C$, weil es als $C$-Vielfaches von $1$ geschrieben werden kann.
    		\item Wegen der Regularität gilt $L\ld_C\overline{C}$ und wegen $C\subseteq K\subseteq\overline{C}$ und dem Lemma \ref{Stapellemma} gilt $L\ld_C K$.
    		\item Seien $k_1,\dots,k_n$ in $K$ algebraisch abhängig über $L$, das heißt, es gibt ein Polynom $$0\neq f(X)=\sum\limits_{\abs{\alpha}\leq m}l_\alpha X^\alpha\text{ in }L[X_1,\dots,X_n]$$ mit $f(k)=0$, also insbesondere $(k^\alpha)_{\abs{\alpha}\leq m}$ linear abhängig über $L$, per Annahme also auch über $C$. Dann existiert aber eine Linearkombination $\sum\limits_{\abs{\alpha}\leq m}c_\alpha k^\alpha=0$ mit $c_\alpha$ in $C$ nicht alle Null, und diese bezeugt die algebraische Abhängigkeit über $C$.
    		\item Diese Aussage ist exakt der Spezialfall von Lemma \ref{Unabhängigkeit acl} in unserem Setting von Körpern als Teilmengen eines algebraisch abgeschlossenen Körpers.
    		\item Wegen algebraischer Unabhängigkeit ist $$\abs{\overline{x}}=\dim(\overline{x}/KL)\leq\dim(\overline{x}/K)\leq\abs{\overline{x}}$$ für alle $\overline{x}$ in $X$, also sind $X$ und $KL$ im modelltheoretischen Sinne unabhängig über $K$. Das gilt dann nach den Regeln aus Kapitel \ref{Kapitel 0} auch für $K\cup X$ und $KL$, für $\acl(K\cup X)$ und $KL$ und auch für $K(X)$ und $KL$, also $K(X)\ad_KKL$.\\
    		Die Erweiterung $K(X)\supseteq K$ ist außerdem regulär, denn $K[X]$ hat als $K$-Basis $$\left\{\prod\limits_{i=1}^kx_i^{n_i}\right\}_{\{x_1,\dots,x_k\}\subseteq X\text,\ n\in\setN^k},$$ wie man wegen algebraischer Unabhängigkeit sieht.\newpage
    		Diese Basis bleibt aber linear unabhängig über $\overline{K}$, denn sonst wäre ein Polynom in $\overline{K}[X_1,X_2,\dots]$ gefunden, was ein $x$ aus $X$ über den anderen Elementen algebraisiert, also $$x\in\acl(X\setminus\{x\}\cup\overline{K})=\acl(X\setminus\{x\}\cup K),$$ demnach wäre $X$ nicht mehr algebraisch unabhängig über $K$. Lemma \ref{Fraktionskörper} besagt dann $$K(X)=\operatorname{Frac}(K[X])\ld_K\overline{K}.$$
    		Also haben wir $K(X)\ad_KKL$ und $K(X)\supseteq K$ regulär, woraus nach Fakt \ref{Das komplizierte Lemma} $K(X)\ld_KKL$ folgt.
    		\item Mit 3. folgt $K\ad_CL$, mit 4. $\overline{K}\ad_C\overline{L}$, mit Lemma \ref{Stapellemma} $K\ad_C\overline{L}$, mit Fakt \ref{Das komplizierte Lemma} $K\ld_C\overline{L}$ (benutze $C\subseteq K$ regulär) und mit noch einmal Lemma \ref{Stapellemma} gilt schließlich für die Einbettungskette $C\subseteq L\subseteq\overline{L}$ die Regularitätsbedingung $LK\ld_L\overline{L}$.
    	\end{enumerate}
    \end{proof}
    
    Es hat nicht nur die algebraische Disjunktheit, sondern auch die Regularität einer Erweiterung und die lineare Disjunktheit eine modelltheoretische Bedeutung.
    
    \begin{lemma}
    	Es sei eine Körpererweiterung $K\subseteq L$ gegeben und ein Tupel $a$ aus $L$. Dann ist $K\subseteq K(a)$ genau dann regulär, wenn $\tp(a/K)$ stationär im Modell $\overline{L}$ von ACF ist.
    \end{lemma}
    \begin{proof}
    	Die minimale Formel dieses Typen ist eine, die beschreibt, welche Elemente algebraisch übereinander sind und wie sie es sind. Ohne Einschränkungen sei $a$ so angeordnet, dass die ersten $k$ Einträge algebraisch unabhängig über $K$ sind und alle späteren im algebraischen Abschluss davon. Die minimale Formel beschreibe dann ohne Einschränkungen Polynome von minimalem Grad derart, dass $a_{k+1}$ über $K[a_1,\dots,a_k]$ algebraisiert wird, $a_{k+2}$ über $K[a_1,\dots,a_{k+1}]$ und so weiter.\\
    	Hierbei seien die auftretenden Potenzen der $a_i$ in den Koeffizienten schon maximal reduziert, also für alle $i$ größer als $k$ auf jeden Fall kleiner als der Grad des Minimalpolynoms von $a_i$ über $K(a_1,\dots,a_{i-1})$, dieser sei mit $n_i$ bezeichnet. Dann ist $\tp(a/K)$ stationär genau dann, wenn keines dieser Polynome über $K[a_1,\dots,a_i]$ beim Übergang zu einer größeren Parametermenge als $K$ reduzibel wird.\newpage
    	Da ACF modellvollständig ist, ist das äquivalent dazu, dass kein solches Polynom beim Übergang zu $\overline{K}$ reduzibel wird und das wiederum ist dasselbe wie die Aussage, dass die Erzeuger $$(a_1^{m_1}\cdot\dots\cdot a_\abs{a}^{m_\abs{a}})_{m_1,\dots,m_k\in\setN,0\leq m_i<n_i\text{ für }i>k}$$ von $K[a]$ über $\overline{K}$ linear unabhängig bleiben, also dass $K(a)$ und $\overline{K}$ linear disjunkt über $K$ sind.
    \end{proof}
    
    \begin{remark}
    	Damit ist die Folgerung \ref{Isomorphismen linear disjunkt} das algebraische Analogon zu folgender Aussage aus der Stabilitätstheorie: Sei $\fM,\fN$ zwei Modelle einer $\omega$-stabilen vollständigen, abzählbaren Theorie und Mengen $S\subseteq M,S'\subseteq N$, sodass es eine elementare Abbildung $$\varphi:S\cong S'$$ gibt, außerdem seien $a,b$ Tupel in $M$ und $a',b'$ Tupel in $N$, sodass gelte: $$\varphi(\tp(a/S))=\tp(a'/S'),\varphi(\tp(b/S))=\tp(b'/S).$$
    	Wenn $a$ und $b$ unabhängig über $S$ sind sowie $a'$ und $b'$ unabhängig über $S'$, außerdem $\tp(b'/S)$ stationär, dann kann der gemeinsame Typ von $a$ und $b$ ebenso durch $\varphi$ übertragen werden: $$\varphi(\tp(a,b/S))=\tp(a',b'/A).$$
    	Dass die Aussagen einander fast entsprechen, sieht man daran, dass die Übertragung der Typen durch $\varphi$ Isomorphismen zwischen den erzeugten Körpern $\langle S,a\rangle$ und $\langle S',a'\rangle$ sowie zwischen $\langle S,b\rangle$ und $\langle S',b'\rangle$ entspricht, die sich auf einen Isomorphismus $$\langle S,a,b\rangle\cong\langle S',a',b'\rangle$$ fortsetzen lassen. Der einzige Unterschied besteht darin, dass in der modelltheoretischen Version die Unabhängigkeit von $a$ und $b$ über $S$ sowie von $a'$ und $b'$ über $S'$ eine algebraische Disjunktheit auf beiden Seiten erzeugt, die wegen Stationarität des Typen nur einseitig zur linearen Disjunktheit wird.
    \end{remark}

    \newpage
    \section{Paare algebraisch abgeschlossener Körper}
    Wir wollen Paare $(K,E_K)$ von Körpern betrachten, wobei $E_K\subseteq K$ ist. In der richtigen Sprache lassen diese sich axiomatisieren, dort haben echte Paare (d.h. $K\neq E_K$) algebraisch abgeschlossener Körper mit fixierter Charakteristik sogar Quantoren\-elimination, sie sind vollständig und in einer kleineren Sprache modellvollständig. Diese Sprachen und einige der Folgerungen für ihre Strukturen wollen wir hier (angelehnt an \cite{Delon}) einführen.
    
    \begin{definition}
    	Wir definieren die Sprachen $\lld:=\{0,1,+,-,\cdot,(l_n)_{n\geq2}\},\lf:=\lld\cup\{f_{i,n}\mid n\geq2,1\leq i\leq n\}$ und $\lfc:=\lf\cup\{^{-1}\}$, wobei die $(l_n)_n$ $n$-stellige Relationen sein sollen und die $(f_{i,n})_{i,n}$ $n+1$-stellige Funktionen.
    \end{definition}
    
    Es kommt im Folgenden zu einem kleineren Problem: Eigentlich benötigt man für $^{-1}$ und die $f_{i,n}$ eine Auffassung als partielle Funktionen und im Folgenden werden sie auch so behandelt. Da das aber grundsätzlich nicht in unseren Axiomen der Logik vorgesehen ist, muss man die Funktionen durch $0$ fortsetzen. Rein formal gesehen gibt es dann den Unterschied zwischen der Interpretation von $$\glqq{}f_{i,n}(\overline{x})=0\grqq{}$$ als logische Formel in der Sprache und als für uns relevante Aussage, der gerade in der Aussage $$\glqq{}\overline{x}\text{ liegt nicht im partiellen Definitionsbereich von }f_{i,n}\grqq{}$$ besteht.\\
    Da aber sowohl der Definitionsbereich von $^{-1}$ als auch von jedem $f_{i,n}$ quantorenfrei\linebreak 0-definierbar ist, wird dieser Unterschied im Folgenden ignoriert werden, weil der Übergang zwischen diesen Interpretationen immer möglich ist, ohne in irgendeiner Form Bedingungen oder Aussagen von Sätzen zu verändern. Diese Tatsache ist stets im Hinterkopf zu behalten, insbesondere auch in Folgerung \ref{Formel-Vereinfachung}. Dort reicht quantorenfrei nämlich nicht aus, mehr dazu aber an der gegebenen Stelle.\\\\
    Nach der Klärung dieser Probleme ist es jetzt möglich, überhaupt zur Bedeutung dieser Sprache zu kommen. Es sei noch festgehalten, dass das weitere Vorgehen auch ohne Benutzung von \glqq{}$^{-1}$\grqq{} möglich wäre. Allerdings besteht das Problem mit den partiellen Funktionen ohnehin, und daher kann man die Behebung auch auf diese Funktion anwenden, wenn man so ein Vorgehen benutzen muss.
    
    \newpage
    
    \begin{lemma}\label{Symbolik}
    	Beliebige Paare $(K,E_K)$ von Körpern werden kanonisch zu $\lld$-Strukturen, indem man folgendes setzt:
    	$$\models l_n(x_1,\dots,x_n):\Leftrightarrow x_1,\dots,x_n\text{ sind linear unabhängig über }E_K.$$
    	Dann kann man die Substruktur $E_K$ auch definieren, da $x$ in $E_K$ ist genau dann, wenn $$\models\neg l_2(1,x)=:E(x)$$ und noch viel weitergehender auch $$y\in\langle\overline{x}\rangle_{E_K}\text{ für } x_1,\dots,x_n\text{ linear unabhängig über }E_K\Leftrightarrow\ \models l_n(\overline{x})\land\neg l_{n+1}(\overline{x},y)=:\phi(\overline{x},y).$$
    	Mit diesem Wissen setzt man jetzt in $\lf$ bzw. $\lfc$
    	$$\models (z=f_{i,n}(y,\overline{x})):\Leftrightarrow\ \models\phi(\overline{x},y)\text{ und }z\text{ ist die }i\text{-te Koordinate von }y\text{ in der Basisdarstellung},$$
    	wobei letzteres durch $$\exists z_1,\dots,z_n(z=z_i\land y=x_1z_1+\dots+x_nz_n\land z_1,\dots,z_n\in E)$$ oder aber auch $$\forall z_1,\dots,z_n(y=x_1z_1+\dots+x_nz_n\land z_1,\dots,z_n\in E\rightarrow z_i=z)$$ definierbar ist.
    \end{lemma}
    
    \begin{lemma}
    	Mit diesen Vorarbeiten sind echte algebraisch abgeschlossene Paare von Körpern definierbar in allen drei Sprachen $\lld,\lf,\lfc$, nenne die Theorien $$\operatorname{ACP}^{\lld},\operatorname{ACP}^{\lf},\operatorname{ACP}^{\lfc}.$$ Besonders zu beachten hierbei ist, dass man in der Theorie sagen muss, dass $\neg l_2(1,x)$ einen Körper definiert.
    \end{lemma}
    \newpage
    Alle verwendeten Sprachen und ihre Interpretationen für Paare von Körpern sind interdefinierbar, aber für bestimmte Fragestellungen sind manche Sprachen passender als andere. Die intuitivste Sprache wäre dabei $\lingua^E:=\lingua\cup\{E(x)\}$ mit der obigen Bedeutung für $E$, aber diese hat leider nicht so starke Eigenschaften (zum Beispiel ist sie nicht modellvollständig, denn lineare Unabhängigkeit über $E$ überträgt sich nicht zwangsläufig auf Obermodelle). Das Ziel ist jetzt, zu beweisen, dass $\operatorname{ACP}^{\lf}$ und $\operatorname{ACP}^{\lfc}$ Quantoren\-eli\-mi\-na\-tion haben sowie dass $\operatorname{ACP}^{\lld}$ immerhin modellvollständig ist.\\
    Dazu müssen wir erst einmal verstehen, wie $\lfc$-Unterstrukturen von $\operatorname{ACP}^{\lfc}$-Modellen aussehen.
    
    \begin{lemma}
    	Betrachte ein Paar von Körpern $(K,E_K)$ und eine Teilmenge ${A\subseteq K}$ sowie eine $\lfc$-Struktur $$\fA:=(A,0,1,+,-,\cdot,^{-1},(l_n)_{n\geq2},(f_{i,n})_{n\geq2,1\leq i\leq n}).$$ Dann ist $\fA$ eine $\lfc$-Unterstruktur von $(K,E_K)$ genau dann, wenn $A$ ein Unterkörper von $K$ ist und außerdem $\fA=(A,E_A)$ für $E_A:=A\cap E_K$ gilt sowie $A$ und $E_K$ linear disjunkt über $E_A$ sind.
    \end{lemma}
    \begin{proof}
    	Die Menge $A$ ist genau dann Unterkörper von $K$, wenn es $0,1$ enthält, unter $+,-,\cdot,^{-1}$ abgeschlossen ist und die entsprechenden Abbildungsvorschriften erbt.\\
    	Außerdem sind $A$ und $E_K$ linear disjunkt über $E_A$ genau dann, wenn für alle $\overline{a}$ in $A$ aus $\overline{a}$ linear abhängig über $E_K$ schon äquivalent zu linearer Abhängigkeit über $E_A$ ist; per kanonischer Definition also genau dann, wenn $$(A,E_A)\models l_{\abs{\overline{a}}}(\overline{a}) \Leftrightarrow (K,E_K)\models l_{\abs{\overline{a}}}(\overline{a}).$$
    	Wenn $\fA$ eine Unterstruktur von $(K,E_K)$ ist, gilt für alle $\overline{a}$ in $A$, dass
    	\begin{align*}
    	&\fA\models l_{\abs{\overline{a}}}(\overline{a})\Leftrightarrow(K,E_K)\models l_{\abs{\overline{a}}}(\overline{a})\\
    	\Leftrightarrow&\ \overline{a}\text{ linear unabhängig über }E_K\Leftrightarrow\overline{a}\text{ linear unabhängig über }E_A,
    	\end{align*}
    	wobei die letzte Äquivalenz davon herrührt, dass in jeder die lineare Abhängigkeit bezeugenden Linearkombination von $\overline{a}$ die Koeffizienten, die nicht $0$ sind, durch Anwendung von Funktionen $f_{i,n}$ auf Teile von $\overline{a}$ extrahiert werden können. Projektionen von Elementen aus $A$ sind aber wegen der Unterstruktureigenschaft wieder in $A$.\newpage
    	Also stimmen die Interpretationen der $l_n$ in $\fA$ mit denen in $(A,E_A)$ überein und es ist $A\ld_{E_A}E_K$. Da die definierende Formel der $(f_{i,n})$ bis auf die Angabe des Bildbereiches eine $\lingua_{\text{Ring}}$-Formel ist, und da $f_{i,n}(\overline{a})$ in $E_K\cap A$ für alle $\overline{a}$ in $A$ und für alle $n,i$, stimmen auch die Interpretationen der $f_{i,n}$ in $\fA$ und in $(A,E_A)$ überein. Damit ist $\fA=(A,E_A)$.\\
    	Die Rückrichtung folgt mit den ersten Zeilen dieses Beweises und, weil die Projektionen $f_{i,n}$ Elemente aus $A$ nach $A$ abbilden. Das erkennt man dadurch, dass man die Koeffizienten einer Linearkombination in eine Abhängigkeitsbedingung umschreiben kann; wenn diese in $E_K$ erfüllt ist, muss sie wegen linearer Disjunktheit auch in $E_A$ erfüllt sein, also sind die Projektionen schon in $E_A$.
    \end{proof}
    
    \begin{lemma}\label{transz Erw}
    	Wenn $(A,E_A)$ eine $\lfc$-Unterstruktur von $(K,E_K)$ ist und $X\subseteq K$ algebraisch unabhängig über $AE_K$, dann erhalten wir die folgende Kette von Inklusionen: $$(A,E_A)\subseteq_{\lfc}(A(X),E_A)\subseteq_{\lfc}(K,E_K).$$
    \end{lemma}
    \begin{proof}
    	Nach dem vorigen Lemma sind $A$ und $E_K$ linear disjunkt über $E_A$, daher gilt mit Lemma \ref{Rechenregeln} (5.) $$A(X)\ld_{E_A}E_K.$$ Da $A(X)$ Unterkörper von $K$ ist, gilt mit der Rückrichtung des letzten Lemmas, dass $$(A(X),E_A)\subseteq_{\lfc}(K,E_K).$$
    	Die Struktur $(A,E_A)$ ist eine $\lfc$-Unterstruktur von $(A(X),E_A)$, weil die Bedingung $A\ld_{E_A} E_A$ immer erfüllt ist.
    \end{proof}
    
    \begin{lemma}\label{E-Erw}
    	Sei $(A,E_A)\subseteq_{\lfc}(K,E_K)$ und $E_A\subseteq B\subseteq E_K$ ein Zwischenkörper. Dann ist $$(A,E_A)\subseteq_{\lfc}(AB,B)\subseteq_{\lfc}(K,E_K).$$
    \end{lemma}
    \begin{proof}
    	Es sind nur die Bedingungen $$A\ld_{E_A}B\text{ und }AB\ld_BE_K$$ zu zeigen. Wegen $(A,E_A)\subseteq_{\lfc}(K,E_K)$ sind $A$ und $E_K$ linear disjunkt über $E_A$ und mit Lemma \ref{Stapellemma} gelten schon beide gesuchten Aussagen.
    \end{proof}
    \newpage
    \begin{lemma}\label{Fortsetzungslemma}
    	Im Falle, dass das vorige Lemma auf die Inklusionen $$(A,E_A)\subseteq_{\lfc}(K,E_K)\text{ und }(\tilde{A},E_{\tilde{A}})\subseteq_{\lfc}(\tilde{K},E_{\tilde{K}})$$ sowie die Zwischenkörper $$E_A\subseteq B\subseteq E_K\text{ und }E_{\tilde{A}}\subseteq \tilde{B}\subseteq E_{\tilde{K}}$$ angewendet wird und dass gilt $$A\cong \tilde{A},B\cong \tilde{B},E_A\cong E_{\tilde{A}}$$ \--- wobei die ersten beiden Isomorphismen den dritten fortsetzen sollen\---, sind $(AB,B)$ und $(\tilde{A}\tilde{B},\tilde{B})$ schon isomorph als $\lfc$-Strukturen.
    \end{lemma}
    \begin{proof}
    	Mit Folgerung \ref{Isomorphismen linear disjunkt} erhält man den Isomorphismus $$A[B]\cong\tilde{A}[\tilde{B}],$$ der sich zu einem Isomorphismus der Quotientenkörper $AB$ und $\tilde{A}\tilde{B}$ fortsetzt und $A$ auf $\tilde{A}$ sowie $B$ auf $\tilde{B}$ und $E_A$ auf $E_{\tilde{A}}$ abbildet. Als Körperisomorphismus überträgt er auch Linearkombinationen, also auch die Interpretationen der $(l_n)_n$ und $(f_{i,n})_{i,n}$, weswegen er ein $\lfc$-Isomorphismus ist.
    \end{proof}
    
    \begin{lemma}\label{Unterstruktur regulär}
    	Sei die Inklusion $(A,E_A)\subseteq_{\lfc}(K,E_K)$ und $(K,E_K)$ ein Paar algebraisch abgeschlossener Körper gegeben, dann ist $E_A\subseteq A$ regulär.
    \end{lemma}
    \begin{proof}
    	Die Aussage folgt aus $A\ld_{E_A}E_K$ und der Körperinklusion $E_A\subseteq\overline{E_A}\subseteq E_K$ mit dem Lemma \ref{Stapellemma}.
    \end{proof}
    
    \begin{lemma}\label{alg Abschl}
    	Unter denselben Bedingungen wie im vorigen Lemma ist $(\overline{A},\overline{E_A})$ Zwischenstruktur.
    \end{lemma}
    \begin{proof}
    	Laut Lemma \ref{E-Erw} ist $(A\overline{E_A},\overline{E_A})$ Zwischenstruktur und damit insbesondere $$A\ld_{E_A}\overline{E_A},A\overline{E_A}\ld_{\overline{E_A}}E_K.$$ Klarerweise ist $$(A,E_A)\subseteq_{\lfc}(\overline{A\overline{E_A}},\overline{E_A})=(\overline{A},\overline{E_A}),$$ weil $A\overline{E_A}$ in der Bedingung $A\ld_{E_A}\overline{E_A}$ gar nicht vorkommt und die Erweiterung deshalb nichts ändert.\newpage
    	Lemma \ref{Rechenregeln} (6.) ergibt wegen der Regularität der Erweiterung $\overline{E_A}\subseteq E_K$ (die gemäß Lemma \ref{Unterstruktur regulär} vorliegt) die Konstellation $$\overline{A}=\overline{A\overline{E_A}}\ld_{\overline{E_A}}E_K,$$ was $(\overline{A},\overline{E_A})\subseteq_{\lfc}(K,E_K)$ beweist.
    \end{proof}
    
    \begin{theorem}\label{QE}
    	Die Theorie $\operatorname{ACP}^{\lfc}$ hat Quantorenelimination und ist vollständig, wenn man eine Charakteristik vorgibt.
    \end{theorem}
    \begin{proof}
    	Gegeben sei eine beliebige unendliche Kardinalzahl $\kappa$.
    	Zeige die Aussage mit dem Back\&Forth-System der Isomorphismen zwischen maximal $\kappa$ großen Unterstrukturen von $\kappa^+$-saturierten Modellen $(K,E_K),(L,E_L)$:
    	Dieses ist nichtleer, denn wenn $\mathbb{P}$ der Primkörper der Charakteristik ist, ist $(\mathbb{P},\mathbb{P})$ Unterstruktur von allen Modellen (wegen Gleichheit des Paares ist lineare Disjunktheit klar), bilde das als Unterstruktur von $K$ auf sich selbst als Unterstruktur von $L$ ab.
    	Sei $(M,E_M)\rightarrow(N,E_N)$ im B\&F-System. Die Erweiterungen $K\supseteq E_K$ und $L\supseteq E_L$ haben Transzendenzgrad $\infty$.\\
    	Dies kann man zum Beispiel feststellen, indem die Erweiterung offenkundig transzendent ist, und man dann jeweils den partiellen Typ über $\emptyset$ betrachtet, der die algebraische Unabhängigkeit von $n$ Elementen über $E_K$ bzw. $E_L$ beschreibt. Dieser hat folgende Gestalt:
    	$$\{\forall \overline{e}\in E\setminus\{0\}(f(\overline{e},\overline{x})\neq0)\mid 0\neq f\in\mathbb{P}[T_1,T_2,\dots,\overline{x}]\}.$$
    	Er ist endlich erfüllbar, da für $m$ größer als der größte Polynomgrad im endlichen Teilfragment und $x$ transzendent über $E_K$ bzw. $E_L$ die Elemente $x,x^m,x^{m^2},\dots$ algebraisch unabhängig über Polynomen von Grad kleiner $m$ sind.\\
    	Ohne Einschränkungen seien $(M,E_M)$ und $(N,E_N)$ jeweils algebraisch abgeschlossene Paare. Das kann man annehmen, denn die Lemmata \ref{E-Erw} und \ref{Fortsetzungslemma} besagen, dass es einen Isomorphismus zwischen den Zwischenstrukturen $$(M\overline{E_M},\overline{E_M})\text{ und }(N\overline{E_N},\overline{E_N}),$$ gibt der sich mit \ref{alg Abschl} auf einen Isomorphismus $$(\overline{M},\overline{E_M})\cong(\overline{N},\overline{E_N})$$ fortsetzt.\\
    	Sei jetzt $a$ in $K$. Wenn $a$ in $M$ liegt, dann kann man die Abbildung auf triviale Weise auf $a$ fortsetzen.\newpage
    	Wenn ansonsten $a$ algebraisch über $E_KM$ ist, ist $a$ in $\acl(E_KM)$, also existiert $X\subseteq E_K$ endlich mit $a$ in $\acl(MX)$. Ohne Einschränkungen sei $X$ jetzt schon ein Oberkörper von $E_M$, wichtig ist nur der endliche Transzendenzgrad über $E_M$. Wegen der Saturation hat die Erweiterung $E_N\subseteq E_L$ Transzendenzgrad $\infty$ und für einen beliebigen algebraisch abgeschlossenen Zwischenkörper $E_N\subseteq Y\subseteq E_L$ von gleichem Transzendenzgrad wie der von $X$ über $E_M$ kann $E_M\cong E_N$ fortgesetzt werden zu einem Isomorphismus $X\cong Y$. Diesen kann man wie oben mit Lemma \ref{Fortsetzungslemma} und Lemma \ref{alg Abschl} fortsetzen zu einem $\lfc$-Isomorphismus zwischen den Zwischenstrukturen $(\overline{MX},X)$ und $(\overline{NY},Y)$, wobei die erste $a$ enthält.\\
    	Wenn $a$ transzendent über $E_KM$ ist, gibt es ein über $E_LN$ transzendentes $b$ in $L$, denn der entsprechende Typ ist konsistent, wenn $\abs{L}$ größer ist als $\abs{E_L}$. So etwas lässt sich aber in einer elementaren Oberstruktur erreichen (für die endliche Konsistenz ist die genaue Kardinalität des Transzendenzgrads von $E_L\subseteq L$ egal) und wenn der Typ dort konsistent ist, dann auch unten.\\
    	Die Elemente $a$ und $b$ erzeugen einen Isomorphismus $C:=\overline{M(a)}\overset{\phi}{\cong}\overline{N(b)}$. Setze $$E:=\overline{M(a)}\cap E_K\cap\phi^{-1}(E_L\cap\overline{N(b)}),$$ dann gilt $(C,E)\cong_{\lfc}(\phi(C),\phi(E))$ und $b$ ist in $\phi(C)$.\\
    	Zu zeigen ist nun nur noch $$(M,E_M)\subseteq_{\lfc}(C,E)\subseteq_{\lfc}(K,E_K)\text{ und }(N,E_N)\subseteq_{\lfc}(\phi(C),\phi(E))\subseteq_{\lfc}(K,E_K).$$\\
    	Aus $(M,E_M)\subseteq_{\lfc}(K,E_K)$ folgt mit Lemma \ref{transz Erw} $$(M,E_M)\subseteq_{\lfc}(M(a),E_M)\subseteq_{\lfc}(K,E_K),$$ daraus mit Lemma \ref{E-Erw} und $E_M\subseteq E\subseteq E_K$ $$(M,E_M)\subseteq_{\lfc}(M(a)E,E)\subseteq_{\lfc}(K,E_K),$$ daraus folgt mit Lemma \ref{alg Abschl} und $E\subseteq\overline{M(a)}=C$ schließlich $$(M,E_M)\subseteq_{\lfc}(C,E)\subseteq_{\lfc}(K,E_K).$$
    	Der Beweis der Behauptung der Zwischenstruktureigenschaft für $(\phi(C),\phi(E))$ geht analog und weil $a$ aus $C$ kommt, haben wir die gesuchte Fortsetzung gefunden.
    \end{proof}
    
    \newpage
    
    \begin{definition}
    	Wenn es keine Rolle spielt, in welcher Sprache man gerade ist, schreibe einfach \textbf{ACP} für die Theorie.
    \end{definition}
    
    Bis jetzt haben wir zwar die Quantorenelimination erreicht, allerdings kann man Formeln modulo ACP noch weiter reduzieren. Insbesondere wäre eine \glqq{}Schachtelung\grqq{} von mehreren $f_{i,n}$ ineinander im Weiteren nicht leicht zu behandeln. Nach weiterer Reduktion der Formeln braucht man das aber auch nicht, da die Funktionen in den kleineren Körper abbilden. Der folgende Fakt lässt sich durch einige Rechnungen und Fallunterscheidungen beweisen, aus Gründen des Leseflusses werden diese aber hier weggelassen.
    
    \begin{fact}\label{Eliminierungsregeln}
    	In jedem Modell $(K,E_K)$ von ACP gelten die nachfolgenden Äquivalenzen für alle $i,n$. Hierbei seien die Variablen aus $K$ beliebig, ausgenommen die $e,e_1,\dots,e_n$, diese seien in $E_K$ beliebig. Im Falle der partiellen Funktionen seien nur Situationen betrachtet, in denen die betrachteten Argumente im partiellen Definitionsbereich liegen.\\
    	Es ist möglich, \glqq{}$^{-1}$\grqq{} in gewissen Situationen zu eliminieren:
    	$$l_n(x_1y_1^{-1},\dots,x_ny_n^{-1})\text{ gilt genau dann, wenn }l_n\left(x_1\prod\limits_{i=1\dots n,i\neq 1}y_i,\dots,x_n\prod\limits_{i=1\dots n,i\neq n}y_i\right)$$
    	$$\text{und }f_{i,n}(x_0y_0^{-1},x_1y_1^{-1},\dots,x_ny_n^{-1})= f_{i,n}\left(x_0\prod\limits_{i=0\dots n,i\neq 0}y_i,\dots,x_n\prod\limits_{i=0\dots n,i\neq n}y_i\right)$$
    	Auch lassen sich Vorfaktoren aus $E_K$ nach außen bringen/eliminieren:
    	$$l_n(e_1x_1,\dots,e_nx_n)\text{ gilt genau dann, wenn }l_n(x_1,\dots,x_n)$$
    	$$\text{und }f_{i,n}(ey,e_1x_1,\dots,e_nx_n)=e{e_i}^{-1}f_{i,n}(y,x_1,\dots,x_n)$$
    	Beim \glqq{}$+$\grqq{} ist die Situation schon komplexer. In der ersten Koordinate ist das noch recht leicht:
    	\begin{align*}
    	&\neg l_n(a+b,x_2,\dots,x_n)\text{ gilt genau dann, wenn }\\&\neg l_{n-1}(x_2,\dots,x_n)\text{ oder }(l_{n-1}(x_2,\dots,x_n)\text{ und }\\&(((l_n(b,x_2,\dots,x_n)\text{ und }\neg l_{n+1}(a,b,x_2,\dots,x_n))\text{ oder }\\&(\neg l_n(b,x_2,\dots,x_n)\text{ und }l_n(a,x_2,\dots,x_n))))),
    	\end{align*}
    	$$\text{außerdem ist }f_{i,n}(a+b,x_1,\dots,x_n)=f_{i,n}(a,x_1,\dots,x_n)+f_{i,n}(b,x_1,\dots,x_n)$$\newpage
    	Bei $l_n$ funktioniert diese Äquivalenz analog auch in den anderen Koordinaten, um \glqq{}$+$\grqq{} aus einem $f_{i,n}$ herauszuziehen, bedarf es aber einer größeren Fallunterscheidung:
    	\begin{align*}
    	&f_{i,n}(z,x_1,\dots,x_{i-1},a+b,x_{i+1},\dots,x_n)=\\
    	&\left\{\begin{array}{ll}
    	f_{i,n+1}(z,x_1,\dots,x_{i-1},a,b,x_{i+1},\dots,x_n)& a,b,\overline{x}\text{ unabhängig}\\
    	f_{i,n}(z,x_1,\dots,x_{i-1},a,x_{i+1},\dots,x_n)&\text{wenn nicht und }b\in\langle\overline{x}\rangle_{E_K}\\
    	f_{i,n-1}(z,x_1,\dots,x_{i-1},x_{i+1},\dots,x_n)&\text{wenn nicht und }z\in\langle\overline{x}\rangle_{E_K}\\
    	\frac{f_{i,n}(z,x_1,\dots,x_{i-1},b,x_{i+1},\dots,x_n)}{1+f_{i,n}(a,x_1,\dots,x_{i-1},b,x_{i+1},\dots,x_n)}&\text{ansonsten}
    	\end{array}\right.,
    	\end{align*}\begin{align*}
    	&f_{j,n}(z,x_1,\dots,x_{i-1},a+b,x_{i+1},\dots,x_n)=\\
    	&\left\{\begin{array}{ll}
    	f_{j+1_{j>i},n+1}(z,x_1,\dots,x_{i-1},a,b,x_{i+1},\dots,x_n)&a,b,\overline{x}\text{ unabhängig}\\\\
    	f_{j,n}(z,x_1,\dots,x_{i-1},a,x_{i+1},\dots,x_n)\\
    	-f_{i,n}(z,x_1,\dots,x_{i-1},a,x_{i+1},\dots,x_n)&\text{wenn nicht und }b\in\langle\overline{x}\rangle_{E_K}\\\cdot f_{j-1_{j>i},n-1}(b,x_1,\dots,x_{i-1},x_{i+1},\dots,x_n)\\\\
    	f_{j,n}(z,x_1,\dots,x_{i-1},b,x_{i+1},\dots,x_n)\\
    	-f_{i,n}(z,x_1,\dots,x_{i-1},a+b,x_{i+1},\dots,x_n)&\text{ansonsten}\\\cdot f_{j,n}(a,x_1,\dots,x_{i-1},b,x_{i+1},\dots,x_n)
    	\end{array}\right.\\
    	&(\text{hierbei sei }j\neq i)
    	\end{align*}
    \end{fact}
    
    Es steht noch eine Erklärung der genauen Verwendung dieses Faktes aus:\\
    Anschaulich betrachtet, kann man damit Formeln, die im Inneren eines $l_n$ oder eines $f_{i,n}$ ein $+,-,^{-1}$ oder ein weiteres $f_{i',n'}$ enthalten, in boolesche Kombinationen von Formeln umwandeln, die diese Art der Schachtelung nicht enthalten.\\
    Also erhält man quantorenfreie Formeln, in denen $+,-,^{-1}$ und alle $f_{i,n}$ wenn überhaupt, dann nur außerhalb aller $l_n$ und $f_{i,n}$ vorkommen. Da man aber quantorenfreie\linebreak$\lingua_\text{Ring}\cup\{^{-1}\}$-Formeln in quantorenfreie $\lingua_\text{Ring}$-Formeln umwandeln kann, ist es möglich, das \glqq{}$^{-1}$\grqq{} vollständig zu eliminieren; man erhält eine leicht handhabbare Form der Formeln.\\
    An dieser Stelle sei darauf hingewiesen, dass hier nicht mehr ausreicht, dass die Definitionsbereiche der $f_{i,n}$ quantorenfrei definierbar sind. Vielmehr wird benötigt, dass sie selbst in einer Form wie in der Folgerung definierbar sind. Das ist aber der Fall, wie man sich weiter vorne versichern kann.
    \newpage
    
    \begin{corollary}\label{Formel-Vereinfachung}
    	Man kann in jedem Modell $(K,E_K)$ jede Formel mit Parametern aus $X\subseteq K$ modulo ACP schreiben als boolesche Kombination aus Formeln der Form $$\glqq{}l_n(\text{Monome mit Koeffizienten von Produkten aus }X)\grqq{}$$ und
    	\begin{align*}
    	\text{\glqq{}}(\text{Polynom in }\mathbb{P}[X])(&f_{i_1,n_1}(\text{Monome in Produkten aus }X),\\\dots,&f_{i_m,n_m}(\text{Monome in Produkten aus }X))=0\grqq{},
    	\end{align*}
        wobei $\mathbb{P}$ den Primkörper der entsprechenden Charakteristik bezeichne.
    \end{corollary}
    \begin{corollary}
    	In jedem Modell $(K,E_K)$ ist jede definierbare Menge $X\subseteq E_K^n$ schon beschreibbar als $X=E_K^n\cap Z$, wobei $Z$ definierbar in der Ringsprache über denselben Parametern ist. Das liegt daran, dass die $(f_{i,n}),(l_n)$ trivial sind, wenn man Werte aus $E_K$ einsetzt.\\ Dementsprechend ist $E(x)$ eine streng minimale Formel und für $e_1,\dots,e_n$ in $E_K$ ist $\RM(\overline{e}/X)=\dim(\overline{e}/X)$ für alle $X\subseteq K$.
    \end{corollary}
    \begin{corollary}
    	Die Theorie $\operatorname{ACP}^{\lf}$ hat für fixierte Charakteristik ebenfalls Quantorenelimination, da nach Folgerung \ref{Formel-Vereinfachung} die Operation \glqq{}$^{-1}$\grqq{} eliminiert werden kann. Da in Lemma \ref{Symbolik} gezeigt wurde, dass die $(f_{i,n})$ sowohl existenziell als auch universell definierbar sind, ist jede Formel modulo $\operatorname{ACP}^{\lld}$ immerhin universell und $\operatorname{ACP}^{\lld}$ ist modellvollständig.
    \end{corollary}
    
    \begin{remark}
    	Man kann sich leicht überlegen, dass ACP$\cup\{\glqq{}\text{Charakteristik}=p\grqq{}\}$ das Primmodell $(\overline{\mathbb{P}(e)},\overline{\mathbb{P}})$ hat für $\mathbb{P}$ als Primkörper und ein beliebiges $e$ transzendent über $\mathbb{P}$. Diese Struktur findet man in einem beliebigen Modell $(K,E_K)$ wieder, indem man ein $z$ in $K\setminus E_K$ wählt und das Paar $(\overline{\mathbb{P}(z)},\overline{\mathbb{P}})$ betrachtet. Klarerweise ist das isomorph zu $(\overline{\mathbb{P}(e)},\overline{\mathbb{P}})$ und es ist in allen betrachteten Sprachen eine Substruktur von $(K,E_K)$, weil $\overline{\mathbb{P}(z)}$ und $E_K$ linear disjunkt über $\overline{\mathbb{P}}$ sind. Dies folgt nämlich mit Lemma \ref{Rechenregeln} (5.), da $z$ transzendent über $E_K$ ist sowie $\mathbb{P}$ und $E_K$ linear disjunkt über $\mathbb{P}$ sind.
    \end{remark}
    \newpage
    Im folgenden Lemma werden einige Polynomumformungen vorgenommen. Weil diese recht technisch sind, sollen sie zuerst an einem Beispiel vorgeführt werden: Seien $a,b,c$ drei über $\setQ$ algebraisch unabhängige Zahlen aus $\setC$, betrachte die ACP-Modelle $$(\overline{\setQ(a)},\overline{\setQ})\preceq(\setC,\overline{\setQ(b,c)})$$ Sei $d$ eine Nullstelle von $$f(X):=X^2-\frac{ac}{b^{-1}+\sqrt{c}}.$$ Man kann dann $f$ durch Multiplikation zu dem Polynom $$b^{-1}X^2+\sqrt{c}X^2-a\sqrt{c}^2$$ umformen, das die Eigenschaft hat, irreduzibel im Polynomring $\setQ(a)[b^{-1},\sqrt{c},X]$ zu sein und $d$ als Nullstelle zu haben. Eine weitere Umformung ergibt das Polynom $$(bc)^{-1}X^2+\sqrt{c}^{-1}X^2-a$$ aus $\setQ(a)[(bc)^{-1},\sqrt{c}^{-1},X]$, das das einzige Polynom von Grad 2 ist, sodass $\xi_1,\xi_2$ aus $\overline{\setQ(b,c)}$ existieren und das Polynom irreduzibel als Element von $\overline{\setQ(a)}[\xi_1,\xi_2,X]$ ist, einen konstanten Term von festgelegtem Wert $a$ ungleich Null hat sowie außerdem $d$ als Nullstelle hat.\\
    Mit dieser Eigenschaft sind seine Koeffizienten als Polynom in $X$ in $\lfc$ definierbar über $\setQ(a)\cup\{d\}$, denn nach TODO ist Irreduzibilität eine elementare Eigenschaft des zugrundeliegenden Körpers. Aus den Koeffizienten sind aber die verwendeten Potenzen von $\xi_1,\xi_2$ über $\overline{\setQ(a)}$ definierbar wegen linearer Disjunktheit von $\overline{\setQ(a)}$ und $\overline{\setQ(b,c)}$, also sind $\xi_1$ und $\xi_2$ algebraisch über $\overline{\setQ(a)}\cup\{d\}$. Da natürlich auch andersherum $d$ algebraisch über $\overline{\setQ(a)}\cup\{\xi_1,\xi_2\}$ ist, gilt für den Morleyrang $$\RM(d/\overline{\setQ(a)})=\RM(\xi_1,\xi_2/\overline{\setQ(a)})=\dim(\xi_1,\xi_2/\overline{\setQ(a)})=\dim(b,c/\overline{\setQ(a)})=2.$$
    Bei diesem Wert handelt es sich genau um die Mächtigkeit des kleinsten Tupels $\overline{e}$ aus $\overline{\setQ(b,c)}$, sodass $d$ algebraisch über $\overline{\setQ(a)},\overline{e}$ ist.\\\\
    Mit diesen Techniken ist es jetzt möglich, die Stabilität unserer Theorie zu beweisen.
    \newpage
    \begin{theorem}
    	Die Theorie ACP ist $\omega$-stabil.
    \end{theorem}
    \begin{proof}
    	Betrachte die Menge der Typen in einem Modell $(K,E_K)$ über einer vorgegebenen Menge $S\subseteq K$ und wähle als Sprache $\lfc$, zunächst sei $S$ keine Teilmenge von $E_K$. Ohne Einschränkungen sei das Modell schon $\abs{S}^+$-saturiert und $S$ sogar $\lfc$-algebraisch abgeschlossen. Insbesondere ist $S$ Träger einer elementaren Unterstruktur, denn $\operatorname{ACP}^{\lfc}$ ist modellvollständig und $S$ ist Modell, da es keine Teilmenge von $E_K$ ist. Aus dem B\&F-System in Satz \ref{QE} geht hervor, dass es die folgenden Typen über $S$ gibt:
    	\begin{itemize}
    		\item Den Typ eines Elementes in $S$
    		\item Die Typen eines Elementes $a$ in $\overline{SE_K}\setminus S$ (bestimmt durch die Größe eines minimalen Tupels $\overline{e}$ aus $E_K$, sodass $a$ in $\overline{S\overline{e}}$ ist, sowie durch den Isomorphietyp des Minimalpoynoms von $a$ über $S\overline{e}$)
    		\item Den Typ eines Elementes in $K\setminus\overline{SE_K}$
    	\end{itemize}
        Der erste Typ hat klarerweise Morleyrang 0. Im Fall des zweiten Typs für ein Element $a$ erhalten wir mit einem analogen Vorgehen zu oben eine Umformung des Minimalpolynoms von $a$ über $SE_K$. Damit ergibt sich der Morleyrang als die Mächtigkeit eines minimalen Tupels aus $E_K$, sodass $a$ algebraisch über $SE_K$ ist.\\
        Der dritte Typ kann keinen Morleyrang größer als $\omega$ haben, denn dann müsste es nach Fakt \ref{Anfangsstück} einen Typen mit Morleyrang $\omega$ geben, die anderen Arten von Typen haben aber endlichen Rang. Sei $(a_n)$ eine Folge von Elementen, sodass $$n=\RM(a_n)\text{ für alle }n\text{ in }\setN.$$ Man könnte zum Beispiel über $S$ algebraisch unabhängige $x_1,x_2,\dots$ aus $E_K\setminus S$ nehmen, die wegen Saturiertheit existieren müssen; ebenso kann man ein $z$ aus $S\setminus E_K$ wählen, das nach unseren Annahmen an $S$ existiert und das transzendent über $E_K$ ist. Die Elemente $1,z,z^2,\dots$ aus $S$ sind dann linear unabhängig über $E_K$ und wir setzen $a_n:=x_1+zx_2+\dots+z^nx_n$. Dann ist $a_n$ interdefinierbar mit $x_1,\dots,x_n$ über $z$, also hat es Morleyrang $n$ wegen der algebraischen Unabhängigkeit der $x_i$.\newpage
        Im Stoneraum gilt \--- bezüglich der in \cite{Lukas} beschriebenen Topologie \--- $$\lim\limits_{n\rightarrow\infty}\tp(a_n)=p$$ für den Typen $p$ der dritten Art: Denn $a_n$ ist transzendent über $SX$ für alle $E_S\subset X$ von Transzendenzgrad $\leq n-1$. Also sind für jede Umgebung, die das Nichterfüllen einer bestimmten Art von Polynom über $SE_K$ beschreibt, fast alle $a_n$ enthalten. Aber diese Umgebungen bestimmen den Typen $p$ eindeutig, also sind für jedes $\phi$ in $p$ fast alle $\tp(a_n)$ in der von $\phi$ erzeugten Umgebung $\fU_\phi$ im Stoneraum.\\
        Jedes $\phi$ in $\fF_1(\lingua_S)$ mit endlichem Rang ist dann nur in endlich vielen $\tp(a_n)$ enthalten (nämlich maximal $\operatorname{RM}(\phi)$ vielen), also ist $\RM(p)\geq\omega$, damit herrscht Gleichheit.\\
        Da alle Typen definierten Morleyrang haben, ist die Theorie $\omega$-stabil  nach Fakt \ref{Stabilität Morleyrang}.\\
        Ein Alternativbeweis wäre auch, dass es nicht mehr Typen über $S$ gibt als $\abs{S}+\aleph_0$, auch das beweist die $\omega$-Stabilität. Und auch mit algebraischen Methoden ist ein Beweis möglich, siehe dazu den Anhang.
    \end{proof}