\documentclass[a4paper, 11pt]{report}
\usepackage[utf8]{inputenc}
\usepackage[ngerman]{babel}
\usepackage{amsmath, amssymb, amsthm}
\usepackage{paralist}
\usepackage{lmodern}
\usepackage[T1]{fontenc}
\usepackage[arrow, matrix, curve]{xy}
\usepackage{graphicx}
\usepackage[percent]{overpic}
\usepackage{cite}
\usepackage{caption}
\usepackage[onehalfspacing]{setspace}
\usepackage{hyperref}
\usepackage{titlesec}
\usepackage{texilikechaps}
\usepackage{color}

\captionsetup[table]{labelformat=empty}


\newcommand{\setk}{\Bbbk}
\newcommand{\ldot}{\,.\,}
\newcommand{\fa}{~\forall}
\newcommand{\ex}{~\exists}
\newcommand{\fA}{\mathcal{A}}
\newcommand{\fB}{\mathcal{B}}
\newcommand{\fC}{\mathcal{C}}
\newcommand{\fM}{\mathcal{M}}
\newcommand{\fN}{\mathcal{N}}
\newcommand{\fF}{\mathcal{F}}
\newcommand{\fU}{\mathcal{U}}
\newcommand{\lingua}{\mathcal{L}}
\newcommand{\lld}{\mathcal{L}^{ld}}
\newcommand{\lf}{\mathcal{L}^f}
\newcommand{\lfc}{\mathcal{L}^{f,c}}
\newcommand{\sF}{\ensuremath{\mathcal{F}} }
\newcommand{\monster}{\mathbb{M}} %Monster-Modell-M 
\newcommand{\MM}{\mathbb{M}} %ebenfalls
\newcommand{\gdw}{\Leftrightarrow}
\newcommand{\Mod}{\mathcal{M}} %geschwungenes Modell-M
\newcommand{\Nod}{\mathcal{N}} %das gleiche für N
\newcommand{\leer}{\emptyset}
\newcommand{\In}{\in}
\newcommand{\setN}{\mathbb{N}}
\newcommand{\setZ}{\mathbb{Z}}
\newcommand{\setQ}{\mathbb{Q}}
\newcommand{\setR}{\mathbb{R}}
\newcommand{\setC}{\mathbb{C}}
\newcommand{\setH}{\mathbb{H}}
\newcommand{\Forall}{~\forall}
\newcommand{\Exists}{~\exists}
\newcommand{\abs}[1]{{\left| #1 \right|}}

\DeclareMathOperator{\ld}{ld}
\DeclareMathOperator{\ad}{ad}
\DeclareMathOperator{\acl}{acl}
\DeclareMathOperator{\dcl}{dcl}
\DeclareMathOperator{\tp}{tp}
\DeclareMathOperator{\tq}{T^2}
\DeclareMathOperator{\td}{T^d}
\DeclareMathOperator{\inn}{int}

\def\Ind#1#2{#1\setbox0=\hbox{$#1x$}\kern\wd0\hbox to 0pt{\hss$#1\mid$\hss}
	\lower.9\ht0\hbox to 0pt{\hss$#1\smile$\hss}\kern\wd0}

\def\ind{\mathop{\mathpalette\Ind{}}}

\def\notind#1#2{#1\setbox0=\hbox{$#1x$}\kern\wd0
	\hbox to 0pt{\mathchardef\nn=12854\hss$#1\nn$\kern1.4\wd0\hss}
	\hbox to 0pt{\hss$#1\mid$\hss}\lower.9\ht0 \hbox to 0pt{\hss$#1\smile$\hss}\kern\wd0}

\def\nind{\mathop{\mathpalette\notind{}}}


\theoremstyle{definition}
\newtheorem{theorem}[subsection]{Satz}
\newtheorem{corollary}[subsection]{Folgerung}
\newtheorem{proposition}[subsection]{Proposition}
\newtheorem{definition}[subsection]{Definition}
\newtheorem*{example}{Beispiel}
\newtheorem*{remark}{Bemerkung}
\newtheorem*{remarks}{Bemerkungen}
\newtheorem*{recall}{Erinnerung}
\newtheorem{satz}[subsection]{Satz}
\newtheorem{kor}[subsection]{Folgerung}
\newtheorem{prop}[subsection]{Proposition}
\newtheorem{lemma}[subsection]{Lemma}
\newtheorem{Def}[subsection]{Definition}
\newtheorem{bsp}[subsection]{Beispiel}
\newtheorem{axiom}[subsection]{Axiom}
\newtheorem{propdef}[subsection]{Proposition/ Definition}
\newtheorem{bemdef}[subsection]{Bemerkung/Definition}
\newtheorem{theocol}[subsection]{Folgerung/Satz}
\newtheorem*{bem}{Bemerkung}
\newtheorem*{erinn}{Erinnerung}

\txisection{chapter}{\normalfont \huge \bfseries }
\txisection{section}{\normalfont \Large \bfseries }

\usepackage[left=3.5cm,right=3cm,top=3.5cm,bottom=5cm]{geometry}
\setlength{\parindent}{0em}

\setlength\abovedisplayskip{4pt}
\setlength\belowdisplayskip{4pt}
\setlength\jot{4pt}

\newcommand{\lra}{\Leftrightarrow}
\newcommand{\xq}{ \bar{x}}

\def\Ind#1#2{#1\setbox0=\hbox{$#1x$}\kern\wd0\hbox to 0pt{\hss$#1\mid$\hss}
	\lower.9\ht0\hbox to 0pt{\hss$#1\smile$\hss}\kern\wd0}
\def\ua{\mathop{\mathpalette\Ind{}}}
\def\notind#1#2{#1\setbox0=\hbox{$#1x$}\kern\wd0
	\hbox to 0pt{\mathchardef\nn=12854\hss$#1\nn$\kern1.4\wd0\hss}
	\hbox to 0pt{\hss$#1\mid$\hss}\lower.9\ht0 \hbox to 0pt{\hss$#1\smile$\hss}\kern\wd0}
\def\nua{\mathop{\mathpalette\notind{}}}

\begin{document}

\section{TODO: Noch woanders einsortieren oder löschen}

\begin{lemma}
	Sei $\fA$ eine o-minimale Erweiterung eines angeordneten Vektorraums über einem angeordneten Körper $F$ und $g:A^{p+1}\rightarrow A$ definierbar, außerdem existiere für unendlich viele $\lambda\in F$ ein $a_\lambda\in A^p$ mit $g(a_\lambda,x)=\lambda x$ für unendlich viele $x\in A$. Dann existiert ein Intervall $I$ in $A$, sodass auf $I$ eine $A$-definierbare Körperstruktur existiert, die mit $<$ kompatibel ist (was automatisch einen reell abgeschlossenen Körper impliziert).
\end{lemma}
\begin{proof}
	TODO: Geht irgendwie aus \cite{PeterStarch} hervor.
\end{proof}

\begin{lemma}
	Es sei $(A,B)\models\td,\ f:B^{n+1}\rightarrow B$ $A$-definierbar in $B$ und $b\in B\setminus A$. Dann enthält $f(A^n\times\{b\})$ kein Intervall um $b$.
\end{lemma}
\begin{proof}
	Nimm an, dass das Gegenteil gelte für das Intervall $J$ (\OE\ mit Randpunkten in $A$): Dann existiert insbesondere für jedes $q\in\setQ$ hinreichend nahe bei $1$ ein $a_q\in A^n$ mit $f(a_q,b)=qb$. Dann existiert wieder ein Intervall $I_q\subseteq J_A$ mit $f(a_q,x)=qx$ für alle $x\in I_q$. \OE\ ist dieses Intervall schon beschränkt und die Randpunkte seien $c_q<d_q$. Definiere dann $$r_q:=\frac{c_q+d_q}{2},s_q:=\frac{d_q-c_q}{2}\in A,$$ $$g:(u,v,x)\mapsto f(u,v+x)-f(u,v)\ \ \ \ u\in A^n,v,x\in A.$$
	Dann gilt für alle $x\in(-s_q,s_q)$ $$g(a_q,r_q,x)=f(a_q,r_q+x)-f(a_k,r_q)=q(r_q+x)-qr_q=qx.$$
	Also existiert nach dem letzten Lemma ein Intervall in $A$ mit einer $A$-definierbaren Körperstruktur als RCF. Durch Translation (benutze Dichtheit) nehme an, dass $b\in I_B$ liegt. Dann existiert nach Lemma \ref{Hilfsaussage Kleinheit} ein Element $c\in I_B\setminus f(A^n\times\{b\})$. \OE\ sei schon $\inf J,\sup J\in I$, sonst ersetze $J$ durch ein kleineres Intervall.\\
	Seien $d,e\in I$ mit $d<c<e$ und $\varphi$ die orientierungserhaltende, $A$-definierbare affine Abbildung in $I$ mit $\varphi(d)=\inf J,\varphi(e)=\sup J$. Dann ist $\varphi(c)\in J\setminus(\phi\circ f)(A^n\times\{b\})$ und da das Verketten mit einer $A$-definierbaren invertierbaren Abbildung nichts an der Aussage ändert, gibt es einen Widerspruch.
\end{proof}

\begin{theorem}
	Wenn $(B,A)\models\td$, dann ist kein Intervall eine kleine Teilmenge.
\end{theorem}
\begin{proof}
	Sei $f:B^n\rightarrow B$ eine durch $\varphi(x,y,b)$ definierbare Abbildung mit $\varphi$ eine\linebreak$\lingua_A$-Formel und $b\in B^m$ für ein $m\in\setN$ definiert. Für $\dim(b/A)=0$ ist $f(A^n)\subseteq A$ klar, deswegen sei \OE\ $\dim(b/A)\geq1$. Definiere
	\begin{align*}
	g(x,z):=\left\{\begin{array}{ll}
	\text{das eindeutige }y\in B &\text{für alle z, für die }\varphi(x,y,z)\\
	\text{ mit }B\models\varphi(x,y,z) &\text{ bei festem }\text{ eine Funktion definiert}\\
	\ &\ \\
	0 &\text{sonst}
	\end{array}\right.,
	\end{align*}
	Dann ist $g$ in $B$ $A$-definierbar und $g(\cdot,b)=f$. Falls $\dim(b/A)>1$, füge genug Komponenten von $b$ zu $A$ hinzu, sodass $\dim(b/A)=1$. Das Hinzufügen ändert nichts, denn $Ab_i$ ist nach den Eingangsbemerkungen Modell von T und $Ab_i$ ist erst recht dicht in, aber nicht gleich $B$ (sonst hätte man die Dimension mit diesem Schritt schon zu sehr verkleinert).\\
	Finde also $b_i$, sodass $A$-definierbare $(h_j)$ existieren mit $b_j=h_j(b_i)$ für alle $j$. Wenn jetzt $J\subseteq f(A^n)=g(A^n,b)=g(A^n,h(b_i))$ für ein Intervall $J$, dann widerspricht das der Aussage des letzten Lemmas für die Funktion $(x,y)\mapsto g(x,h(y))$.
\end{proof}

\begin{proof}
	Wenn $D=C$, ist $D\preceq A$ und die Aussage daher klar nach der vorigen Folgerung. Die Inklusion $(AD,A)\subseteq(B,A)$ ist trivialerweise frei (zwei gleiche Mengen in der Unabhängigkeit), außerdem ist $A\preceq AD$ dicht (da $A$ dicht in $B\supseteq AD$) und eine echte Inklusion, da für $D=A$ wegen Unabhängigkeit von $D$ und $A$ ansonsten $D=C$ folgen würde. Nach Lemma \ref{freie Inklusionen} ist also $(AD,A)\preceq(B,A)$ und daher ist $\dcl(D)\subseteq\dcl(AD)=AD$, da $AD$ definierbar abgeschlossen nach Lemma \ref{A definierbar abgeschl}.\\
	Sei jetzt $d\in AD$ $\lingua^P$-definierbar über $D$ und $a\in A^n$ minimal mit $d\in Da$ (insbesondere ist $a$ unabhängig über $D$). Im Folgenden wird gezeigt, dass dann $a$ schon das leere Tupel, also $d\in D$ ist.\\
	Nimm an, dass $n>0$ und sei $f:B^n\rightarrow B$ die definierende Funktion von $d$, also ist sie $D$-definierbar und $f(a)=d$. Seien $$S_1:=\{x\in B^n\mid f(x_1,\dots,x_{n-1},\cdot)\text{ ist streng monoton wachsend auf einem Intervall um }x_n\},$$ $$S_2:=\{x\in B^n\mid f(x_1,\dots,x_{n-1},\cdot)\text{ ist streng monoton fallend auf einem Intervall um }x_n\},$$ $$S_3:=\{x\in B^n\mid f(x_1,\dots,x_{n-1},\cdot)\text{ ist konstant auf einem Intervall um }x_n\}.$$
	$S_1\cup S_2\cup S_3$ ist groß, denn wenn eine offene Menge $U\subseteq B^n\setminus(S_1\cup S_2\cup S_3)$ existiert, wähle $x\in U$ beliebig und ein Intervall $I$ um $x_n$ mit $\{(x_1,\dots,x_{n-1})\}\times I\subset U$. Nach der Charakterisierung o-minimaler definierbarer Funktionen existiert ein Subintervall $J\subseteq I$, sodass $f(x_1,\dots,x_{n-1},\cdot)$ entweder streng monoton wachsend, fallend oder konstant ist auf $J$. Also ist $x\in S_1\cup S_2\cup S_3$ im Widerspruch zu $x\in U$.\newpage
	Da $a$ generisch ist, muss es also in der großen Menge liegen.
	\begin{itemize}
		\item Wenn $a$ in $S_1$ liegt, nehmen wir an, dass $(B,A)$ schon hinreichend saturiert ist (das ändert nichts, da $(B,A)$ ja nur irgendeine Oberstruktur und Modell von $\td$ sein muss) und finden in $A\setminus Da_1\dots a_{n-1}$ ein $a'\neq a_n$ mit demselben Ordnungstyp über $Da_1\dots a_{n-1}$ (ansonsten wäre $a_n$ definierbar über $a_1,\dots,a_{n-1}$). Insbesondere ist $a_1,\dots,a_{n-1},a'\in S_1$, weil die Menge aller solchen Elemente $a'$ $Da_1\dots a_{n-1}$-definierbar ist und daher eine $Da_1\dots a_{n-1}$-definierbare Umgebung von $a_n$ dort drin liegt, in der $a'$ liegen muss. Da $f$ streng monoton ist, ist $f(a_1,\dots,a_{n-1},a')\neq f(a)=d\in D$.\\
		Allerdings ist $d$ $\lingua^P$-definierbar über $D$, also ist $$f(a_1,\dots,a_{n-1},x)=d\in\tp_{\lingua^P}(a/Da_1\dots a_{n-1})\setminus\tp_{\lingua^P}(a'/Da_1\dots a_{n-1})$$ (oder zumindest mit der definierenden Formel für $d$ eingesetzt), die Typen sind daher nicht gleich.\\
		Da $a_n,a'\in A$ aber den gleichen Ordnungstyp über $Da_1\dots a_{n-1}$ haben, haben sie auch den gleichen $\lingua$-Typ über $Da_1\dots a_{n-1}$ nach dem Beweis von Satz \ref{BackForth}. Außerdem ist $(Da_1\dots a_{n-1},Ca_1\dots a_{n-1})\subseteq(B,A)$ nach Lemma \ref{Unabhängigkeitsregeln} (6.) frei, weswegen aus Lemma \ref{Gemeinsame Unterstruktur} folgt, dass $a_n,a'$ denselben $\lingua^P$-Typ über $Da_1\dots a_{n-1}$ haben - Widerspruch!
		\item Das Fall $a\in S_2$ geht analog, es wurde eben auch nur streng monoton benutzt.
		\item Im Falle $a\in S_3$ ist $d$ $\lingua$-definierbar über $Da_1\dots a_{n-1}$ durch $$\glqq{}d=f(a_1,\dots,a_{n-1},x)\text{ für irgendein }(a_1,\dots,a_{n-1},x)\in S_3.$$
	\end{itemize}
\end{proof}

\begin{lemma}
	Für jede $\lingua$-definierbare Menge $S\subseteq B^m$ und Funktion $g:B^m\rightarrow B^k$ gibt es eine $\lingua$-definierbare Teilmenge $S'\subseteq S$, sodass $$A^m\cap S\cap g^{-1}(A^k)=A^m\cap S'.$$
\end{lemma}
\begin{proof}
	Für $S=\emptyset$, wähle $S'=\emptyset$. Ansonsten führen wir eine Induktion über $(m,k,\dim S)$ mit elementweiser Halbordnung (die ist fundiert):\\
	Wenn $m=0,k=0$ oder $\dim S=0$, ist $A^m\cap S\cap g^{-1}(A^k)$ endlich und daher $\lingua$-definierbar, also kann man $S'=A^m\cap S\cap g^{-1}(A^k)$ wählen. Sei also $(m,k,\dim S)>(0,0,0)$.
	\begin{itemize}
		\item Wenn $k>1$ gilt und $g$ die Koordinatenfunktionen $g_1,\dots,g_k$ hat, so existieren $(S'_i)_{i\leq k}$ mit $S'_i\subseteq S$ und $A^m\cap S\cap g_k^{-1}(A)=A^m\cap S'_i$ für alle $i$ per Induktionsvoraussetzung. Dann gilt
		\begin{align*}
		A^m\cap S\cap g^{-1}(A^k)=\bigcap\limits_{i=1}^k A^m\cap S\cap g_i^{-1}(A)=\bigcap\limits_{i=1}^k A^m\cap S'_i=A^m\cap(\bigcap\limits_{i=1}^k S'_i),
		\end{align*}
		also erfüllt $S':=\bigcap\limits_{i=1}^k S'_i$ das Gewünschte.
		\item Wenn $k=1$ gilt, zerlege $S$ in Zellen $(Z_i)$, deren Dimension natürlich $\leq\dim S$ ist. Wenn man da das Problem löst (induktiv bzw. von Hand) und jeweils ein passendes $S'_i$ findet, löst $\bigcup\limits_i S'_i$ das Problem für $S$.\\ Sei also $S$ jetzt schon eine Zelle.
		\begin{itemize}
			\item Wenn $n:=\dim S<m$ ist und $\pi$ die entsprechende homöomorphe Projektion auf eine offene Zelle in $B^n$  bzw. eine $\lingua$-definierbare Fortsetzung davon auf ganz $B^m$, sei $\lambda$ eine $\lingua$-definierbare Fortsetzung der Umkehrfunktion dieser Projektion. Wähle die Fortsetzung $\lambda$ dabei so, dass $\lambda(\pi(S))$ und $\lambda(B^n\setminus\pi(S))$ disjunkt sind. Das ermöglicht die Gleichheit $\lambda(C\cap D)=\lambda(C)\cap\lambda(D)$ für $C\subseteq\pi(S)$. Löse dann mit einem $\lingua$-definierbaren $S''\subseteq \pi(S)$ das Problem $$A^n\cap\pi(S)\cap\lambda^{-1}(A^m)\cap (g\circ\lambda)^{-1}(A)=A^n\cap S''.$$
			Das Problem entspricht im Übrigen den Anforderungen, weil man\linebreak $\lambda^{-1}(A^m)\cap (g\circ\lambda)^{-1}(A)$ wie im Fall $k>1$ umschreiben kann. Schneidet man das mit $\lambda^{-1}(A^m)$ und wendet darauf $\lambda$ an, erhält man (mit schrittweiser Verwendung des $\cap$-Herausziehens)
			\begin{align*}
			\lambda(A^n)\cap S\cap A^m\cap g^{-1}(A)&=\lambda(A^n\cap\pi(S)\cap\lambda^{-1}(A^m)\cap (g\circ\lambda)^{-1}(A))\\
			&=\lambda(\lambda^{-1}(A^m)\cap A^n\cap S'')\\
			&=A^m\cap\lambda(A^n)\cap\lambda(S''),
			\end{align*}
			wegen $A^m\cap S\subseteq\lambda(A^n)$ aufgrund der Projektionseigenschaft von $\pi$, kann man $\lambda(A^n)$ weglassen und erhält $$A^m\cap S\cap g^{-1}(A)=A^m\cap\lambda(S''),$$
			also löst $\lambda(S'')$ das Problem für $S$.
			\newpage
			\item Wenn $\dim S=m$, finde eine $\lingua_A$-definierbare Funktion $G:B^{m+n}\rightarrow B$ mit $g=G(\cdot,b)$ für ein über $A$ unabhängiges Tupel $b\in B^n$. Als nächstes betreiben wir Induktion über $n$. Wenn $n=0$, dann ist nichts zu tun, weil dann $g$ schon $A$-definierbar ist, also $g^{-1}(A)=A^m$ und man dann $S'=S$ wählen kann. Ansonsten zerlege $S$ in die Mengen
			\begin{align*}
			S_1:=\{x\in B^{m+n}\mid&G(x_1,\dots,x_{m+n-1},\cdot)\text{ ist streng monoton wachsend }\\&\text{auf einem Intervall um }x_n\},\\S_2:=\{x\in B^{m+n}\mid&G(x_1,\dots,x_{m+n-1},\cdot)\text{ ist streng monoton fallend }\\&\text{auf einem Intervall um }x_n\},\\S_3:=\{x\in B^{m+n}\mid&G(x_1,\dots,x_{m+n-1},\cdot)\text{ ist konstant auf einem Intervall um }x_n\}
			\end{align*}
			und den Rest $S_4:=B^{m+n}\setminus(S_1\cup S_2\cup S_3)$.
			$S_1\cup S_2\cup S_3$ ist groß, denn wenn eine offene Menge $U\subseteq S_4$ existiert, wähle $x\in U$ beliebig und ein Intervall $I$ um $x_n$ mit $\{(x_1,\dots,x_{n-1})\}\times I\subset U$. Nach der Charakterisierung o-minimaler definierbarer Funktionen existiert ein Subintervall $J\subseteq I$, sodass $G(x_1,\dots,x_{m+n-1},\cdot)$ entweder streng monoton wachsend, fallend oder konstant ist auf $J$. Also ist $x\in S_1\cup S_2\cup S_3$ im Widerspruch zu $x\in U$.
			Partitioniere diese Mengen dann noch in $A$-definierbare Zellen $(Z_i)_i$ und definiere $Z'_i:=\{x\in B\mid (x,b)\in Z_i\}$ für alle $i$. Dann ist für jede offene Zelle $G$ in der letzten Koordinate entweder streng monoton steigend, fallend oder konstant jeweils auf der ganzen Zelle; das folgt, indem offene Zellen schon Teilmenge von $S_1,S_2$ oder $S_3$ sind. Die lokale Definition dieser Mengen überträgt sich durch Supremumsbildung oder definierbaren Zusammenhang auf die gesamte Zelle.\\
			Löse das Problem jetzt für alle $(Z'_i)_i$, wegen $S:=\bigcup\limits_i Z'_i$ ist es dann auch für $S$ gelöst: Für nicht-offene Zellen geht das per Induktion bzw. genauso wie im vorigen Unterpunkt. Wenn $Z'_i$ nun eine offene Zelle ist, gilt für ein generisches Element $x$ über $A,b$, dass $(x,b)$ generisch von $B^{m+n}$ ist, also in $S_1\cup S_2\cup S_3$. Also ist $Z_i$ entweder in $S_1,S_2$ oder $S_3$ enthalten.
			\begin{itemize}
				\item Wenn $Z_i\subseteq S_3$ ist, definiere $$\tilde{G}(\overline{x})=z:\Leftrightarrow z=G(\overline{x},y)\text{ für ein }y\text{ mit }(\overline{x},y)\in Z_i,$$ dann gilt $g=\tilde{G}(\cdot,b_1,\dots,b_{n-1})$ und per Induktion kann man das Problem für $n-1$ lösen.
				\item Wenn $Z_i\subseteq S_1,S_2$, also $G$ auf $Z_i$ injektiv in der letzten Koordinate ist, wird das Problem durch $\emptyset$ gelöst: Denn sei $a\in A^m\cap S'_i\cap g^{-1}(A)$, also existiert $a'\in A$ mit $a'=g(a)=G(a,b)$, weil $a\in Z'_i$ ist, ist $(a,b)\in Z_i$, also ist wegen Injektivität von $G$ in der letzten Koordinate $b_n$ eindeutig bestimmt mit $(a,b)\in Z_i$ und $a'=G(a,b)$. Das ist aber $A,b_1,\dots,b_{n-1}$-definierbar, also ist $b$ nicht unabhängig über $A$.
			\end{itemize}
		\end{itemize}
	\end{itemize}
\end{proof}

\begin{definition}
	Sei $\fA$ eine Struktur in einer Sprache $\lingua$ und $\fB$ eine Struktur in einer Sprache $\lingua'$. Dann heißt $\fB$ Erweiterung von $\fA$, wenn $A=B$ und die Interpretation von $\lingua$ in $\fA$ schon $\lingua'$-definierbar in $\fB$ ist. TODO: Muss es sogar schon 0-definierbar sein? sonst später Problem mit dem Typ...
\end{definition}

\begin{lemma}\label{Erweiterung definierbare Mengen}
	Sei $\fA$ eine unendliche Struktur, und sei $\fB$ eine $\aleph_0$-saturierte o-minimale Erweiterung von $\fA$, , sodass Definable Choice gilt und alle in $\fB$ definierbaren Funktionen $A\rightarrow A$ schon in $\fA$ definierbar sind. Dann sind die in $\fA$ und $\fB$ definierbaren Mengen die gleichen.
\end{lemma}
TODO: Eigentlich musste $\fB$ nicht o-minimal sein, dafür $\fA$. Fehler?
\begin{proof}
	Per Definition einer Erweiterung sind alle $\fA$-definierbaren Mengen auch $\fB$-definierbar.\\
	Sei $S\subseteq A^n$ definierbar in $\fB$. Wenn $n=1$ ist, ist die charakteristische Funktion $\chi_S:A\rightarrow A$ definierbar in $\fB$, also per Voraussetzung auch in $\fA$, also ist auch $S=\{\chi_S=1\}$ definierbar in $\fA$. Wenn $0,1$ nicht in $A$ enthalten sind, muss man sich stattdessen ein Analogon mit zwei bestimmten Elementen aus $A$ basteln.\\
	Wenn $n>1$ ist, dann zerlege $S$ in $\fB$-Zellen. Es reicht daher, die Aussage für eine beliebige $\fB$-Zelle $S$ zu beweisen, genauer reicht es sogar aus, die $\fA$-Definierbarkeit für die definierende(n) partiellen Funktion(en) $x\mapsto\sup S_x,x\mapsto\inf S_x$ zu beweisen; nach wählen eines willkürlichen noch nicht angenommenen Funktionswertes, reicht es, die $\fA$-Definierbarkeit für beliebige $\fB$-definierbare Funktionen $f:A^{n-1}\rightarrow A$ zu beweisen.\\
	Wenn $n=2$ ist, gilt das auch schon per Voraussetzung. Wenn $n>2$ ist, definiere die $\fB$-definierbaren Funktionen $f_a:A^{n-2}\rightarrow A,x\mapsto f(a,x)$. Nach Induktionsvoraussetzung ist jede davon $\fA$ definierbar, sei also $f_a=F_a(c_a,\cdot)$ für 0-$\fA$-definierbare Funktionen $F_a:A^{m_a+n-2}\rightarrow A$ und passende $m_a\in\setN,c_a\in A^{n_a}$. Es ist möglich, bloß endlich viele unterschiedliche $F_a$ zu verwenden: Ansonsten ist nämlich $$\{\forall c(f(a,x)\neq F(c,x))\mid F:A^{m+n-2}\rightarrow A\ \text{0-}\fA\text{-definierbar,}m\in\setN\}$$ konsistent in $\fB$, der Erfüller davon darf nach obigen Erkenntnissen aber nicht existieren.\\
	Also existieren 0-$\fA$-definierbare Funktionen $F_i:A^{m_i+n-2}\rightarrow A$ für $i=1,\dots,k$, sodass für alle $a\in A$ ein $i\leq k$ und ein $c\in A^{m_i}$ existiert mit $f_a=F_i(c_i,\cdot)$. Für ein $b\in A$ und $z$ von der Dimension $\max\limits_i m_i$ sei
	$$F(z,y,x):=\left\{\begin{array}{ll}
	F_i(y_1,\dots,y_{m_i},x)&i\text{ ist das einzige }j\text{ mit }z_j=b\\
	b&\text{sonst}
	\end{array}\right..$$
	Dann ist $F$ definierbar in $\fA$ und es gilt, dass für alle $a\in A$ ein $(z,y)\in A^{k+\max\limits_i m_i}$ mit $f_a=F(z,y,\cdot)$. Da in $\fB$ Definable Choice gilt, existiert eine $\fB$-definierbare Funktion $g$, sodass $f(a,x)=F(g(a),x)$ gilt. Nach der Voraussetzung sind alle Koordinatenfunktionen von $g$ definierbar in $\fA$, also auch $g$ selbst, also auch $f$.
\end{proof}

\begin{theorem}
	Sei $(B,A)\models\td,\operatorname{RCF}\subseteq\operatorname{T},\setR\subseteq B$ und $X\subseteq \setR^n$. Wenn $X$ offen und $\lingua^P$-definierbar ist, ist es $\lingua$-definierbar.
\end{theorem}
\begin{remark}
	Für eindimensionale Mengen ist das trivial, denn in der Darstellung von Satz \ref{Satz 4} kann der Fall $X\cap I$ dicht und kodicht nicht auftreten, weil offene Mengen niemals kodicht sind. Also sind die definierbaren offenen Teilmengen von $B$ gerade die endlichen Vereinigungen von Intervallen und das ist $\lingua$-definierbar.
\end{remark}
\begin{proof}[Beweis des Satzes]
	Füge zunächst zu $\lingua$ $n$-stellige Relationen $O_\varphi$ für jede $\lingua^P_B$-Formel $\varphi$, die in $(B,A)$ eine offene Teilmenge von $\setR^n$ definiert, in dieser Sprache $\lingua'$ sei $\tilde{B}$ die Erweiterung von $B$ durch kanonische Interpretation als $O_\varphi(B):=\varphi((B,A))$. Nach der Bemerkung oben sind die eindimensionalen offenen $\lingua^P$-definierbaren Teilmengen von $\setR$ die Vereinigungen von Intervallen in $\setR$ und daher (TODO: warum?) folgt nach \cite{MillSpeiss}, dass $\tilde{B}$ o-minimal ist. Es sei $(D,C)$ eine $\aleph_0$-saturierte Elementarerweiterung von $(B,A)$ und $\tilde{D}$ die Erweiterung von $D$ auf $\lingua'$ durch kanonische Interpretation der $O_\varphi$. Dann muss $\tilde{B}\preceq\tilde{D}$ gelten, denn alle $\lingua'_B$-Formeln gehen auf $\lingua^P_B$-Formeln zurück und für die gilt per Konstruktion die Elementarität der Inklusion. Also ist $\tilde{D}$ auch o-minimal, daher ist jede $\lingua'$-definierbare Funktion $f:D\rightarrow D$ stückweise stetig. Weil sie dann auch $\lingua^P$-definierbar ist, gilt nach Lemma \ref{Stückweise stetige Abbildungen}, dass jede $\lingua'$-definierbare Funktion $f:D\rightarrow D$ schon $\lingua$-definierbar ist. $\tilde{D}$ ist o-minimal und hat Definable Choice, deswegen erfüllt die Erweiterung $\tilde{D}/D$ die Voraussetzungen des Lemma \ref{Erweiterung definierbare Mengen} und die definierbaren Mengen in $D$ und $\tilde{D}$ sind die gleichen.\\
	Wenn $X$ jetzt eine offene Teilmenge von $\setR^n$ ist, die durch $\chi$ in $(B,A)$ definiert werde, dann ist $X_{(D,C)}=\chi((D,C))$ eine definierbare Teilmenge in $\tilde{D}$, also definierbar in $D$ durch eine Formel $\psi(x,d)$ für ein $d\in D^m$. In $(D,C)$ gilt also $\exists y(\chi(x)\leftrightarrow\psi(x,y))$, also existiert wegen $(B,A)\preceq(D,C)$ ein $b\in B^m$ mit $X=\chi((B,A))=\psi((B,A),d)=\psi(B,d)$. Das heißt, $X$ ist definierbar in $\lingua$.
\end{proof}
\begin{corollary}
	Auch abgeschlossene $\lingua^P$-definierbare Teilmengen von $\setR$ sind $\lingua$-definierbar. Die Definition läuft in diesem Fall über das Komplement.
\end{corollary}
\begin{corollary}
	Sei $(B,A)$ wie oben und $S\subseteq\setR^n$ $\lingua^P$-definierbar. Dann
	\begin{itemize}
		\item sind $\inn S,\overline{S}$ definierbar in $B$ nach dem Satz und der Folgerung als offene bzw. abgeschlossene $\lingua^P$-definierbare Mengen.
		\item ist $S$ genau dann $\lingua$-definierbar, wenn es eine boolesche Kombination von offenen/abgeschlossenen Teilmengen von $\setR^n$ ist, von denen jede $\lingua^P$-definierbar ist. Die Rückrichtung folgt dabei aus dem Satz und der Folgerung, da dann jede einzelne Menge in der Kombination $\lingua$-definierbar ist; die Hinrichtung folgt per Zellzerlegung.
	\end{itemize}
\end{corollary}

\end{document}