%!TEX root = DieLoesungAllerMilleniumsprobleme.tex
\addcontentsline{toc}{section}{Vorwort}
\chapter*{Vorwort}
In vielen Bereichen der Algebra und Zahlentheorie betrachtet man Körpererweiterungen, wobei insbesondere die Körper $\setC,\overline{\setQ},\setR,\overline{\setQ}\cap\setR$ von Interesse sind, die die bekanntesten Modelle der Theorien ACF und RCF darstellen. Es liegt nahe, auch die Theorien der Erweiterungen $$\overline{\setQ}\subset\setC,\overline{\setQ}\cap\setR\subset\setR$$ modelltheoretisch zu untersuchen, bzw. allgemeiner echte Körpererweiterungen algebraisch abgeschlossener Körper und o-minimaler Körper.\\
Erstere sind $\omega$-stabil und vollständig, wenn man eine Charakteristik vorgibt; in einer passend gewählten Sprache haben sie außerdem Quantorenelimination, die Sprache wird hierbei so gewählt, dass die Unterstrukturen von Modellen passender Saturiertheit ein Back\&Forth-System bilden.\\
Für o-minimale Körper muss man nicht die Charakteristik vorgeben, sondern stattdessen, dass die Erweiterung dicht ist. Das weitere Vorgehen mit dem Finden einer neuen Sprache könnte man zwar analog wieder durchziehen, da wir die ursprüngliche Sprache allerdings nicht eindeutig festlegen wollen, wäre die neue Sprache aber etwas umständlich zu verstehen. Indem man die Sprache sparsam erweitert und ein etwas modifiziertes, vom Verfahren her analoges B\&F-System wählt, hat man zwar keine Quantorenelimination, aber doch eine erhebliche Reduktion der Formeln modulo der Theorie. Des weiteren erhält man über das Back\&Forth-System einige Aussagen, unter welchen Bedingungen Elemente den selben Typ über bestimmten Mengen haben. Mit einer passenden Definition von Kleinheit folgen einige interessante Aussagen über dichte Paare o-minimaler Körper:
\begin{theorem}
	Intervalle sind nicht klein.
\end{theorem}

\begin{lemma}
	Wenn $(B,A)$ ein dichtes Paar ist und $X\subseteq A^n$ eine im Paar definierbare Menge, dann ist $X=A^n\cap Z$ für eine im Redukt definierbare Menge $Z$.
\end{lemma}

\begin{theorem}
	In dichten Paaren stimmen definierbare Mengen und Funktionen bis auf kleine Mengen mit definierbaren Mengen und Funktionen im Redukt überein.
\end{theorem}

\begin{theorem}
	Definierbare Mengen in dichten Paaren lassen sich partitionieren in Punkte, Intervalle und Mengen, die dicht und kodicht in Intervallen liegen.
\end{theorem}

\begin{lemma}
	Offene und abgeschlossene definierbare Mengen in einem dichten Paar sind schon im Redukt definierbar.
\end{lemma}