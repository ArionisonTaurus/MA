%!TEX root = DieLoesungAllerMilleniumsprobleme.tex
\addcontentsline{toc}{section}{Danksagung}
\chapter*{Danksagung}
Zu aller erst möchte ich Amador Martin-Pizarro danken, nicht nur für die hervorragende Betreuung dieser Arbeit und dafür, dass er immer für Fragen ansprechbar war, sondern auch für die unzähligen Kurse, Seminare und Vorlesungen, die ich in den letzten acht Semestern bei ihm hören konnte und für die Vermittlung der Freude an Back\&Forth-Systemen.\\
Danken möchte ich außerdem meinen Eltern für ihre Unterstützung in den letzten 23 Jahren und natürlich insbesondere während des Studiums.\\
Ich danke ebenso allen Korrekturleserinnen, vor allem Meike Dünnweber und meiner Schwester Tabea.
\addcontentsline{toc}{section}{Eigenständigkeitserklärung}
\chapter*{Eigenständigkeitserklärung}
\vspace{0.8cm}
Hiermit versichere ich, Max Aaron Vollprecht, dass ich diese Masterarbeit selbstständig verfasst habe. Es wurden keine anderen als die angegebenen Quellen und Hilfsmittel benutzt.
\\ \\
Freiburg, den \today

\newpage
\thispagestyle{empty}
\mbox{}
\newpage

\addcontentsline{toc}{section}{Vorwort}
\chapter*{Vorwort}
In vielen Bereichen der Algebra und Zahlentheorie betrachtet man Körpererweiterungen, wobei insbesondere die Körper $\setC,\overline{\setQ},\setR$ und $\overline{\setQ}\cap\setR$ von Interesse sind, die die bekanntesten Modelle der Theorien ACF und RCF darstellen. Es liegt nahe, auch die Theorien der Erweiterungen $$\overline{\setQ}\subset\setC,\overline{\setQ}\cap\setR\subset\setR$$ modelltheoretisch zu untersuchen, bzw. allgemeiner echte Körpererweiterungen algebraisch abgeschlossener Körper und o-minimaler Körper.\\
Erstere sind $\omega$-stabil und vollständig, wenn man eine Charakteristik vorgibt. In einer Sprache, die die lineare Abgebra der Körpererweiterung beschreibt, haben sie außerdem Quantorenelimination; die Sprache wird hierbei so gewählt, dass die Unterstrukturen von Modellen passender Saturiertheit ein Back\&Forth-System bilden.\\
Für o-minimale Körper muss man nicht die Charakteristik vorgeben, sondern stattdessen, dass die Erweiterung dicht ist. Das weitere Vorgehen mit dem Finden einer neuen Sprache wäre zwar wieder möglich; da wir die ursprüngliche Sprache allerdings nicht eindeutig festlegen wollen, wäre die neue Sprache aber etwas umständlich zu verstehen. Indem man die Sprache sparsam erweitert und ein etwas modifiziertes, vom Verfahren her analoges B\&F-System wählt, hat man zwar keine Quantorenelimination, aber doch eine erhebliche Reduktion der Formeln modulo der Theorie. Des weiteren erhält man über das Back\&Forth-System einige Aussagen, unter welchen Bedingungen Elemente den selben Typ über bestimmten Mengen haben. Mit einer passenden Definition von Kleinheit folgen interessante Aussagen über echte dichte Paare o-minimaler Körper:
\begin{satzleer}
	Intervalle sind nicht klein.
\end{satzleer}

\begin{satzleer}
	Wenn $(B,A)$ ein dichtes Paar ist und $X\subseteq A^n$ eine im Paar definierbare Menge, dann ist $X=A^n\cap Z$ für eine im Redukt definierbare Menge $Z$.
\end{satzleer}

\begin{satzleer}
	In echten dichten Paaren stimmen definierbare Mengen und Funktionen bis auf kleine Mengen mit definierbaren Mengen und Funktionen im Redukt überein.
\end{satzleer}

\begin{satzleer}
	Definierbare Mengen in echten dichten Paaren lassen sich partitionieren in Punkte, Intervalle und Mengen, die dicht und kodicht in Intervallen liegen.
\end{satzleer}

\begin{satzleer}
	Offene und abgeschlossene definierbare Mengen in einem echten dichten Paar sind schon im Redukt definierbar.
\end{satzleer}

Der letzte Satz wird hier nur mit Abstrichen bewiesen werden können, mehr dazu findet sich dann in dem entsprechenden Kapitel.