%!TEX root = DieLoesungAllerMilleniumsprobleme.tex
\chapter{Dichte Paare o-minimaler Strukturen}
\section{Allgemeine Betrachtungen und Anforderungen an die Theorien}

Viele Techniken und Gedanken aus dem vorigen Kapitel werden jetzt auf o-minimale Theorien übertragen: Im Folgenden halten wir eine vollständige o-minimale Theorie T mit dichter Ordnung in der Sprache $\lingua$ fest und betrachten die Theorie $\tq$ in der Sprache $\lingua_P:=\lingua\cup\{P(x)\}$, sodass die Modelle von $\tq$ Modelle von T sind und in jedem Modell $\fM$ die Menge $P(M)$ ebenfalls Modell von T ist. Schreibe so ein Paar dann als $(B,A)$ mit $A=P(B)$.\\
Wir setzen voraus, dass T RCF erweitert, insbesondere gilt Definable Choice. Dann sind die Skolemfunktionen definierbar und \OE\ ist $\lingua$ schon so eine definitorische Erweiterung, dass $T$ Quantorenelimination hat und universell axiomatisierbar ist (wobei bei einzelnen Theorien die Frage interessant wäre, welche Skolemfunktionen man dafür überhaupt hinzufügen muss). TODO: braucht man das? Außerdem seien alle Modelle genug saturiert, dass die üblichen Rechenregeln für die Dimension gelten.\\
Aus T universell mit Quantorenelimination folgt, dass Unterstrukturen von Modellen von T schon elementare Unterstrukturen sind. Also ist für jede Teilmenge $S$ eines Modells $\dcl(S)$ schon eine elementare Substruktur; zur Vereinfachung bezeichne in Zukunft $AB:=\dcl(A\cup B)$ für zwei Teilmengen $A,B$ eines Modells.\\
$P$ beschreibt also eine elementare Unterstruktur, mit $\td$ wird nun die Theorie beschrieben, die ausdrückt, dass $P$ eine dichte echte Unterstruktur ist (diese zwei Sachen oder deren Gegenteil müssen auf jeden Fall von der Theorie beschrieben werden, wenn sie vollständig sein soll). Klar ist dann, dass Unterstrukturen von $\td$ automatisch Modelle von $\tq$ sind.\\
Der Arbeit \cite{VanDenDries} folgend, wird die Vollständigkeit und eine Art von Quantorenelimination für $\td$ gezeigt, die mittels eines Back\&Forth-Systems bewiesen wird, bevor genauer auf einige Eigenschaften von definierbaren Mengen und Abbildungen in $\td$ eingegangen wird. Zunächst ist aber eine genauere Betrachtung von sogenannten kleinen Mengen vonnöten.

\section{Kleine Mengen}
\begin{definition}
	Sei $(B,A)\models\td$, dann ist eine $\lingua_P$-definierbare Menge $S\subseteq B$ \textbf{klein}, wenn eine $\lingua$-definierbare Funktion $f:B^n\rightarrow B$ existiert mit $S\subseteq f(A^n)$.
\end{definition}

\newpage
In dem folgenden Lemma meint $+,\cdot$ nicht die Operationen aus $\lingua$, sondern neue, beliebige Operationen.

\begin{lemma}\label{Hilfsaussage Kleinheit}
	Seien $A\prec B\models\operatorname{T},\ f:B^{n+1}\rightarrow B\ A\text{-definierbar},\ b\in B\setminus A,\ \beta,\gamma\in A\cup\{\pm\infty\}$ mit $\beta<b<\gamma$ und einer angeordneten $A$-definierbaren Körperstruktur $(\cdot,+)$ auf $(\beta,\gamma)=:I$. Dann existieren $a_0,\dots,a_n\in I_A$ mit $$a_nb^n+a_{n-1}b^{n-1}+\dots,a_0\in I\setminus f(A^n\times\{b\}).$$
\end{lemma}
\begin{proof}
	Wenn die Aussage nicht gilt, dann gilt mit $p(x,y):=x_ny^n+x_{n-1}y^{n-1}+\dots,x_0$, dass für jedes $a\in (I_A)^{n+1}$ ein $\alpha\in A^n$ existiert mit $p(a,b)=f(\alpha,b)$. Es muss für festes $a\in I_A$ ein Intervall um $b$ in $I_A$ mit dieser Eigenschaft geben, denn sonst wäre $b\in\dcl(A)=A$.\\
	Sei jetzt $a$ nicht mehr fixiert, dann existiert mit Definable Choice in $I_A$ eine definierbare Zuordnung $a\mapsto\alpha(a)$, sodass $p(a,\cdot)=f(\alpha(a),\cdot)$ auf einem Intervall gilt. Da jedes $a$ $n+1$ viele Einträge hat und jedes $\alpha(a)$ $n$ viele, müssen unendlich viele $a\in (I_A)^{n+1}$ existieren, die durch $\alpha$ auf das selbe Element abgebildet werden. Denn wenn das nicht so wäre, wäre ein generisches Element aus $(I_A)^{n+1}$ algebraisch über einem Element aus $A^n$, was der Generizität widerspricht. Da es unendlich viele Elemente gibt, sodass $\alpha$ auf ihnen konstant ist, gibt es schon eine Zelle von Dimension $>0$ mit der Eigenschaft und damit insbesondere eine Zelle $E$ von Dimension 1 (Als Teilmenge einer Zelle lässt sich immer eine von kleinerer Dimension finden). Nenne den konstanten Wert dann $\alpha^*$.\\
	Da also gilt: Für alle $a\in E$ existiert ein Intervall $J$ mit $p(a,\cdot)=f(\alpha^*,\cdot)$ auf $J$; existieren mit Definable Choice $\beta^*,\gamma^*:E\rightarrow I_A$, sodass $p(a,\cdot)=f(\alpha^*,\cdot)$ auf $(\beta^*(a),\gamma^*(a))$ gilt. \OE\ seien $\beta^*$ und $\gamma^*$ jetzt schon stetig auf $E$ und ein $e\in E$ beliebig· Dann existiert für $\varepsilon$ hinreichend klein eine $E$-Umgebung $U$ um $e$, sodass $$\beta^*<\frac{1}{2}(\beta^*(e)+\gamma^*(e))-\varepsilon,\ \frac{1}{2}(\beta^*(e)+\gamma^*(e))+\varepsilon<\gamma^*$$ auf $U$, also $$p(a,x)=f(\alpha^*,x)\text{ für alle }a\in U,x\in(\frac{1}{2}(\beta^*(e)+\gamma^*(e))-\varepsilon,\frac{1}{2}(\beta^*(e)+\gamma^*(e))+\varepsilon)$$ gilt. Es kann aber nicht $p(a-a',x)=p(a,x)-p(a',x)=f(\alpha^*,x)-f(\alpha^*,x)=0$ für $a,a'\in U$ verschieden und unendlich viele $x$ sein, weil ein Nichtnullpolynom nicht unendlich viele Nullstellen haben kann.
\end{proof}

\newpage

\begin{corollary}
	Es sei $(A,B)\models\td,\ f:B^{n+1}\rightarrow B$ $A$-definierbar in $B$ und $b\in B\setminus A$. Dann enthält $f(A^n\times\{b\})$ kein Intervall um $b$.
\end{corollary}
\begin{proof}
	Nimm an, dass das Gegenteil gelte für das Intervall $J$ (\OE\ mit Randpunkten $c<d\in A$): Dann ist $x\mapsto\frac{1}{c-x}+\frac{1}{d-x}$ eine ordnungstreue $A$-definierbare Bijektion $(c,d)\rightarrow A$. Diese erzeugt in $A$ eine definierbare angeordnete Körperstruktur auf $(c,d)=J_A$, also auch auf $J$ eine $A$-definierbare angeordnete Körperstruktur. Mit dem vorigen Lemma existiert ein Element aus $J\setminus f(A^n\times\{b\})$.
\end{proof}

\begin{theorem}\label{Kleinheit}
	Wenn $(B,A)\models\td$, dann ist kein Intervall eine kleine Teilmenge.
\end{theorem}
\begin{proof}
	Sei $f:B^n\rightarrow B$ eine durch $\varphi(x,y,b)$ definierbare Abbildung mit $\varphi$ eine\linebreak$\lingua_A$-Formel und $b\in B^m$ für ein $m\in\setN$ definiert. Für $\dim(b/A)=0$ ist $f(A^n)\subseteq A$ klar, deswegen sei \OE\ $\dim(b/A)\geq1$. Definiere
	\begin{align*}
	g(x,z):=\left\{\begin{array}{ll}
	\text{das eindeutige }y\in B &\text{für alle z, für die }\varphi(x,y,z)\\
	\text{ mit }B\models\varphi(x,y,z) &\text{ bei festem }\text{ eine Funktion definiert}\\
	\ &\ \\
	0 &\text{sonst}
	\end{array}\right.,
	\end{align*}
	Dann ist $g$ in $B$ $A$-definierbar und $g(\cdot,b)=f$. Falls $\dim(b/A)>1$, füge genug Komponenten von $b$ zu $A$ hinzu, sodass $\dim(b/A)=1$. Das Hinzufügen einer Komponente $b_j$ ändert nichts, denn $Ab_j$ ist nach den Eingangsbemerkungen Modell von T und $Ab_j$ ist erst recht dicht in, aber nicht gleich $B$ (sonst hätte man die Dimension mit diesem Schritt schon zu sehr verkleinert).\\
	Finde also $b_i$, sodass $A$-definierbare $(h_j)$ existieren mit $b_j=h_j(b_i)$ für alle $j$. Wenn jetzt $J\subseteq f(A^n)=g(A^n,b)=g(A^n,h(b_i))$ für ein Intervall $J$, dann widerspricht das der Aussage des letzten Lemmas für die Funktion $(x,y)\mapsto g(x,h(y))$.
\end{proof}

\begin{definition}
	Schreibe ab jetzt $P(\overline{x}):=\bigwedge\limits_{i=1}^\abs{x}P(x_i)$.
\end{definition}

\newpage

\begin{lemma}
	Wenn $(B,A)$ für ein unendliches $\kappa>\abs{\operatorname{T}}$ ein $\kappa$-saturiertes Modell von $\td$ ist, ist $\dim(B/A)\geq\kappa$.
\end{lemma}
\begin{proof}
	Sei $S$ eine Basis von $B/A$ mit $\abs{S}<\kappa$; zeige nun, dass es kein Erzeugendensystem sein kann. Das folgt aus der Saturation angewandt auf den partiellen Typen $$\{\forall\overline{y}\in P(x\neq t(\overline{y}))\mid t\ \lingua_S\text{-Term}\},$$ der endlich erfüllbar ist, weil die Negation jeder dieser Formeln \glqq{}$x$ ist in einer kleinen Menge\grqq{} impliziert. Wenn der Typ also nicht endlich erfüllbar wäre, würde eine endliche Vereinigung von kleinen Mengen ganz $B$ überdecken. Das kann aber nicht gelten, denn eine endliche Vereinigung von kleinen Mengen ist wieder klein (in der das bezeugenden Abbildung kann man das durch Erhöhen der Dimension des Definitionsbereiches und Fallunterscheidung über eine Koordinate beweisen). TODO: ausführen!
\end{proof}

\begin{corollary}\label{Finden transz Elte}
	Da Intervalle nicht klein sind, zeigt der Beweis sogar, dass in einem $\kappa$-saturiertem Modell $(B,A)\models\td$, gegeben Menge, $S,S',S''\subset B$ mit $\abs{S},\abs{S'},\abs{S''}<\kappa$, ein transzendentes Element $b$ über $SA$ gefunden werden kann mit $a<b$ für alle $a\in S'$ und $b<c$ für alle $c\in S''$, sofern dieser Ordnungstyp von $b$ überhaupt konsistent ist.
\end{corollary}

\newpage

\section{Formelreduzierung in $\td$}
In diesem Abschnitt wird gezeigt, dass sich $\lingua_P$-Formeln modulo $\td$ sehr stark vereinfachen lassen. Indem dieses mit einem Back\&Forth-System gezeigt wird, erhält man zusätzlich eine sehr große Klasse von elementaren Abbildungen zwischen Modellen von $\td$.\\
In diesem Kontext wird wieder die $\acl$-Unabhängigkeit in einem Modell von T relevant, die im ersten Kapitel mit \glqq{}algebraisch disjunkt\grqq{} bezeichnet wurde. Man kann sich dafür folgende (teilweise schon bekannte) Fakten überlegen.

\begin{lemma}\label{Unabhängigkeitsregeln}
	Seien $A,B,C,D$ Mengen in irgendeinem Modell von $T$.
	\begin{enumerate}
		\item Wenn $A$ und $B$ unabhängig über $C$ sind, sind $B$ und $A$ unabhängig über $C$ und $A\cap B\subseteq\acl(C)$ (in fast allen betrachteten Fällen wird sowieso $A,B\supseteq C$ und $C=\acl(C)$ gelten).
		\item Wenn $A$ und $B$ unabhängig über $C$ sind und $S\subseteq B$, dann sind auch $A\cup S$ und $B$ unabhängig über $C\cup S$.
		\item Wenn $A$ und $B$ unabhängig über $C$ sind, $A\subseteq S\subseteq\acl(A),B\subseteq S'\subseteq\acl(B)$, dann sind $S$ und $S'$ unabhängig über $C$.
		\item Wenn $A$ und $B$ unabhängig über $C$ sind und $D$ (algebraisch) unabhängig über $AB$, dann sind $A\cup D$ und $B$ unabhängig über $C$.
		\item Wenn $(D,C)\preceq(B,A)\models\tq$, dann sind $A$ und $D$ unabhängig über $C$.
		\item Wenn $(D,C)\subseteq(B,A)\models\tq,\ S\subseteq A$ und $A$ und $D$ unabhängig über $C$ sind, dann sind $A$ und $DS$ unabhängig über $CS$, $\langle D\cup S\rangle_{\lingua_P}=(DS,CS)$ und $$(D,C)\subseteq(DS,CS)\subseteq(B,A).$$
		\item Wenn $(D,C)\subseteq(B,A)\models\tq$ und $S\subseteq B$ unabhängig über $DA$ ist, dann sind $A$ und $DS$ unabhängig über $C$, $\langle D\cup S\rangle_{\lingua_P}=(DS,C)$ und $$(D,C)\subseteq(DS,C)\subseteq(B,A).$$
	\end{enumerate}
\end{lemma}
\newpage
\begin{proof}
	1.-4. sind bekannt.
	\item[5.] Wenn $\overline{d}\in D$ algebraisch unabhängig über $C$ ist, aber nicht über $A$, dann existiert eine $\lingua_A$-Formel $\varphi(\overline{x},\overline{a})$, sodass \OE\ $d_1$ von $\varphi(x_1,d_2,d_3\dots,\overline{a})$ algebraisiert wird (\OE\ wird $d_1$ schon durch $\varphi$ definiert). Also erfüllt $\overline{d}$ die $\lingua_P$-Formel $$\exists \overline{y}\in P(\varphi(\overline{x},\overline{y})\land\forall z_2,z_3,\dots\exists! z_1(\varphi(\overline{z},\overline{y})))$$ in $(B,A)$, also auch in $(D,C)$. Es existiert also $\overline{c}\in C$ mit $$B\models\varphi(\overline{d},\overline{c})\land\forall z_2,z_3,\dots\exists! z_1(\varphi(\overline{z},\overline{c})),$$ was im Widerspruch zur Unabhängigkeit von $\overline{d}$ über $C$ steht.
	\item[6.] Dass $A$ und $DS$ unabhängig über $CS$ sind, ergibt sich in der Kombination von 2. und dann 3.\\
	Dass die Trägermenge von $\langle D\cup S\rangle_{\lingua_P}$ die Menge $DS$ ist, ergibt sich direkt per Definition als $DS=\dcl(D\cup S)=\langle D\cup S\rangle_\lingua$. Weil $A$ und $DS$ unabhängig über $CS$ sind, folgt $$P(\langle D\cup S\rangle_{\lingua_P})=DS\cap P(B)=DS\cap A=CS.$$
	\item[7] Es ergibt sich aus 4. und 3. dass $A$ und $DS$ unabhängig über $CS$ sind. Der Rest geht analog zu 6.
\end{proof}

Zu bemerken ist, dass ein Spezialfall von Unabhängigkeit viele nützliche Eigenschaften hat. Auf diesen wird später noch oft zurückgegriffen werden.
\begin{definition}
	Seien $(D,C)\subseteq(B,A)$ zwei Modelle von $\tq$. Dann heiße diese Inklusion \textbf{frei}, wenn $D$ und $A$ unabhängig über $C$ sind.
\end{definition}

\begin{lemma}\label{Kodichte von A}
	Sei $(B,A)\models\td$. Dann ist $A$ auch kodicht in $B$.
\end{lemma}
\begin{proof}
	Zu zeigen ist, dass für alle $a,c\in B$ ein $b\in B\setminus A$ existiert mit $a<b<c$. Durch Translation und additive Inversion kann man annehmen, dass $a=0$. Wähle jetzt ein $d\in B\setminus A$ beliebig und $e\in A$ mit $d-c<e<d$. Dann ist $d-e$ nicht in $A$ (denn sonst wäre es $d$) und $0=e-e<d-e<d-(d-c)=c$.
\end{proof}

\newpage
Für die Konstruktion des gewünschten Back\&Forth-Systems sei $\kappa>\abs{T}$ eine beliebige, aber feste Kardinalzahl und $(B,A),(D,C)\models\td$ zwei $\kappa$-saturierte Modelle.
\begin{theorem}\label{BackForth}
	Sei $S$ die Menge aller partiellen Isomorphismen zwischen Unterstrukturen $(B',A')$ von $(B,A)$ und $(D',C')$ von $(D,C)$ der Mächtigkeit $<\kappa$, sodass die Inklusionen frei sind. Dann bildet $S$ ein nichtleeres B\&F-System und $\td$ ist insbesondere vollständig.
\end{theorem}
\begin{proof}
	Das System ist nichtleer, denn es gibt ein Primmodell $\fM$ von $T$, weil $T$ vollständig ist und in jedem Modell $A$ alle Eigenschaften von $\fM_A:=\langle\emptyset\rangle_\lingua$ in $T$ beschrieben werden. Klarerweise ist $\abs{M}=\abs{T}<\kappa$. Der Isomorphismus $(\fM_A,\fM_A)\cong(\fM_C,\fM_C)$ liegt in $S$, denn Unabhängigkeit ist bei zwei gleichen Mengen offensichtlich.\\
	Sei jetzt $S\ni i:(B',A')\rightarrow(D',C')$ und $b\in B$. Wenn $b\in B'$ ist, ist nichts zu zeigen. Wenn $b\in A\setminus B'$, betrachte den partiellen Typ über $D'$ $$\{\alpha<x\mid i^{-1}(\alpha)<b\}\cup\{x<\beta\mid b<i^{-1}(\beta)\}\cup\{P(x)\}.$$
	Dieser ist konsistent, da $i$ ein Isomorphismus ist und $C$ dicht in $D$; mit Saturation existiert ein $d\in C\setminus D'$ mit diesem Ordnungstyp. $i$ setzt sich dann eindeutig zu einem Isomorphismus $i':(B'b,A'b)\rightarrow(D'd,C'd)$ mit $i(b)=d$ fort, der gegeben ist durch die Abbildung $t(b)\mapsto i(t)(d)$ für $t$ einen $\lingua_{B'}$-Term und $i(t)$ den durch $i$ geshifteten Term. Die Surjektivität dieser Abbildung ist klar, ebenso dass $i'(A'b)=C'd$. Wohldefiniertheit, Injektivität und Isomorphismuseigenschaft gelten, denn:\\
	$Rt_1(b)\dots t_n(b)$ gilt für $\lingua_{B'}$-Terme $t_1,\dots,t_n$ und eine Relation $R$ genau dann, wenn es ein $B'$-definierbares Intervall $I$ um $b$ mit dieser Eigenschaft gibt (denn sonst wäre $b$ definierbar über $B'$ und somit in $B'$). Schickt man $I\cap B'$ mit $i$ nach $J:=i(I\cap B')$, so gilt für alle Elemente $z\in J$, dass $Ri(t_1)(z)\dots i(t_n)(z)$, da $i$ ein Isomorphismus ist. Wäre jetzt nicht $Ri(t_1)(d)\dots i(t_n)(d)$, so gäbe es ein $D'$-definierbares Intervall $I'$ um $d$, sodass das nicht gilt; insbesondere ist $I'$ disjunkt zu $J$. Allerdings ist $$d\in I'\cap\operatorname{convex}(J)=I'\cap(i(\inf I),i(\sup I)),$$ also können $I'$ und $J$ nicht disjunkt sein. Es gilt also $Ri(t_1)(d)\dots i(t_n)(d)$.\\
	Die Rückrichtung geht analog.\\
	Zu zeigen ist nun, dass $B'b$ und $A$ frei über $A'b$ sowie $D'd$ und $C$ frei über $C'd$ sind, ebenso zu zeigen ist noch, dass $(B'b,A'b)\subseteq(B,A),(D'd,C'd)\subseteq(D,C)$. Das alles folgt aber aus Lemma \ref{Unabhängigkeitsregeln} (6.). Außerdem gilt $\abs{D'd}=\abs{B'b}=\abs{B'}+\abs{T}<\kappa$.\\
	Sei jetzt $b\in B'A\setminus(A\cup B')$. Dann gibt es $\overline{a}\in A$ mit $b\in B'\overline{a}$. Erweitere wie schon bekannt $i$, sodass $\overline{a}\in\operatorname{dom}(i)$; dann ist schon ganz $B'\overline{a}\subseteq\operatorname{dom}(i)$, also auch $b$.\\
	Abschließend sei $b\in B\setminus B'A$;  wie oben erfülle dann den mit $i$ geshifteten Ordnungstyp von $b$ über $B'$ mit einem Element $d\in D\setminus D'C$ (mit Folgerung \ref{Finden transz Elte} geht das). Wie oben kann $i$ dann auf einen Isomorphismus $(B'b,A')\rightarrow(D'd,C')$ fortgesetzt werden und nach Lemma \ref{Unabhängigkeitsregeln} (7.) erfüllen $(B'b,A'),(D'd,C')$ auch die hinreichenden Eigenschaften.
\end{proof}

Dieses B\&F-System beweist die Formelreduzierung in $\td$.
\begin{theorem}
	Jede $\lingua_P$-Formel ist modulo $\td$ äquivalent zu einer booleschen Kombination von Formeln der Gestalt
	$$\exists\overline{y}\in P(\phi(\overline{x},\overline{y}))$$
	für $\phi$ eine $\lingua$-Formel. Nenne eine solche boolesche Kombination eine \textbf{gute Formel} und eine Formel der Gestalt wie beschrieben eine \textbf{gute Formel in Reinform}.
\end{theorem}
\begin{proof}
	\underline{Hilfsaussage:}\\
	Es reicht zu zeigen, dass für alle Modelle $(B,A),(D,C)\models\td$ und für alle $b\in B^n,d\in D^n$ gilt: Wenn $b$ und $d$ dieselben guten Formeln erfüllen, sind ihre Typen in $(B,A)$ und $(D,C)$ dieselben.\\
	Dass dies ausreicht, erkennt man mit dem Ziegler'schen Trennungslemma: Sei $\psi\in\fF_n(\lingua_P)$ nicht äquivalent zu einer guten Formel und nenne die Menge aller guten Formeln in $n$ freien Koordinaten $K$. Dann ist $K$ abgeschlossen unter $\land,\lor$ und enthält $\top,\bot$. Wenn $\psi$ nicht äquivalent zu einer Formel aus $K$ ist, sind $\td\cup\{\psi\}$ und $\td\cup\{\neg\psi\}$ nicht durch $K$ trennbar, also existieren $(B,A),(D,C)\models\td,b\in B^n,d\in D^n$, sodass $(B,A)\models\psi(b)$ und $(D,C)\models\neg\psi(d)$, aber $(B,A)\models\chi(b)$ genau dann, wenn $(D,C)\models\chi(d)$ für alle $\chi\in K$. Dann erfüllen $b$ und $d$ dieselben guten Formeln, aber haben nicht denselben Typ - ein Widerspruch!\\
	\begin{proof}[Beweis der Hilfsaussage]
		Seien $b,d$ wie verlangt und $(B,A),(D,C)$ schon \OE\linebreak $\abs{T}^+$-saturiert (das ändert nichts an Typen und dem Erfüllen von guten Formeln). Sei $a\in A^m$ für ein hinreichend großes $m$, mit der Eigenschaft dass $\dim(b/a)\leq\dim(b/A)$ (es folgt dann Gleichheit, da über einer kleineren Menge nicht mehr interdefinierbar werden kann). Für $A':=\dcl(a),B':=\dcl(a,b)$ gilt dann, dass $A$ und $B'$ unabhängig über $A'$ sind. Es sind nämlich per Definition von $a$ die Mengen $A$ und $b$ unabhängig über $a$ (eben wegen $\dim(b/a)=\dim(b/A)$), mit Lemma \ref{Unabhängigkeitsregeln} (2.) sind dann auch $A$ und $b\cup(A')$ unabhängig über $A'$ und mit 3. sind $A$ und $B'=\dcl(b\cup A')$ unabhängig über $A'$. Außerdem sind $A'$ und $B'$ maximal $\abs{T}$ groß.\newpage
		Wenn man den partiellen $\lingua_P$-Typ $\tp_\lingua(a/b)\cup\{P(\overline{x})\}$ betrachtet, bleibt er konsistent unter der Ersetzung $b\rightarrowtail d$ in den Formeln. Seien nämlich $\psi_1(\overline{x},b),\dots,\psi_n(\overline{x},b)\in\tp_\lingua(a/b)$, dann ist $$\exists\overline{x}\in P(\bigwedge\limits_{i=1}^n\psi_i(\overline{x},\overline{y}))$$ eine gute Formel, die von $b$ und daher auch von $d$ erfüllt wird. Also ist der ersetzte partielle Typ endlich konsistent, wegen Saturation habe er den Erfüller $c\in C$ und es gilt $\tp_\lingua(a,b)=\tp_\lingua(c,d)$. Wegen der Typengleichheit folgt insbesondere $\dim(b/a)=\dim(d/c)$; es bleibt noch zu zeigen, dass $\dim(b/A)=\dim(d/C)$, damit dann gilt $\dim(d/C)=\dim(b/A)=\dim(b/a)=\dim(d/c)$ und wie oben $C$ und $D':=\dcl(c,d)$ frei über $C':=\dcl(c)$ sind. Die Gleichheit $\dim(b/A)=\dim(d/C)$ gilt aber, da für jede $\lingua$-Formel $\psi$ und $j_1,\dots,j_n\in\setN$ die Formel zu $$\glqq{}\text{es existiert }\overline{y}\in P\text{, sodass }\psi(\overline{x},\overline{y})\ x_i\text{ über }x_{j_1},\dots,x_{j_m}\text{ definiert}\grqq{}$$ eine gute Formel ist, die also genau dann von $b$ erfüllt wird, wenn sie von $d$ erfüllt wird.\\
		Da $(a,b)$ und $(c,d)$ den gleichen $\lingua$-Typ haben, gibt es einen partiellen Isomorphismus $i$ von $B'=\dcl(a,b)$ nach $D'=\dcl(c,d)$ mit $i((a,b))=(c,d)$, die Einschränkung auf $A'=\dcl(a)$ bildet einen Isomorphismus nach $C'=\dcl(c)$. Also ist $i$ partieller Isomorphismus $(B,A)\rightarrow(D,C)$, damit im B\&F-System, also elementare Abbildung, weswegen $b$ und $d$ denselben $\lingua_P$-Typen haben.
	\end{proof}
\end{proof}

\begin{corollary}\label{Definierbarkeit aus A}
	Für ein dichtes Paar $(B,A)$ und $S\subseteq B^n$ eine $A_0$-definierbare Menge in $\lingua_P$ (wobei $A_0\subseteq A$) ist $S\cap A^n$ eine $A_0$-definierbare Menge in $\lingua$.
\end{corollary}
\begin{proof}
	Nach der Formelreduzierung sei $S$ \OE\ durch eine gute Formel definiert. Da die Definierbarkeit abgeschlossen unter booleschen Kombinationen ist, reicht es, eine Formel in Reinform zu betrachten.\newpage
	Da aber für jede $\lingua_{A_0}$-Formel $\varphi(x,y,a')$ und jedes $a\in A^n$ die Aussagen $$\glqq{}\text{Es existiert ein }y\in A^m\text{ mit }(B,A)\models\varphi(a,y,a')\grqq{},$$ $$\glqq{}\text{Es existiert ein }y\in A^m\text{ mit }B\models\varphi(a,y,a')\grqq{},$$ $$\glqq{}\text{Es existiert ein }y\in A^m\text{ mit }A\models\varphi(a,y,a')\grqq{}$$ äquivalent sind wegen $\varphi$ als $\lingua$-Formel und $A\prec B$, folgt, dass $$\exists y\in P(\varphi(x,y,a'))(B)\cap A^n=\exists y(\varphi(x,y,a'))(A).$$
\end{proof}

\section{Folgen der Existenz des B\&F-Systems}
Im Folgenden werden einige Anordnungen von wechselseitigen Inklusionen von Modellen von T betrachtet, in der Gleichheit von bestimmten Typen folgt.

\begin{lemma}\label{freie Inklusionen}
	Für dichte Paare $(B,A),(D,C)$ mit $(D,C)\subseteq(B,A)$ sind folgende Eigenschaften äquivalent:
	\begin{enumerate}
		\item $(D,C)\preceq(B,A)$
		\item Die Inklusion ist frei.
	\end{enumerate}
\end{lemma}
\begin{proof}
	$\glqq{}1.\Rightarrow2.\grqq{}:$ Diese Richtung ist schon aus Lemma \ref{Unabhängigkeitsregeln} (5.) bekannt.\\
	$\glqq{}2.\Rightarrow1.\grqq{}:$ Finde $(\abs{B}+\abs{T})^+$-saturierte Strukturen $$(B,A)\preceq(B',A'),(D,C)\preceq(D',C');$$ es ist dann $(D,C)$ eine gemeinsame Unterstruktur und $(D,C)\subseteq(D',C')$ ist frei nach dem Beweis der Gegenrichtung. Außerdem sind nach Voraussetzung $D$ und $A$ unabhängig über $C$, da aber Unabhängigkeit von Tupeln in $D$ über $A$ auch über $A'$ erhalten bleibt (da $(B',A')$ elementare Oberstruktur), ist auch $(D,C)\subseteq(B',A')$ frei. Also ist die Identität auf $(D,C)$ im Back\&Forth-System, daher elementare Abbildung. Daraus folgt für alle $(\lingua_P)_D$-Formeln $\varphi$, dass $$(D,C)\models\phi\Leftrightarrow(D',C')\models\varphi\Leftrightarrow(B',A')\models\varphi\Leftrightarrow(B,A)\models\varphi.$$
\end{proof}

\begin{lemma}\label{Gemeinsame Unterstruktur}
	Seien $(B_1,A_1),(B_2,A_2)\models\td$ und $(B,A)$ eine gemeinsame Unterstruktur, sodass die Inklusionen frei sind. Wenn $a\in (A_1)^n$ und $b\in (A_2)^n$ denselben $\lingua$-Typen über $B$ erfüllen, erfüllen sie auch denselben $\lingua_P$-Typen über $B$.
\end{lemma}
\begin{proof}
	\OE\ seien $(B_1,A_1)$ und $(B_2,A_2)$ schon genügend saturiert, das ändert nichts an Typen über $B$ und (nach derselben Argumentation wie im vorigen Lemma) auch nichts an der Unabhängigkeit. Da $a$ und $b$ denselben Typen über $B$ erfüllen, kann man wieder $\lingua_B$-Terme mit eingesetztem $a$ auf $\lingua_B$-Terme mit eingesetztem $b$ abbilden (Wohldefiniertheit und Injektivität wird durch die Typengleichheit ermöglicht) und bekommt einen partiellen Isomorphismus $i:Ba\cong Bb$, dessen Einschränkung auf die $\lingua_A$-Terme einen partiellen Isomorphismus $Aa\cong Ab$ induziert und sodass $i(a)=b$. Also gilt $i:(Ba,Aa)\cong(Bb,Ab)$, da außerdem die Inklusionen $(Ba,Aa)\subseteq(B_1,A_1)$ und $(Bb,Ab)\subseteq(B_2,A_2)$ frei sind nach Lemma \ref{Unabhängigkeitsregeln} (6.), ist $i$ im Back\&Forth-System, also elementar, also haben $a$ und $b$ denselben $\lingua_P$-Typen über $B$.
\end{proof}

\begin{corollary}
	Wenn man sich solch ein Paar $(a,b)$ beliebig wählt (z.B. $a=b=0$), sind in dem Typen auch die parameterfreien $(\lingua_P)_B$-Formeln, die in $(B_1,A_1)$ bzw. $(B_2,A_2)$ gelten. Also gelten dieselben Formeln, was als $(B_1,A_1)\equiv_B(B_2,A_2)$ geschrieben wird.
\end{corollary}

\begin{lemma}\label{selber Schnitt}
	Seien $(B_1,A_1),(B_2,A_2)$ zwei dichte Paare und $A\subseteq A_1\cap A_2$ eine gemeinsame Substruktur, sowie $a\in B_1\setminus A_1,\ b\in B_2\setminus A_2$, die den gleichen Ordnungstyp über $A$ haben. Dann haben $a$ und $b$ sogar den gleichen $\lingua_P$-Typ über $A$.
\end{lemma}
\begin{proof}
	Es sind trivialer Weise $A_i$ und $A$ unabhängig über $A$ für $i=1,2$, außerdem ist $a$ transzendent über $A_1$ und $b$ transzendent über $A_2$. Nach Lemma \ref{Unabhängigkeitsregeln} (4.) sind also die Einbettungen $(Aa,A)\subseteq(B_1,A_1)$ und $(Ab,A)\subseteq(B_2,A_2)$ frei. Nach dem Beweis zu Satz \ref{BackForth} gibt es also einen Isomorphismus $Aa\cong Ab$, der $A\cong A$ fortsetzt und für den $i(a)=b$ gilt, also gibt es einen Isomorphismus $i:(Aa,A)\cong(Ab,A)$.\\ Wenn \OE\ die beiden Modelle von $\td$ genügend saturiert sind, ist $i$ im B\&F-System, also erfüllen $a$ und $b$ dieselben Formeln.
\end{proof}

\newpage

\section{Definierbare Teilmengen von $A^n$}
Wir interessieren uns für die Gestalt von $\lingua_P$-definierbaren Teilmengen von $A^n$. Dafür braucht man zuerst eine Hilfsaussage für definierbare Mengen in o-minimalen Strukturen.

\begin{lemma}
	Sei $\fM$ eine o-minimale Struktur, die eine angeordnete Gruppenoperation $+$ mit positivem Element 1 hat und $Y\subseteq M^n$ definierbar. Dann ist $Y$ eine endliche Vereinigung von Mengen der Form $\{f(b,\cdot)=0,g(b,\cdot)>0\}$, wobei $b\in M^m$ und $f,g$ stetige, 0-definierbare Abbildungen $M^{m+n}\rightarrow M$ sind.
\end{lemma}
\begin{proof}
	Schreibe $Y=\phi(b,\fM)$ für ein $b\in M^m$ und definiere $Z:=\phi(\fM)$. Wenn man $Z$ in Zellen $(Z_i)_i$ zerlegt, erhält man $Y$ als endliche Vereinigung von $((Z_i)_b)_i$.  Es sei also o.B.d.A. $Z$ schon eine 0-definierbare Zelle.\\
	Definiere $$f(x):=\left\{\begin{array}{ll}
	\inf\{\abs{x-d}\mid d\in Z\}&Z\text{ nichtleer}\\
	1&\text{sonst}
	\end{array}\right.,$$
	$$g(x):=\left\{\begin{array}{ll}
	\inf\{\abs{x-d}\mid d\in \overline{Z}\setminus Z\}&Z\text{ nichtleer}\\
	1&\text{sonst}
	\end{array}\right.,$$ das sind lipschitzstetige Funktionen.\\
	Klar ist, dass $\overline{Z}=\{f=0\}$; da Zellen lokal abgeschlossen sind, ist $\overline{Z}\setminus Z=\overline{\overline{Z}\setminus Z}=\{g=0\}$. Also erhalten wir $$Z=\overline{Z}\setminus(\overline{Z}\setminus Z)=\{f=0\}\setminus\{g=0\}=\{f=0\}\cap\{g>0\}$$ und $$Y=Z_b=\{f(b,\cdot)=0\}\cap\{g(b,\cdot)>0\}.$$
\end{proof}

\begin{theorem}\label{Definierbare Mengen}
	Für ein dichtes Paar $(B,A)$ und $Y\subseteq A^n$ ist folgendes äquivalent:
	\begin{enumerate}
		\item $Y$ ist $\lingua_P$-definierbar.
		\item Es existiert ein $\lingua$-definierbares $Z\subseteq B^n$, sodass $Y=Z\cap A^n$.
		\item $Y$ ist definierbar in $(A,(R_b)_{b\in B})$ mit der Interpretation $A\models R_b(a)$ genau dann, wenn $0<a<b$ in $B$.
	\end{enumerate}
\end{theorem}
\newpage
\begin{proof}
	$\glqq{}1.\Rightarrow 2.\grqq{}:$ Sei $\varphi$ eine $(\lingua_P)_B$-Formel mit $\varphi(B)=Y$. Zu zeigen ist, dass eine $\lingua_B$-Formel $\psi$ existiert mit $(B,A)\models P(x)\rightarrow(\varphi(x)\leftrightarrow\psi(x))$; das ist genau dann der Fall, wenn $\mathfrak{Th}(B,A)_B\cup\{P(x)\}\cup\{\varphi(x)\}$ und $\mathfrak{Th}(B,A)_B\cup\{P(x)\}\cup\{\neg\varphi(x)\}$ in $\lingua_B$ getrennt werden können. Nach dem Trennungslemma gilt das genau dann, wenn für alle $(B,A)\preceq(D_1,C_1),(D_2,C_2)$ und alle $c_i\in C_i\ (i=1,2)$ mit $(D_1,C_1)\models\varphi(c_1),(D_2,C_2)\models\neg\varphi(c_2)$ eine $\lingua_B$-Formel $\chi$ existiert mit $(D_1,C_1)\models\chi(c_1),(D_2,C_2)\models\neg\chi(c_2)$.\\
	Seien solche $(D_i,C_i)$ und $c_i$, die die Voraussetzungen von oben erfüllen. Dann ist das die Situation aus Lemma \ref{Gemeinsame Unterstruktur}, denn elementare Erweiterungen sind frei. Also muss ein trennendes $\chi$ wie verlangt existieren, denn ansonsten würden $c_1$ und $c_2$ denselben $\lingua$-Typ erfüllen, aber nicht denselben $\lingua_P$-Typ.\\
	$\glqq{}2.\Rightarrow 3.\grqq{}:$ Sei $Y=Z\cap A^n$. Nach dem letzten Lemma ist $Z$ eine boolesche Kombination aus Mengen der Form $\{f(b,\cdot)=0\}$ und $\{g(b,\cdot)>0\}$ für stetige 0-$\lingua$-definierbare Funktionen $f,g$ und passende $b\in B^m$. Es reicht also die Aussage für Mengen in diesen Formen zu zeigen. Wegen der Stetigkeit der Funktionen und $A$ dicht in $B$ gilt aber in $B$
	\begin{align*}
	f(b,z)=0\Leftrightarrow\ &\text{Für alle }0<\varepsilon\in A\text{ existiert }A^m\ni a<b\text{ (koordinatenweise),}\\&\text{sodass für alle }a'\in A^m\text{ mit }a<a'<b\text{ (koordinatenweise)}\\&\text{gilt, dass }\abs{f(a',z)}<\varepsilon,\\
	g(b,z)>0\Leftrightarrow\ &\text{Es existiert ein }0<\varepsilon\in A\text{ und ein }A^m\ni a<b\text{ (koordinatenweise),}\\&\text{sodass für alle }a'\in A^m\text{ mit }a<a'<b\text{ (koordinatenweise)}\\&\text{gilt, dass }\abs{f(a',z)}>\varepsilon.
	\end{align*}
	Die rechten Bedingungen sind jeweils in $(A,(R_b)_{b\in B})$ definierbar.\\
	$\glqq{}3.\Rightarrow 1.\grqq{}:$ Da $A$und alle $R_b$ in $(B,A)$ definierbar sind, ist $Y$ auch in $(B,A)$ definierbar.
\end{proof}

\section{Definierbare eindimensionale Mengen}
In diesem Abschnitt wird eine zur o-Minimalität ähnliche Charakterisierung von eindimensionalen Mengen in Modellen von $\td$ hergeleitet. Hierbei sei $(B,A)$ vorerst ein beliebiges dichtes Paar, als Konvention nehmen wir an, dass $A^0=\{0\}$.
TODO: Das noch überarbeiten, \cite{Piz} erwähnen

\begin{lemma}
	Sei $(U_y)_{y\in Y}, Y\subseteq B^n$ eine Familie von offenen, eindimensionalen, in $(B,A)$ uniform $\lingua$-definierbaren Mengen. Dann ist $$S:=\bigcup\limits_{y\in Y\cap A^n}U_y$$ auch $\lingua$-definierbar.
\end{lemma}
\begin{proof}
	Wir führen eine Induktion über $\dim Y$, man kann schon annehmen, dass alle $U_y$ nichtleer sind, sonst muss man entsprechend aus $Y$ aussondern. Wenn $\dim Y=0$ ist, ist $Y$ endlich und es ist schon $Y\cap A^n$ definierbar, also auch $S$.\\
	Zerlege $Y$ in Zellen, das ändert nichts an irgendwelchen Definierbarkeiten, also kann man annehmen, dass $Y$ selbst schon Zelle ist. Wenn $Y$ keine offene Zelle ist und $\pi:Y\rightarrow Z$ der kanonische homöomorphe Projektion zu einer offenen Zelle von Dimension $m<\dim Y$, existiert nach Satz \ref{Definierbare Mengen} ein $\lingua$-definierbares $Z'\subseteq B^m$ mit $\pi(Y\cap A^n)=Z'\cap A^m$, da $\pi(Y\cap A^n)\subseteq A^m$ eine $\lingua_P$-definierbare Menge ist. Definiere dann $Y':=Z\cap Z'$, sodass man die Menge umparametrisieren kann:
	$$\bigcup\limits_{y'\in Y'\cap A^m}U_{\pi^{-1}(y')}=\bigcup\limits_{y\in\pi^{-1}(Y'\cap A^m)}U_y=\bigcup\limits_{y\in Y\cap A^n}U_y=S$$
	Da $(U_{\pi^{-1}(y')})_{y'\in Y'}$ die gleichen Dinge erfüllt wie $(U_y)_{y\in Y}$ und $\dim Y'\leq\dim Z<\dim Y$, gilt die Aussage per Induktion.\\
	Für den Beweis für offene Zellen wird die Endlichkeit einer bestimmten Menge benötigt:
	
	\begin{lemma}
		Seien die drei $\lingua$-definierbare Mengen $$U:=\bigcup\limits_{y\in Y}U_y,$$ $$Y_x:=\{y\in Y\mid x\in U_y\},$$ $$B:=\{x\in U\mid\inn(Y_x)\neq\emptyset\},$$ dann ist $B$ endlich.
	\end{lemma}
	\begin{proof}
		Wenn $B$ nicht endlich ist, hat es als definierbare Teilmenge von $B$ nichtleeres Inneres und daher ist $A:=\bigcup\limits_{x\in\inn B}Y_x$ nichtleer, denn jedes $Y_x$ ist nichtleer (sonst wäre $x\notin U\supseteq B$). Es gilt außerdem $$A=\bigcup\limits_{x\in U}\{y\in Y\mid x\in U_y\cap\inn B\}=\{y\in Y\mid U_y\cap\inn B\neq\emptyset\}.$$
		Wir wollen jetzt den Widerspruch herleiten, dass $A$ leer sein muss. Seien dafür mit Definable Choice die $\lingua$-definierbaren Funktionen $f,g_1,g_2:A\rightarrow M$ gegeben mit $$f(y)\in(g_1(y),g_2(y))\subseteq U_y\cap\inn B.$$
		Sei \OE\ $A$ offen (TODO: warum ist $\inn A$ nichtleer?) und $f,g_1,g_2$ stetig auf $A$. Sei außerdem $d,e\in U_y\cap\inn B, (c,k)\subseteq U_y\cap\inn B$ (das ist offen), sodass $$c<g_1(y)<d<f(y)<e<g_2(y)<k.$$ Setze $$V:=g_1^{-1}((c,d))\cap f^{-1}((d,e))\cap g_2^{-1}((e,k)),$$ das ist dann eine offene Umgebung um $y$ in $B^n$. Für alle $z\in V$ ist $$f(y)\in(d,e)\subseteq(g_1(z),g_2(z))\subseteq U_z,$$ also ist $z\in Y_{f(y)}$. Das heißt, es gilt $V\subseteq Y_{f(y)}$; da aber $f(y)\in B$, ist das unmöglich, weil $V$ offen ist.
	\end{proof}
	
	Da $B$ endlich ist, ist auch seine Teilmenge $$B':=B\cap\bigcup\limits_{y\in Y\cap A^n}U_y$$ endlich und damit $\lingua$-definierbar. Es gilt $$\bigcup\limits_{y\in Y\cap A^n}(U_y\setminus B)=\bigcup\limits_{y\in Y}(U_y\setminus B):$$ In einer Richtung herrscht per Konstruktion eine Inklusion, für die andere Richtung sei $x\in U_y\setminus B$ für ein $y\in Y$. Dann ist $\inn Y_x$ nichtleer, also existiert wegen der Dichte des Paares ein $z\in Y_x\cap A^n$, also ein $z\in Y\cap A^n$ mit $x\in U_z$. Also ist $x\in\bigcup\limits_{y\in Y\cap A^n}(U_y\setminus B)$. Da $\bigcup\limits_{y\in Y}(U_y\setminus B)$ definierbar ist, ist auch $$S=\bigcup\limits_{y\in Y\cap A^n}U_y=\bigcup\limits_{y\in Y\cap A^n}(U_y\setminus B)\cup\bigcup\limits_{y\in Y\cap A}(U_y\cap B)=\bigcup\limits_{y\in Y}(U_y\setminus B)\cup B'$$ definierbar.
\end{proof}

\begin{lemma}
	Sei eine $\lingua_P$-definierbare Menge $X\subseteq B$. Dann stimmt $X$ bis auf eine kleine Menge mit einer $\lingua$-definierbaren Menge $X'$ überein.
\end{lemma}
\begin{proof}
	Sei zunächst $X$ gegeben durch $\exists y\in P(\psi(x,y))$ für eine Formel $\psi\in\fF_{1+n}(\lingua_B)$. Die Mengen $F_y:=\psi(B,y)\cap\partial\psi(B,y)$ sind endlich für jedes $y\in M^n$ und weil o-minimale Theorien $\exists^\infty$ eliminieren, ist deren Mächtigkeit uniform beschränkt, sagen wir durch $k$. Sei $Y:=\exists x\psi(x,B)$ und $\lingua$-definierbare Funktionen $g_1,\dots,g_k:Y\rightarrow F_y$, sodass $F_y=\{g_1(y),\dots,g_k(y)\}$ für alle $y\in Y$. Als diese Funktionen kann man zum Beispiel die angeordnete Aufzählung der Elemente in $F_y$ nehmen. Dann gilt
	\begin{align*}
	X&=\exists y\in P(\psi(x,y))(B,A)=\bigcup\limits_{y\in A^n}\psi(B,y)=\bigcup\limits_{y\in Y\cap A^n}\psi(B,y)=\bigcup\limits_{y\in Y\cap A^n}\psi(B,y)\\&=\bigcup\limits_{y\in Y\cap A^n}(\psi(B,y\cap \partial\psi(B,y))\cap \partial\psi(B,y))\cup\bigcup\limits_{y\in Y\cap A^n}\inn\psi(B,y)\\&=\bigcup\limits_{y\in Y\cap A^n}\{g_1(y),\dots,g_k(y)\}\cup\bigcup\limits_{y\in Y\cap A^n}\inn\psi(B,y)=\bigcup\limits_{i=1}^kg_i(Y\cap A^n)\cup \bigcup\limits_{y\in Y\cap A^n}\inn\psi(B,y).
	\end{align*}
	Da $X':=\bigcup\limits_{i=1}^kg_i(Y\cap A^n)$ klein ist und $S:=\bigcup\limits_{y\in Y\cap A^n}\inn\psi(B,y)$ nach dem letzten Lemma $\lingua$-definierbar, stimmen $X$ und $X'$ bis auf die kleine Menge $S$ überein. Da Darstellungen \glqq{}$(X\setminus  S')\cup S,X\ \lingua$-definierbar, $S,S'$ klein\grqq{} unter booleschen Kombinationen erhalten bleiben, folgt die Aussage für gute Formeln und daher für alle Mengen.
\end{proof}

\begin{lemma}
	Für jede $\lingua$-definierbare Menge $S\subseteq B^m$ und Funktion $g:B^m\rightarrow B^k$ gibt es eine $\lingua$-definierbare Teilmenge $S'\subseteq S$, sodass $$A^m\cap S\cap g^{-1}(A^k)=A^m\cap S'.$$
\end{lemma}
\begin{proof}
	Für $S=\emptyset$, wähle $S'=\emptyset$. Ansonsten führen wir eine Induktion über $(m,k,\dim S)$ mit elementweiser Halbordnung (die ist fundiert):\\
	Wenn $m=0,k=0$ oder $\dim S=0$, ist $A^m\cap S\cap g^{-1}(A^k)$ endlich und daher $\lingua$-definierbar, also kann man $S'=A^m\cap S\cap g^{-1}(A^k)$ wählen. Sei also $(m,k,\dim S)>(0,0,0)$.
	\begin{itemize}
		\item Wenn $k>1$ gilt und $g$ die Koordinatenfunktionen $g_1,\dots,g_k$ hat, so existieren $(S'_i)_{i\leq k}$ mit $S'_i\subseteq S$ und $A^m\cap S\cap g_k^{-1}(A)=A^m\cap S'_i$ für alle $i$ per Induktionsvoraussetzung. Dann gilt
		\begin{align*}
		A^m\cap S\cap g^{-1}(A^k)=\bigcap\limits_{i=1}^k A^m\cap S\cap g_i^{-1}(A)=\bigcap\limits_{i=1}^k A^m\cap S'_i=A^m\cap(\bigcap\limits_{i=1}^k S'_i),
		\end{align*}
		also erfüllt $S':=\bigcap\limits_{i=1}^k S'_i$ das Gewünschte.
		\item Wenn $k=1$ gilt, zerlege $S$ in Zellen $(Z_i)$, deren Dimension natürlich $\leq\dim S$ ist. Wenn man da das Problem löst (induktiv bzw. von Hand) und jeweils ein passendes $S'_i$ findet, löst $\bigcup\limits_i S'_i$ das Problem für $S$.\\ Sei also $S$ jetzt schon eine Zelle.
		\begin{itemize}
			\item Wenn $n:=\dim S<m$ ist und $\pi$ die entsprechende homöomorphe Projektion auf eine offene Zelle in $B^n$  bzw. eine $\lingua$-definierbare Fortsetzung davon auf ganz $B^m$, sei $\lambda$ eine $\lingua$-definierbare Fortsetzung der Umkehrfunktion dieser Projektion. Wähle die Fortsetzung $\lambda$ dabei so, dass $\lambda(\pi(S))$ und $\lambda(B^n\setminus\pi(S))$ disjunkt sind. Das ermöglicht die Gleichheit $\lambda(C\cap D)=\lambda(C)\cap\lambda(D)$ für $C\subseteq\pi(S)$. Löse dann mit einem $\lingua$-definierbaren $S''\subseteq \pi(S)$ das Problem $$A^n\cap\pi(S)\cap\lambda^{-1}(A^m)\cap (g\circ\lambda)^{-1}(A)=A^n\cap S''.$$
			Das Problem entspricht im Übrigen den Anforderungen, weil man\linebreak $\lambda^{-1}(A^m)\cap (g\circ\lambda)^{-1}(A)$ wie im Fall $k>1$ umschreiben kann. Schneidet man das mit $\lambda^{-1}(A^m)$ und wendet darauf $\lambda$ an, erhält man (mit schrittweiser Verwendung des $\cap$-Herausziehens)
			\begin{align*}
			\lambda(A^n)\cap S\cap A^m\cap g^{-1}(A)&=\lambda(A^n\cap\pi(S)\cap\lambda^{-1}(A^m)\cap (g\circ\lambda)^{-1}(A))\\
			&=\lambda(\lambda^{-1}(A^m)\cap A^n\cap S'')\\
			&=A^m\cap\lambda(A^n)\cap\lambda(S''),
			\end{align*}
			wegen $A^m\cap S\subseteq\lambda(A^n)$ aufgrund der Projektionseigenschaft von $\pi$, kann man $\lambda(A^n)$ weglassen und erhält $$A^m\cap S\cap g^{-1}(A)=A^m\cap\lambda(S''),$$
			also löst $\lambda(S'')$ das Problem für $S$.
			\newpage
			\item Wenn $\dim S=m$, finde eine $\lingua_A$-definierbare Funktion $G:B^{m+n}\rightarrow B$ mit $g=G(\cdot,b)$ für ein über $A$ unabhängiges Tupel $b\in B^n$. Als nächstes betreiben wir Induktion über $n$. Wenn $n=0$, dann ist nichts zu tun, weil dann $g$ schon $A$-definierbar ist, also $g^{-1}(A)=A^m$ und man dann $S'=S$ wählen kann. Ansonsten zerlege $S$ in die Mengen
			\begin{align*}
			S_1:=\{x\in B^{m+n}\mid&G(x_1,\dots,x_{m+n-1},\cdot)\text{ ist streng monoton wachsend }\\&\text{auf einem Intervall um }x_n\},\\S_2:=\{x\in B^{m+n}\mid&G(x_1,\dots,x_{m+n-1},\cdot)\text{ ist streng monoton fallend }\\&\text{auf einem Intervall um }x_n\},\\S_3:=\{x\in B^{m+n}\mid&G(x_1,\dots,x_{m+n-1},\cdot)\text{ ist konstant auf einem Intervall um }x_n\}
			\end{align*}
			und den Rest $S_4:=B^{m+n}\setminus(S_1\cup S_2\cup S_3)$.
			$S_1\cup S_2\cup S_3$ ist groß, denn wenn eine offene Menge $U\subseteq S_4$ existiert, wähle $x\in U$ beliebig und ein Intervall $I$ um $x_n$ mit $\{(x_1,\dots,x_{n-1})\}\times I\subset U$. Nach der Charakterisierung o-minimaler definierbarer Funktionen existiert ein Subintervall $J\subseteq I$, sodass $G(x_1,\dots,x_{m+n-1},\cdot)$ entweder streng monoton wachsend, fallend oder konstant ist auf $J$. Also ist $x\in S_1\cup S_2\cup S_3$ im Widerspruch zu $x\in U$.
			Partitioniere diese Mengen dann noch in $A$-definierbare Zellen $(Z_i)_i$ und definiere $Z'_i:=\{x\in B\mid (x,b)\in Z_i\}$ für alle $i$. Dann ist für jede offene Zelle $G$ in der letzten Koordinate entweder streng monoton steigend, fallend oder konstant jeweils auf der ganzen Zelle; das folgt, indem offene Zellen schon Teilmenge von $S_1,S_2$ oder $S_3$ sind. Die lokale Definition dieser Mengen überträgt sich durch Supremumsbildung oder definierbaren Zusammenhang auf die gesamte Zelle.\\
			Löse das Problem jetzt für alle $(Z'_i)_i$, wegen $S:=\bigcup\limits_i Z'_i$ ist es dann auch für $S$ gelöst: Für nicht-offene Zellen geht das per Induktion bzw. genauso wie im vorigen Unterpunkt. Wenn $Z'_i$ nun eine offene Zelle ist, gilt für ein generisches Element $x$ über $A,b$, dass $(x,b)$ generisch von $B^{m+n}$ ist, also in $S_1\cup S_2\cup S_3$. Also ist $Z_i$ entweder in $S_1,S_2$ oder $S_3$ enthalten.
			\begin{itemize}
				\item Wenn $Z_i\subseteq S_3$ ist, definiere $$\tilde{G}(\overline{x})=z:\Leftrightarrow z=G(\overline{x},y)\text{ für ein }y\text{ mit }(\overline{x},y)\in Z_i,$$ dann gilt $g=\tilde{G}(\cdot,b_1,\dots,b_{n-1})$ und per Induktion kann man das Problem für $n-1$ lösen.
				\item Wenn $Z_i\subseteq S_1,S_2$, also $G$ auf $Z_i$ injektiv in der letzten Koordinate ist, wird das Problem durch $\emptyset$ gelöst: Denn sei $a\in A^m\cap S'_i\cap g^{-1}(A)$, also existiert $a'\in A$ mit $a'=g(a)=G(a,b)$, weil $a\in Z'_i$ ist, ist $(a,b)\in Z_i$, also ist wegen Injektivität von $G$ in der letzten Koordinate $b_n$ eindeutig bestimmt mit $(a,b)\in Z_i$ und $a'=G(a,b)$. Das ist aber $A,b_1,\dots,b_{n-1}$-definierbar, also ist $b$ nicht unabhängig über $A$.
			\end{itemize}
		\end{itemize}
	\end{itemize}
\end{proof}

\begin{lemma}
	Sei $X\subseteq B$ eine kleine Teilmenge. Dann ist $X$ eine endliche Vereinigung von Mengen $f(A^n\cap E)$ für $E$ eine offene Zelle und $f:E\rightarrow B$ eine stetige $\lingua$-definierbare Funktion.
\end{lemma}
\begin{proof}
	Wenn $X$ klein ist, existiert ein $\lingua$-definierbares $g:B^m\rightarrow B$, sodass $X\subseteq g(A^m)$. Setze $X':=g^{-1}(X)\cap A^m=(g\upharpoonright A^m)^{-1}(X)$, das ist $\lingua_P$-definierbar und es gilt $g(X')=X$ wegen $X\subseteq\operatorname{im}(g\upharpoonright A^m)$.\\
	Beweise die Aussage jetzt induktiv über $m$: Wenn $m=0$ ist, ist $X$ maximal einelementig und entweder gleich $f(\{0\})$ für eine konstante Funktion $f$ oder schon die leere Vereinigung.\\
	Wenn $m>0$ ist, schreibe $X'$ als Teilmenge von $A^m$ wegen Satz \ref{Definierbare Mengen} in der Form $Y\cap A^m$ für ein $\lingua$-definierbares $Y$. Sei eine Zerlegung $\mathfrak{Z}$ von $Y$ in Zellen gegeben, auf denen $g$ jeweils stetig ist, dann ist $$X=g(Y\cap A^m)=\bigcup\limits_{Z\in\mathfrak{Z}}g(Z\cap A^m).$$
	Für offene Zellen $Z$ ist so eine Darstellung also schon gefunden. Sei $Z$ nun eine Zelle der Dimension $d<m$ und $\pi:B^d\rightarrow B^m$ eine $\lingua$-definierbare Fortsetzung des kanonischen Homöomorphismus der entsprechenden offenen Zelle $Z'$ nach $Z$, die so gewählt ist, dass $\pi(Z')$ und $\pi(M^d\setminus Z')$ disjunkt sind. Dann gilt mit derselben Argumentation wie im letzten Beweis und mit Anwendung des daraus resultierenden Lemmas $$f(Z\cap A^m)=(f\circ\pi)(A^d\cap Z\cap\pi^{-1}(A^m))=(f\circ\pi)(A^d\cap S)$$ für ein passendes $\lingua$-definierbares $S$. Das ist dann aber schon der Fall eines kleineren $m$ und mit der Induktionsbehauptung folgt die Aussage.
\end{proof}

\begin{theorem}\label{Satz 4}
	Sei $X\subseteq B$ eine $\lingua_P$-definierbare Menge. Dann existiert eine endliche Unterteilung von $B$, sodass für jedes dadurch erzeugte offene Intervall $I$ genau einer der folgenden Fälle gilt:
	\begin{itemize}
		\item $I$ ist disjunkt zu $X$.
		\item $I$ ist Teilmenge von $X$.
		\item $X\cap I$ ist dicht \& kodicht in $I$ und entweder ist $X\cap I$ klein oder $I\setminus X$.
	\end{itemize}
	Für kleine $X$ entfallen die Fälle \glqq{}Teilmenge\grqq{} und \glqq{}koklein\grqq{} natürlich.
\end{theorem}
\begin{proof}
	Die gesuchte Eigenschaft bleibt unter endlichen Vereinigungen und Differenzen von Mengen erhalten (die Eigenschaft für kleine Mengen nur unter Vereinigungen), man muss nur eine Verfeinerung der Unterteilung durchführen. Deswegen sei $X$ zunächst klein und nach dem letzten Lemma gegeben als $X=f(A^n\cap E)$ für ein stetiges, $\lingua$-definierbares $f:E\rightarrow B$ und eine offene Zelle $E\subseteq B^n$. Da definierbarer Zusammenhang unter definierbaren stetigen Funktionen erhalten bleibt, ist $I:=f(E)$ auch definierbar zusammenhängend, weil es $\lingua$-definierbar ist, hat es auch ein Supremum und Infimum in $B\cup\{\pm\infty\}$, ist also ein Intervall (ausnahmsweise sei hier auch ein nichtoffenes Intervall mitgemeint). Wenn $I$ endlich ist, ist auch $X$ endlich und es ist nichts zu zeigen. $X$ ist disjunkt zu $B\setminus I$, nun muss nur noch gezeigt werden, dass $X=X\cap I$ dicht und kodicht in $I$ ist, denn Dichte und Kodichte in nichtoffenen unendlichen Intervallen ist äquivalent zu Dichte und Kodichte in deren Innerem. Aber für dichte Teilmengen werden durch stetige Abbildungen auf dichte Teilmengen abgebildet; da $A^n$ dicht in $B^n$ ist, ist auch $A^n\cap E$ dicht in $E$ und folglich $X$ dicht in $I$. Andererseits muss auch das Komplement von $X$ dicht in $I$ sein, da es sonst ein Intervall ganz in $X$ gäbe, was der Kleinheit mit Lemma \ref{Kleinheit} widerspricht.\\
	Sei jetzt $X$ nicht mehr klein, dann stimmt es aber bis auf eine kleine Menge mit einer $\lingua$-definierbaren Menge $X'$ überein. Schreibe also $X=(X'\setminus Y)\cup Z$ für $Y$ und $Z$ klein. $X'$ hat als $\lingua$-definierbare Menge sowieso schon die gewünschte Gestalt (sogar nur unter Benutzung der ersten zwei Fälle), $Y$ und $Z$ nach dem ersten Teil des Beweises auch. Da die Darstellung unter Differenz und Vereinigung erhalten bleibt, hat auch $X$ die gewünschte Form.
\end{proof}

\begin{lemma}
	$\td$ eliminiert $\exists^\infty$: Wenn $S\subseteq B^{m+n}$ eine $\lingua_P$-definierbare Menge ist mit $S_x$ endlich für alle $x\in B^m$, dann ist $(\abs{S_x})_{x\in B^m}$ beschränkt.
\end{lemma}
\begin{proof}
	Für $n=1$ gilt das nach Bemerkung 5.33. aus \cite{Lukas}. Denn $S_x$ ist endlich, genau dann, wenn $S_x$ diskret in $B$ ist; und das ist uniform $\lingua_P$-ausdrückbar. Die Äquivalenz zur Diskretheit sieht man ein, indem man eine Aufteilung von $S_x$ wie im letzten Satz vornimmt. Dann ist $S_x$ genau dann endlich, wenn nur der Fall $\glqq{}S_x\cap I=\emptyset\grqq{}$ vorkommt; weil dichte Mengen und Intervalle nicht diskret sind, gilt das wiederum genau dann, wenn $S_x$ diskret ist.\\
	Sei $n>1$. Dann sind für $\pi_{i_1,\dots,i_k}$ als Projektionsabbildung auf die Koordinaten $i_1,\dots,i_k$ jeweils auch $$Y_x:=(\pi_{1,\dots,m+1}(S))_x=(\pi_{m+1}(S_x))$$ und $$Z_x:=(\pi_{1,\dots,m,m+2,\dots,m+n}(S))_x=(\pi_{m+2,\dots,m+n}(S_x))$$ endlich und daher ist nach Induktionsvoraussetzung die Mächtigkeit jeweils uniform beschränkt durch irgendwelche $K,L\in\setN$. Dann gilt aber $$\abs{S_x}\leq \abs{Y_x\times Z_x}=\abs{Y_x}\abs{Z_x},$$ was uniform durch $KL$ beschränkt ist.
\end{proof}

\section{Definierbare Funktionen}
Um definierbare Funktionen besser zu verstehen, ist es notwendig, sich mit dem definierbaren Abschluss zu beschäftigen. Damit kann man zeigen, dass definierbare Funktionen in einer Variablen \glqq{}fast überall\grqq{} $\lingua$-definierbar sind.

\begin{lemma}\label{A definierbar abgeschl}
	In jedem dichten Paar $(B,A)$ ist $A$ definierbar abgeschlossen.
\end{lemma}
\begin{proof}
	Sei $b\in B\setminus A$ und $(B,A)\preceq(D,C)$ eine genügend saturierte Elementarerweiterung. Dann wird der Ordnungstyp von $b$ über $A$ auch von einem Element $D\setminus C\ni d\neq b$ realisiert wegen Dichtheit von $D\setminus C$ in $D$ und Saturation.\newpage
	Nach Lemma \ref{selber Schnitt} haben $b$ und $d$ dann den selben $\lingua_P$-Typen über $A$, weswegen $b$ nicht definierbar über $A$ in $(D,C)$ sein kann, also auch nicht in $(B,A)$.
\end{proof}

\begin{corollary}
	Sei $(B,A)$ ein dichtes Paar und $A_0\preceq A$. Dann ist $A_0$ definierbar abgeschlossen.
\end{corollary}
\begin{proof}
	Sei $b$ definierbar über $A_0$. Dann ist $b$ insbesondere definierbar über $A$, also in $A$. Nach Folgerung \ref{Definierbarkeit aus A} ist dann $\{a\}=\{a\}\cap A$ schon $\lingua$-definierbar aus $A_0$, also in $A_0$, da $A_0$ elementare Substruktur ist.
\end{proof}

\begin{lemma}\label{Freie Definierbarkeit}
	Sei $(D,C)\subseteq(B,A)$ frei und $(B,A)$ dichtes Paar. Dann ist $D$ definierbar abgeschlossen in $(B,A)$.
\end{lemma}
\begin{proof}
	Für eine beliebige Struktur $(B',A')\succeq (B,A)$ gelten die Voraussetzungen ebenso, da $\acl$-Abhängigkeit über $A$ als Teil vom Typen äquivalent ist zu $\acl$-Abhängigkeit über $A'$. Ebenso ist $D$ in $(B,A)$ definierbar abgeschlossen genau dann, wenn es in $(B',A')$ definierbar abgeschlossen ist. Also sei $(B,A)$ jetzt schon \OE\ hinreichend saturiert und $b\in B\setminus D$.\\
	Nach dem Beweis zur Existenz des B\&F-Systems kann der partielle Isomorphismus $(B,A)\supseteq(D,C)\cong(D,C)\subseteq(B,A)$ insbesondere auf mehrere Weisen auf $b$ fortgesetzt werden: Wenn $b\in A$ oder $b\in B\setminus AD$, ging es nur um die Erfüllung von transzendenten Ordnungstypen, da hat man also viele Optionen. Wenn $b\in AD\setminus(A\cup D)$ ist und $a\in A^n$ unabhängig über $D$ mit $b\in Da$, dann finde $a'_1$ transzendent über $Da$ von passendem Ordnungstyp über $D$, $a'_2$ transzendent über $Daa'_1$ von passendem Ordnungstyp über $Da'_1$, usw. So kann man den Isomorphismus auf $Da$ fortsetzen mit Bild $Da'$; da aber $a$ und $a'$ per Konstruktion unabhängig über $D$ waren, sind auch $Da$ und $Da'$ unabhängig über $D$, also $Da\cap Da'=D$ und das Bild von $b$ kann nicht $b$ selbst sein. Also gibt es auch in diesem Fall mehrere Möglichkeiten für eine Fortsetzung, also mehrere elementare Abbildungen, also kann $b$ nicht definierbar über $D$ sein.
\end{proof}

\begin{lemma}
	Sei $(B,A)\models\td$ und sei $F:A^n\rightarrow A$ eine $\lingua_P$-definierbare Funktion. Dann gibt es in $A$ definierbare $f_1,\dots,f_k:A^n\rightarrow A$, sodass für alle $a\in A^n$ ein $f_i$ existiert mit $F(a)=f_i(a)$.
\end{lemma}
\begin{proof}
	Wenn die Aussage nicht gilt, gilt für alle $k\in\setN$ und alle in $A$ definierbaren $f_1,\dots,f_k:A^n\rightarrow A$, dass ein $a\in A$ existiert mit $f_i(a)\neq F(a)$ für alle $i$. Also ist der partielle Typ $$\{P(x)\}\cup\{F(x)\neq f(x)\mid f:A^n\rightarrow A\ \lingua_A\text{-definierbar}\}$$ konsistent und es existiert $(B,A)\preceq(B',A')$ und $a'\in A'^n$ mit $F(a')\neq f(a')$ für alle in $A'$ $\lingua_A$-definierbaren $f:A'^n\rightarrow A'$.\\
	Allerdings ist nach Lemma \ref{Unabhängigkeitsregeln} (6.) wegen $a'\in A'^n$ die Inklusion $(Ba',Aa')\subseteq(B',A')$ frei, nach Lemma \ref{Freie Definierbarkeit} ist $Ba'$ also $\lingua_P$-definierbar abgeschlossen. Da $F(a')$ $\lingua_P$-definierbar über $Ba'$ ist, ist es in $Ba'$, wegen $F:A'^n\rightarrow A'$ ist $F(a')\in A'$. Wegen Unabhängigkeit liegt also $F(a')\in Ba'\cap A'=Aa'$ und es gibt eine $\lingua_A$-definierbare Abbildung $f:A'^n\rightarrow A'$ mit $f(a')=F(a')$ - ein Widerspruch!
\end{proof}

\begin{lemma}
	Sei $(B,A)\models\td$ und sei $F:B\rightarrow B$ eine $\lingua_P$-definierbare Funktion. Dann gibt es $\lingua$-definierbare $f_1,\dots,f_k:B\rightarrow B$ und eine kleine Menge $X\subset B$, sodass für alle $b\in B\setminus X$ ein $f_i$ existiert mit $F(b)=f_i(b)$.
\end{lemma}
\begin{proof}
	Wenn die Aussage nicht gilt, gilt für alle $k\in\setN$, alle kleinen Mengen $X\subset B$ und alle $\lingua$-definierbaren $f_1,\dots,f_k:B\rightarrow B$, dass ein $b\in B\setminus X$ existiert mit $f_i(b)\neq F(b)$ für alle $i$. Also ist der partielle Typ $$\{x\notin X\mid X\text{ klein in }B\}\cup\{F(x)\neq f(x)\mid f:B\rightarrow B\ \lingua_B\text{-definierbar}\}$$ konsistent und es existiert $(B,A)\preceq(B',A')$ und $$b'\in B'\setminus\bigcup\limits_{f:B'^n\rightarrow B'\ \lingua_B\text{-definierbar}}f(A'^n)=B'\setminus A'B$$ mit $F(b')\neq f(b')$ für alle $\lingua_B$-definierbaren $f:B'\rightarrow B'$.\\
	Allerdings ist nach Lemma \ref{Unabhängigkeitsregeln} (7.) wegen $b'\in B'\setminus A'B$ die Inklusion $(Bb',A)\subseteq(B',A')$ frei, nach Lemma \ref{Freie Definierbarkeit} ist $Bb'$ also $\lingua_P$-definierbar abgeschlossen. Da $F(b')$ $\lingua_P$-definierbar über $Bb'$ ist, ist es in $Bb'$, also existiert eine $\lingua_B$-definierbare Abbildung $f:B'\rightarrow B'$ mit $f(b')=F(b')$ - ein Widerspruch!
\end{proof}

\begin{theorem}\label{Satz 3}
	Sei $(B,A)\models\td$ und sei $F:B\rightarrow B$ eine $\lingua_P$-definierbare Funktion. Dann stimmt $F$ auf bis auf eine kleine Menge mit einer $\lingua$-definierbaren Funktion überein.
\end{theorem}
\begin{proof}
	TODO: Muss noch warten
\end{proof}

\begin{lemma}\label{Stückweise stetige Abbildungen}
	Sei $f:B\rightarrow B$ stückweise stetig und $\lingua_P$-definierbar. Dann ist $f$ schon $\lingua$-definierbar.
\end{lemma}
\begin{proof}
	Nach Satz $\ref{Satz 3}$ stimmt $f$ bis auf eine kleine Menge $X$ mit einer $\lingua$-definierbaren Funktion $f'$ überein. Wegen o-Minimalität von $T$ ist $f'$ auch stückweise stetig. Unterteile $X$ wie in Satz \ref{Satz 4} und verfeinere die Unterteilung, so dass $f,f'$ auf jedem Intervall stetig sind. Für jedes Intervall $I$ dieser Unterteilung gilt dann entweder, dass $X\cap I$ dicht und kodicht in $I$ ist oder, dass $X\cap I=\emptyset$. Der erste Fall kann nie eintreten, da zwei stetige Funktionen, die auf der dichten Teilmenge $I\setminus X$ übereinstimmen, schon auf ganz $I$ übereinstimmen. Also ist $X$ endlich und $f$ kann mit $\lingua$ definiert werden (nämlich durch $f'$ außerhalb von $X$ und ansonsten manuell).
\end{proof}

\section{Offene Teilmengen von $\setR^n$}
Nach der Beschreibung der Struktur von $\lingua_P$-definierbaren Teilmengen von $B$ liegt die Vermutung nahe, dass offene und abgeschlossene $\lingua_P$-definierbare Mengen schon $\lingua$-definierbar sind. Zumindest für Teilmengen von $\setR^n$ kann man das beweisen.

\begin{definition}
	Sei $\fA$ eine Struktur in einer Sprache $\lingua$ und $\fB$ eine Struktur in einer Sprache $\lingua'$. Dann heißt $\fB$ Erweiterung von $\fA$, wenn $A=B$ und die Interpretation von $\lingua$ in $\fA$ schon $\lingua'$-definierbar in $\fB$ ist. TODO: Muss es sogar schon 0-definierbar sein? sonst später Problem mit dem Typ...
\end{definition}

\begin{lemma}\label{Erweiterung definierbare Mengen}
	Sei $\fA$ eine unendliche Struktur, und sei $\fB$ eine $\aleph_0$-saturierte o-minimale Erweiterung von $\fA$, , sodass Definable Choice gilt und alle in $\fB$ definierbaren Funktionen $A\rightarrow A$ schon in $\fA$ definierbar sind. Dann sind die in $\fA$ und $\fB$ definierbaren Mengen die gleichen.
\end{lemma}
TODO: Eigentlich musste $\fB$ nicht o-minimal sein, dafür $\fA$. Fehler?
\begin{proof}
	Per Definition einer Erweiterung sind alle $\fA$-definierbaren Mengen auch $\fB$-definierbar.\\
	Sei $S\subseteq A^n$ definierbar in $\fB$. Wenn $n=1$ ist, ist die charakteristische Funktion $\chi_S:A\rightarrow A$ definierbar in $\fB$, also per Voraussetzung auch in $\fA$, also ist auch $S=\{\chi_S=1\}$ definierbar in $\fA$. Wenn $0,1$ nicht in $A$ enthalten sind, muss man sich stattdessen ein Analogon mit zwei bestimmten Elementen aus $A$ basteln.\\
	Wenn $n>1$ ist, dann zerlege $S$ in $\fB$-Zellen. Es reicht daher, die Aussage für eine beliebige $\fB$-Zelle $S$ zu beweisen, genauer reicht es sogar aus, die $\fA$-Definierbarkeit für die definierende(n) partiellen Funktion(en) $x\mapsto\sup S_x,x\mapsto\inf S_x$ zu beweisen; nach wählen eines willkürlichen noch nicht angenommenen Funktionswertes, reicht es, die $\fA$-Definierbarkeit für beliebige $\fB$-definierbare Funktionen $f:A^{n-1}\rightarrow A$ zu beweisen.\\
	Wenn $n=2$ ist, gilt das auch schon per Voraussetzung. Wenn $n>2$ ist, definiere die $\fB$-definierbaren Funktionen $f_a:A^{n-2}\rightarrow A,x\mapsto f(a,x)$. Nach Induktionsvoraussetzung ist jede davon $\fA$ definierbar, sei also $f_a=F_a(c_a,\cdot)$ für 0-$\fA$-definierbare Funktionen $F_a:A^{m_a+n-2}\rightarrow A$ und passende $m_a\in\setN,c_a\in A^{n_a}$. Es ist möglich, bloß endlich viele unterschiedliche $F_a$ zu verwenden: Ansonsten ist nämlich $$\{\forall c(f(a,x)\neq F(c,x))\mid F:A^{m+n-2}\rightarrow A\ \text{0-}\fA\text{-definierbar,}m\in\setN\}$$ konsistent in $\fB$, der Erfüller davon darf nach obigen Erkenntnissen aber nicht existieren.\\
	Also existieren 0-$\fA$-definierbare Funktionen $F_i:A^{m_i+n-2}\rightarrow A$ für $i=1,\dots,k$, sodass für alle $a\in A$ ein $i\leq k$ und ein $c\in A^{m_i}$ existiert mit $f_a=F_i(c_i,\cdot)$. Für ein $b\in A$ und $z$ von der Dimension $\max\limits_i m_i$ sei
	$$F(z,y,x):=\left\{\begin{array}{ll}
	F_i(y_1,\dots,y_{m_i},x)&i\text{ ist das einzige }j\text{ mit }z_j=b\\
	b&\text{sonst}
	\end{array}\right..$$
	Dann ist $F$ definierbar in $\fA$ und es gilt, dass für alle $a\in A$ ein $(z,y)\in A^{k+\max\limits_i m_i}$ mit $f_a=F(z,y,\cdot)$. Da in $\fB$ Definable Choice gilt, existiert eine $\fB$-definierbare Funktion $g$, sodass $f(a,x)=F(g(a),x)$ gilt. Nach der Voraussetzung sind alle Koordinatenfunktionen von $g$ definierbar in $\fA$, also auch $g$ selbst, also auch $f$.
\end{proof}

\begin{theorem}
	Sei $(B,A)\models\td,\operatorname{RCF}\subseteq\operatorname{T},\setR\subseteq B$ und $X\subseteq \setR^n$. Wenn $X$ offen und $\lingua_P$-definierbar ist, ist es $\lingua$-definierbar.
\end{theorem}
\begin{remark}
	Für eindimensionale Mengen ist das trivial, denn in der Darstellung von Satz \ref{Satz 4} kann der Fall $X\cap I$ dicht und kodicht nicht auftreten, weil offene Mengen niemals kodicht sind. Also sind die definierbaren offenen Teilmengen von $B$ gerade die endlichen Vereinigungen von Intervallen und das ist $\lingua$-definierbar.
\end{remark}
\begin{proof}[Beweis des Satzes]
	Füge zunächst zu $\lingua$ $n$-stellige Relationen $O_\varphi$ für jede $(\lingua_P)_B$-Formel $\varphi$, die in $(B,A)$ eine offene Teilmenge von $\setR^n$ definiert, in dieser Sprache $\lingua'$ sei $\tilde{B}$ die Erweiterung von $B$ durch kanonische Interpretation als $O_\varphi(B):=\varphi((B,A))$. Nach der Bemerkung oben sind die eindimensionalen $\lingua_P$-definierbaren Teilmengen von $\setR$ die Vereinigungen von Intervallen in $\setR$ und daher (TODO: warum?) folgt nach \cite{MillSpeiss}, dass $\tilde{B}$ o-minimal ist. Es sei $(D,C)$ eine $\aleph_0$-saturierte Elementarerweiterung von $(B,A)$ und $\tilde{D}$ die Erweiterung von $D$ auf $\lingua'$ durch kanonische Interpretation der $O_\varphi$. Dann muss $\tilde{B}\preceq\tilde{D}$ gelten, denn alle $\lingua'_B$-Formeln gehen auf $(\lingua_P)_B$-Formeln zurück und für die gilt per Konstruktion die Elementarität der Inklusion. Also ist $\tilde{D}$ auch o-minimal, daher ist jede $\lingua'$-definierbare Funktion $f:D\rightarrow D$ stückweise stetig. Weil sie dann auch $\lingua_P$-definierbar ist, gilt nach Lemma \ref{Stückweise stetige Abbildungen}, dass jede $\lingua'$-definierbare Funktion $f:D\rightarrow D$ schon $\lingua$-definierbar ist. $\tilde{D}$ ist o-minimal und hat Definable Choice, deswegen erfüllt die Erweiterung $\tilde{D}/D$ die Voraussetzungen des Lemma \ref{Erweiterung definierbare Mengen} und die definierbaren Mengen in $D$ und $\tilde{D}$ sind die gleichen.\\
	Wenn $X$ jetzt eine offene Teilmenge von $\setR^n$ ist, die durch $\chi$ in $(B,A)$ definiert werde, dann ist $X_{(D,C)}=\chi((D,C))$ eine definierbare Teilmenge in $\tilde{D}$, also definierbar in $D$ durch eine Formel $\psi(x,d)$ für ein $d\in D^m$. In $(D,C)$ gilt also $\exists y(\chi(x)\leftrightarrow\psi(x,y))$, also existiert wegen $(B,A)\preceq(D,C)$ ein $b\in B^m$ mit $X=\chi((B,A))=\psi((B,A),d)=\psi(B,d)$. Das heißt, $X$ ist definierbar in $\lingua$.
\end{proof}
\begin{corollary}
	Auch abgeschlossene $\lingua_P$-definierbare Teilmengen von $\setR$ sind $\lingua$-definierbar. Die Definition läuft in diesem Fall über das Komplement.
\end{corollary}
\begin{corollary}
	Sei $(B,A)$ wie oben und $S\subseteq\setR^n$ $\lingua_P$-definierbar. Dann
	\begin{itemize}
		\item sind $\inn S,\overline{S}$ definierbar in $B$ nach dem Satz und der Folgerung als offene bzw. abgeschlossene $\lingua_P$-definierbare Mengen.
		\item ist $S$ genau dann $\lingua$-definierbar, wenn es eine boolesche Kombination von offenen/abgeschlossenen Teilmengen von $\setR^n$ ist, von denen jede $\lingua_P$-definierbar ist. Die Rückrichtung folgt dabei aus dem Satz und der Folgerung, da dann jede einzelne Menge in der Kombination $\lingua$-definierbar ist; die Hinrichtung folgt per Zellzerlegung.
	\end{itemize}
\end{corollary}


\newpage
\section{TODO: Noch woanders einsortieren oder löschen}

\begin{lemma}
	Sei $\fA$ eine o-minimale Erweiterung eines angeordneten Vektorraums über einem angeordneten Körper $F$ und $g:A^{p+1}\rightarrow A$ definierbar, außerdem existiere für unendlich viele $\lambda\in F$ ein $a_\lambda\in A^p$ mit $g(a_\lambda,x)=\lambda x$ für unendlich viele $x\in A$. Dann existiert ein Intervall $I$ in $A$, sodass auf $I$ eine $A$-definierbare Körperstruktur existiert, die mit $<$ kompatibel ist (was automatisch einen reell abgeschlossenen Körper impliziert).
\end{lemma}
\begin{proof}
	TODO: Geht irgendwie aus \cite{PeterStarch} hervor.
\end{proof}

\begin{lemma}
	Es sei $(A,B)\models\td,\ f:B^{n+1}\rightarrow B$ $A$-definierbar in $B$ und $b\in B\setminus A$. Dann enthält $f(A^n\times\{b\})$ kein Intervall um $b$.
\end{lemma}
\begin{proof}
	Nimm an, dass das Gegenteil gelte für das Intervall $J$ (\OE\ mit Randpunkten in $A$): Dann existiert insbesondere für jedes $q\in\setQ$ hinreichend nahe bei $1$ ein $a_q\in A^n$ mit $f(a_q,b)=qb$. Dann existiert wieder ein Intervall $I_q\subseteq J_A$ mit $f(a_q,x)=qx$ für alle $x\in I_q$. \OE\ ist dieses Intervall schon beschränkt und die Randpunkte seien $c_q<d_q$. Definiere dann $$r_q:=\frac{c_q+d_q}{2},s_q:=\frac{d_q-c_q}{2}\in A,$$ $$g:(u,v,x)\mapsto f(u,v+x)-f(u,v)\ \ \ \ u\in A^n,v,x\in A.$$
	Dann gilt für alle $x\in(-s_q,s_q)$ $$g(a_q,r_q,x)=f(a_q,r_q+x)-f(a_k,r_q)=q(r_q+x)-qr_q=qx.$$
	Also existiert nach dem letzten Lemma ein Intervall in $A$ mit einer $A$-definierbaren Körperstruktur als RCF. Durch Translation (benutze Dichtheit) nehme an, dass $b\in I_B$ liegt. Dann existiert nach Lemma \ref{Hilfsaussage Kleinheit} ein Element $c\in I_B\setminus f(A^n\times\{b\})$. \OE\ sei schon $\inf J,\sup J\in I$, sonst ersetze $J$ durch ein kleineres Intervall.\\
	Seien $d,e\in I$ mit $d<c<e$ und $\varphi$ die orientierungserhaltende, $A$-definierbare affine Abbildung in $I$ mit $\varphi(d)=\inf J,\varphi(e)=\sup J$. Dann ist $\varphi(c)\in J\setminus(\phi\circ f)(A^n\times\{b\})$ und da das Verketten mit einer $A$-definierbaren invertierbaren Abbildung nichts an der Aussage ändert, gibt es einen Widerspruch.
\end{proof}

\begin{theorem}
	Wenn $(B,A)\models\td$, dann ist kein Intervall eine kleine Teilmenge.
\end{theorem}
\begin{proof}
	Sei $f:B^n\rightarrow B$ eine durch $\varphi(x,y,b)$ definierbare Abbildung mit $\varphi$ eine\linebreak$\lingua_A$-Formel und $b\in B^m$ für ein $m\in\setN$ definiert. Für $\dim(b/A)=0$ ist $f(A^n)\subseteq A$ klar, deswegen sei \OE\ $\dim(b/A)\geq1$. Definiere
	\begin{align*}
	g(x,z):=\left\{\begin{array}{ll}
	\text{das eindeutige }y\in B &\text{für alle z, für die }\varphi(x,y,z)\\
	\text{ mit }B\models\varphi(x,y,z) &\text{ bei festem }\text{ eine Funktion definiert}\\
	\ &\ \\
	0 &\text{sonst}
	\end{array}\right.,
	\end{align*}
	Dann ist $g$ in $B$ $A$-definierbar und $g(\cdot,b)=f$. Falls $\dim(b/A)>1$, füge genug Komponenten von $b$ zu $A$ hinzu, sodass $\dim(b/A)=1$. Das Hinzufügen ändert nichts, denn $Ab_i$ ist nach den Eingangsbemerkungen Modell von T und $Ab_i$ ist erst recht dicht in, aber nicht gleich $B$ (sonst hätte man die Dimension mit diesem Schritt schon zu sehr verkleinert).\\
	Finde also $b_i$, sodass $A$-definierbare $(h_j)$ existieren mit $b_j=h_j(b_i)$ für alle $j$. Wenn jetzt $J\subseteq f(A^n)=g(A^n,b)=g(A^n,h(b_i))$ für ein Intervall $J$, dann widerspricht das der Aussage des letzten Lemmas für die Funktion $(x,y)\mapsto g(x,h(y))$.
\end{proof}

\begin{proof}
	Wenn $D=C$, ist $D\preceq A$ und die Aussage daher klar nach der vorigen Folgerung. Die Inklusion $(AD,A)\subseteq(B,A)$ ist trivialerweise frei (zwei gleiche Mengen in der Unabhängigkeit), außerdem ist $A\preceq AD$ dicht (da $A$ dicht in $B\supseteq AD$) und eine echte Inklusion, da für $D=A$ wegen Unabhängigkeit von $D$ und $A$ ansonsten $D=C$ folgen würde. Nach Lemma \ref{freie Inklusionen} ist also $(AD,A)\preceq(B,A)$ und daher ist $\dcl(D)\subseteq\dcl(AD)=AD$, da $AD$ definierbar abgeschlossen nach Lemma \ref{A definierbar abgeschl}.\\
	Sei jetzt $d\in AD$ $\lingua_P$-definierbar über $D$ und $a\in A^n$ minimal mit $d\in Da$ (insbesondere ist $a$ unabhängig über $D$). Im Folgenden wird gezeigt, dass dann $a$ schon das leere Tupel, also $d\in D$ ist.\\
	Nimm an, dass $n>0$ und sei $f:B^n\rightarrow B$ die definierende Funktion von $d$, also ist sie $D$-definierbar und $f(a)=d$. Seien $$S_1:=\{x\in B^n\mid f(x_1,\dots,x_{n-1},\cdot)\text{ ist streng monoton wachsend auf einem Intervall um }x_n\},$$ $$S_2:=\{x\in B^n\mid f(x_1,\dots,x_{n-1},\cdot)\text{ ist streng monoton fallend auf einem Intervall um }x_n\},$$ $$S_3:=\{x\in B^n\mid f(x_1,\dots,x_{n-1},\cdot)\text{ ist konstant auf einem Intervall um }x_n\}.$$
	$S_1\cup S_2\cup S_3$ ist groß, denn wenn eine offene Menge $U\subseteq B^n\setminus(S_1\cup S_2\cup S_3)$ existiert, wähle $x\in U$ beliebig und ein Intervall $I$ um $x_n$ mit $\{(x_1,\dots,x_{n-1})\}\times I\subset U$. Nach der Charakterisierung o-minimaler definierbarer Funktionen existiert ein Subintervall $J\subseteq I$, sodass $f(x_1,\dots,x_{n-1},\cdot)$ entweder streng monoton wachsend, fallend oder konstant ist auf $J$. Also ist $x\in S_1\cup S_2\cup S_3$ im Widerspruch zu $x\in U$.\newpage
	Da $a$ generisch ist, muss es also in der großen Menge liegen.
	\begin{itemize}
		\item Wenn $a$ in $S_1$ liegt, nehmen wir an, dass $(B,A)$ schon hinreichend saturiert ist (das ändert nichts, da $(B,A)$ ja nur irgendeine Oberstruktur und Modell von $\td$ sein muss) und finden in $A\setminus Da_1\dots a_{n-1}$ ein $a'\neq a_n$ mit demselben Ordnungstyp über $Da_1\dots a_{n-1}$ (ansonsten wäre $a_n$ definierbar über $a_1,\dots,a_{n-1}$). Insbesondere ist $a_1,\dots,a_{n-1},a'\in S_1$, weil die Menge aller solchen Elemente $a'$ $Da_1\dots a_{n-1}$-definierbar ist und daher eine $Da_1\dots a_{n-1}$-definierbare Umgebung von $a_n$ dort drin liegt, in der $a'$ liegen muss. Da $f$ streng monoton ist, ist $f(a_1,\dots,a_{n-1},a')\neq f(a)=d\in D$.\\
		Allerdings ist $d$ $\lingua_P$-definierbar über $D$, also ist $$f(a_1,\dots,a_{n-1},x)=d\in\tp_{\lingua_P}(a/Da_1\dots a_{n-1})\setminus\tp_{\lingua_P}(a'/Da_1\dots a_{n-1})$$ (oder zumindest mit der definierenden Formel für $d$ eingesetzt), die Typen sind daher nicht gleich.\\
		Da $a_n,a'\in A$ aber den gleichen Ordnungstyp über $Da_1\dots a_{n-1}$ haben, haben sie auch den gleichen $\lingua$-Typ über $Da_1\dots a_{n-1}$ nach dem Beweis von Satz \ref{BackForth}. Außerdem ist $(Da_1\dots a_{n-1},Ca_1\dots a_{n-1})\subseteq(B,A)$ nach Lemma \ref{Unabhängigkeitsregeln} (6.) frei, weswegen aus Lemma \ref{Gemeinsame Unterstruktur} folgt, dass $a_n,a'$ denselben $\lingua_P$-Typ über $Da_1\dots a_{n-1}$ haben - Widerspruch!
		\item Das Fall $a\in S_2$ geht analog, es wurde eben auch nur streng monoton benutzt.
		\item Im Falle $a\in S_3$ ist $d$ $\lingua$-definierbar über $Da_1\dots a_{n-1}$ durch $$\glqq{}d=f(a_1,\dots,a_{n-1},x)\text{ für irgendein }(a_1,\dots,a_{n-1},x)\in S_3.$$
	\end{itemize}
\end{proof}