%!TEX root = DieLoesungAllerMilleniumsprobleme.tex
\chapter{Dichte Paare o-minimaler Strukturen}
\section{Allgemeine Betrachtungen und Anforderungen an die Theorien}
Viele Techniken und Gedanken aus dem vorigen Kapitel werden jetzt auf o-minimale Theorien übertragen. Einführend folgen hier zunächst die wichtigsten Erkenntnisse über diese Theorien, basierend auf \cite{vdDZellzerlegung}.

\begin{definition}
	Eine $\lingua$-Struktur $\fM$ heißt \textbf{o-minimal}, wenn $\lingua$ eine zweistellige Relation $<$ enthält, deren Interpretation die einer linearen Ordnung ist, und die definierbaren Teilmengen von $\fM$ endliche Vereinigungen von Intervallen und Punkten sind.
\end{definition}

Induktiv kann man den Begriff einer Zelle in einer solchen Struktur definieren.
\begin{definition}
	Sei $\sigma$ eine $\{0,1\}$-wertige endliche Folge der Länge $n$. Eine $\sigma$-Zelle $Z$ in der o-minimalen Struktur $\fM$ ist eine (definierbare) Teilmenge von $M^n$, sodass genau eine der folgenden Möglichkeiten gilt:
	\begin{itemize}
		\item $\sigma$ ist die leere Folge, $n=0$ und $Z=B^0=\{0\}$.
		\item $n=1$ und $Z$ ist entweder Intervall ($\sigma=(1)$) oder einelementig ($\sigma=0$).
		\item Es existiert ein $\sigma'$, eine $\sigma'$-Zelle Z' und eine definierbare stetige Funktion $f:Z'\rightarrow M$, sodass $Z=\operatorname{Graph}(f)$. Außerdem ist $\sigma=\sigma'\textasciicircum0$.
		\item Es existiert ein $\sigma'$, eine $\sigma'$-Zelle Z' und zwei definierbare stetige Funktionen $f,g:M^{Z'}\cup\{\pm\infty\}$, sodass $f<g$ überall und $$Z=\{(x,y)\in Z'\times M\mid f(x)<y<g(x)\}.$$ Außerdem ist $\sigma=\sigma'\textasciicircum1$.
	\end{itemize}
\end{definition}
\begin{definition}
	Sei $\fM$ o-minimal. Eine Menge $Z\subseteq M^n$ heißt \textbf{Zelle}, wenn es eine $\{0,1\}$-Folge $\sigma$ der Länge $n$ gibt, sodass $Z$ eine $\sigma$-Zelle ist.
\end{definition}

Es gibt \glqq{}schöne\grqq{} Zellen, nämlich die offenen. Glücklicherweise ist jede Zelle definierbar homöomorph zu einer offenen Zelle.
\begin{lemma}
	Jede $\sigma$-Zelle ist homöomorph zu einer offenen Zelle. Der kanonische Homöomorphismus entsteht dabei durch weglassen der Koordinaten, in denen $\sigma$ den Wert 0 hat. Insbesondere ist der Homöomorphismus definierbar über der selben Menge wie die Zelle.
\end{lemma}

Wichtigste Grundlage der Arbeit mit o-minimalen Theorien sind die folgenden Aussagen zur \textbf{Zellzerlegung}. Sie stellen quasi eine Verallgemeinerung der Definition von O-Minimalität auf höhere Dimensionen dar und geben die Struktur definierbarer Funktionen an.
\begin{theorem}
	In einer o-minimalen Struktur $\fM$ ist jede definierbare Menge eine endliche disjunkte Vereinigung von Zellen. Für jede definierbare (möglicherweise partielle) Funktion gibt es eine Zerlegung des Definitionsbereiches in Zellen, sodass die Funktion auf jeder Zelle stetig ist. Wenn es eine Funktion in einer Variable ist, ist sie nicht nur stückweise stetig, sondern es gibt auch eine Zerlegung, sodass die Funktion auf jedem Stück entweder streng monoton oder konstant ist.
\end{theorem}
\begin{remark}
	Wenn die Menge bzw. die Funktion über einer bestimmten Menge definierbar sind, kann man sich die Zerlegung ebenso über dieser Menge definierbar wählen. Außerdem kann man endlich viele Mengen simultan zerlegen und sie sogar partitionieren. Eine Partition sei dabei definiert als
	\begin{itemize}
		\item Eine Zerlegung einer Teilmenge von $M$ oder
		\item Eine Zerlegung einer Teilmenge von $M^{n+1}$, so dass die Projektion auf die ersten $n$ Koordinaten eine Partition einer Teilmenge von $M^n$ erzeugt.
	\end{itemize}
\end{remark}

\begin{corollary}
	O-Minimale Strukturen eliminieren $\exists^\infty$. Damit ist für jede definierbare Menge in einer Variable die Menge der zu vereinigendenden Punkte und Intervalle uniform beschränkt. Also lässt sich o-Minimalität als Formelmenge beschreiben und ist somit Eigenschaft einer vollständigen Theorie.
\end{corollary}

\begin{definition}
	Eine vollständige Theorie sei o-minimal, wenn eines ihrer Modelle es ist.
\end{definition}

Ab jetzt geben wir uns ein o-minimales Modell $\fM$ einer vollständigen $\lingua$-Theorie T so vor, dass die Ordnung dicht ist.
\begin{lemdef}
	Die endliche Folge $\sigma$ zur jeweiligen Zelle $Z$ ist eindeutig bestimmt und daher der Wert $\dim(Z)=\sum\limits_{i=1}^n\sigma(i)$ wohldefiniert. Er wird \textbf{Dimension der Zelle} genannt. Für allgemeine definierbare nichtleere Mengen $X$ sei die \textbf{Dimension} $\dim(X)$ definiert als das Maximum der Zelldimensionen in einer beliebigen Zellzerlegung, die Dimension der leeren Menge sei 0. Das ist wohldefiniert und rein topologisch definierbar: $\dim(X)\leq n$ genau dann, wenn eine Projektion von $X$ nach $M^n$ nichtleeres Inneres hat. Damit ist die Dimension auch $\lingua$-definierbar.
\end{lemdef}

Man kann die Dimension auch anders definieren:
\begin{lemma}
	In o-minimalen Strukturen erfüllt $\acl=\dcl$ das Austauschprinzip und induziert also einen Dimensionsbegriff für Tupel. Wenn $\fM$ hinreichend saturiert ist und $X\subseteq M^n$ definierbar über $Y$, ist $\dim(X)=\max\{\dim(a/Y)\mid a\in X\}$.
\end{lemma}

Im Folgenden halten wir eine vollständige o-minimale Theorie T mit dichter Ordnung in der Sprache $\lingua$ fest und betrachten die Theorie $\tq$ in der Sprache $\lingua^P:=\lingua\cup\{P(x)\}$, sodass die Modelle von $\tq$ Modelle von T sind und in jedem Modell $\fM$ die Menge $P(M)$ ebenfalls Modell von T ist. Schreibe so ein Paar dann als $(B,A)$ mit $A=P(B)$.\\
Wir setzen voraus, dass T RCF erweitert, insbesondere gilt Definable Choice. Dann sind Skolemfunktionen definierbar und \OE\ ist $\lingua$ schon so eine definitorische Erweiterung, dass $T$ Quantorenelimination hat und universell axiomatisierbar ist (wobei bei einzelnen Theorien die Frage interessant wäre, welche Skolemfunktionen man dafür überhaupt hinzufügen muss). Außerdem seien alle Modelle genug saturiert, dass die üblichen Rechenregeln für die Dimension gelten.\\
Aus T universell mit Quantorenelimination folgt, dass Unterstrukturen von Modellen von T schon elementare Unterstrukturen sind. Also ist für jede Teilmenge $S$ eines Modells $\dcl(S)$ schon eine elementare Substruktur; zur Vereinfachung bezeichne in Zukunft $AB:=\dcl(A\cup B)$ für zwei Teilmengen $A,B$ eines Modells.\\
$P$ beschreibt also eine elementare Unterstruktur, mit $\td$ wird nun die Theorie beschrieben, die ausdrückt, dass $P$ eine dichte echte Unterstruktur ist (diese zwei Sachen oder deren Gegenteil müssen auf jeden Fall von der Theorie beschrieben werden, wenn sie vollständig sein soll). Klar ist dann, dass Unterstrukturen von $\td$ automatisch Modelle von $\tq$ sind.\\
Der Arbeit \cite{VanDenDries} folgend, wird in den nächsten Abschnitten die Vollständigkeit und eine Art von Quantorenelimination für $\td$ gezeigt, die mittels eines Back\&Forth-Systems bewiesen wird. Danach werden einige Erkenntnisse aus dem B\&F-System gezogen, bevor die Hauptaussagen dieser Arbeit bewiesen werden. Zunächst ist aber eine genauere Betrachtung von sogenannten kleinen Mengen vonnöten.

\section{Kleine Mengen}
\begin{definition}
	Sei $(B,A)\models\td$, dann ist eine $\lingua^P$-definierbare Menge $S\subseteq B$ \textbf{klein}, wenn eine $\lingua$-definierbare Funktion $f:B^n\rightarrow B$ existiert mit $S\subseteq f(A^n)$.
\end{definition}

\begin{lemma}
	Kleine Mengen in einem dichten Paar bilden ein Ideal bzgl. $\glqq{}\subseteq\grqq{}$.
\end{lemma}
\begin{proof}
	Es gibt kleine Mengen, zum Beispiel Punktmengen oder der kleinere Teil eines Paares, außerdem sind Teilmengen von kleinen Mengen offensichtlich wieder klein.\\
	Sei jetzt $(B,A)$ ein dichtes Paar und $X,Y$ kleine Mengen, seien $f,g:B^n\rightarrow B$ definierbar mit $X\subseteq f(A^n),Y\subseteq g(A^n)$. \OE\ bilden $f,g$ dabei schon von demselben $B^n$ ab, ansonsten muss man beide mit Dummyvariablen vergrößern. Definiere
	\begin{align*}
	h:B^{n+1}&\rightarrow B,\\
	h(\overline{x},y):=&\left\{\begin{array}{ll}
	f(\overline{x})&y=0\\
	g(\overline{x})&\text{sonst}\\
	\end{array}\right.,
	\end{align*}
	das ist $\lingua$-definierbar und es gilt $X\cup Y\subseteq h(A^{n+1})$, also ist $X\cup Y$ auch klein.\\
	Dass nicht jede Menge klein ist, wird im Folgenden noch bewiesen werden.
\end{proof}

In dem folgenden Lemma meint $+,\cdot$ nicht unbedingt die Operationen aus $\lingua$, sondern neue, beliebige Operationen. Die Anordnung hingegen soll die gewöhnliche bleiben.

\begin{lemma}\label{Hilfsaussage Kleinheit}
	Seien $A\prec B\models\operatorname{T},\ f:B^{n+1}\rightarrow B\ A\text{-definierbar},\ b\in B\setminus A,\ \beta,\gamma\in A\cup\{\pm\infty\}$ mit $\beta<b<\gamma$ und einer angeordneten $A$-definierbaren Körperstruktur $(\cdot,+)$ auf $(\beta,\gamma)=:I$. Dann existieren $a_0,\dots,a_n\in I_A$ mit $$a_nb^n+a_{n-1}b^{n-1}+\dots,a_0\in I\setminus f(A^n\times\{b\}).$$
\end{lemma}
\begin{proof}
	Wenn die Aussage nicht gilt, dann gilt mit $p(x,y):=x_ny^n+x_{n-1}y^{n-1}+\dots,x_0$, dass für jedes $a\in (I_A)^{n+1}$ ein $\alpha\in A^n$ existiert mit $p(a,b)=f(\alpha,b)$. Es muss für festes $a\in I_A$ ein Intervall um $b$ in $I_A$ mit dieser Eigenschaft geben, denn sonst wäre $b\in\dcl(A)=A$.\\
	Sei jetzt $a$ nicht mehr fixiert, dann existiert mit Definable Choice in $I_A$ eine definierbare Zuordnung $a\mapsto\alpha(a)$, sodass $p(a,\cdot)=f(\alpha(a),\cdot)$ auf einem Intervall gilt. Da jedes $a$ $n+1$ viele Einträge hat und jedes $\alpha(a)$ $n$ viele, müssen unendlich viele $a\in (I_A)^{n+1}$ existieren, die durch $\alpha$ auf das selbe Element abgebildet werden. Denn wenn das nicht so wäre, wäre ein generisches Element aus $(I_A)^{n+1}$ algebraisch über einem Element aus $A^n$, was der Generizität widerspricht. Da es unendlich viele Elemente gibt, sodass $\alpha$ auf ihnen konstant ist, gibt es schon eine Zelle von Dimension $>0$ mit der Eigenschaft und damit insbesondere eine Zelle $E$ von Dimension 1 (Als Teilmenge einer Zelle lässt sich immer eine von kleinerer Dimension finden). Nenne den konstanten Wert dann $\alpha^*$.\\
	Da also gilt: für alle $a\in E$ existiert ein Intervall $J$ mit $p(a,\cdot)=f(\alpha^*,\cdot)$ auf $J$, existieren mit Definable Choice $\beta^*,\gamma^*:E\rightarrow I_A$, sodass $p(a,\cdot)=f(\alpha^*,\cdot)$ auf $(\beta^*(a),\gamma^*(a))$ gilt. \OE\ seien $\beta^*$ und $\gamma^*$ jetzt schon stetig auf $E$ und ein $e\in E$ beliebig· Dann existiert für $\varepsilon$ hinreichend klein eine $E$-Umgebung $U$ um $e$, sodass $$\beta^*<\frac{1}{2}(\beta^*(e)+\gamma^*(e))-\varepsilon,\ \frac{1}{2}(\beta^*(e)+\gamma^*(e))+\varepsilon<\gamma^*$$ auf $U$, also $$p(a,x)=f(\alpha^*,x)\text{ für alle }a\in U,x\in(\frac{1}{2}(\beta^*(e)+\gamma^*(e))-\varepsilon,\frac{1}{2}(\beta^*(e)+\gamma^*(e))+\varepsilon)$$ gilt. Es kann aber nicht $p(a-a',x)=p(a,x)-p(a',x)=f(\alpha^*,x)-f(\alpha^*,x)=0$ für $a,a'\in U$ verschieden und unendlich viele $x$ sein, weil ein Nichtnullpolynom nicht unendlich viele Nullstellen haben kann.
\end{proof}

\newpage

\begin{corollary}
	Es sei $(A,B)\models\td,\ f:B^{n+1}\rightarrow B$ $A$-definierbar in $B$ und $b\in B\setminus A$. Dann enthält $f(A^n\times\{b\})$ kein Intervall um $b$.
\end{corollary}
\begin{proof}
	Nimm an, dass das Gegenteil gelte für das Intervall $J$ (\OE\ mit Randpunkten $c<d\in A$): Dann ist $x\mapsto\frac{1}{c-x}+\frac{1}{d-x}$ eine ordnungstreue $A$-definierbare Bijektion $(c,d)\rightarrow A$. Diese erzeugt in $A$ eine definierbare angeordnete Körperstruktur auf $(c,d)=J_A$, also auch auf $J$ eine $A$-definierbare angeordnete Körperstruktur. Mit dem vorigen Lemma existiert ein Element aus $J\setminus f(A^n\times\{b\})$.
\end{proof}

\begin{theorem}\label{Kleinheit}
	Wenn $(B,A)\models\td$, dann ist kein Intervall eine kleine Teilmenge.
\end{theorem}
\begin{proof}
	Sei $f:B^n\rightarrow B$ eine durch $\varphi(x,y,b)$ definierbare Abbildung mit $\varphi$ eine\linebreak$\lingua_A$-Formel und $b\in B^m$ für ein $m\in\setN$ definiert. Für $\dim(b/A)=0$ ist $f(A^n)\subseteq A$ klar, deswegen sei \OE\ $\dim(b/A)\geq1$. Definiere
	\begin{align*}
	g(x,z):=\left\{\begin{array}{ll}
	\text{das eindeutige }y\in B &\text{für alle z, für die }\varphi(x,y,z)\\
	\text{ mit }B\models\varphi(x,y,z) &\text{ bei festem }z\text{ eine Funktion definiert}\\
	\ &\ \\
	0 &\text{sonst}
	\end{array}\right.,
	\end{align*}
	Dann ist $g$ in $B$ $A$-definierbar und $g(\cdot,b)=f$. Falls $\dim(b/A)>1$, füge genug Komponenten von $b$ zu $A$ hinzu, sodass $\dim(b/A)=1$. Das Hinzufügen einer Komponente $b_j$ zu $A$ ändert nichts, denn $Ab_j$ ist nach den Eingangsbemerkungen Modell von T und $Ab_j$ ist erst recht dicht in, aber nicht gleich $B$ (sonst hätte man die Dimension mit diesem Schritt schon zu sehr verkleinert).\\
	Finde also $b_i$, sodass $A$-definierbare $(h_j)$ existieren mit $b_j=h_j(b_i)$ für alle $j$. Wenn jetzt $J\subseteq f(A^n)=g(A^n,b)=g(A^n,h(b_i))$ für ein Intervall $J$, dann widerspricht das der Aussage des letzten Lemmas für die Funktion $(x,y)\mapsto g(x,h(y))$.
\end{proof}

\begin{definition}
	Schreibe ab jetzt $P(\overline{x}):=\bigwedge\limits_{i=1}^\abs{x}P(x_i)$.
\end{definition}

\newpage

\begin{lemma}
	Wenn $(B,A)$ für ein unendliches $\kappa>\abs{\operatorname{T}}$ ein $\kappa$-saturiertes Modell von $\td$ ist, ist $\dim(B/A)\geq\kappa$.
\end{lemma}
\begin{proof}
	Sei $S$ eine Basis von $B/A$ mit $\abs{S}<\kappa$; zeige nun, dass es kein Erzeugendensystem sein kann. Das folgt aus der Saturation angewandt auf den partiellen Typen $$\{\forall\overline{y}\in P(x\neq t(\overline{y}))\mid t\ \lingua_S\text{-Term}\},$$ der endlich erfüllbar ist, weil die Negation jeder dieser Formeln \glqq{}$x$ ist in einer kleinen Menge\grqq{} impliziert. Wenn der Typ also nicht endlich erfüllbar wäre, würde eine endliche Vereinigung von kleinen Mengen ganz $B$ überdecken. Das kann aber nicht gelten, denn eine endliche Vereinigung von kleinen Mengen ist wieder klein.
\end{proof}

\begin{corollary}\label{Finden transz Elte}
	Da Intervalle nicht klein sind, zeigt der Beweis sogar, dass in einem $\kappa$-saturiertem Modell $(B,A)\models\td$, gegeben Mengen $S,S',S''\subset B$ mit $\abs{S},\abs{S'},\abs{S''}<\kappa$, ein transzendentes Element $b$ über $SA$ gefunden werden kann mit $a<b$ für alle $a\in S'$ und $b<c$ für alle $c\in S''$, sofern dieser Ordnungstyp von $b$ überhaupt konsistent ist.
\end{corollary}

\newpage

\section{Formelreduzierung in $\td$}
In diesem Abschnitt wird gezeigt, dass sich $\lingua^P$-Formeln modulo $\td$ sehr stark vereinfachen lassen. Indem dieses mit einem Back\&Forth-System gezeigt wird, erhält man zusätzlich eine sehr große Klasse von elementaren Abbildungen zwischen Modellen von $\td$.\\
In diesem Kontext wird wieder die $\acl$-Unabhängigkeit in einem Modell von T relevant, die im ersten Kapitel mit \glqq{}algebraisch disjunkt\grqq{} bezeichnet wurde. Man kann sich dafür folgende (teilweise schon bekannte) Fakten überlegen.
\colorbox{red}{Bis hierhin und nicht weiter}
\begin{lemma}\label{Unabhängigkeitsregeln}
	Seien $A,B,C,D$ Mengen in irgendeinem Modell von $T$.
	\begin{enumerate}
		\item Wenn $A$ und $B$ unabhängig über $C$ sind, sind $B$ und $A$ unabhängig über $C$ und $A\cap B\subseteq\acl(C)$ (in fast allen betrachteten Fällen wird sowieso $A,B\supseteq C$ und $C=\acl(C)$ gelten).
		\item Wenn $A$ und $B$ unabhängig über $C$ sind und $S\subseteq B$, dann sind auch $A\cup S$ und $B$ unabhängig über $C\cup S$.
		\item Wenn $A$ und $B$ unabhängig über $C$ sind, $A\subseteq S\subseteq\acl(A),B\subseteq S'\subseteq\acl(B)$, dann sind $S$ und $S'$ unabhängig über $C$.
		\item Wenn $A$ und $B$ unabhängig über $C$ sind und $D$ (algebraisch) unabhängig über $AB$, dann sind $A\cup D$ und $B$ unabhängig über $C$.
		\item Wenn $(D,C)\preceq(B,A)\models\tq$, dann sind $A$ und $D$ unabhängig über $C$.
		\item Wenn $(D,C)\subseteq(B,A)\models\tq,\ S\subseteq A$ und $A$ und $D$ unabhängig über $C$ sind, dann sind $A$ und $DS$ unabhängig über $CS$, $\langle D\cup S\rangle_{\lingua^P}=(DS,CS)$ und $$(D,C)\subseteq(DS,CS)\subseteq(B,A).$$
		\item Wenn $(D,C)\subseteq(B,A)\models\tq$ und $S\subseteq B$ unabhängig über $DA$ ist, dann sind $A$ und $DS$ unabhängig über $C$, $\langle D\cup S\rangle_{\lingua^P}=(DS,C)$ und $$(D,C)\subseteq(DS,C)\subseteq(B,A).$$
	\end{enumerate}
\end{lemma}
\newpage
\begin{proof}
	1.-4. sind bekannt.
	\item[5.] Wenn $\overline{d}\in D$ algebraisch unabhängig über $C$ ist, aber nicht über $A$, dann existiert eine $\lingua_A$-Formel $\varphi(\overline{x},\overline{a})$, sodass \OE\ $d_1$ von $\varphi(x_1,d_2,d_3\dots,\overline{a})$ algebraisiert wird (\OE\ wird $d_1$ schon durch $\varphi$ definiert). Also erfüllt $\overline{d}$ die $\lingua^P$-Formel $$\exists \overline{y}\in P(\varphi(\overline{x},\overline{y})\land\forall z_2,z_3,\dots\exists! z_1(\varphi(\overline{z},\overline{y})))$$ in $(B,A)$, also auch in $(D,C)$. Es existiert also $\overline{c}\in C$ mit $$B\models\varphi(\overline{d},\overline{c})\land\forall z_2,z_3,\dots\exists! z_1(\varphi(\overline{z},\overline{c})),$$ was im Widerspruch zur Unabhängigkeit von $\overline{d}$ über $C$ steht.
	\item[6.] Dass $A$ und $DS$ unabhängig über $CS$ sind, ergibt sich in der Kombination von 2. und dann 3.\\
	Dass die Trägermenge von $\langle D\cup S\rangle_{\lingua^P}$ die Menge $DS$ ist, ergibt sich direkt per Definition als $DS=\dcl(D\cup S)=\langle D\cup S\rangle_\lingua$. Weil $A$ und $DS$ unabhängig über $CS$ sind, folgt $$P(\langle D\cup S\rangle_{\lingua^P})=DS\cap P(B)=DS\cap A=CS.$$
	\item[7] Es ergibt sich aus 4. und 3. dass $A$ und $DS$ unabhängig über $CS$ sind. Der Rest geht analog zu 6.
\end{proof}

Zu bemerken ist, dass ein Spezialfall von Unabhängigkeit viele nützliche Eigenschaften hat. Auf diesen wird später noch oft zurückgegriffen werden.
\begin{definition}
	Seien $(D,C)\subseteq(B,A)$ zwei Modelle von $\tq$. Dann heiße diese Inklusion \textbf{frei}, wenn $D$ und $A$ unabhängig über $C$ sind.
\end{definition}

\begin{lemma}\label{Kodichte von A}
	Sei $(B,A)\models\td$. Dann ist $A$ auch kodicht in $B$.
\end{lemma}
\begin{proof}
	Zu zeigen ist, dass für alle $a,c\in B$ ein $b\in B\setminus A$ existiert mit $a<b<c$. Durch Translation und additive Inversion kann man annehmen, dass $a=0$. Wähle jetzt ein $d\in B\setminus A$ beliebig und $e\in A$ mit $d-c<e<d$. Dann ist $d-e$ nicht in $A$ (denn sonst wäre es $d$) und $0=e-e<d-e<d-(d-c)=c$.
\end{proof}

\newpage
Für die Konstruktion des gewünschten Back\&Forth-Systems sei $\kappa>\abs{T}$ eine beliebige, aber feste Kardinalzahl und $(B,A),(D,C)\models\td$ zwei $\kappa$-saturierte Modelle.
\begin{theorem}\label{BackForth}
	Sei $S$ die Menge aller partiellen Isomorphismen zwischen Unterstrukturen $(B',A')$ von $(B,A)$ und $(D',C')$ von $(D,C)$ der Mächtigkeit $<\kappa$, sodass die Inklusionen frei sind. Dann bildet $S$ ein nichtleeres B\&F-System und $\td$ ist insbesondere vollständig.
\end{theorem}
\begin{proof}
	Das System ist nichtleer, denn es gibt ein Primmodell $\fM$ von $T$, weil $T$ vollständig ist und in jedem Modell $A$ alle Eigenschaften von $\fM_A:=\langle\emptyset\rangle_\lingua$ in $T$ beschrieben werden. Klarerweise ist $\abs{M}=\abs{T}<\kappa$. Der Isomorphismus $(\fM_A,\fM_A)\cong(\fM_C,\fM_C)$ liegt in $S$, denn Unabhängigkeit ist bei zwei gleichen Mengen offensichtlich.\\
	Sei jetzt $S\ni i:(B',A')\rightarrow(D',C')$ und $b\in B$. Wenn $b\in B'$ ist, ist nichts zu zeigen. Wenn $b\in A\setminus B'$, betrachte den partiellen Typ über $D'$ $$\{\alpha<x\mid i^{-1}(\alpha)<b\}\cup\{x<\beta\mid b<i^{-1}(\beta)\}\cup\{P(x)\}.$$
	Dieser ist konsistent, da $i$ ein Isomorphismus ist und $C$ dicht in $D$; mit Saturation existiert ein $d\in C\setminus D'$ mit diesem Ordnungstyp. $i$ setzt sich dann eindeutig zu einem Isomorphismus $i':(B'b,A'b)\rightarrow(D'd,C'd)$ mit $i(b)=d$ fort, der gegeben ist durch die Abbildung $t(b)\mapsto i(t)(d)$ für $t$ einen $\lingua_{B'}$-Term und $i(t)$ den durch $i$ geshifteten Term. Die Surjektivität dieser Abbildung ist klar, ebenso dass $i'(A'b)=C'd$. Wohldefiniertheit, Injektivität und Isomorphismuseigenschaft gelten, denn:\\
	$Rt_1(b)\dots t_n(b)$ gilt für $\lingua_{B'}$-Terme $t_1,\dots,t_n$ und eine Relation $R$ genau dann, wenn es ein $B'$-definierbares Intervall $I$ um $b$ mit dieser Eigenschaft gibt (denn sonst wäre $b$ definierbar über $B'$ und somit in $B'$). Schickt man $I\cap B'$ mit $i$ nach $J:=i(I\cap B')$, so gilt für alle Elemente $z\in J$, dass $Ri(t_1)(z)\dots i(t_n)(z)$, da $i$ ein Isomorphismus ist. Wäre jetzt nicht $Ri(t_1)(d)\dots i(t_n)(d)$, so gäbe es ein $D'$-definierbares Intervall $I'$ um $d$, sodass das nicht gilt; insbesondere ist $I'$ disjunkt zu $J$. Allerdings ist $$d\in I'\cap\operatorname{convex}(J)=I'\cap(i(\inf I),i(\sup I)),$$ also können $I'$ und $J$ nicht disjunkt sein. Es gilt also $Ri(t_1)(d)\dots i(t_n)(d)$.\\
	Die Rückrichtung geht analog.\\
	Zu zeigen ist nun, dass $B'b$ und $A$ frei über $A'b$ sowie $D'd$ und $C$ frei über $C'd$ sind, ebenso zu zeigen ist noch, dass $(B'b,A'b)\subseteq(B,A),(D'd,C'd)\subseteq(D,C)$. Das alles folgt aber aus Lemma \ref{Unabhängigkeitsregeln} (6.). Außerdem gilt $\abs{D'd}=\abs{B'b}=\abs{B'}+\abs{T}<\kappa$.\\
	Sei jetzt $b\in B'A\setminus(A\cup B')$. Dann gibt es $\overline{a}\in A$ mit $b\in B'\overline{a}$. Erweitere wie schon bekannt $i$, sodass $\overline{a}\in\operatorname{dom}(i)$; dann ist schon ganz $B'\overline{a}\subseteq\operatorname{dom}(i)$, also auch $b$.\\
	Abschließend sei $b\in B\setminus B'A$;  wie oben erfülle dann den mit $i$ geshifteten Ordnungstyp von $b$ über $B'$ mit einem Element $d\in D\setminus D'C$ (mit Folgerung \ref{Finden transz Elte} geht das). Wie oben kann $i$ dann auf einen Isomorphismus $(B'b,A')\rightarrow(D'd,C')$ fortgesetzt werden und nach Lemma \ref{Unabhängigkeitsregeln} (7.) erfüllen $(B'b,A'),(D'd,C')$ auch die hinreichenden Eigenschaften.
\end{proof}

Dieses B\&F-System beweist die Formelreduzierung in $\td$.
\begin{theorem}\label{Formelreduzierung}
	Jede $\lingua^P$-Formel ist modulo $\td$ äquivalent zu einer booleschen Kombination von Formeln der Gestalt
	$$\exists\overline{y}\in P(\phi(\overline{x},\overline{y}))$$
	für $\phi$ eine $\lingua$-Formel. Nenne eine solche boolesche Kombination eine \textbf{gute Formel} und eine Formel der Gestalt wie beschrieben eine \textbf{gute Formel in Reinform}.
\end{theorem}
\begin{proof}
	\underline{Hilfsaussage:}\\
	Es reicht zu zeigen, dass für alle Modelle $(B,A),(D,C)\models\td$ und für alle $b\in B^n,d\in D^n$ gilt: Wenn $b$ und $d$ dieselben guten Formeln erfüllen, sind ihre Typen in $(B,A)$ und $(D,C)$ dieselben.\\
	Dass dies ausreicht, erkennt man mit dem Ziegler'schen Trennungslemma: Sei $\psi\in\fF_n\lingua^P$ nicht äquivalent zu einer guten Formel und nenne die Menge aller guten Formeln in $n$ freien Koordinaten $K$. Dann ist $K$ abgeschlossen unter $\land,\lor$ und enthält $\top,\bot$. Wenn $\psi$ nicht äquivalent zu einer Formel aus $K$ ist, sind $\td\cup\{\psi\}$ und $\td\cup\{\neg\psi\}$ nicht durch $K$ trennbar, also existieren $(B,A),(D,C)\models\td,b\in B^n,d\in D^n$, sodass $(B,A)\models\psi(b)$ und $(D,C)\models\neg\psi(d)$, aber $(B,A)\models\chi(b)$ genau dann, wenn $(D,C)\models\chi(d)$ für alle $\chi\in K$. Dann erfüllen $b$ und $d$ dieselben guten Formeln, aber haben nicht denselben Typ - ein Widerspruch!\\
	\begin{proof}[Beweis der Hilfsaussage]
		Seien $b,d$ wie verlangt und $(B,A),(D,C)$ schon \OE\linebreak $\abs{T}^+$-saturiert (das ändert nichts an Typen und dem Erfüllen von guten Formeln). Sei $a\in A^m$ für ein hinreichend großes $m$, mit der Eigenschaft dass $\dim(b/a)\leq\dim(b/A)$ (es folgt dann Gleichheit, da über einer kleineren Menge nicht mehr interdefinierbar werden kann). Für $A':=\dcl(a),B':=\dcl(a,b)$ gilt dann, dass $A$ und $B'$ unabhängig über $A'$ sind. Es sind nämlich per Definition von $a$ die Mengen $A$ und $b$ unabhängig über $a$ (eben wegen $\dim(b/a)=\dim(b/A)$), mit Lemma \ref{Unabhängigkeitsregeln} (2.) sind dann auch $A$ und $b\cup(A')$ unabhängig über $A'$ und mit 3. sind $A$ und $B'=\dcl(b\cup A')$ unabhängig über $A'$. Außerdem sind $A'$ und $B'$ maximal $\abs{T}$ groß.\newpage
		Wenn man den partiellen $\lingua^P$-Typ $\tp_\lingua(a/b)\cup\{P(\overline{x})\}$ betrachtet, bleibt er konsistent unter der Ersetzung $b\rightarrowtail d$ in den Formeln. Seien nämlich $\psi_1(\overline{x},b),\dots,\psi_n(\overline{x},b)\in\tp_\lingua(a/b)$, dann ist $$\exists\overline{x}\in P(\bigwedge\limits_{i=1}^n\psi_i(\overline{x},\overline{y}))$$ eine gute Formel, die von $b$ und daher auch von $d$ erfüllt wird. Also ist der ersetzte partielle Typ endlich konsistent, wegen Saturation habe er den Erfüller $c\in C$ und es gilt $\tp_\lingua(a,b)=\tp_\lingua(c,d)$. Wegen der Typengleichheit folgt insbesondere $\dim(b/a)=\dim(d/c)$; es bleibt noch zu zeigen, dass $\dim(b/A)=\dim(d/C)$, damit dann gilt $\dim(d/C)=\dim(b/A)=\dim(b/a)=\dim(d/c)$ und wie oben $C$ und $D':=\dcl(c,d)$ frei über $C':=\dcl(c)$ sind. Die Gleichheit $\dim(b/A)=\dim(d/C)$ gilt aber, da für jede $\lingua$-Formel $\psi$ und $j_1,\dots,j_n\in\setN$ die Formel zu $$\glqq{}\text{es existiert }\overline{y}\in P\text{, sodass }\psi(\overline{x},\overline{y})\ x_i\text{ über }x_{j_1},\dots,x_{j_m}\text{ definiert}\grqq{}$$ eine gute Formel ist, die also genau dann von $b$ erfüllt wird, wenn sie von $d$ erfüllt wird.\\
		Da $(a,b)$ und $(c,d)$ den gleichen $\lingua$-Typ haben, gibt es einen partiellen Isomorphismus $i$ von $B'=\dcl(a,b)$ nach $D'=\dcl(c,d)$ mit $i((a,b))=(c,d)$, die Einschränkung auf $A'=\dcl(a)$ bildet einen Isomorphismus nach $C'=\dcl(c)$. Also ist $i$ partieller Isomorphismus $(B,A)\rightarrow(D,C)$, damit im B\&F-System, also elementare Abbildung, weswegen $b$ und $d$ denselben $\lingua^P$-Typen haben.
	\end{proof}
\end{proof}

\begin{corollary}\label{Definierbarkeit aus A}
	Für ein dichtes Paar $(B,A)$ und $S\subseteq B^n$ eine $\lingua^P_{A_0}$-definierbare Menge (wobei $A_0\subseteq A$) ist $S\cap A^n$ eine $\lingua_{A_0}$-definierbare Menge.
\end{corollary}
\begin{proof}
	Nach der Formelreduzierung sei $S$ \OE\ durch eine gute Formel definiert. Da die Definierbarkeit abgeschlossen unter booleschen Kombinationen ist, reicht es, eine Formel in Reinform zu betrachten.\newpage
	Da aber für jede $\lingua_{A_0}$-Formel $\varphi(x,y,a')$ und jedes $a\in A^n$ die Aussagen $$\glqq{}\text{Es existiert ein }y\in A^m\text{ mit }(B,A)\models\varphi(a,y,a')\grqq{},$$ $$\glqq{}\text{Es existiert ein }y\in A^m\text{ mit }B\models\varphi(a,y,a')\grqq{},$$ $$\glqq{}\text{Es existiert ein }y\in A^m\text{ mit }A\models\varphi(a,y,a')\grqq{}$$ äquivalent sind wegen $\varphi$ als $\lingua$-Formel und $A\prec B$, folgt, dass $$\exists y\in P(\varphi(x,y,a'))(B)\cap A^n=\exists y(\varphi(x,y,a'))(A).$$
\end{proof}

\section{Folgen der Existenz des B\&F-Systems}
Im Folgenden werden einige Anordnungen von wechselseitigen Inklusionen von Modellen von T betrachtet, in der Gleichheit von bestimmten Typen folgt.

\begin{lemma}\label{freie Inklusionen}
	Für dichte Paare $(B,A),(D,C)$ mit $(D,C)\subseteq(B,A)$ sind folgende Eigenschaften äquivalent:
	\begin{enumerate}
		\item $(D,C)\preceq(B,A)$
		\item Die Inklusion ist frei.
	\end{enumerate}
\end{lemma}
\begin{proof}
	$\glqq{}1.\Rightarrow2.\grqq{}:$ Diese Richtung ist schon aus Lemma \ref{Unabhängigkeitsregeln} (5.) bekannt.\\
	$\glqq{}2.\Rightarrow1.\grqq{}:$ Finde $(\abs{B}+\abs{T})^+$-saturierte Strukturen $$(B,A)\preceq(B',A'),(D,C)\preceq(D',C');$$ es ist dann $(D,C)$ eine gemeinsame Unterstruktur und $(D,C)\subseteq(D',C')$ ist frei nach dem Beweis der Gegenrichtung. Außerdem sind nach Voraussetzung $D$ und $A$ unabhängig über $C$, da aber Unabhängigkeit von Tupeln in $D$ über $A$ auch über $A'$ erhalten bleibt (da $(B',A')$ elementare Oberstruktur), ist auch $(D,C)\subseteq(B',A')$ frei. Also ist die Identität auf $(D,C)$ im Back\&Forth-System, daher elementare Abbildung. Daraus folgt für alle $\lingua^P_D$-Formeln $\varphi$, dass $$(D,C)\models\phi\Leftrightarrow(D',C')\models\varphi\Leftrightarrow(B',A')\models\varphi\Leftrightarrow(B,A)\models\varphi.$$
\end{proof}

\begin{lemma}\label{Gemeinsame Unterstruktur}
	Seien $(B_1,A_1),(B_2,A_2)\models\td$ und $(B,A)$ eine gemeinsame Unterstruktur, sodass die Inklusionen frei sind. Wenn $a\in (A_1)^n$ und $b\in (A_2)^n$ denselben $\lingua$-Typen über $B$ erfüllen, erfüllen sie auch denselben $\lingua^P$-Typen über $B$.
\end{lemma}
\begin{proof}
	\OE\ seien $(B_1,A_1)$ und $(B_2,A_2)$ schon genügend saturiert, das ändert nichts an Typen über $B$ und (nach derselben Argumentation wie im vorigen Lemma) auch nichts an der Unabhängigkeit. Da $a$ und $b$ denselben Typen über $B$ erfüllen, kann man wieder $\lingua_B$-Terme mit eingesetztem $a$ auf $\lingua_B$-Terme mit eingesetztem $b$ abbilden (Wohldefiniertheit und Injektivität wird durch die Typengleichheit ermöglicht) und bekommt einen partiellen Isomorphismus $i:Ba\cong Bb$, dessen Einschränkung auf die $\lingua_A$-Terme einen partiellen Isomorphismus $Aa\cong Ab$ induziert und sodass $i(a)=b$. Also gilt $i:(Ba,Aa)\cong(Bb,Ab)$, da außerdem die Inklusionen $(Ba,Aa)\subseteq(B_1,A_1)$ und $(Bb,Ab)\subseteq(B_2,A_2)$ frei sind nach Lemma \ref{Unabhängigkeitsregeln} (6.), ist $i$ im Back\&Forth-System, also elementar, also haben $a$ und $b$ denselben $\lingua^P$-Typen über $B$.
\end{proof}

\begin{corollary}
	Wenn man sich solch ein Paar $(a,b)$ beliebig wählt (z.B. $a=b=0$), sind in dem Typen auch die parameterfreien $\lingua^P_B$-Formeln, die in $(B_1,A_1)$ bzw. $(B_2,A_2)$ gelten. Also gelten dieselben Formeln, was als $(B_1,A_1)\equiv_B(B_2,A_2)$ geschrieben wird.
\end{corollary}

\begin{lemma}\label{selber Schnitt}
	Seien $(B_1,A_1),(B_2,A_2)$ zwei dichte Paare und $A\subseteq A_1\cap A_2$ eine gemeinsame Substruktur, sowie $a\in B_1\setminus A_1,\ b\in B_2\setminus A_2$, die den gleichen Ordnungstyp über $A$ haben. Dann haben $a$ und $b$ sogar den gleichen $\lingua^P$-Typ über $A$.
\end{lemma}
\begin{proof}
	Es sind trivialer Weise $A_i$ und $A$ unabhängig über $A$ für $i=1,2$, außerdem ist $a$ transzendent über $A_1$ und $b$ transzendent über $A_2$. Nach Lemma \ref{Unabhängigkeitsregeln} (4.) sind also die Einbettungen $(Aa,A)\subseteq(B_1,A_1)$ und $(Ab,A)\subseteq(B_2,A_2)$ frei. Nach dem Beweis zu Satz \ref{BackForth} gibt es also einen Isomorphismus $Aa\cong Ab$, der $A\cong A$ fortsetzt und für den $i(a)=b$ gilt, also gibt es einen Isomorphismus $i:(Aa,A)\cong(Ab,A)$.\\ Wenn \OE\ die beiden Modelle von $\td$ genügend saturiert sind, ist $i$ im B\&F-System, also erfüllen $a$ und $b$ dieselben Formeln.
\end{proof}

\begin{lemma}\label{Speziell definierbare kleine Mengen}
	Sei $(D,C)\subseteq(B,A)$ frei und $(B,A)\models\td$, sowie $X\subseteq B$ eine kleine $\lingua^P_D$-definierbare Menge. Dann gibt es sogar eine $\lingua_D$-definierbare Funktion $f:B^n\rightarrow B$, die die Kleinheit von $X$ bezeugt.
\end{lemma}
\begin{proof}
	Man kann für $(B,A)$ schon hinreichende Saturiertheit annehmen. Wenn es keine solche Funktion gäbe, wäre $$\{x\in X\}\cup\{x\notin f(A^n)\mid f:B^n\rightarrow B\ D\text{-definierbar}\}$$ eine konsistente Formelmenge über $D$; ihr Erfüller sei $b\in B\setminus DA$. Da $X$ klein ist, kann man den Ordnungstyp von $b$ über $D$ auch durch ein Element $\tilde{b}\in B\setminus(X\cup DA)$ erfüllen. Nach Konstruktion des B\&F-Systems haben $b$ und $\tilde{b}$ dann aber den selben $\lingua^P_D$-Typ über $D$, was im Widerspruch zu $b\in X,\tilde{b}\notin X$ steht.
\end{proof}

\section{Definierbare Teilmengen von $A^n$}
Wir interessieren uns für die Gestalt von $\lingua^P$-definierbaren Teilmengen von $A^n$. Dafür braucht man zuerst eine Hilfsaussage für definierbare Mengen in o-minimalen Strukturen.

\begin{lemma}
	Sei $\fM$ eine o-minimale Struktur, die eine angeordnete Gruppenoperation $+$ mit positivem Element 1 hat und $Y\subseteq M^n$ definierbar. Dann ist $Y$ eine endliche Vereinigung von Mengen der Form $\{f(b,\cdot)=0,g(b,\cdot)>0\}$, wobei $b\in M^m$ und $f,g$ stetige, 0-definierbare Abbildungen $M^{m+n}\rightarrow M$ sind.
\end{lemma}
\begin{proof}
	Schreibe $Y=\phi(b,\fM)$ für ein $b\in M^m$ und definiere $Z:=\phi(\fM)$. Wenn man $Z$ in Zellen $(Z_i)_i$ zerlegt, erhält man $Y$ als endliche Vereinigung von $((Z_i)_b)_i$.  Es sei also o.B.d.A. $Z$ schon eine 0-definierbare Zelle.\\
	Definiere $$f(x):=\left\{\begin{array}{ll}
	\inf\{\abs{x-d}\mid d\in Z\}&Z\text{ nichtleer}\\
	1&\text{sonst}
	\end{array}\right.,$$
	$$g(x):=\left\{\begin{array}{ll}
	\inf\{\abs{x-d}\mid d\in \overline{Z}\setminus Z\}&Z\text{ nichtleer}\\
	1&\text{sonst}
	\end{array}\right.,$$ das sind lipschitzstetige Funktionen.\\
	Klar ist, dass $\overline{Z}=\{f=0\}$; da Zellen lokal abgeschlossen sind, ist $\overline{Z}\setminus Z=\overline{\overline{Z}\setminus Z}=\{g=0\}$. Also erhalten wir $$Z=\overline{Z}\setminus(\overline{Z}\setminus Z)=\{f=0\}\setminus\{g=0\}=\{f=0\}\cap\{g>0\}$$ und $$Y=Z_b=\{f(b,\cdot)=0\}\cap\{g(b,\cdot)>0\}.$$
\end{proof}

\begin{theorem}\label{Definierbare Mengen}
	Für ein dichtes Paar $(B,A)$ und $Y\subseteq A^n$ ist folgendes äquivalent:
	\begin{enumerate}
		\item $Y$ ist $\lingua^P$-definierbar.
		\item Es existiert ein $\lingua$-definierbares $Z\subseteq B^n$, sodass $Y=Z\cap A^n$.
		\item $Y$ ist definierbar in $(A,(R_b)_{b\in B})$ mit der Interpretation $A\models R_b(a)$ genau dann, wenn $0<a<b$ in $B$.
	\end{enumerate}
\end{theorem}
\begin{proof}
	$\glqq{}1.\Rightarrow 2.\grqq{}:$ Sei $\varphi$ eine $\lingua^P_B$-Formel mit $\varphi(B)=Y$. Zu zeigen ist, dass eine $\lingua_B$-Formel $\psi$ existiert mit $(B,A)\models P(x)\rightarrow(\varphi(x)\leftrightarrow\psi(x))$; das ist genau dann der Fall, wenn $\mathfrak{Th}(B,A)_B\cup\{P(x)\}\cup\{\varphi(x)\}$ und $\mathfrak{Th}(B,A)_B\cup\{P(x)\}\cup\{\neg\varphi(x)\}$ in $\lingua_B$ getrennt werden können. Nach dem Trennungslemma gilt das genau dann, wenn für alle $(B,A)\preceq(D_1,C_1),(D_2,C_2)$ und alle $c_i\in C_i\ (i=1,2)$ mit $(D_1,C_1)\models\varphi(c_1),(D_2,C_2)\models\neg\varphi(c_2)$ eine $\lingua_B$-Formel $\chi$ existiert mit $(D_1,C_1)\models\chi(c_1),(D_2,C_2)\models\neg\chi(c_2)$.\\
	Seien solche $(D_i,C_i)$ und $c_i$, die die Voraussetzungen von oben erfüllen. Dann ist das die Situation aus Lemma \ref{Gemeinsame Unterstruktur}, denn elementare Erweiterungen sind frei. Also muss ein trennendes $\chi$ wie verlangt existieren, denn ansonsten würden $c_1$ und $c_2$ denselben $\lingua$-Typ erfüllen, aber nicht denselben $\lingua^P$-Typ.\\
	$\glqq{}2.\Rightarrow 3.\grqq{}:$ Sei $Y=Z\cap A^n$. Nach dem letzten Lemma ist $Z$ eine boolesche Kombination aus Mengen der Form $\{f(b,\cdot)=0\}$ und $\{g(b,\cdot)>0\}$ für stetige 0-$\lingua$-definierbare Funktionen $f,g$ und passende $b\in B^m$. Es reicht also die Aussage für Mengen in diesen Formen zu zeigen. Wegen der Stetigkeit der Funktionen und $A$ dicht in $B$ gilt aber in $B$
	\begin{align*}
	f(b,z)=0\Leftrightarrow\ &\text{Für alle }0<\varepsilon\in A\text{ existiert }A^m\ni a<b\text{ (koordinatenweise),}\\&\text{sodass für alle }a'\in A^m\text{ mit }a<a'<b\text{ (koordinatenweise)}\\&\text{gilt, dass }\abs{f(a',z)}<\varepsilon,\\
	g(b,z)>0\Leftrightarrow\ &\text{Es existiert ein }0<\varepsilon\in A\text{ und ein }A^m\ni a<b\text{ (koordinatenweise),}\\&\text{sodass für alle }a'\in A^m\text{ mit }a<a'<b\text{ (koordinatenweise)}\\&\text{gilt, dass }\abs{f(a',z)}>\varepsilon.
	\end{align*}
	Die rechten Bedingungen sind jeweils in $(A,(R_b)_{b\in B})$ definierbar.\\
	$\glqq{}3.\Rightarrow 1.\grqq{}:$ Da $A$ und alle $R_b$ in $(B,A)$ definierbar sind, ist $Y$ auch in $(B,A)$ definierbar.
\end{proof}

\begin{definition}
	Sei $(B,A)\models\tq$, dann heißt $S\subseteq B$ speziell, wenn $S$ und $A$ frei über $A\cap S$ sind.
\end{definition}

\begin{lemma}\label{Definierbare Mengen Spezialität}
	Der Fall 2. aus dem vorigen Lemma lässt sich wie folgt verallgemeinern: Wenn $Y\subseteq A^n$ eine $\lingua^P_D$-definierbare Menge ist für ein spezielles $D\subseteq B$, dann existiert ein $\lingua_D$-definierbares $Z\subseteq B^n$ mit $Y=Z\cap A^n$.
\end{lemma}
\begin{proof}
	Da eine Ersetzung $D\leftrightarrow\dcl(D)$ nichts ändert, können wir gleich davon ausgehen, dass $D$ schon ein Modell von T ist.\\
	Der Beweis geht dann analog, nur müssen diesmal $\mathfrak{Th}(B,A)_D\cup\{P(x)\}\cup\{\varphi(x)\}$ und $\mathfrak{Th}(B,A)_D\cup\{P(x)\}\cup\{\neg\varphi(x)\}$ in $\lingua_D$ getrennt werden. Das ergibt anstatt von $(B,A)\preceq(D_1,C_1),(D_2,C_2)$ hier $(D_1,C_1),(D_2,C_2)\models\mathfrak{Th}(B,A)_D$.\\
	Wegen der Spezialität von $D$ gilt, dass die Inklusionen $(D,D\cap A)\subseteq(D_1,C_1),(D_2,C_2)$ frei sind: Wenn $\varphi(c,x)$ nämlich in $(D_i,C_i)$ für ein $i$ die Abhängigkeit von $d\in D^n$ über $C_i$ bezeugt, muss $$\glqq{}\text{Es existiert ein }a'\in P\text{ mit }\varphi(a',x)\text{ bezeugt die Abhängigkeit von }d\text{ über }P\grqq{}$$ in $\mathfrak{Th}(B,A)_D$ gelten, also ist $d$ abhängig über $A$, also wegen Spezialität von $D$ auch abhängig über $D\cap A$. Ab dann kann man den Beweis wie bekannt weiterführen und muss nur $(B,A)$ durch $(D,D\cap A)$ ersetzen.
\end{proof}

\section{Definierbare Mengen in einer Variablen}
In diesem Abschnitt wird eine zur o-Minimalität ähnliche Charakterisierung von Mengen in einer Variable in Modellen von $\td$ hergeleitet. Ab jetzt sei $(B,A)$ ein beliebiges dichtes Paar und $S\subseteq B$ eine spezielle Menge, als Konvention nehmen wir an (TODO: nach vorne), dass $A^0=\{0\}$. Da $B$ selbst eine spezielle Menge ist, gelten die folgenden Aussagen insbesondere alle auch für Definierbarkeit unabhängig von irgendwelchen Parametermengen. Angelehnt ist die folgende Vorgehensweise (auch im nächsten Abschnitt) an die Modifikation in \cite{Piz} von \cite{VanDenDries}, Teile konnten noch präziser für spezielle Mengen gemacht werden.

\begin{lemma}
	Sei $Y\subseteq B^n$ $\lingua_S$-definierbar und $(U_y)_{y\in Y}\subseteq B$ eine Familie von offenen, uniform $\lingua_S$-definierbaren Mengen. Dann ist $$X:=\bigcup\limits_{y\in Y\cap A^n}U_y$$ auch $\lingua_S$-definierbar.
\end{lemma}
\begin{proof}
	Wir führen eine Induktion über $\dim Y$, man kann schon annehmen, dass alle $U_y$ nichtleer sind, sonst muss man entsprechend aus $Y$ aussondern. Wenn $\dim Y=0$ ist, ist $Y$ endlich und es ist schon $Y\cap A^n$ definierbar mit Parametern aus $S$ (zähle z.B. auf, das wievielte Element aus $Y$ man nimmt), also auch $X$.\\
	Zerlege $Y$ in Zellen, das ändert nichts an irgendwelchen Definierbarkeiten, also kann man annehmen, dass $Y$ selbst schon Zelle ist. Wenn $Y$ keine offene Zelle ist und $\pi:Y\rightarrow Z$ der kanonische homöomorphe Projektion zu einer offenen Zelle von Dimension $m<\dim Y$, existiert nach Satz \ref{Definierbare Mengen Spezialität} ein $\lingua_S$-definierbares $Z'\subseteq B^m$ mit $\pi(Y\cap A^n)=Z'\cap A^m$, da $\pi(Y\cap A^n)\subseteq A^m$ eine $\lingua^P_S$-definierbare Menge ist. Definiere dann $Y':=Z\cap Z'$, sodass man die Menge umparametrisieren kann:
	$$\bigcup\limits_{y'\in Y'\cap A^m}U_{\pi^{-1}(y')}=\bigcup\limits_{y\in\pi^{-1}(Y'\cap A^m)}U_y=\bigcup\limits_{y\in Y\cap A^n}U_y=S$$
	Da $(U_{\pi^{-1}(y')})_{y'\in Y'}$ die gleichen Dinge erfüllt wie $(U_y)_{y\in Y}$ und $\dim Y'\leq\dim Z<\dim Y$, gilt die Aussage per Induktion.\\
	Für den Beweis für offene Zellen definiere die vier $\lingua_S$-definierbaren Mengen $$U:=\bigcup\limits_{y\in Y}U_y,$$ $$Y_x:=\{y\in Y\mid x\in U_y\},$$ $$C:=\{x\in U\mid\inn(Y_x)=\emptyset\},$$ $$D:=\bigcup\limits_{x\in\inn C}Y_x.$$
	Es gilt $$D=\bigcup\limits_{x\in U}\{y\in Y\mid x\in U_y\cap\inn C\}=\{y\in Y\mid U_y\cap\inn C\neq\emptyset\}.$$
	Wir wollen jetzt herleiten, dass $\inn D$ leer sein muss. Seien dafür mit Definable Choice die $\lingua_S$-definierbaren Funktionen $f,g_1,g_2:D\rightarrow B$ gegeben mit $$f(y)\in(g_1(y),g_2(y))\subseteq U_y\cap\inn C.$$
	Wenn $\inn D\neq\emptyset$, schränke $D$ ein, sodass es offen ist und $f,g_1,g_2$ stetig auf $D$ sind. Sei außerdem $d,e\in U_y\cap\inn C, (c,k)\subseteq U_y\cap\inn C$ (das ist offen), sodass $$c<g_1(y)<d<f(y)<e<g_2(y)<k.$$ Setze $$V:=g_1^{-1}((c,d))\cap f^{-1}((d,e))\cap g_2^{-1}((e,k)),$$ das ist dann eine offene Umgebung um $y$ in $B^n$. Für alle $z\in V$ ist $$f(y)\in(d,e)\subseteq(g_1(z),g_2(z))\subseteq U_z,$$ also ist $z\in Y_{f(y)}$. Das heißt, es gilt $V\subseteq Y_{f(y)}$; da aber $f(y)\in C$, ist das unmöglich, weil $V$ offen in dem (eingeschränkten) Definitionsbereich, also auch in $B^n$ ist und also $Y_{f(y)}$ nichtleeres Inneres hätte.\\
	$D$ hat als Menge ohne Inneres eine kleinere Dimension als $n$. Also ist induktiv auch $$\bigcup\limits_{y\in D\cap A^n}U_y$$ definierbar in $B$. Es ist außerdem
	\begin{align}\label{align 1}
	\bigcup\limits_{y\in (Y\setminus D)\cap A^n}U_y\cup((C\setminus\inn C)\cap\bigcup\limits_{y\in Y\setminus D}U_y)=\bigcup\limits_{y\in Y\setminus D}U_y.
	\end{align}
	Dass die linke Seite Teilmenge der Rechten ist, ist sowieso klar per Definition; sei nun $$x\in\bigcup\limits_{y\in Y\setminus D}U_y\setminus (C\setminus\inn C)$$ und $y\in Y\setminus D$ mit $x\in U_y$. Wegen $x\notin C\setminus\inn C$ hat $Y_x$ nichtleeres Inneres ($x\notin C$) oder $x\in\inn C$ und $y\in D$. Da zweiteres ausgeschlossen wurde, hat $Y_x$ nichtleeres Inneres, also auch $Y_x\setminus D$ (denn $D$ war niedrigdimensional) und enthält damit ein Element $z\in A^n\cap(Y_x\setminus D)$ wegen Dichtheit. $z$ bezeugt, dass $x\in\bigcup\limits_{y\in (Y\setminus D)\cap A^n}U_y$ liegt.\\
	Aus \ref{align 1} folgt, dass $$\bigcup\limits_{y\in (Y\setminus D)\cap A^n}U_y=\bigcup\limits_{y\in Y\setminus D}U_y\setminus(C\setminus\inn C)\cup((C\setminus\inn C)\cap\bigcup\limits_{y\in (Y\setminus D)\cap A^n}U_y)$$ $\lingua_S$-definierbar ist (der letzte Teil ist es wegen Endlichkeit von $C\setminus\inn C$). Also ist auch $$X=\bigcup\limits_{y\in Y\cap A^n}U_y=\bigcup\limits_{y\in (Y\setminus D)\cap A^n}U_y\cup\bigcup\limits_{y\in D\cap A^n}U_y$$ $\lingua_S$-definierbar in $B$.
\end{proof}

\begin{theorem}
	Sei eine $\lingua^P_S$-definierbare Menge $X\subseteq B$. Dann stimmt $X$ bis auf eine kleine $\lingua^P_S$-definierbare Menge mit einer $\lingua_S$-definierbaren Menge $X'$ überein.
\end{theorem}
\begin{proof}
	Sei zunächst $X$ gegeben durch $\exists y\in P(\psi(x,y))$ für eine Formel $\psi\in\fF_{1+n}(\lingua_S)$. Die Mengen $F_y:=\psi(B,y)\cap\partial\psi(B,y)$ sind endlich für jedes $y\in M^n$ und weil o-minimale Theorien $\exists^\infty$ eliminieren, ist deren Mächtigkeit uniform beschränkt, sagen wir durch $k$. Sei $Y:=\exists x\psi(x,B)$ und $\lingua_S$-definierbare Funktionen $g_1,\dots,g_k:Y\rightarrow F_y$, sodass $F_y=\{g_1(y),\dots,g_k(y)\}$ für alle $y\in Y$. Als diese Funktionen kann man zum Beispiel die angeordnete Aufzählung der Elemente in $F_y$ nehmen. Dann gilt
	\begin{align*}
	X&=\exists y\in P(\psi(x,y))(B,A)=\bigcup\limits_{y\in A^n}\psi(B,y)=\bigcup\limits_{y\in Y\cap A^n}\psi(B,y)\\&=\bigcup\limits_{y\in Y\cap A^n}(\psi(B,y)\cap \partial\psi(B,y))\cup\bigcup\limits_{y\in Y\cap A^n}\inn\psi(B,y)\\&=\bigcup\limits_{y\in Y\cap A^n}\{g_1(y),\dots,g_k(y)\}\cup\bigcup\limits_{y\in Y\cap A^n}\inn\psi(B,y)\\&=\bigcup\limits_{i=1}^kg_i(Y\cap A^n)\cup \bigcup\limits_{y\in Y\cap A^n}\inn\psi(B,y).
	\end{align*}
	Da $X':=\bigcup\limits_{i=1}^kg_i(Y\cap A^n)$ klein ist und $S'':=\bigcup\limits_{y\in Y\cap A^n}\inn\psi(B,y)$ nach dem letzten Lemma $\lingua_S$-definierbar, stimmen $X$ und $X'$ bis auf die kleine $\lingua^P_S$-definierbare Menge $S''$ überein. Da Darstellungen \glqq{}$(X\setminus  S')\cup S'',X\ \lingua_S$-definierbar, $S',S''$ klein, $\lingua^P_S$-definierbar\grqq{} unter booleschen Kombinationen erhalten bleiben, folgt die Aussage für gute Formeln und daher für alle Mengen. Dabei ist zu beachten, dass Satz \ref{Formelreduzierung} auch hergibt, dass $\lingua^P_S$-Formeln zu guten $\lingua^P_S$-Formeln äquivalent sind.
\end{proof}

\begin{lemma}
	Sei $X\subseteq B$ eine kleine $\lingua^P_S$-definierbare Teilmenge. Dann ist $X$ eine endliche Vereinigung von Mengen $f(A^n\cap E)$ für $E$ eine offene $\lingua_S$-definierbare Zelle und $f:E\rightarrow B$ eine stetige $\lingua_S$-definierbare Funktion.
\end{lemma}
\begin{proof}
	Wenn $X$ klein ist, existiert ein $\lingua$-definierbares $g:B^m\rightarrow B$, sodass $X\subseteq g(A^m)$, nach Lemma \ref{Speziell definierbare kleine Mengen} kann man schon annehmen, dass $g$ über $\dcl(S)$ definierbar ist (denn $(\dcl(S),\dcl(S)\cap A)\subseteq (B,A)$ ist normal), also auch über $S$. Setze $X':=g^{-1}(X)\cap A^m=(g\upharpoonright A^m)^{-1}(X)$, das ist $\lingua^P_S$-definierbar und es gilt $g(X')=X$ wegen $X\subseteq\operatorname{im}(g\upharpoonright A^m)$.\\
	Beweise die Aussage jetzt induktiv über $m$: Wenn $m=0$ ist, ist $X$ maximal einelementig und entweder gleich $f(\{0\})$ für eine konstante Funktion $f$ oder schon die leere Vereinigung. Wegen Endlichkeit gilt $X\subseteq\dcl(S)$, also ist $f$ \--- sofern es definiert werden muss \--- über $S$ definierbar.\\
	Wenn $m>0$ ist, schreibe $X'$ als Teilmenge von $A^m$ wegen Lemma \ref{Definierbare Mengen Spezialität} in der Form $Y\cap A^m$ für ein $\lingua_S$-definierbares $Y$. Sei eine Zerlegung $\mathfrak{Z}$ von $Y$ in $\lingua_S$-definierbare Zellen gegeben, auf denen $g$ jeweils stetig ist, dann ist $$X=g(Y\cap A^m)=\bigcup\limits_{Z\in\mathfrak{Z}}g(Z\cap A^m).$$
	Für offene Zellen $Z$ ist so eine Darstellung also schon gefunden. Sei $Z$ nun eine Zelle der Dimension $d<m$ und $\pi:B^d\rightarrow B^m$ eine $\lingua_S$-definierbare Fortsetzung des kanonischen Homöomorphismus der entsprechenden offenen Zelle $Z'$ nach $Z$. Indem man mit Lemma \ref{Definierbare Mengen Spezialität} ein $\lingua$-definierbares $E\subseteq B^d$ findet mit $A^d\cap Z'\cap\pi^{-1}(A^m)=A^d\cap E$, erhält man 
	\begin{align*}
	g(Z\cap A^m)&=g(\pi|_{Z'}(\pi^{-1}(A^m)))=(g\circ\pi)(Z'\cap\pi^{-1}(A^m))\\&=(g\circ\pi)(A^d\cap Z'\cap\pi^{-1}(A^m))=(g\circ\pi)(A^d\cap E),
	\end{align*}
	wobei die mittlere Gleichheit gilt, da $A^d\cap Z'\supseteq Z'\cap\pi^{-1}(A^m)$, weil $\pi|_{Z'}$ die Umkehrabbildung einer Projektion ist.\\
	Die Gleichung ist dann aber schon der Fall eines kleineren $m$ und mit der Induktionsbehauptung folgt die Aussage.
\end{proof}

\begin{theorem}\label{Satz 4}
	Sei $X\subseteq B$ eine $\lingua^P_S$-definierbare Menge. Dann existiert eine endliche $\lingua_S$-definierbare Unterteilung von $B$, sodass für jedes dadurch erzeugte offene Intervall $I$ genau einer der folgenden Fälle gilt:
	\begin{itemize}
		\item $I$ ist disjunkt zu $X$.
		\item $I$ ist Teilmenge von $X$.
		\item $X\cap I$ ist dicht \& kodicht in $I$ und entweder ist $X\cap I$ klein oder $I\setminus X$.
	\end{itemize}
	Für kleine $X$ entfallen die Fälle \glqq{}Teilmenge\grqq{} und \glqq{}koklein\grqq{}.
\end{theorem}
\begin{proof}
	Sei $X$ zunächst klein und gegeben als $X=f(A^n\cap E)$ für ein stetiges, $\lingua_S$-definierbares $f:E\rightarrow B$ und eine offene $\lingua_S$-definierbare Zelle $E\subseteq B^n$. Da definierbarer Zusammenhang unter definierbaren stetigen Funktionen erhalten bleibt, ist $I:=f(E)$ auch definierbar zusammenhängend, weil es $\lingua_S$-definierbar ist, hat es auch ein Supremum und Infimum in $B\cup\{\pm\infty\}$, ist also ein Intervall (ausnahmsweise sei hier auch ein nichtoffenes Intervall mitgemeint). Wenn $I$ endlich ist, ist auch $X$ endlich und es ist nichts zu zeigen. $X$ ist disjunkt zu $B\setminus I$, nun muss nur noch gezeigt werden, dass $X=X\cap I$ dicht und kodicht in $I$ ist, denn Dichte und Kodichte in nichtoffenen unendlichen Intervallen ist äquivalent zu Dichte und Kodichte in deren Innerem. Aber für dichte Teilmengen werden durch stetige Abbildungen auf dichte Teilmengen abgebildet; da $A^n$ dicht in $B^n$ ist, ist auch $A^n\cap E$ dicht in $E$ und folglich $X$ dicht in $I$. Andererseits muss auch das Komplement von $X$ dicht in $I$ sein, da es sonst ein Intervall ganz in $X$ gäbe, was der Kleinheit mit Satz \ref{Kleinheit} widerspricht.\\
	Die gesuchte Eigenschaft für kleine Mengen bleibt unter Vereinigungen erhalten. Man muss nur eine Verfeinerung der Unterteilung durchführen und dann ausnutzen, dass endliche Vereinigungen von kleinen Mengen klein sind. Also gilt für $Y$ und $Z$ als kleine Teilmengen eines Intervalls $I$:
	\begin{itemize}
		\item $Y\cup Z$ klein
		\item $Y\cup Z=Y$, wenn $Z=\emptyset$
		\item $Y\cup Z$ dicht in $I$, wenn $Y$ und $Z$ dicht in $I$
		\item $Y\cup Z$ kodicht in $I$ als kleine Menge (s. Begr. oben)
	\end{itemize}
    Also gilt mit dem letzten Lemma die Aussage für kleine Mengen allgemein.\\
	Sei jetzt $X$ nicht mehr klein, dann stimmt es aber bis auf eine kleine Menge mit einer $\lingua$-definierbaren Menge $X'$ überein. Schreibe also $X=(X'\cup Y)\setminus Z$ für $Y$ und $Z$ klein und disjunkt. $X'$ ist als $\lingua$-definierbare Menge eine endliche Vereinigung von Punkten und Intervallen, da $Y,Z$  als kleine Mengen die oben beschriebene Gestalt haben, entspricht $X'\cup Y$ der Darstellung ohne den Fall koklein. In der Darstellung der drei Fälle wird der erste durch Subtraktion von $Z$ nicht geändert, der zweite bleibt oder wird zum dritten (2. Teil) und der dritte bleibt auch, da er nur von $Y$ kommt, aber $Y$ und $Z$ disjunkt sind. Also hat $X$ auch so eine Darstellung.
\end{proof}

\begin{corollary}
	Jede $\lingua^P_S$-definierbare Menge $X\subseteq B$ hat ein Supremum und Infimum in $\dcl(S)\cup\{\pm\infty\}$.
\end{corollary}

\begin{lemma}
	$\td$ eliminiert $\exists^\infty$: Wenn $S\subseteq B^{m+n}$ eine $\lingua^P$-definierbare Menge ist mit $S_x$ endlich für alle $x\in B^m$, dann ist $(\abs{S_x})_{x\in B^m}$ beschränkt.
\end{lemma}
\begin{proof}
	Für $n=1$ gilt das nach Bemerkung 5.33. aus \cite{Lukas}. Denn $S_x$ ist endlich, genau dann, wenn $S_x$ diskret in $B$ ist; und das ist uniform $\lingua^P$-ausdrückbar. Die Äquivalenz zur Diskretheit sieht man ein, indem man eine Aufteilung von $S_x$ wie im letzten Satz vornimmt. Dann ist $S_x$ genau dann endlich, wenn nur der Fall $\glqq{}S_x\cap I=\emptyset\grqq{}$ vorkommt; weil dichte Mengen und Intervalle nicht diskret sind, gilt das wiederum genau dann, wenn $S_x$ diskret ist.\\
	Sei $n>1$. Dann sind für $\pi_{i_1,\dots,i_k}$ als Projektionsabbildung auf die Koordinaten $i_1,\dots,i_k$ jeweils auch $$Y_x:=(\pi_{1,\dots,m+1}(S))_x=(\pi_{m+1}(S_x))$$ und $$Z_x:=(\pi_{1,\dots,m,m+2,\dots,m+n}(S))_x=(\pi_{m+2,\dots,m+n}(S_x))$$ endlich und daher ist nach Induktionsvoraussetzung die Mächtigkeit jeweils uniform beschränkt durch irgendwelche $K,L\in\setN$. Dann gilt aber $$\abs{S_x}\leq \abs{Y_x\times Z_x}=\abs{Y_x}\abs{Z_x},$$ was uniform durch $KL$ beschränkt ist.
\end{proof}

\section{Definierbare Funktionen und definierbarer Abschluss}
Um definierbare Funktionen besser zu verstehen, ist es notwendig, sich mit dem definierbaren Abschluss in $\lingua^P$ zu beschäftigen. Damit kann man zeigen, dass definierbare Funktionen in einer Variablen \glqq{}fast überall\grqq{} $\lingua$-definierbar sind. Wir erinnern uns, dass ein dichtes Paar $(B,A)$ fixiert war und $S\subseteq B$ speziell. Mit \glqq{}definierbar abgeschlossen\grqq{} ist im Folgenden \glqq{}$\lingua^P$-definierbar abgeschlossen\grqq{} gemeint.

\begin{lemma}\label{A definierbar abgeschl}
	$A$ definierbar abgeschlossen.
\end{lemma}
\begin{proof}
	Sei $b\in B\setminus A$ und $(B,A)\preceq(D,C)$ eine genügend saturierte Elementarerweiterung. Dann wird der Ordnungstyp von $b$ über $A$ auch von einem Element $D\setminus C\ni d\neq b$ realisiert wegen Dichtheit von $D\setminus C$ in $D$ und Saturation.\\
	Nach Lemma \ref{selber Schnitt} haben $b$ und $d$ dann den selben $\lingua^P$-Typen über $A$, weswegen $b$ nicht definierbar über $A$ in $(D,C)$ sein kann, also auch nicht in $(B,A)$.
\end{proof}

\begin{corollary}
	Sei $A_0\preceq A$. Dann ist $A_0$ definierbar abgeschlossen.
\end{corollary}
\begin{proof}
	Sei $b$ definierbar über $A_0$. Dann ist $b$ insbesondere definierbar über $A$, also in $A$. Nach Folgerung \ref{Definierbarkeit aus A} ist dann $\{b\}=\{b\}\cap A$ schon $\lingua$-definierbar aus $A_0$, also in $A_0$, da $A_0$ elementare Substruktur ist.
\end{proof}

\begin{lemma}\label{Freie Definierbarkeit}
	Sei $(D,C)\subseteq(B,A)$ frei, dann ist $D$ definierbar abgeschlossen in $(B,A)$.
\end{lemma}
\begin{proof}
	Für eine beliebige Struktur $(B',A')\succeq (B,A)$ gelten die Voraussetzungen ebenso, da $\acl$-Abhängigkeit über $A$ als Teil vom Typen äquivalent ist zu $\acl$-Abhängigkeit über $A'$. Ebenso ist $D$ in $(B,A)$ definierbar abgeschlossen genau dann, wenn es in $(B',A')$ definierbar abgeschlossen ist. Also sei $(B,A)$ jetzt schon \OE\ hinreichend saturiert und $b\in B\setminus D$.\\
	Nach dem Beweis zur Existenz des B\&F-Systems kann der partielle Isomorphismus $(B,A)\supseteq(D,C)\cong(D,C)\subseteq(B,A)$ insbesondere auf mehrere Weisen auf $b$ fortgesetzt werden: Wenn $b\in A$ oder $b\in B\setminus AD$, ging es nur um die Erfüllung von transzendenten Ordnungstypen, da hat man also viele Optionen. Wenn $b\in AD\setminus(A\cup D)$ ist und $a\in A^n$ unabhängig über $D$ mit $b\in Da$, dann finde $a'_1$ transzendent über $Da$ von passendem Ordnungstyp über $D$, $a'_2$ transzendent über $Daa'_1$ von passendem Ordnungstyp über $Da'_1$, usw. So kann man den Isomorphismus auf $Da$ fortsetzen mit Bild $Da'$; da aber $a$ und $a'$ per Konstruktion unabhängig über $D$ waren, sind auch $Da$ und $Da'$ unabhängig über $D$, also $Da\cap Da'=D$ und das Bild von $b$ kann nicht $b$ selbst sein. Also gibt es auch in diesem Fall mehrere Möglichkeiten für eine Fortsetzung, also mehrere elementare Abbildungen, also kann $b$ nicht definierbar über $D$ sein.
\end{proof}

\begin{lemma}
	Sei $F:A^n\rightarrow A$ eine $\lingua^P_S$-definierbare Funktion. Dann gibt es $\dcl(S)\cap A$-definierbare (TODO: das gilt doch bestimmt auch ohne $\dcl$??) $f_1,\dots,f_k:A^n\rightarrow A$ in $A$, sodass für alle $a\in A^n$ ein $f_i$ existiert mit $F(a)=f_i(a)$.
\end{lemma}
\begin{proof}
	Man kann annehmen, dass $S$ schon Modell von T ist, ansonsten geht man zu $\dcl(S)$ über.\\
	Wenn die Aussage nicht gilt, gilt für alle $k\in\setN$ und alle in $A$ über $S\cap A$ definierbaren $f_1,\dots,f_k:A^n\rightarrow A$, dass ein $a\in A$ existiert mit $f_i(a)\neq F(a)$ für alle $i$. Also ist der partielle Typ $$\{P(x)\}\cup\{F(x)\neq f(x)\mid f:A^n\rightarrow A\ \lingua_{S\cap A}\text{-definierbar}\}$$ konsistent und es existiert $(B,A)\preceq(B',A')$ und $a'\in A'^n$ mit $F(a')\neq f(a')$ für alle $\lingua_{S\cap A}$-definierbaren $f:A'^n\rightarrow A'$.\\
	Allerdings ist die Inklusion $(S,S\cap A)\subseteq (B,A)$ frei, wegen Elementarität auch $(S,S\cap A)\subseteq (B',A')$. Nach Lemma \ref{Unabhängigkeitsregeln} (6.) wegen $a'\in A'^n$ und die Inklusion $(Sa',(S\cap A)a')\subseteq(B',A')$ frei, nach Lemma \ref{Freie Definierbarkeit} ist $Sa'$ also $\lingua^P$-definierbar abgeschlossen. Da $F(a')$ $\lingua^P$-definierbar über $Sa'$ ist, liegt es in $Sa'$, wegen $F:A'^n\rightarrow A'$ ist $F(a')\in A'$. Wegen Unabhängigkeit liegt also $F(a')\in Sa'\cap A'=(S\cap A)a'$ und es gibt eine $\lingua_{S\cap A}$-definierbare Abbildung $f:A'^n\rightarrow A'$ mit $f(a')=F(a')$ - ein Widerspruch!
\end{proof}

\begin{lemma}
	Sei $F:B\rightarrow B$ eine $\lingua^P_S$-definierbare Funktion. Dann gibt es $\lingua_S$-definierbare $f_1,\dots,f_k:B\rightarrow B$ und eine kleine $\lingua^P_S$-definierbare Menge $X\subseteq B$, sodass für alle $b\in B\setminus X$ ein $f_i$ existiert mit $F(b)=f_i(b)$.
\end{lemma}
\begin{proof}
	\OE\ ist wieder $S$ ein Modell.\\
	Wenn die Aussage nicht gilt, gilt für alle $k\in\setN$, alle kleinen $\lingua^P_S$-definierbaren Mengen $X\subset B$ und alle $\lingua_S$-definierbaren $f_1,\dots,f_k:B\rightarrow B$, dass ein $b\in B\setminus X$ existiert mit $f_i(b)\neq F(b)$ für alle $i$. Also ist der partielle Typ $$\{x\notin X\mid X\text{ klein, }\lingua^P_S\text{-definierbar}\}\cup\{F(x)\neq f(x)\mid f:B\rightarrow B\ \lingua_S\text{-definierbar}\}$$ (endlich) konsistent und es existiert $(B,A)\preceq(B',A')$ und $$b'\in B'\setminus\bigcup\limits_{f:B'^n\rightarrow B'\ \lingua_S\text{-definierbar}}f(A'^n)=B'\setminus A'S$$ mit $F(b')\neq f(b')$ für alle $\lingua_S$-definierbaren $f:B'\rightarrow B'$.\\
	Allerdings ist wie im letzten Beweis $(S,S\cap A)\subseteq(B',A')$ frei und nach Lemma \ref{Unabhängigkeitsregeln} (7.) wegen $b'\in B'\setminus A'S$ die Inklusion $(Sb',S\cap A)\subseteq(B',A')$ ebenso, nach Lemma \ref{Freie Definierbarkeit} ist $Sb'$ also $\lingua^P$-definierbar abgeschlossen. Da $F(b')$ $\lingua^P$-definierbar über $Sb'$ ist, ist es in $Sb'$, also existiert eine $\lingua_S$-definierbare Abbildung $f:B'\rightarrow B'$ mit $f(b')=F(b')$ - ein Widerspruch!
\end{proof}

\begin{theorem}\label{Satz 3}
	Sei $F:B\rightarrow B$ eine $\lingua^P_S$-definierbare Funktion. Dann stimmt $F$ auf bis auf eine $\lingua^P_S$-definierbare kleine Menge mit einer $\lingua_S$-definierbaren Funktion überein.
\end{theorem}
\begin{proof}
	Nach dem vorigen Lemma existieren $\lingua_S$-definierbare $f_1,\dots,f_k:B\rightarrow B$ und ein kleines $\lingua^P_S$-definierbares $X$, sodass für alle $b\in B\setminus X$ ein $i$ existiert mit $F(b)=f_i(b)$. Wenn $k=1$ ist, ist das die gewünschte Aussage, wenn nicht, zeige dass man $k$ weiter reduzieren kann. Dafür partitioniere $B$ mit Satz \ref{Satz 4} in $\lingua_S$-definierbare $E,T,D,K,C\subseteq B$, sodass $E$ endlich ist, alle anderen Mengen dafür offen, 
	$$T\subseteq\{F=f_1\}, D\cap\{F=f_1\}=\emptyset,K':=K\cap\{F=f_1\}\text{ klein, dicht und kodicht in }K$$ $$\text{sowie }C':=C\cap\{F=f_1\}\text{ koklein, dicht und kodicht in }C.$$
	Definiere dann $$f:x\mapsto\left\{\begin{array}{ll}
	f_1(x)&x\in T\cup C\\
	f_2(x)&\text{sonst}
	\end{array}\right.$$
	Die Menge $X':=X\cup E\cup K'\cup (C\setminus C')$ ist klein als Vereinigung von kleinen Mengen, außerdem $\lingua^P_S$-definierbar, und wenn $x\in B\setminus X'$ mit $F(x)\neq f_i(x)$ für $i=3,\dots,k$, dann gibt es folgende Möglichkeiten:
	\begin{itemize}
		\item $x\in T\cup C'\subseteq\{F=f_1\}$: Dann ist $F(x)=f_1(x)=f(x)$.
		\item $x\in D\cup(K\setminus K')$, also insbesondere $x\notin\{F=f_1\}$: Dann ist $f(x)=f_2(x)$ und wegen $F(x)\neq f_1(x)$ ist $F(x)=f(x)$.
	\end{itemize}
    Also nimmt $F$ auf $B\setminus X'$ immer die Werte von $f,f_3,\dots,f_k$ an und induktiv ist die Aussage gezeigt.
\end{proof}

\begin{lemma}\label{Stückweise stetige Abbildungen}
	Sei $f:B\rightarrow B$ stückweise stetig und $\lingua^P_S$-definierbar. Dann ist $f$ schon $\lingua_S$-definierbar.
\end{lemma}
\begin{proof}
	Nach Satz $\ref{Satz 3}$ stimmt $f$ bis auf eine kleine $\lingua^P_S$-definierbare Menge $X$ mit einer $\lingua_S$-definierbaren Funktion $f'$ überein. Wegen o-Minimalität von $T$ ist $f'$ auch stückweise stetig. Unterteile $X$ wie in Satz \ref{Satz 4} und verfeinere die Unterteilung, so dass $f,f'$ auf jedem Intervall stetig sind. Für jedes Intervall $I$ dieser Unterteilung gilt dann entweder, dass $X\cap I$ dicht und kodicht in $I$ ist oder, dass $X\cap I=\emptyset$. Der erste Fall kann nie eintreten, da zwei stetige Funktionen, die auf der dichten Teilmenge $I\setminus X$ übereinstimmen, schon auf ganz $I$ übereinstimmen. Also ist $X$ endlich und $f$ kann mit $\lingua$ definiert werden (nämlich durch $f'$ außerhalb von $X$ und ansonsten manuell). Da diese Unterteilung $\lingua_S$-definierbar ist, ist auch $f$ schon $\lingua_S$-definierbar.
\end{proof}

\section{Offene und abgeschlossene $\lingua^P$-definierbare Mengen}
Nach der Beschreibung der Struktur von $\lingua^P$-definierbaren Teilmengen von $B$ liegt die Vermutung nahe, dass offene und abgeschlossene $\lingua^P$-definierbare Mengen schon $\lingua$-definierbar sind.\\
Zum Beweis dessen benötigt man einen Satz aus \cite{Piz}.
\begin{theorem}\label{Kurven}
	Sei $F:B^n\rightarrow B$ eine $\lingua^P_S$-definierbare Funktion. Dann ist $F$ definierbar in $\lingua_S$ genau dann, wenn für alle speziellen $X\supseteq S$ und alle $\lingua_X$-definierbaren partiellen Funktionen $\alpha: Y\rightarrow B^n$ mit $Y$ offen in $B$ auch $F\circ\alpha$ schon $\lingua_X$-definierbar ist.
\end{theorem}

Als erstes ist festzuhalten, dass die Aussage für Mengen in einer Variable leicht zu sehen ist und man sie nur für offene Mengen zeigen muss.
\begin{lemma}
	\begin{itemize}
		\item Sei $Z\subseteq B$ offen und $\lingua^P_S$-definierbar. Dann ist $Z$ schon $\lingua_S$-definierbar.
		\item Wenn alle offenen $\lingua^P_S$-definierbaren Teilmengen von $B^n$ auch $\lingua_S$-definierbar sind, sind es alle abgeschlossenen solchen Mengen auch.
	\end{itemize}
\end{lemma}
\begin{proof}
	\begin{itemize}
		\item In der Darstellung von Satz \ref{Satz 4} kann der Fall $Z\cap I$ dicht und kodicht nicht auftreten, weil offene Mengen niemals kodicht sind. Also sind die $\lingua^P_S$-definierbaren offenen Teilmengen von $B$ gerade die endlichen Vereinigungen von Intervallen mit Rand aus $\dcl(S)\cup\{\pm\infty\}$ und Punkten aus $\dcl(S)$ und das ist $\lingua_S$-definierbar.
		\item Wenn $Z\subseteq B^n$ abgeschlossen und $\lingua^P_S$-definierbar ist, ist $Z^c$ offen und $\lingua^P_S$-definierbar, also $\lingua_S$-definierbar per Voraussetzung. Damit ist $Z$ dann selbst $\lingua_S$-definierbar.
	\end{itemize}
\end{proof}

\begin{theorem}
	Offene und abgeschlossene $\lingua^P_S$-definierbare Mengen sind $\lingua_S$-definierbar.
\end{theorem}
\begin{proof}
	Nach dem vorigen Lemma reicht es, das für offene Mengen zu zeigen, genauer reicht es, die Definierbarkeit für charakteristische Funktionen solcher Mengen zu zeigen. Sei $Z\subseteq B^n$ eine offene $\lingua^P_S$-definierbare Menge und $X\supseteq S$ speziell, $\alpha:Y\rightarrow B^n$ eine $\lingua_X$-definierbare, partielle Funktion und $Y\subseteq B$ offen. Zu zeigen ist für die Anwendung von Satz \ref{Kurven}, dass $\chi_Z\circ\alpha$ definierbar in $\lingua_X$ ist. Man kann annehmen, dass $\alpha$ stetig ist, sonst zerlege $Y$ in Intervalle, auf denen $\alpha$ stetig ist und Punkte; das ändert nichts an irgendwelchen Definierbarkeiten. Man kann dann feststellen, dass $$1=\chi_Z\circ\alpha(x)\Leftrightarrow\alpha(x)\in Z\Leftrightarrow x\in\alpha^{-1}(Z).$$ Da aber $\alpha^{-1}(Z)$ eine $\lingua^P_X$-definierbare Teilmenge von $B$ ist, offen wegen Stetigkeit von $\alpha$ und Offenheit von $Z$ und zusätzlich $X$ speziell ist, ist $\alpha^{-1}(Z)$ nach dem vorigen Lemma $\lingua_X$-definierbar. Also ist $\chi_Z\circ\alpha$ definierbar in $\lingua_X$ durch $$\chi_Z\circ\alpha(x)=\left\{\begin{array}{ll}
	1&x\in\alpha^{-1}(Z)\\
	0&\text{sonst}
	\end{array}\right..$$
\end{proof}

\begin{corollary}
	Der Abschluss und das Innere von $\lingua^P_S$-definierbaren Mengen sind $\lingua_S$-definierbar.
\end{corollary}