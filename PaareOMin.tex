%!TEX root = DieLoesungAllerMilleniumsprobleme.tex
\chapter{Dichte Paare o-minimaler Strukturen}
\section{Allgemeine Betrachtungen und Anforderungen an die Theorien}

Viele Techniken und Gedanken aus dem vorigen Kapitel werden jetzt auf o-minimale Theorien übertragen: Im Folgenden halten wir einfach eine vollständige o-minimale Theorie T in der Sprache $\lingua$ fest und betrachten die Theorie $\tq$ in der Sprache $\lingua_P:=\lingua\cup\{P(x)\}$, sodass die Modelle von $\tq$ Modelle von T sind und in jedem Modell $\fM$ die Menge $P(M)$ ebenfalls Modell von T ist. Schreibe so ein Paar dann auch als $(B,A)$ mit $A=P(B)$.\\
Wir setzen voraus, dass $T$ eine durch $+$ dicht und linear angeordnete abelsche Gruppe mit einem positiven Element $1$ beschreibt, sodass Definable Choice gilt. Dann sind die Skolemfunktionen definierbar und \OE\ ist $\lingua$ schon so eine definitorische Erweiterung, dass $T$ Quantorenelimination hat und universell axiomatisierbar ist (wobei bei einzelnen Theorien die Frage interessant wäre, welche Skolemfunktionen man dafür überhaupt hinzufügen muss). Außerdem seien alle Modelle genug saturiert, dass die üblichen Rechenregeln für die Dimension gelten.\\
Aus T universell mit Quantorenelimination folgt, dass Unterstrukturen von Modellen von T schon elementare Unterstrukturen sind. Also ist für jede Teilmenge $S$ eines Modells $\dcl(S)$ schon eine elementare Substruktur; zur Vereinfachung bezeichne in Zukunft $AB:=\dcl(A\cup B)$ für zwei Teilmengen $A,B$ eines Modells.\\
$P$ beschreibt also eine elementare Unterstruktur, mit $\td$ wird nun die Theorie beschrieben, die ausdrückt, dass $P$ eine dichte echte Unterstruktur ist (diese zwei Sachen oder deren Gegenteil müssen auf jeden Fall von der Theorie beschrieben werden, wenn sie vollständig sein soll). Klar ist dann, dass Unterstrukturen von $\td$ automatisch Modelle von $\tq$ sind.\\
Dem Verlauf in \cite{VanDenDries} folgend, wird zunächst die Vollständigkeit und eine Art von Quantorenelimination für $\td$ gezeigt, wofür aber eine genauere Betrachtung von sogenannten kleinen Mengen vonnöten ist.

\section{Kleine Mengen}
\begin{definition}
	Sei $(B,A)\models\td$, dann ist eine definierbare Menge $S\subseteq B$ \textbf{klein}, wenn eine $B$-definierbare Funktion $f:B^n\rightarrow B$ existiert mit $S\subseteq f(A^n)$.
\end{definition}

Das Ziel ist jetzt, zu zeigen, dass definierbare Intervalle nicht klein sind.\\
Im folgenden Lemma meint $(\cdot,+)$ nicht unbedingt die Operationen in $\lingua$, sondern nur irgendwelche definierbaren Verknüpfungen.

\begin{lemma}\label{Hilfsaussage Kleinheit}
	Seien $A\prec B\models\operatorname{T},\ f:B^{n+1}\rightarrow B\ A\text{-definierbar},\ b\in B\setminus A,\ {\beta,\gamma\in A}$ mit $\beta<b<\gamma$ und einer angeordneten $A$-definierbaren Körperstruktur $(\cdot,+)$ auf $(\beta,\gamma)=:I$. Dann existieren $a_0,\dots,a_n\in I_A$ mit $$a_nb^n+a_{n-1}b^{n-1}+\dots,a_0\in I\setminus f(A^n\times\{b\}).$$
\end{lemma}
\begin{proof}
	Wenn die Aussage nicht gilt, dann gilt mit $p(x,y):=x_ny^n+x_{n-1}y^{n-1}+\dots,x_0$, dass für jedes $a\in (I_A)^{n+1}$ ein $\alpha\in A^n$ existiert mit $p(a,b)=f(\alpha,b)$. Es muss für festes $a\in I_A$ ein Intervall um $b$ in $I_A$ mit dieser Eigenschaft geben, denn sonst wäre $b\in\dcl(A)=A$.\\
	Sei jetzt $a$ nicht mehr fixiert, dann existiert mit Definable Choice eine definierbare Zuordnung $a\mapsto\alpha(a)$, sodass $p(a,\cdot)=f(\alpha(a),\cdot)$ auf einem Intervall gilt. Da jedes $a$ $n+1$ viele Einträge hat und jedes $\alpha(a)$ $n$ viele, müssen unendlich viele $a\in (I_A)^{n+1}$ existieren, die durch $\alpha$ auf das selbe Element abgebildet werden. Denn wenn das nicht so wäre, wäre ein generisches Element aus $(I_A)^{n+1}$ algebraisch über einem Element aus $A^n$, was der Generizität widerspricht. Da es unendlich viele Elemente gibt, sodass $\alpha$ auf ihnen konstant ist, gibt es schon eine Zelle von Dimension $>0$ mit der Eigenschaft und damit insbesondere eine Zelle $E$ von Dimension 1 (Als Teilmenge einer Zelle lässt sich immer eine von kleinerer Dimension finden). Nenne den konstanten Wert dann $\alpha^*$.\\
	Da also gilt: Für alle $a\in E$ existiert ein Intervall $J$ mit $p(a,\cdot)=f(\alpha^*,\cdot)$ auf $J$; existieren mit Definable Choice $\beta^*,\gamma^*:E\rightarrow I_A$, sodass $p(a,\cdot)=f(\alpha^*,\cdot)$ auf $(\beta^*(a),\gamma^*(a))$ gilt. \OE\ seien $\beta^*$ und $\gamma^*$ jetzt schon stetig auf $E$ und ein $e\in E$ beliebig· Dann existiert für $\varepsilon$ hinreichend klein eine $E$-Umgebung $U$ um $e$, sodass $$\beta^*<\frac{1}{2}(\beta^*(e)+\gamma^*(e))-\varepsilon,\ \frac{1}{2}(\beta^*(e)+\gamma^*(e))+\varepsilon<\gamma^*$$ auf $U$, also $$p(a,x)=f(\alpha^*,x)\text{ für alle }a\in U,x\in(\frac{1}{2}(\beta^*(e)+\gamma^*(e))-\varepsilon,\frac{1}{2}(\beta^*(e)+\gamma^*(e))+\varepsilon)$$ gilt. Es kann aber nicht $p(a-a',x)=p(a,x)-p(a',x)=f(\alpha^*,x)-f(\alpha^*,x)=0$ für $a,a'\in U$ verschieden und unendlich viele $x$ sein, weil ein Nichtnullpolynom nicht unendlich viele Nullstellen haben kann.
\end{proof}

\begin{lemma}
	Sei $\fA$ eine o-minimale Erweiterung eines angeordneten Vektorraums über einem angeordneten Körper $F$ und $g:A^{p+1}\rightarrow A$ definierbar, außerdem existiere für unendlich viele $\lambda\in F$ ein $a_\lambda\in A^p$ mit $g(a_\lambda,x)=\lambda x$ für unendlich viele $x\in A$. Dann existiert ein Intervall $I$ in $A$, sodass auf $I$ eine $A$-definierbare Körperstruktur existiert, die mit $<$ kompatibel ist (was automatisch einen reell abgeschlossenen Körper impliziert).
\end{lemma}
\begin{proof}
	TODO: Geht irgendwie aus \cite{PeterStarch} hervor.
\end{proof}

\begin{lemma}
	Es sei $(A,B)\models\td,\ f:B^{n+1}\rightarrow B$ $A$-definierbar in $B$ und $b\in B\setminus A$. Dann enthält $f(A^n\times\{b\})$ kein Intervall um $b$.
\end{lemma}
\begin{proof}
	Nimm an, dass das Gegenteil gelte für das Intervall $J$ (\OE\ mit Randpunkten in $A$): Dann existiert insbesondere für jedes $q\in\setQ$ hinreichend nahe bei $1$ ein $a_q\in A^n$ mit $f(a_q,b)=qb$. Dann existiert wieder ein Intervall $I_q\subseteq J_A$ mit $f(a_q,x)=qx$ für alle $x\in I_q$. \OE\ ist dieses Intervall schon beschränkt und die Randpunkte seien $c_q<d_q$. Definiere dann $$r_q:=\frac{c_q+d_q}{2},s_q:=\frac{d_q-c_q}{2}\in A,$$ $$g:(u,v,x)\mapsto f(u,v+x)-f(u,v)\ \ \ \ u\in A^n,v,x\in A.$$
	Dann gilt für alle $x\in(-s_q,s_q)$ $$g(a_q,r_q,x)=f(a_q,r_q+x)-f(a_k,r_q)=q(r_q+x)-qr_q=qx.$$
	Also existiert nach dem letzten Lemma ein Intervall in $A$ mit einer $A$-definierbaren Körperstruktur als RCF. Durch Translation (benutze Dichtheit) nehme an, dass $b\in I_B$ liegt. Dann existiert nach Lemma \ref{Hilfsaussage Kleinheit} ein Element $c\in I_B\setminus f(A^n\times\{b\})$. \OE\ sei schon $\inf J,\sup J\in I$, sonst ersetze $J$ durch ein kleineres Intervall.\\
	Seien $d,e\in I$ mit $d<c<e$ und $\varphi$ die orientierungserhaltende, $A$-definierbare affine Abbildung in $I$ mit $\varphi(d)=\inf J,\varphi(e)=\sup J$. Dann ist $\varphi(c)\in J\setminus(\phi\circ f)(A^n\times\{b\})$ und da das Verketten mit einer $A$-definierbaren invertierbaren Abbildung nichts an der Aussage ändert, gibt es einen Widerspruch.
\end{proof}

\newpage

\begin{theorem}
	Wenn $(B,A)\models\td$, dann ist kein Intervall eine kleine Teilmenge.
\end{theorem}
\begin{proof}
	Sei $f:B^n\rightarrow B$ eine durch $\varphi(x,y,b)$ definierbare Abbildung mit $\varphi$ eine\linebreak$\lingua_A$-Formel und $b\in B^m$ für ein $m\in\setN$ definiert. Für $\dim(b/A)=0$ ist $f(A^n)\subseteq A$ klar, deswegen sei \OE\ $\dim(b/A)\geq1$. Definiere
	\begin{align*}
	g(x,z):=\left\{\begin{array}{ll}
	\text{das eindeutige }y\in B &\text{für alle z, für die }\varphi(x,y,z)\\
	\text{ mit }B\models\varphi(x,y,z) &\text{ bei festem }\text{ eine Funktion definiert}\\
	\ &\ \\
	0 &\text{sonst}
	\end{array}\right.,
	\end{align*}
	Dann ist $g$ in $B$ $A$-definierbar und $g(\cdot,b)=f$. Falls $\dim(b/A)>1$, füge genug Komponenten von $b$ zu $A$ hinzu, sodass $\dim(b/A)=1$. Das Hinzufügen ändert nichts, denn $Ab_i$ ist nach den Eingangsbemerkungen Modell von T und $Ab_i$ ist erst recht dicht in $B$.\\
	Finde also $b_i$, sodass $A$-definierbare $(h_j)$ existieren mit $b_j=h_j(b_i)$ für alle $j$. Wenn jetzt $J\subseteq f(A^n)=g(A^n,b)=g(A^n,h(b_i))$ für ein Intervall $J$, dann widerspricht das der Aussage des letzten Lemmas für die Funktion $(x,y)\mapsto g(x,h(y))$.
\end{proof}

\begin{definition}
	Schreibe ab jetzt $P(\overline{x}):=\bigwedge\limits_{i=1}^\abs{x}P(x_i)$.
\end{definition}

\begin{lemma}
	Wenn $(B,A)$ für ein unendliches $\kappa>\abs{\operatorname{T}}$ ein $\kappa$-saturiertes Modell von $\td$ ist, ist $\dim(B/A)\geq\kappa$.
\end{lemma}
\begin{proof}
	Sei $S$ eine Basis von $B/A$ mit $\abs{S}<\kappa$; zeige nun, dass es kein Erzeugendensystem sein kann. Das folgt aus der Saturation angewandt auf den partiellen Typen $$\{\forall\overline{y}\in P(x\neq t(\overline{y}))\mid t\ \lingua_E\text{-Term}\},$$ der endlich erfüllbar ist, weil die Negation jeder dieser Formeln \glqq{}$x$ ist in einer kleinen Menge\grqq{} impliziert. Wenn der Typ also nicht endlich erfüllbar wäre, würde eine endliche Vereinigung von kleinen Mengen ganz $B$ überdecken. Das kann aber nicht gelten, denn eine endliche Vereinigung von kleinen Mengen ist wieder klein (in der das bezeugenden Abbildung kann man das durch Erhöhen der Dimension des Urbildes und Fallunterscheidung über eine Koordinate beweisen).
\end{proof}
\begin{corollary}\label{Finden transz Elte}
	Der Beweis zeigt sogar, dass in einem $\kappa$-saturiertem Modell $(B,A)\models\td$, gegeben Menge, $S,S',S''\subset B$ mit $\abs{S},\abs{S'},\abs{S''}<\kappa$, ein transzendentes Element $b$ über $SA$ gefunden werden kann mit $a<b$ für alle $a\in S'$ und $b<c$ für alle $c\in S''$, sofern dieser Ordnungstyp von $b$ überhaupt konsistent ist.
\end{corollary}

\section{Formelreduzierung in $\td$}
In diesem Abschnitt wird gezeigt, dass sich $\lingua_P$-Formeln modulo $\td$ sehr stark vereinfachen lassen. Indem dieses mit einem Back\&Forth-System gezeigt wird, erhält man zusätzlich eine sehr große Klasse von elementaren Abbildungen zwischen Modellen von $\td$.\\
In diesem Kontext wird wieder die $\acl$-Unabhängigkeit in einem Modell von T relevant, die im ersten Kapitel mit \glqq{}algebraisch disjunkt\grqq{} bezeichnet wurde. Man kann sich dafür folgende (teilweise schon bekannte) Fakten überlegen.

\begin{lemma}\label{Unabhängigkeitsregeln}
	Seien $A,B,C,D$ Mengen in irgendeinem Modell von $T$.
	\begin{enumerate}
		\item Wenn $A$ und $B$ unabhängig über $C$ sind, sind $B$ und $A$ unabhängig über $C$ und $A\cap B\subseteq C$ (in fast allen Fällen wird sowieso $A,B\supseteq C$ gelten).
		\item Wenn $A$ und $B$ unabhängig über $C$ sind und $S\subseteq B$, dann sind auch $A\cup S$ und $B$ unabhängig über $C\cup S$.
		\item Wenn $A$ und $B$ unabhängig über $C$ sind, $A\subseteq S\subseteq\acl(A),B\subseteq S'\subseteq\acl(B)$, dann sind $S$ und $S'$ unabhängig über $C$.
		\item Wenn $A$ und $B$ unabhängig über $C$ sind und $D$ unabhängig über $AB$, dann sind $A\cup D$ und $B$ unabhängig über $C$.
		\item Wenn $(D,C)\preceq(B,A)\models\tq$, dann sind $A$ und $D$ unabhängig über $C$.
		\item Wenn $(D,C)\subseteq(B,A)\models\tq,\ S\subseteq A$ und $A$ und $D$ unabhängig über $C$ sind, dann sind $A$ und $DS$ unabhängig über $CS$, $\langle D\cup S\rangle_{\lingua_P}=(DS,CS)$ und $(D,C)\subseteq(DS,CS)\subseteq(B,A)$.
		\item Wenn $(D,C)\subseteq(B,A)\models\tq$ und $S\subseteq B$ unabhängig über $DA$ ist, dann sind $A$ und $DS$ unabhängig über $C$, $\langle D\cup S\rangle_{\lingua_P}=(DS,C)$ und $(D,C)\subseteq(DS,C)\subseteq(B,A)$.
	\end{enumerate}
\end{lemma}
\begin{proof}
	1.-4. sind bekannt.
	\item[5.] Wenn $\overline{d}\in D$ algebraisch unabhängig über $C$ ist, aber nicht über $A$, dann existiert eine $\lingua_A$-Formel $\varphi(\overline{x},\overline{a})$, sodass \OE\ $d_1$ von $\varphi(x_1,d_2,d_3\dots,\overline{a})$ algebraisiert wird (\OE\ wird $d_1$ schon durch $\varphi$ definiert). Also erfüllt $\overline{d}$ die $\lingua_P$-Formel $$\exists \overline{y}\in P(\varphi(\overline{x},\overline{y})\land\forall z_2,z_3,\dots\exists! z_1(\varphi(\overline{z},\overline{y})))$$ in $(B,A)$, also auch in $(D,C)$. Es existiert also $\overline{c}\in C$ mit $$B\models\varphi(\overline{d},\overline{c})\land\forall z_2,z_3,\dots\exists! z_1(\varphi(\overline{z},\overline{c})),$$ was im Widerspruch zur Unabhängigkeit von $\overline{d}$ über $C$ steht.
	\item[6.] Dass $A$ und $DS$ unabhängig über $CS$ sind, ergibt sich in der Kombination von 2. und dann 3.\\
	Dass die Trägermenge von $\langle D\cup S\rangle_{\lingua_P}$ die Menge $DS$ ist, ergibt sich direkt per Definition als $DS=\dcl(D\cup S)=\langle D\cup S\rangle_\lingua$. Weil $A$ und $DS$ unabhängig über $CS$ sind, folgt $$P(\langle D\cup S\rangle_{\lingua_P})=DS\cap P(B)=DS\cap A=CS.$$
	\item[7] Es ergibt sich aus 4. und 3. dass $A$ und $DS$ unabhängig über $CS$ sind. Der Rest geht analog zu 6.
\end{proof}

Zu bemerken ist, dass ein Spezialfall von Unabhängigkeit viele nützliche Eigenschaften hat. Auf diesen wird später noch oft zurückgegriffen werden.
\begin{definition}
	Seien $(D,C)\subseteq(B,A)$ zwei Modelle von $\tq$. Dann heiße diese Erweiterung \textbf{frei}, wenn $D$ und $A$ unabhängig über $C$ sind.
\end{definition}

\begin{lemma}
	Sei $(B,A)\models\td$. Dann ist $A$ auch kodicht in $B$.
\end{lemma}
\begin{proof}
	Zu zeigen ist, dass für alle $a,c\in B$ ein $b\in B\setminus A$ existiert mit $a<b<c$. Durch Translation und Inversion kann man annehmen, dass $a=0$. Wähle jetzt ein $d\in B\setminus A$ beliebig und $e\in A$ mit $d-c<e<d$. Dann ist $d-e$ nicht in $A$ (denn sonst wäre es $d$) und $0=e-e<d-e<d-(d-c)=c$.
\end{proof}

Für die Konstruktion des gewünschten Back\&Forth-Systems sei $\kappa>\abs{T}$ eine beliebige, aber feste Kardinalzahl und $(B,A),(D,C)\models\td$ zwei $\kappa$-saturierte Modelle.
\begin{theorem}
	Sei $S$ die Menge aller partiellen Isomorphismen zwischen Unterstrukturen $(B',A')$ von $(B,A)$ und $(D',C')$ von $(D,C)$ der Mächtigkeit $<\kappa$, sodass die Erweiterungen frei sind. Dann bildet $S$ ein nichtleeres B\&F-System und $\td$ ist insbesondere vollständig.
\end{theorem}
\begin{proof}
	Das System ist nichtleer, denn es gibt ein Primmodell $\fM$ von $T$, weil $T$ vollständig ist und in jedem Modell $A$ alle Eigenschaften von $\fM_A:=\langle\emptyset\rangle_\lingua$ in $T$ beschrieben werden. Klarerweise ist $\abs{M}=\abs{T}<\kappa$. Der Isomorphismus $(\fM_A,\fM_A)\cong(\fM_C,\fM_C)$ liegt in $S$, denn Unabhängigkeit ist bei zwei gleichen Mengen offensichtlich.\\
	Sei jetzt $S\ni i:(B',A')\rightarrow(D',C')$ und $b\in B$. Wenn $b\in B'$ ist, ist nichts zu zeigen. Wenn $b\in A\setminus B'$, betrachte den partiellen Typ $$\{\alpha<x\mid i^{-1}(\alpha)<b\}\cup\{x<\beta\mid b<i^{-1}(\beta)\}\cup\{P(x)\}.$$
	Dieser ist konsistent, da $i$ ein Isomorphismus ist und $C$ dicht in $D$; mit Saturation existiert ein $d\in C\setminus D'$ mit diesem Ordnungstyp. $i$ setzt sich dann eindeutig zu einem Isomorphismus $i':(B'b,A'b)\rightarrow(D'd,C'd)$ mit $i(b)=d$ fort, der gegeben ist durch die Abbildung $t(b)\mapsto i(t)(d)$ für $t$ einen $\lingua_{B'}$-Term und $i(t)$ den durch $i$ geshifteten Term. Die Surjektivität dieser Abbildung ist klar, ebenso dass $i'(A'b)=C'd$. Wohldefiniertheit, Injektivität und Isomorphismuseigenschaft gelten, denn:\\
	$Rt_1(b)\dots t_n(b)$ gilt für $\lingua_{B'}$-Terme $t_1,\dots,t_n$ und eine Relation $R$ genau dann, wenn es ein $B'$-definierbares Intervall $I$ um $b$ mit dieser Eigenschaft gibt (denn sonst wäre $b$ definierbar über $B'$ und somit in $B'$). Schickt man $I\cap B'$ mit $i$ nach $J:=i(I\cap B')$, so gilt für alle Elemente $z\in J$, dass $Ri(t_1)(z)\dots i(t_n)(z)$, da $i$ ein Isomorphismus ist. Wäre jetzt nicht $Ri(t_1)(d)\dots i(t_n)(d)$, so gäbe es ein $D'$-definierbares Intervall $I'$ um $d$, sodass das nicht gilt; insbesondere ist $I'$ disjunkt zu $J$. Allerdings ist $$b\in I'\cap\operatorname{convex}(J)=I'\cap(i(\inf I),i(\sup I)),$$ also können $I'$ und $J$ nicht disjunkt sein. Es gilt also $Ri(t_1)(d)\dots i(t_n)(d)$.\\
	Die Rückrichtung geht analog.\\
	Zu zeigen ist nun, dass $B'b$ und $A$ frei über $A'b$ sowie $D'd$ und $C$ frei über $C'd$ sind, ebenso zu zeigen ist noch, dass $(B'b,A'b)\subseteq(B,A),(D'd,C'd)\subseteq(D,C)$. Das alles folgt aber aus Lemma \ref{Unabhängigkeitsregeln} (6.). Außerdem gilt $\abs{D'd}=\abs{B'b}=\abs{B'}+\abs{T}<\kappa$.\\
	Sei jetzt $b\in B'A\setminus(A\cup B')$. Dann gibt es $\overline{a}\in A$ mit $b\in B'\overline{a}$. Erweitere wie schon bekannt $i$, sodass $\overline{a}\in\operatorname{dom}(i)$; dann ist schon ganz $B'\overline{a}\subseteq\operatorname{dom}(i)$, also auch $b$.\\
	Abschließend sei $b\in B\setminus B'A$;  wie oben erfülle dann den mit $i$ geshifteten Ordnungstyp von $b$ über $B'$ mit einem Element $d\in D\setminus D'C$ (mit Folgerung \ref{Finden transz Elte} geht das). Wie oben kann $i$ dann auf einen Isomorphismus $(B'b,A')\rightarrow(D'd,C')$ fortgesetzt werden und nach Lemma \ref{Unabhängigkeitsregeln} (7.) erfüllen $(B'b,A'),(D'd,C')$ auch die hinreichenden Eigenschaften.
\end{proof}

Dieses B\&F-System beweist die Formelreduzierung in $\td$.
\begin{theorem}
	Jede $\lingua_P$-Formel ist modulo $\td$ äquivalent zu einer booleschen Kombination von Formeln der Gestalt
	$$\exists\overline{y}\in P(\phi(\overline{x},\overline{y}))$$
	für $\phi$ eine $\lingua$-Formel. Nenne eine solche boolesche Kombination eine \textbf{gute Formel} und eine Formel der Gestalt wie beschrieben eine \textbf{gute Formel in Reinform}.
\end{theorem}
\begin{proof}
	\underline{Hilfsaussage:}\\
	Es reicht zu zeigen, dass für alle Modelle $(B,A),(D,C)\models\td$ und für alle $b\in B^n,d\in D^n$ gilt: Wenn $b$ und $d$ dieselben guten Formeln erfüllen, sind ihre Typen in $(B,A)$ und $(D,C)$ dieselben.\\
	Dass dies ausreicht, erkennt man mit dem Ziegler'schen Trennungslemma: Sei $\psi\in\fF_n(\lingua_P)$ nicht äquivalent zu einer guten Formel und nenne die Menge aller guten Formeln in $n$ freien Koordinaten $K$. Dann ist $K$ abgeschlossen unter $\land,\lor$ und enthält $\top,\bot$. Wenn $\psi$ nicht äquivalent zu einer Formel aus $K$ ist, sind $\td\cup\{\psi\}$ und $\td\cup\{\neg\psi\}$ nicht durch $K$ trennbar, also existieren $(B,A),(D,C)\models\td,b\in B^n,d\in D^n$, sodass $(B,A)\models\psi(b)$ und $(D,C)\models\neg\psi(d)$, aber $(B,A)\models\chi(b)$ genau dann, wenn $(D,C)\models\chi(d)$ für alle $\chi\in K$. Dann erfüllen $b$ und $d$ dieselben guten Formeln, aber haben nicht denselben Typ - ein Widerspruch!\\
	\begin{proof}[Beweis der Hilfsaussage]
		Seien $b,d$ wie verlangt und $(B,A),(D,C)$ schon \OE\linebreak $\abs{T}^+$-saturiert (das ändert nichts an Typen und dem Erfüllen von guten Formeln). Sei $a\in A^m$ für ein hinreichend großes $m$, mit der Eigenschaft dass $\dim(b/a)\leq\dim(b/A)$ (es folgt dann Gleichheit, da über einer kleineren Menge nicht mehr interdefinierbar werden kann). Für $A':=\dcl(a),B':=\dcl(a,b)$ gilt dann, dass $A$ und $B'$ unabhängig über $A'$ sind. Es sind nämlich per Definition von $a$ die Mengen $A$ und $b$ unabhängig über $a$ (eben wegen $\dim(b/a)=\dim(b/A)$), mit Lemma \ref{Unabhängigkeitsregeln} (2.) sind dann auch $A$ und $b\cup(A')$ unabhängig über $A'$ und mit 3. sind $A$ und $B'=\dcl(b\cup A')$ unabhängig über $A'$. Außerdem sind $A'$ und $B'$ maximal $\abs{T}$ groß.\\
		Wenn man den partiellen $\lingua_P$-Typ $\tp_\lingua(a/b)\cup\{P(\overline{x})\}$ betrachtet, bleibt er konsistent unter der Ersetzung $b\rightarrowtail d$ in den Formeln. Seien nämlich $\psi_1(\overline{x},b),\dots,\psi_n(\overline{x},b)\in\tp_\lingua(a/b)$, dann ist $$\exists\overline{x}\in P(\bigwedge\limits_{i=1}^n\psi_i(\overline{x},\overline{y}))$$ eine gute Formel, die von $b$ und daher auch von $d$ erfüllt wird. Also ist der ersetzte partielle Typ endlich konsistent, wegen Saturation habe er den Erfüller $c\in C$ und es gilt $\tp_\lingua(a,b)=\tp_\lingua(c,d)$. Wegen der Typengleichheit folgt insbesondere $\dim(b/a)=\dim(d/c)$; es bleibt noch zu zeigen, dass $\dim(b/A)=\dim(d/C)$, damit dann gilt $\dim(d/C)=\dim(b/A)=\dim(b/a)=\dim(d/c)$ und wie oben $C$ und $D':=\dcl(c,d)$ frei über $C':=\dcl(c)$ sind. Die Gleichheit $\dim(b/A)=\dim(d/C)$ gilt aber, da für jede $\lingua$-Formel $\psi$ und $j_1,\dots,j_n\in\setN$ die Formel zu $$\glqq{}\text{es existiert }\overline{y}\in P\text{, sodass }\psi(\overline{x},\overline{y})\ x_i\text{ über }x_{j_1},\dots,x_{j_m}\text{ definiert}\grqq{}$$ eine gute Formel ist, die also genau dann von $b$ erfüllt wird, wenn sie von $d$ erfüllt wird.\\
		Da $(a,b)$ und $(c,d)$ den gleichen $\lingua$-Typ haben, gibt es einen partiellen Isomorphismus $i$ von $B'=\dcl(a,b)$ nach $D'=\dcl(c,d)$ mit $i((a,b))=(c,d)$, die Einschränkung auf $A'=\dcl(a)$ bildet einen Isomorphismus nach $C'=\dcl(c)$. Also ist $i$ partieller Isomorphismus $(B,A)\rightarrow(D,C)$, damit im B\&F-System, also elementare Abbildung, weswegen $b$ und $d$ denselben $\lingua_P$-Typen haben.
	\end{proof}
\end{proof}

\begin{corollary}
	Für ein dichtes Paar $(B,A)$ und $S\subseteq B^n$ eine $A_0$-definierbare Menge in $\lingua_P$ (wobei $A_0\subseteq A$) ist $S\cap A^n$ eine $A_0$-definierbare Menge in $A$.
\end{corollary}
\begin{proof}
	Nach der Formelreduzierung sei $S$ \OE\ durch eine gute Formel definiert. Da die Definierbarkeit abgeschlossen unter booleschen Kombinationen ist, reicht es, eine Formel in Reinform zu betrachten.\\
	Da aber für jede $\lingua_{A_0}$-Formel $\varphi(x,y,a')$ und jedes $a\in A^n$ die Aussagen $$\glqq{}\text{Es existiert ein }y\in A^m\text{ mit }(B,A)\models\varphi(a,y,a')\grqq{},$$ $$\glqq{}\text{Es existiert ein }y\in A^m\text{ mit }B\models\varphi(a,y,a')\grqq{},$$ $$\glqq{}\text{Es existiert ein }y\in A^m\text{ mit }A\models\varphi(a,y,a')\grqq{}$$ äquivalent sind wegen $\varphi$ als $\lingua$-Formel und $A\prec B$, folgt, dass $$\exists y\in P(\varphi(x,y,a'))(B)\cap A^n=\exists y(\varphi(x,y,a'))(A).$$
\end{proof}

\section{Folgen der Existenz des B\&F-Systems}
Im Folgenden werden einige Anordnungen von wechselseitigen Inklusionen von Modellen von T betrachtet, in der Gleichheit von bestimmten Typen folgt.

\begin{lemma}
	Für dichte Paare $(B,A),(D,C)$ mit $(D,C)\subseteq(B,A)$ sind folgende Eigenschaften äquivalent:
	\begin{enumerate}
		\item $(D,C)\preceq(B,A)$
		\item Die Erweiterung ist frei.
	\end{enumerate}
\end{lemma}
\begin{proof}
	$\glqq{}1.\Rightarrow2.\grqq{}:$ Diese Richtung ist schon aus Lemma \ref{Unabhängigkeitsregeln} (5.) bekannt.\\
	$\glqq{}2.\Rightarrow1.\grqq{}:$ Finde $(\abs{B}+\abs{T})^+$-saturierte Strukturen $$(B,A)\preceq(B',A'),(D,C)\preceq(D',C');$$ es ist dann $(D,C)$ eine gemeinsame Unterstruktur und $(D,C)\subseteq(D',C')$ ist frei nach dem Beweis der Gegenrichtung. Außerdem sind nach Voraussetzung $D$ und $A$ unabhängig über $C$, da aber Unabhängigkeit von Tupeln in $D$ über $A$ auch über $A'$ erhalten bleibt (da $(B',A')$ elementare Oberstruktur), ist auch $(D,C)\subseteq(B',A')$ frei. Also ist die Identität auf $(D,C)$ im Back\&Forth-System, daher elementare Abbildung. Daraus folgt für alle $(\lingua_P)_D$-Formeln $\varphi$, dass $$(D,C)\models\phi\Leftrightarrow(D',C')\models\varphi\Leftrightarrow(B',A')\models\varphi\Leftrightarrow(B,A)\models\varphi.$$
\end{proof}

\begin{lemma}
	Seien $(B_1,A_1),(B_2,A_2)\models\td$ und $(B,A)$ eine gemeinsame Unterstruktur, sodass die Inklusionen frei sind. Wenn $a\in (A_1)^n$ und $b\in (A_2)^n$ denselben $\lingua$-Typen über $B$ erfüllen, erfüllen sie auch denselben $\lingua_P$-Typen über $B$.
\end{lemma}
\begin{proof}
	\OE\ seien $(B_1,A_1)$ und $(B_2,A_2)$ schon genügend saturiert, das ändert nichts an Typen über $B$ und (nach derselben Argumentation wie im vorigen Lemma) auch nichts an der Unabhängigkeit. Da $a$ und $b$ denselben Typen über $B$ erfüllen, kann man wieder $\lingua_B$-Terme mit eingesetztem $a$ auf $\lingua_B$-Terme mit eingesetztem $b$ abbilden (Wohldefiniertheit und Injektivität wird durch die Typengleichheit ermöglicht) und bekommt einen partiellen Isomorphismus $i:Ba\cong Bb$, dessen Einschränkung auf die $\lingua_A$-Terme einen partiellen Isomorphismus $Aa\cong Ab$ induziert und sodass $i(a)=b$. Also gilt $i:(Ba,Aa)\cong(Bb,Ab)$, da außerdem die Erweiterungen $(Ba,Aa)\subseteq(B_1,A_1)$ und $(Bb,Ab)\subseteq(B_2,A_2)$ frei sind nach Lemma \ref{Unabhängigkeitsregeln} (6.), ist $i$ im Back\&Forth-System, also elementar, also haben $a$ und $b$ denselben $\lingua_P$-Typen über $B$.
\end{proof}

\begin{corollary}
	Wenn man sich solch ein Paar $(a,b)$ beliebig wählt (z.B. $a=b=0$), sind in dem Typen auch die parameterfreien $(\lingua_P)_B$-Formeln, die in $(B_1,A_1)$ bzw. $(B_2,A_2)$ gelten. Also gelten dieselben Formeln, was als $(B_1,A_1)\equiv_B(B_2,A_2)$ geschrieben wird.
\end{corollary}